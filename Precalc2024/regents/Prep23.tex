\documentclass[12pt, twoside]{article}
\documentclass[12pt, twoside]{article}
\usepackage[letterpaper, margin=1in, headsep=0.2in]{geometry}
\setlength{\headheight}{0.6in}
%\usepackage[english]{babel}
\usepackage[utf8]{inputenc}
\usepackage{microtype}
\usepackage{amsmath}
\usepackage{amssymb}
%\usepackage{amsfonts}
\usepackage{siunitx} %units in math. eg 20\milli\meter
\usepackage{yhmath} % for arcs, overparenth command
\usepackage{tikz} %graphics
\usetikzlibrary{quotes, angles}
\usepackage{graphicx} %consider setting \graphicspath{{images/}}
\usepackage{parskip} %no paragraph indent
\usepackage{enumitem}
\usepackage{multicol}
\usepackage{venndiagram}

\usepackage{fancyhdr}
\pagestyle{fancy}
\fancyhf{}
\renewcommand{\headrulewidth}{0pt} % disable the underline of the header
\raggedbottom
\hfuzz=2mm %suppresses overfull box warnings

\usepackage{hyperref}
\usepackage{float}

\title{Algebra 2}
\author{Chris Huson}
\date{June 2024}

\fancyhead[LE]{\thepage}
\fancyhead[RO]{\thepage \\ Name: \hspace{1.5cm} \,\\}
\fancyhead[LO]{BECA/Huson/Algebra 2: Regents Preparation \\* 5 June 2024}

\begin{document}
\subsubsection*{Prep \#23 Calculator solutions}
\begin{enumerate}[itemsep=0.5cm]

\item Simplify each complex expression to the form $a+bi$, with real numbers $a$ and $b$.
\begin{multicols}{2}
\begin{enumerate}
    \item $\frac{1}{2}(8+2i)(3\sqrt{-48})=$
    \item $(2+3i)^2-2i=$
\end{enumerate}
\end{multicols} \vspace{4cm}

\item Solve each equation. Write your solution in $a+bi$ form.
\begin{multicols}{2}
\begin{enumerate}[itemsep=0.5cm]
    \item $x^2-3x+6=0$
    \item $2x^2-6x+7=0$
\end{enumerate}
\end{multicols} \vspace{4cm}

\item Solve each system of equations.
\begin{multicols}{2}
    \begin{enumerate}
        \item 
        \begin{align*}
            2x+5y +2z &= -38 \\
            3x-2y +4z &= 17 \\
            -6x +y -7z &= -12 \\
        \end{align*}
        \item 
        \begin{align*}
            3x -9z &= 33 \\
            7x-4y -z &= -15 \\
            4x+6y +5z &= -6 \\
        \end{align*}
    \end{enumerate}
\end{multicols}

\newpage
\item Factor completely $2d^4+6d^3-18d^2-54d$. \vspace{4cm}

\item Determine which expressions are equivalent to $\displaystyle \frac{x^3+2x^2+x+6}{x+2}$. \\[0.25cm]
    (hint: substitute $x=0$ and $x=1$)
    \begin{multicols}{2}
    \begin{enumerate}
        \item $x^2+3$
        \item $\displaystyle x^2+1 +\frac{4}{x+2}$
        \item $2x^2+x+6$
        \item $\displaystyle 2x^2+1 + \frac{4}{x+2}$
    \end{enumerate}
    \end{multicols} \vspace{3cm}

\item Convert between radical and rational exponent forms. (assume $x > 0$)
    \begin{multicols}{2}
    \begin{enumerate}
        \item $\displaystyle \frac{(4x^2)^{\frac{5}{2}}}{x^{3}} =$
        \item $\displaystyle \frac{4\sqrt{x^5}}{\sqrt[4]{16x^2}} = $
    \end{enumerate}
    \end{multicols} \vspace{2cm}

\newpage
\item Write an explicit formula for the sequence $\frac{27}{8}$, $\frac{9}{4}$, $\frac{3}{2}$, $1$, $\ldots$ \vspace{3cm}

\item Write a recursive formula for the sequence 1.55, 2.05, 2.55, 3.05, $\ldots$ \vspace{3cm}


\item Given the sequence beginning  $4$, 2, $1$, $\frac{1}{2}$, $\ldots$, find the sum of the first 7 terms, rounded to the \emph{nearest hundredth}. \vspace{3cm}

\item The first two terms of an arithmetic sequence are shown in the table. Complete the table and write a recursive definition for the sequence.
\begin{center}
\begin{tabular}{|p{1cm}|p{1cm}|p{1cm}|p{1cm}|p{1cm}|p{1cm}|}
    \hline
    $n$ & 1 & 2 & 3 & 4 & 5 \\
    \hline
    $a_n$ & 3 & 9 & & & \\[0.25cm]
    \hline
\end{tabular}
\end{center}

\newpage
\item Given events $A$ and $B$, such that $P(A) = 0.6$, $P(B) = 0.5$, and $P(A \cap B) = 0.3$, determine whether $A$ and $B$ are independent or dependent. \vspace{3cm}

\item The set of data in the table below shows the results of a survey on the number of messages that people of different ages text on their cell phones each month.
\begin{center}
    \begin{tabular}{|c|c|c|c|}
        \hline
        Age Group & 0-10 & 11-50 & Over 50 \\
        \hline
        15-18 & 4 & 37 & 68 \\[0.25cm]
        \hline
        19-22 & 6 & 25 & 87 \\[0.25cm]
        \hline
        23-60 & 25 & 47 & 157 \\[0.25cm]
        \hline
    \end{tabular}
\end{center}
If a person from this survey is selected at random, what is the probability that the person texts over 50 messages per month given that the person is between the ages of 23 and 60?  \vspace{3cm}

\item The lifespan of a 60-watt lightbulb produced by a company is
normally distributed with a mean of 1450 hours and a standard
deviation of 8.5 hours. If a 60-watt lightbulb produced by this
company is selected at random, what is the probability that its lifespan
will be between 1440 and 1465 hours? \vspace{3cm}



\end{enumerate}
\end{document}