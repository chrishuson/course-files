\documentclass[12pt, twoside]{article}
\documentclass[12pt, twoside]{article}
\usepackage[letterpaper, margin=1in, headsep=0.2in]{geometry}
\setlength{\headheight}{0.6in}
%\usepackage[english]{babel}
\usepackage[utf8]{inputenc}
\usepackage{microtype}
\usepackage{amsmath}
\usepackage{amssymb}
%\usepackage{amsfonts}
\usepackage{siunitx} %units in math. eg 20\milli\meter
\usepackage{yhmath} % for arcs, overparenth command
\usepackage{tikz} %graphics
\usetikzlibrary{quotes, angles}
\usepackage{graphicx} %consider setting \graphicspath{{images/}}
\usepackage{parskip} %no paragraph indent
\usepackage{enumitem}
\usepackage{multicol}
\usepackage{venndiagram}

\usepackage{fancyhdr}
\pagestyle{fancy}
\fancyhf{}
\renewcommand{\headrulewidth}{0pt} % disable the underline of the header
\raggedbottom
\hfuzz=2mm %suppresses overfull box warnings

\usepackage{hyperref}
\usepackage{siunitx}

\title{IB Mathematics}
\author{Chris Huson}
\date{May 2024}

%\fancyhead[LE]{\thepage}
\fancyhead[RO]{\thepage \\ Name: \hspace{1cm} \,\\}
\fancyhead[LO]{BECA/Huson/Algebra II: Regents Prep \\* 8 May 2024}

\begin{document}

\subsubsection*{Practice Regents problems \#8}
AII-F.BF.6 Represent and evaluate the sum of a finite arithmetic
or finite geometric series, using summation (sigma) notation. For geometric series:
$$\sum_{k=1}^{n} a_k = a_1 + a_2 + \ldots + a_n = a_1 \left( \frac{1-r^n}{1-r} \right)$$

\begin{enumerate}
\item Given the sequence $a$: $4 \frac{1}{2}$, 6, 8, $10 \frac{2}{3}$, $\ldots$
    %$\frac{3}{2}$, 3, $\frac{9}{2}$, 6, $\ldots$
\begin{enumerate}[itemsep=2cm]
    \item State whether the sequence is arithmetic, geometric, or neither. Justify your answer by showing the calculation of the common difference $d$ or ratio $r$.
    \item Write a recursive formula for $a$.
    \item Write an explicit formula for the sequence.
    \item Find the sum of the first eight terms the sequence.
\end{enumerate} \vspace{3cm}

\item Express each of the following in simplest radical form. 
\begin{multicols}{2}
    \begin{enumerate}
        \item $(4x)^{\frac{1}{2}}$
        \item $9x^{-\frac{1}{2}}$
    \end{enumerate}
\end{multicols}
\vspace{3cm}

\newpage
AII-F.LE.2: Construct a linear or exponential function symbolically given: a graph, a description of the relationship, or two input-output pairs (include reading these from a table).

\item Two functions are compared, a linear function $f(x)$ and the exponential function $g(x)$.
\begin{enumerate}%[itemsep=1cm]
    \item Fill out the table for $f(x)$ and write an explicit formula for the linear function.
    \begin{center}
    \begin{tabular}{|p{1cm}|p{1cm}|p{1cm}|p{1cm}|p{1cm}|p{1cm}|}
        \hline
        Days & 0 & 1 & 2 & 3 & 4 \\
        \hline
        Area & 30 & & 60 & & \\[0.25cm]
        \hline
    \end{tabular}
    \end{center} \vspace{2cm}
    \item The geometric function is defined by $\displaystyle g(x) = 20 \cdot e^{\frac{x}{2}}$. On the grid below, sketch both functions, $f(x)$ and $g(x)$.
    \begin{center}
        \begin{tikzpicture}[x=1cm, y=0.075cm, xscale=2]
            \draw [thin, color=lightgray, xstep=1cm,ystep=0.75cm] (0,0) grid (4,151);
            \draw [thick, ->] (0,0) -- (+4.3,0) node [below]{$x$};
            \draw [thick, ->] (0,0) -- (0,153) node [right]{$y$};        
            \foreach \x in {0,1,...,4}
                \draw (\x cm,5pt) -- (\x cm,-5pt) node[below] {$\x$};
            \foreach \y in {0,30,...,150}
                \draw[shift={(0,\y)}] (2pt,0pt)--(-2pt,0pt) node[left]{$\y$};
            %\draw [thick, ->, smooth,domain=0.:4.1] plot(\x,{120*(0.333^(\x/3))});
            %\fill (0,90) ellipse [x radius=1pt, y radius=2.5pt]  node [right] {$(0,32)$};
            %\fill (2,30) ellipse [x radius=1pt, y radius=2.5pt] node [above right] {$(1.5,4)$};
        \end{tikzpicture}
        \end{center}
    \item Mark the intersection of the two functions on the graph as an ordered pair, rounding to the \emph{nearest tenth}.
\end{enumerate}


\end{enumerate}
\end{document}