\documentclass[12pt, twoside]{article}
% \documentclass[12pt, twoside]{article}
\usepackage[letterpaper, margin=1in, headsep=0.2in]{geometry}
\setlength{\headheight}{0.6in}
%\usepackage[english]{babel}
\usepackage[utf8]{inputenc}
\usepackage{microtype}
\usepackage{amsmath}
\usepackage{amssymb}
%\usepackage{amsfonts}
\usepackage[nomessages]{fp} %\FPeval{\var-name}{2*sin(pi/6)}
\usepackage{siunitx} %units in math. eg 20\milli\meter
\usepackage{yhmath} % for arcs, overparenth command
\usepackage{tikz} %graphics
\usetikzlibrary{quotes, angles, arrows, arrows.meta}
\usepackage{graphicx} %consider setting \graphicspath{{images/}}
\usepackage{parskip} %no paragraph indent
\usepackage{enumitem}
\usepackage{multicol}
\usepackage{venndiagram}

\usepackage{fancyhdr}
\pagestyle{fancy}
\fancyhf{}
\renewcommand{\headrulewidth}{0pt} % disable the underline of the header
\raggedbottom
\hfuzz=2mm %suppresses overfull box warnings

\usepackage{hyperref}
\usepackage{float}

\title{Algebra 2}
\author{Chris Huson}
\date{December 2023}

\fancyhead[RO]{\\ Name: \hspace{4cm} \,\\}
\fancyhead[LO]{BECA / Huson / Algebra 2: Polynomials \\* 12 December 2023}

\begin{document}

\subsubsection*{2.21 Homework: Polynomials exam review}
\begin{enumerate}
\item Which expression is equivalent to $2(5x-2)(x+1)(x-3)$? \vspace{0.25cm}
    \begin{enumerate}
        \item $5x^3-24x^2-22x-12$ 
        \item $10x^3-24x^2-22x+6$ 
        \item $2x^3-24x^2-22x+12$ 
        \item $10x^3-24x^2-22x+12$ 
    \end{enumerate} \vspace{0.5cm}

\item The polynomial $p$ is a function of $x$. The graph of $p$ has three zeros at $7$, $\frac{2}{3}$, and $-1$. Select $\bf{all}$ the expressions that could represent $p$. \vspace{0.25cm}
    \begin{multicols}{2}
    \begin{enumerate}
        \item $(x-7)(x-\frac{2}{3})(x+1)$
        \item $(x-7)(3x-2)(x-1)$
        \item $3(x-7)(x-\frac{2}{3})(x+1)$
        \item $3x(x+7)(x+\frac{2}{3})(x-1)^2$
        \item $(x-7)(x+\frac{2}{3})(x-1)$
        \item $(x-7)(3x-2)(x+1)$
        \item $3(x-7)(x-\frac{2}{3})(x-1)$
        \item $3x(x+7)(x-\frac{2}{3})(x+1)^2$
    \end{enumerate}
    \end{multicols}
        \vspace{0.5cm}

\item Let $f$ be a polynomial function of $x$ where $f(x)=4x^3-11x^2-6x+9$. If $x-3$ is a factor of $f$, write an equation for $f$ as a product of linear factors.
\vspace{5cm}

\newpage
\item Let $P$ be a polynomial function of $x$, and $P(x)=x^3+dx^2-5x+6$. If $x-1$ is a factor of $P$, what is the value of $d$? Explain or show how you know.
\vspace{6cm}

\item Let $j(x)=-x(x+4)(x-3)^2$ be a polynomial function. 
    \begin{center}
    \begin{tikzpicture}[xscale=0.7, yscale=0.7]
        \draw [thick, ->] (-7.2,0) -- (7.5,0) node [above] {$x$};
        \draw [thick, ->] (0,-6.2)--(0,6.5) node [right] {$y$};
        \foreach \x in {-7,...,7} \draw (\x cm,5pt) -- (\x cm,-5pt);
    \end{tikzpicture}
    \end{center}
    \begin{enumerate}[itemsep=0.25cm]
        \item Sketch a graph of the function.
        \item Name all horizontal and vertical intercepts of the graph.
        \item State the end behavior of $j$.
    \end{enumerate}
        

\end{enumerate}
\end{document}