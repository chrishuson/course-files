\documentclass[12pt, twoside]{article}
\usepackage[letterpaper, margin=1in, headsep=0.5in]{geometry}
\usepackage[english]{babel}
\usepackage[utf8]{inputenc}
\usepackage{amsmath}
\usepackage{amsfonts}
\usepackage{amssymb}
\usepackage{tikz}
%\usetikzlibrary{quotes, angles}

\usepackage{graphicx}
\usepackage{enumitem}
\usepackage{multicol}

\usepackage{fancyhdr}
\pagestyle{fancy}
\fancyhf{}
\renewcommand{\headrulewidth}{0pt} % disable the underline of the header

\fancyhead[RE]{\thepage}
\fancyhead[RO]{\thepage \\ Name: \hspace{3cm}}
\fancyhead[L]{BECA / Dr. Huson / 12.1 IB Math SL\\* 22 January 2019}

\begin{document}
\subsubsection*{Spiral Review: 1-1 P1 (No Calculator) Algebra Sequences}
 \begin{enumerate}

  \item 17M.1.sl.TZ2.1
  \begin{enumerate}
    \item In an arithmetic sequence, the first term is 3 and the second term is 7.\\
    Find the common difference. [2 marks]
    \item Find the tenth term. [2 marks]
    \item Find the sum of the first ten terms of the sequence. [2 marks]
  \end{enumerate}

  \item 17N.1.sl.TZ0.2
  \begin{enumerate}
    \item In an arithmetic sequence, the first term is 8 and the second term is 5.\\
    Find the common difference. [2 marks]
    \item Find the tenth term. [2 marks]
    \item Find the sum of the first ten terms. [2 marks]
  \end{enumerate}

  \item 14N.1.sl.TZ0.2
  \begin{enumerate}
    \item In an arithmetic sequence, the first term is 2 and the second term is 5.\\
    Find the common difference. [2 marks]
    \item Find the eighth term. [2 marks]
    \item Find the sum of the first eight terms of the sequence. [2 marks]
  \end{enumerate}

  \item 14M.1.sl.TZ1.2
  \begin{enumerate}
    \item In an arithmetic sequence, the third term is 10 and the fifth term is 16.\\
    Find the common difference. [2 marks]
    \item Find the first term. [2 marks]
    \item Find the sum of the first 20 terms of the sequence. [3 marks]
  \end{enumerate}

  \item 11M.1.sl.TZ2.1
  \begin{enumerate}
    \item In an arithmetic sequence, $u_1=2$ and $u_3=8$.\\
    Find $d$. [2 marks]
    \item Find $u_{20}$. [2 marks]
    \item Find $S_{20}$. [2 marks]
  \end{enumerate}

  \item Three consecutive terms of a geometric sequence are $x-3$, $6$ and $x+2$.\\
  Find the possible values of $x$.\\
  16M.1.sl.TZ2.4  [6 marks]

  \item 08N.1.sl.TZ0.1
  \begin{enumerate}
    \item Consider the infinite geometric sequence $3, 3(0.9), 3(0.9)^2, 3(0.9)^3,\dots$.\\
    Write down the 10th term of the sequence. Do not simplify your answer. [1 mark]
    \item Find the sum of the infinite sequence. [4 marks]
  \end{enumerate}

  \item 10N.1.sl.TZ0.1
  \begin{enumerate}
    \item The first three terms of an infinite geometric sequence are 32, 16 and 8.\\
    Write down the value of r. [2 marks]
    \item Find $u_6$. [2 marks]
    \item Find the sum to infinity of this sequence. [2 marks]
  \end{enumerate}

  \item 16M.1.sl.TZ1.4
  \begin{enumerate}
    \item Consider the following sequence of figures.\\
    Figure 1 contains 5 line segments.\\
    Given that Figure $n$ contains 801 line segments, show that $n=200$.[3 marks]
    \item Find the total number of line segments in the first 200 figures. [3 marks]
  \end{enumerate}

  \item 16M.1.sl.TZ1.4
  \begin{enumerate}
    \item Consider the arithmetic sequence $2, 5, 8, 11, \dots$.\\
    Find $u_{101}$.[3 marks]
    \item Find the value of $n$ so that $u_n = 152$. [3 marks]
  \end{enumerate}



  \newpage




\end{enumerate}
\end{document}
