
\documentclass[12pt, twoside]{article}
\usepackage[letterpaper, margin=1in, headsep=0.5in]{geometry}
\usepackage[english]{babel}
\usepackage[utf8]{inputenc}
\usepackage{amsmath}
\usepackage{amsfonts}
\usepackage{amssymb}
\usepackage{tikz} %graphics
\usepackage{yhmath} % for arcs, overparenth command
\usetikzlibrary{quotes, angles}
\usepackage{graphicx}
\usepackage{enumitem}
\usepackage{multicol}

\usepackage{fancyhdr}
\pagestyle{fancy}
\fancyhf{}
\renewcommand{\headrulewidth}{0pt} % disable the underline of the header
\raggedbottom

\usepackage[nomessages]{fp}% http://ctan.org/pkg/fp


\begin{document}
usepackage[nomessages]{fp}\% http://ctan.org/pkg/fp

\begin{enumerate}
\item \FPeval{\result}{(5*6)}
$5*6=\result$
\par This is the result: $\result$ (use clip or round)

\item \FPeval{\result}{clip(5+6)}
$5+6=\result$
\par This is the result: $\result$

\item \FPeval{\sumA}{clip(5+6)} \par
$5+6=\sumA$
\par This is the sumA: $\sumA$

\item \FPeval{\result}{round(2+3/5*pi,5)}
$2+3/5\times\pi=\result$
\par This is the result: $\result$

\item \FPeval{\result}{round(36*3.14159/180,5)}
$36*3.14159/180=\result$
\par This is the result: $\result$

\item \FPeval{\result}{(36*pi/180)}
$36*\pi/180=\result$
 \par This is the result: $\result$

\item \FPsin{\result2}{(36*pi/180)}%{{36*3.14159/180}}
$\sin 36 = \result2$
\par This is the result2: $\result2$

\item \FPsin{\temp2}{(pi/6)} 
$\sin \pi/6 = \temp2$

\item \FPsin{\temp3}{30} 
$\sin 30 = \temp3$

\item \FPeval{\temp4}{(pi/6)} 
$\sin \pi/6 = \temp4$

\item \FPeval{\temp5}{round(\temp4,6)} 
$round \pi/6 = \temp5$  rounded

\item \FPsin{\temp6}{(\temp5)} 
$\sin temp5 = \temp6$


\item \FPeval{\temp7}{sin(0.5236)} 
$\sin (0.5236) = \temp7$

\item \FPeval{\temp8}{sin(pi/6)} 
$\sin \pi/6 = \temp8$

\item \FPeval{\temp9}{sin(36*pi/180)} 
$\sin 36*\pi/180 = \temp9$

\FPeval{\inscr}{0.5*10*sin(36*pi/180)} 
Inscribed area: $\inscr$ \par 

\FPeval{\mul}{1/cos(18*pi/180)}
Cos inv multiplier: $\mul$

\FPeval{\circum}{\inscr * \mul * \mul} 
Circumscribed area: $\circum$

\end{enumerate}
\end{document}