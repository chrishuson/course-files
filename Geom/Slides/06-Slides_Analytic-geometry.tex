\documentclass{beamer}
\usepackage{geometry}
\usepackage[english]{babel}
\usepackage[utf8]{inputenc}
\usepackage{amsmath}
\usepackage{amsfonts}
\usepackage{amssymb}
\usepackage{tikz}
\usetikzlibrary{quotes, angles}
\usepackage{graphicx}

%\usepackage{pgfplots}
%\pgfplotsset{width=10cm,compat=1.9}
%\usepackage{pgfplotstable}

\usepackage{fancyhdr}
\pagestyle{fancy}
\setlength{\headheight}{12pt}%doesn't seem to fix warning
\fancyhf{}

%\rhead{\small{10 September 2018}}
\lhead{\small{BECA / Dr. Huson / Geometry Unit 6: Analytic Geometry}}

\renewcommand{\headrulewidth}{0pt}

\title{Mathematics Class Slides}
\subtitle{Bronx Early College Academy}
\author{Chris Huson}
\date{25 November 2019}

\begin{document}
\frame{\titlepage}
\section[Outline]{}
\frame{\tableofcontents}

\section{6.1 Intro to the coordinate plane and linear functions, 25 November}
  \frame
  {
    \frametitle{GQ: How do we plot lines on the coordinate plane?}
    \framesubtitle{CCSS: HSG.CO.A.1 Know precise geometric definitions \hfill \alert{6.1 Monday 25 Nov}}

    \begin{block}{Do Now: Plotting points and lines}
    \begin{enumerate}
      \item Modeling geometric situations with an algebraic equation
      \item Slope-intercept form of linear equations
      \item Dilation of a line centered at the origin
    \end{enumerate}
    \end{block}
    Review exam results \\
    Lesson: Perpendicular and parallel slopes \\*[5pt]
    Homework: Test corrections due \alert{tomorrow}
  }

\section{6.2 Laptop practice - Geogebra graphing functions on coordinate plane, 26 November}
\frame
{
\frametitle{GQ: How do we communicate examples of dilations?}
\framesubtitle{CCSS: HSG.CO.A.1 Know precise geometric definitions \hfill \alert{6.2 Tuesday 26 Nov}}

\begin{block}{Do Now Quiz: Deltamath dilation calculations \\[0.5cm]
  Project: Examples of dilation on the coordinate plane}
\begin{enumerate}
  \item Use Geogebra \& MS Word to write a 1+ page paper
  \item The Geogebra \emph{Graphing Calculator} works with $x$-$y$ coordinates
  \item Include the following graphs
  \begin{enumerate}
    \item A triangle in standard position dilated centered at the origin
    \item A polygon dilated with a center not on the origin
    \item A line and its image after a dilation centered at the origin\\
    Spicy: State the equations of the two lines
  \end{enumerate}
  \item Use the equation editor and captions. Follow MLA.
  \item Email pdf and MS Word .docx files, with the subject line Dilation assignment
\end{enumerate}
\end{block}
10.1 meets in Room 414 first period (advisory schedule)\\
Homework: Complete exploration paper (10:00 deadline)
}

\section{6.3 Coordinate geometry practice, 25 November}
  \frame
  {
    \frametitle{GQ: How do we plot lines on the coordinate plane?}
    \framesubtitle{CCSS: HSG.CO.A.1 Know precise geometric definitions \hfill \alert{6.3 Wednesday 27 Nov}}

    \begin{block}{Do Now: Plotting points and lines}
    \begin{enumerate}
      \item Modeling geometric situations with an algebraic equation
      \item Slope-intercept form of linear equations
      \item Dilation of a line centered at the origin
    \end{enumerate}
    \end{block}
    Review exam results \\
    Lesson: Perpendicular and parallel slopes \\*[5pt]
    Homework: Test corrections due \alert{tomorrow}
  }

\end{document}

