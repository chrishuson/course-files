\documentclass{beamer}
\usepackage{geometry}
\usepackage[english]{babel}
\usepackage[utf8]{inputenc}
\usepackage{amsmath}
\usepackage{amsfonts}
\usepackage{amssymb}
\usepackage{tikz}
\usetikzlibrary{quotes, angles}
\usepackage{graphicx}
\usepackage{multicol}

%\usepackage{pgfplots}
%\pgfplotsset{width=10cm,compat=1.9}
%\usepackage{pgfplotstable}

\setlength{\headheight}{26pt}%doesn't seem to fix warning

\usepackage{fancyhdr}
\pagestyle{fancy}
\fancyhf{}

%\rhead{\small{3 September 2019}}
\lhead{\small{BECA / Dr. Huson / Geometry Unit 2}}

\renewcommand{\headrulewidth}{0pt}

\title{Mathematics Class Slides}
\subtitle{Bronx Early College Academy}
\author{Christopher J. Huson PhD}
\date{13 October 2020}

\begin{document}
\frame{\titlepage}
\section[Outline]{}
\frame{\tableofcontents}

\section{2.1 Angle vocabulary, 13 October}
  \frame
  {
    \frametitle{GQ: How do we measure angles?}
    \framesubtitle{CCSS: HSG.CO.A.1 Know precise geometric definitions  \hfill \alert{2.1 Tuesday 13 Oct}}
  
    \begin{block}{Do Now: Write down angle definitions}
    \begin{enumerate}
        \item \emph{Right angles} measure $90^\circ$
        \item \emph{Perpendicular} lines meet at right angles. $\overline{AB} \perp \overline{CD}$
        \item \emph{Acute} angles are $< 90^\circ$
        \item \emph{Obtuse} angles are $90^\circ < \angle m < 180^\circ$
        \item A \emph{straight line} or angle measures $180^\circ$
    \end{enumerate}
    \end{block}
    \begin{multicols}{2}
    \begin{tikzpicture}[scale=0.8]
      \draw [<->, thick]
        (3,-1) coordinate (a) node[right] {P}
        -- (0,0) coordinate (b) node[left] {Q}
        -- (2,2) coordinate (c) node[above right] {R}
        pic["1", <->, draw=black, angle eccentricity=1.2, angle radius=1cm]
        {angle=a--b--c};
    \end{tikzpicture} \\
    Point $Q$ is the \emph{vertex} \\[0.25cm]
    The sides or \emph{legs} are $\overrightarrow{QR}$, $\overrightarrow{QP}$ \\[0.25cm]
    Written $\angle 1$, $\angle Q$, or $\angle PQR$
  \end{multicols}
  }

  \section{2.1 Angle relationships, 14 October}
  \frame
  {
    \frametitle{GQ: How do we relate angles to each other?}
    \framesubtitle{CCSS: HSG.CO.A.1 Know precise geometric definitions  \hfill \alert{2.2 Thursday 15 Oct}}
  
    \begin{block}{Do Now: Write down angle definitions}
    \begin{enumerate}
        \item \emph{Adjacent} measure $90^\circ$
        \item \emph{Complementary} lines meet at right angles. $\overline{AB} \perp \overline{CD}$
        \item \emph{Supplementary} angles are $< 90^\circ$
        \item \emph{Vertical} or opposite angles are $90^\circ < \angle m < 180^\circ$
        \item A \emph{straight line} or angle measures $180^\circ$
    \end{enumerate}
    \end{block}
    \begin{multicols}{2}
    \begin{tikzpicture}[scale=0.8]
      \draw [<->, thick]
        (3,-1) coordinate (a) node[right] {P}
        -- (0,0) coordinate (b) node[left] {Q}
        -- (2,2) coordinate (c) node[above right] {R}
        pic["1", <->, draw=black, angle eccentricity=1.2, angle radius=1cm]
        {angle=a--b--c};
    \end{tikzpicture} \\
    Point $Q$ is the \emph{vertex} \\[0.25cm]
    The sides or \emph{legs} are $\overrightarrow{QR}$, $\overrightarrow{QP}$ \\[0.25cm]
    Written $\angle 1$, $\angle Q$, or $\angle PQR$
  \end{multicols}
  }


  \end{document}