\documentclass{beamer}
\usepackage{geometry}
\usepackage[english]{babel}
\usepackage[utf8]{inputenc}
\usepackage{amsmath}
\usepackage{amsfonts}
\usepackage{amssymb}
\usepackage{tikz}
\usetikzlibrary{quotes, angles}
\usepackage{graphicx}
\usepackage{multicol}

%\usepackage{pgfplots}
%\pgfplotsset{width=10cm,compat=1.9}
%\usepackage{pgfplotstable}

\setlength{\headheight}{26pt}%doesn't seem to fix warning

\usepackage{fancyhdr}
\pagestyle{fancy}
\fancyhf{}

%\rhead{\small{3 September 2019}}
\lhead{\small{BECA / Dr. Huson / Geometry Unit 1}}

\renewcommand{\headrulewidth}{0pt}

\title{Mathematics Class Slides}
\subtitle{Bronx Early College Academy}
\author{Christopher J. Huson PhD}
\date{21-25 September 2020}

\begin{document}
\frame{\titlepage}
\section[Outline]{}
\frame{\tableofcontents}

\section{1.1 1st day of Geometry, Segment addition, 21 Sept}
\frame
{
  \frametitle{GQ: How do we define the basic elements of geometry?}
  \framesubtitle{CCSS: HSG.CO.A.1 Know precise geometric definitions \hfill \alert{1.1 Monday 21-22 Sept}}

  Welcome back to school
  \begin{block}{Do Now: Algebra skills check}
  \begin{enumerate}
      \item Remote learning attendance
      \item Take out notebooks (or blank paper)
      \item Complete Do Now on Google Classroom
  \end{enumerate}
  \end{block}
  Supply list: Composition book, folder, looseleaf, pencils \& pens, \\*
  compass and ruler, calculator \\
  Lesson: Points, line segments, length; Segment addition postulate \\
  Homework: Begin Khan Academy unit (due Friday)
}

  \frame
  {
    \frametitle{Take class notes in a composition book}
    \begin{block}{Use this notebook format (required)}
      \begin{enumerate}
        \item In the front, write your name, my contact info, your passwords
        \item Each page in the top left corner: \\ \qquad First+Last Name \\
        \qquad 21 September 2020 \\ \qquad 1.1 Segment addition postulate \vspace{0.25cm}
        \item Copy definitions using your own words
        \item Write down example diagrams and problems
      \end{enumerate}
      \end{block}
    Point: a location, a dot, has no size; label with capital letter, $P$ \\[0.25cm]
    Line segment: two points and all the points between them; label with \emph{end points} and a bar, $\overline{AB}$ \\
  }

  \frame
  {
    \frametitle{Example: Points and line segments} \vspace{0.2cm}
    Shown points $P$, $A$, $B$, $C$, line segments $\overline{AB}$, $\overline{BC}$\\[0.15in]
       \begin{tikzpicture}
        \draw [fill] (1,2) circle [radius=0.05] node[below]{$P$};
        \draw [-, thick] (0,1)--(3,1);
        \draw [fill] (0,1) circle [radius=0.05] node[below]{$A$};
        \draw [fill] (3,1) circle [radius=0.05] node[below]{$B$};
        \draw [-, thick] (3,0)--(7,0);
        \draw [fill] (3,0) circle [radius=0.05] node[below]{$B$};
        \draw [fill] (7,0) circle [radius=0.05] node[below]{$C$};
      \end{tikzpicture} \\ \vspace{0.5cm}
      Given $AB=3$, $BC=4$.\\[0.5cm]
      Notation: the length of a line segment is written as the two end points without a bar over them, $AB$.
  }

  \frame
  {
    \frametitle{Example: Points and line segments} \vspace{0.2cm}
    \framesubtitle{Segment Addition Postulate}
    Shown \emph{collinear} points $A$, $B$, $C$. Given $AB=3$, $BC=4$. \\[0.15in]
    Find $AC$. \\[0.25in]
       \begin{tikzpicture}
        \draw [fill] (0,0) circle [radius=0.05] node[below]{$A$};
        \draw [-, thick] (0,0)--(7,0);
        \draw [fill] (3,0) circle [radius=0.05] node[below]{$B$};
        \draw [fill] (7,0) circle [radius=0.05] node[below]{$C$};
        \node at (1.5,0) [above]{$3$};
        \node at (5,0) [above]{$4$};
        \draw [<->, dashed] (0,-1)--(7,-1);
        \node at (3.5,-1) [below]{$?$};
      \end{tikzpicture} \\ \vspace{1cm}
      
      Definition: Points are \emph{collinear} when they lie on a straight line.
  }

  \frame
  {
    \frametitle{Example 2: Points and line segments} \vspace{0.2cm}
    \framesubtitle{Segment Addition Postulate}
    Given collinear points $Q$, $R$, $S$, with $QR=11$, $QS=20$. \\[0.15in]
    Find $RS$. \\[0.25in]
       \begin{tikzpicture}
        \draw [fill] (0,0) circle [radius=0.05] node[below]{$Q$};
        \draw [-, thick] (0,0)--(7,0);
        \draw [fill] (3.8,0) circle [radius=0.05] node[below]{$R$};
        \draw [fill] (7,0) circle [radius=0.05] node[below]{$S$};
        \node at (1.7,0) [above]{$11$};
        \node at (5.2,0) [above]{$x$};
        \draw [<->, dashed] (0,-1)--(7,-1);
        \node at (3.5,-1) [below]{$20$};
      \end{tikzpicture} \\ \vspace{0.2cm}
      \begin{enumerate}
        \item How would you check your answer?
        \item Which equation represents the situation?
        \begin{multicols}{2}
          $11 + x = 20$ \\
          $x = 20 - 11$
        \end{multicols}
      \end{enumerate}
      
  }

  \frame
  {
    \frametitle{Example 3: Segment addition postulate}
    
    Given $\overline{JKL}$, $JK=2x+3$, $KL=5$, $JL=12$. Find ${x}$.\\[0.15in]
       \begin{tikzpicture}
        \draw [-, thick] (0,0)--(7,0);
        \draw [fill] (0,0) circle [radius=0.05] node[below]{$J$};
        \draw [fill] (4,0) circle [radius=0.05] node[below]{$K$};
        \draw [fill] (7,0) circle [radius=0.05] node[below]{$L$};
        \node at (1.7,0) [above]{$2x+3$};
        \node at (5.5,0) [above]{$5$};
        \draw [<->, dashed] (0,-1)--(7,-1);
        \node at (3.5,-1) [below]{$12$};
      \end{tikzpicture} %\vspace{1cm}
  \begin{enumerate}
      \item Write down an equation to represent the situation. \vspace{1cm}
      \item Solve for $x$. \vspace{1cm}
      \item Check your answer.
    \end{enumerate}
  }

  \frame
  {
    \frametitle{Example 4: Segment addition postulate}
    
    Given $\overline{ABC}$, $AB=3x-7$, $BC=x+5$, $AC=14$. Find ${AB}$.\\[0.5in]
       \begin{tikzpicture}
        \draw [-, thick] (0,0)--(7,0);
        \draw [fill] (0,0) circle [radius=0.05] node[below]{$A$};
        \draw [fill] (3,0) circle [radius=0.05] node[below]{$B$};
        \draw [fill] (7,0) circle [radius=0.05] node[below]{$C$};
      \end{tikzpicture} \vspace{1cm}
  \begin{enumerate}
      \item<2-> Sketch and label the situation\\
      \item<2-> Write a geometric equation\\
      \item<2-> Substitute algebraic values\\
      \item<2-> Solve for the unknown\\
      \item<2-> Answer the question\\
      \item<2-> Check your answer
    \end{enumerate}
  }

  \section{1.2 Segment addition, midpoint, 23 Sept}
  \frame
  {
    \frametitle{GQ: How do we solve for segment lengths?}
    \framesubtitle{CCSS: HSG.CO.A.1 Know precise geometric definitions  \hfill \alert{1.2 Wedn 23-24 Sept}}
  
    \begin{block}{Do Now: Complete Google Form in G-Classroom}
    %\begin{enumerate}
        %\item
    %\end{enumerate}
    \end{block}
    Lesson: \\ Point, line segment, end point, collinear, distance or length; \\ line, ray, plane, coplanar, congruent, angle, vertex \\*[5pt]
    Midpoints, bisectors, practice segment addition situations
  }

  \frame
  {
    \frametitle{How do we add lengths? Segment addition postulate}
    
    Given $\overline{ABC}$, $AB=3x-7$, $BC=x+5$, $AC=14$. Find ${AB}$.\\[0.5in]
       \begin{tikzpicture}
        \draw [-, thick] (0,0)--(7,0);
        \draw [fill] (0,0) circle [radius=0.05] node[below]{$A$};
        \draw [fill] (3,0) circle [radius=0.05] node[below]{$B$};
        \draw [fill] (7,0) circle [radius=0.05] node[below]{$C$};
      \end{tikzpicture} \vspace{1cm}
  \begin{enumerate}
      \item<2-> Sketch and label the situation\\
      \item<2-> Write a geometric equation\\
      \item<2-> Substitute algebraic values\\
      \item<2-> Solve for the unknown\\
      \item<2-> Answer the question\\
      \item<2-> Check your answer
    \end{enumerate}
  }

\section{1.3 Equilateral triangle construction, 25, 29 Sept}
\frame
{
  \frametitle{GQ: How do we construct an equilateral triangle?}
  \framesubtitle{CCSS: HSG.CO.D.13 Construct an equilateral triangle \hfill \alert{1.3 Friday 25, 29 Sept}}

  \begin{block}{Do Now: $x=0$ vs $y=0$. Copy into notebook, do problems}
  \begin{enumerate}
      \item $x=0$, starting point, $y$-intercept, $b$, initial condition, $f(0)$
      \item $y=0$, $x$-intercept, the solution, the zeros, $f(x)=0$
  \end{enumerate}
  \end{block}
  Lesson: Circle notation; ``Sketch", ``draw", ``construct"; ``Given"\\[5pt]
  Euclid's first construction
  \begin{enumerate}
      \item Steps in the construction
      \item Logic: Why does it work?
      \item MLA headings: First+Last Name / Dr. Huson \\
      10.x Geometry / 9 September 2019
      \item Assessment criteria: precision, correct \& complete, elegance
  \end{enumerate}
  Homework: Measurement, terminology, and algebra practice\\
  Due: Compass, ruler, protractor, calculator
}

\section{1.5 Angle terminology, 11 Sept}
  \frame
  {
    \frametitle{GQ: How do we measure angles?}
    \framesubtitle{CCSS: HSG.CO.A.1 Know precise geometric definitions \hfill \alert{1.5 Wednesday 11 Sept}}
 
    \begin{block}{Do Now: How big is a football field?}
      \begin{enumerate}
          \item On lined scrap paper, calculate the area of a football field
          \item 100 yards long, $53 \frac{1}{3}$ yards wide
          \item What is the area of the end zone? (10 yards deep)
          \item Spicy: What is the area in square feet?
      \end{enumerate}
      \end{block}
    Lesson: Measuring angles, making angles of a given measure\\
    Angle terminology: legs, vertex, interior, exterior, right, acute, obtuse; adjacent, opposite or vertical angles\\ \vspace{0.5cm}
    Homework: Pretest handout, \alert{Test Friday}
  }
    
\section{1.6 Angle terminology, quiz review, 12 Sept}
\frame
{
  \frametitle{GQ: How do we measure angles?}
  \framesubtitle{CCSS: HSG.CO.A.1 Know precise geometric definitions \hfill \alert{1.6 Thursday 12 Sept}}
  \begin{block}{Do Now handout}
  \begin{enumerate}
      \item Measuring angles
      \item Protractor use
      \item Making angles of a given measure
  \end{enumerate}
  \end{block}
  Angle terminology: legs, vertex, interior, exterior, right, acute, obtuse\\
  Review for \alert{test tomorrow} \\
  Homework: Study for test
}

\end{document}