\documentclass{beamer}
\usepackage{geometry}
\usepackage[english]{babel}
\usepackage[utf8]{inputenc}
\usepackage{amsmath}
\usepackage{amsfonts}
\usepackage{amssymb}
\usepackage{tikz}
\usepackage{graphicx}
\usepackage{venndiagram}

%\usepackage{pgfplots}
%\pgfplotsset{width=10cm,compat=1.9}
%\usepackage{pgfplotstable}

\setlength{\headheight}{26pt}%doesn't seem to fix warning

\usepackage{fancyhdr}
\pagestyle{fancy}
\renewcommand{\headrulewidth}{0pt}

\lhead{\small{BECA / Dr. Huson / 11.1 IB Math - Unit 8 Descriptive Statistics}}

\title{11.1 IB Math - Unit 8 Descriptive Statistics}
  \subtitle{Bronx Early College Academy}
  \author{Christopher J. Huson PhD}
  \date{6 May 2019}

\begin{document}
  \frame{\titlepage}
  \section[Outline]{}
  \frame{\tableofcontents}

\section{10.1 Exponential function \& applications Tuesday 28 May}
  \frame
  {
    \frametitle{GQ: How do we apply geometric growth to situations?}
    \framesubtitle{CCSS: HSG.CO.D.12 Congruence, geometric constructions \hfill \alert{10.1 Tuesday 28 May}}

    \begin{block}{Do Now: Handout}
      \begin{enumerate}
        \item Using scale factors
        \item Real world situations
      \end{enumerate}
    \end{block}
    Guest teacher, Mr. Segal. Applications of exponential functions in finance.\\[0.25cm]
    Homework: Problem set, test corrections due Thursday
  }

\section{10.2 Polynomials introduction Wednesday 29 May}
  \frame
  {
    \frametitle{GQ: How do we work with polynomial functions?}
    \framesubtitle{CCSS: HSG.CO.D.12 Congruence, geometric constructions \hfill \alert{10.2 Wednesday 29 May}}

    \begin{block}{Do Now: Solve for the relevant parameters, $j$, $k$, etc.}
      \begin{enumerate}
        \item $2x^2+7=2x^2+j$
        \item $kx^2+5x+4=3x^2+mx+4$
        \item $x^3+x^2+5x+4=(x+1)(x^2+nx+4)+p$
      \end{enumerate}
    \end{block}
    Polynomial functions\\[0.25cm]
    Homework: Problem set, test corrections due tomorrow
  }

  \section{10.3 Polynomial zeros \& graphs Thursday 30 May}
    \frame
    {
      \frametitle{GQ: How do we work with polynomial functions?}
      \framesubtitle{CCSS: HSG.CO.D.12 Congruence, geometric constructions \hfill \alert{10.3 Thursday 30 May}}

      \begin{block}{Do Now: Solve for the relevant parameters, $j$, $k$, etc.}
        \begin{enumerate}
          \item $2x^2+7=2x^2+j$
          \item $kx^2+5x+4=3x^2+mx+4$
          \item $x^3+3x^2+6x+8=(x+1)(x^2+nx+4)+p$
        \end{enumerate}
      \end{block}
      Polynomial functions\\[0.25cm]
      Homework: Problem set, test corrections due\\
      Reminder: Regents review at Melrose Library 9:00-10:30 Monday
    }

    \section{10.4 Polynomial zeros \& graphs Wednesday 5 June}
      \frame
      {
        \frametitle{GQ: How do we work with polynomial functions?}
        \framesubtitle{CCSS: HSG.CO.D.12 Congruence, geometric constructions \hfill \alert{10.4 Wednesday 5 June}}

        \begin{block}{Do Now: Given the function $f(x)=x^3-3x^2-x+3$}
          \begin{enumerate}
            \item Sketch $f$. Mark the intercepts and extrema (local max, min)
            \item Write $f(x)$ in factored form.
            \item Characterize its \emph{end behavior}
            \item Mark its increasing/decreasing behavior on an axis using plusses and minusus
          \end{enumerate}
        \end{block}
        Review homework\\
        Polynomial functions\\[0.25cm]
        Homework: Problem set\\
        Reminder: Last day for work in this marking period is \alert{Friday}
      }
\end{document}
