\documentclass{beamer}
\usepackage{geometry}
\usepackage[english]{babel}
\usepackage[utf8]{inputenc}
\usepackage{amsmath}
\usepackage{amsfonts}
\usepackage{amssymb}
\usepackage{tikz}
\usepackage{graphicx}
\usepackage{venndiagram}

%\usepackage{pgfplots}
%\pgfplotsset{width=10cm,compat=1.9}
%\usepackage{pgfplotstable}

\setlength{\headheight}{26pt}%doesn't seem to fix warning

\usepackage{fancyhdr}
\pagestyle{fancy}
\renewcommand{\headrulewidth}{0pt}

\lhead{\small{BECA / Dr. Huson / 11.1 IB Math - Unit 8 Descriptive Statistics}}

\title{11.1 IB Math - Unit 8 Descriptive Statistics}
  \subtitle{Bronx Early College Academy}
  \author{Christopher J. Huson PhD}
  \date{6 May 2019}

\begin{document}
  \frame{\titlepage}
  \section[Outline]{}
  \frame{\tableofcontents}

\section{8.1 Introduction and definitions Monday 6 May}
  \frame
  {
    \frametitle{GQ: How do we determine the features of a population?}
    \framesubtitle{HSS.ID.A.1-4 Summarize, represent, and interpret data on a single measurement variable \hfill \alert{8.1 Monday 6 May}}

    \begin{block}{Do Now: Skills Check p. 254}
    \end{block}
    Lesson: Qualitative \& quantitative data, graphing, definitions p.255-9\\[1cm]
    Homework: Exercises 8B p. 259
  }

\section{8.2 Deltamath summary statistics, Tuesday 6 May}
  \frame
  {
    \frametitle{GQ: How do we determine the features of a population?}
    \framesubtitle{CCSS: HSS.ID.A.1-4 Summarize, represent, and interpret data on a single measurement variable \hfill \alert{8.2 Tuesday 7 May}}

    \begin{block}{Deltamath probability practice}
    \end{block}
    Homework: Complete Deltamath exercises
  }

\section{8.3 Central tendency Wednesday 8 May}
  \frame
  {
    \frametitle{GQ: How do we determine the ``center" of a population?}
    \framesubtitle{HSS.ID.A.1-4 Summarize, represent, and interpret data on a single measurement variable \hfill \alert{8.3 Wednesday 8 May}}

    \begin{block}{Do Now: Sequences review}
      \begin{enumerate}
        \item An arithmetic sequence begins $4, k, 10, ...$. Find $k$.
        \item Find the value of the 8th term of the sequence.
        \item The sum of the first $n$ terms in the sequence is 589. Find $n$.
    \end{enumerate}
    \end{block}
    Lesson: Measures of central tendency: mean, median, \& mode p. 260-7\\[0.5cm]
    Homework: Exercises 8C, 8D, 8E. Select an appropriate number of problems.
  }


\end{document}
