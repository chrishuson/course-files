\documentclass[12pt, oneside]{article}
\usepackage[letterpaper, margin=1in, headsep=0.5in]{geometry}
\usepackage[english]{babel}
\usepackage[utf8]{inputenc}
\usepackage{amsmath}
\usepackage{amsfonts}
\usepackage{amssymb}
\usepackage{tikz}
\usetikzlibrary{quotes, angles}
\usepackage{graphicx}
%\usepackage{pgfplots}
%\pgfplotsset{width=10cm,compat=1.9}
%\usepgfplotslibrary{statistics}
%\usepackage{pgfplotstable}
%\usepackage{tkz-fct}
%\usepackage{venndiagram}

\usepackage{fancyhdr}
\pagestyle{fancy}
\fancyhf{}
\rhead{\thepage \\Name: \hspace{1.5in}.\\}
\lhead{BECA / Dr. Huson / 10th Grade Geometry\\* 15 November 2018}

\renewcommand{\headrulewidth}{0pt}

\begin{document}
\subsubsection*{Do Now: Applications of altitude constructions}
Use only a compass and straightedge for these classical constructions.
  \begin{enumerate}

  \item Construct a perpendicular to $\overline{AB}$ though $C$.\\
  Hint: Start with a circle centered on $C$ that intersects $\overleftrightarrow{AB}$ in two places.
    %\hspace{1cm} Given the line  $l$ and point $P$.
    \vspace{2cm}
    \begin{center}
    \begin{tikzpicture}
      \draw [-, thick] (4,0)--(8,0);
      \draw [<->, dashed] (0,0)--(11,0);
      \draw [fill] (4,0) circle [radius=0.05] node[below]{$A$};
      \draw [fill] (8,0) circle [radius=0.05] node[below]{$B$};
      \draw [fill] (7,3) circle [radius=0.05] node[above right]{$C$};
    \end{tikzpicture}
  \end{center} \vspace{2cm}

\subsubsection*{Construct a triangle's orthocenter}

  \item Construct a perpendicular to each of the leg of the triangle from the opposite vertex. Show their intersection, the orthocenter. Hint: you may extend the triangle sides as has been done for you on one side.%\\[0.2cm]
      %\hspace{1cm} Given the line  $l$ and point $P$.
      \vspace{3cm}
      \begin{center}
      \begin{tikzpicture}[scale=0.8]
        \draw [<->, thick] (0,0)--(9,0)--(5.5,5)--cycle;
        \draw [<->, dashed] (-1,0)--(11,0);

        %\draw [fill] (2,3) circle [radius=0.05] node[right]{$P$};
        %\node at (8.5,-0.4){$l$};
        %\draw [fill] (6,0) circle [radius=0.05] node[below]{$Q$};
      \end{tikzpicture}
      \end{center}

  \newpage
  \subsubsection*{Spicy: Construct a hexagon inscribed in a circle}

    \item Construct an equilateral triangle on $\overline{AB}$ by drawing a circle centered on $A$. Continue with a second equilateral triangle on  $\overline{AC}$ by drawing a circle centered on $C$. Work around the circle $B$ four more times to construct the hexagon.
        %\hspace{1cm} Given the line  $l$ and point $P$.
        \vspace{3cm}
        \begin{center}
        \begin{tikzpicture}
          \draw [-, thick] (-6,0) node[left]{$A$}--(0,0);
          \draw  (0,0) circle [radius=6] node[right]{$B$};
          \draw [-, dashed] (120:6) node[above left]{$C$}--(0,0);
          %\node at (8.5,-0.4){$l$};
          %\draw [fill] (6,0) circle [radius=0.05] node[below]{$Q$};
        \end{tikzpicture}
        \end{center}

\end{enumerate}
\end{document}
