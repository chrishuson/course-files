\documentclass{beamer}
\usepackage{geometry}
\usepackage[english]{babel}
\usepackage[utf8]{inputenc}
\usepackage{amsmath}
\usepackage{amsfonts}
\usepackage{amssymb}
\usepackage{tikz}
\usetikzlibrary{quotes, angles}
\usepackage{graphicx}

%\usepackage{pgfplots}
%\pgfplotsset{width=10cm,compat=1.9}
%\usepackage{pgfplotstable}

\setlength{\headheight}{26pt}%doesn't seem to fix warning

\usepackage{fancyhdr}
\pagestyle{fancy}
\fancyhf{}

%\rhead{\small{24 March 2019}}
\lhead{\small{BECA / Dr. Huson / Geometry - Capstone Unit: Design}}

\renewcommand{\headrulewidth}{0pt}

\title{10th Grade Geometry - Capstone Unit: Design}
\subtitle{Bronx Early College Academy}
\author{Christopher J. Huson PhD}
\date{13 May 2019}

\begin{document}
\frame{\titlepage}
\section[Outline]{}
\frame{\tableofcontents}


\section{Capstone.1 Introduction \& draft design Monday 13 May}
  \frame
  {
    \frametitle{GQ: How do we design a household lamp?}
    \framesubtitle{CCSS: MP.4 Modeling with mathematics \hfill \alert{Capstone.1 Monday 13 May}}

    \begin{block}{Do Now: Solids handout}
      \begin{enumerate}
        \item Simplify compound object into basic shapes
        \item Use formulas to calculate volume
        \item Convert to weight using density and cost using price
      \end{enumerate}
    \end{block}
    Lesson: Steps in design, manufacture, \& sale of consumer goods\\[0.25cm]
    Capstone paper: Geogebra diagram (optional hand drawing), text description, volume, weight, \& cost calculations. (MLA format)\\
    \alert{Outline \& sketch due today in notebook}\\[0.25cm]
    Homework: Problem set handout, test corrections due Wednesday\\
    Capstone inspiration image \& reflection (email to husonbeca@gmail.com)
  }

\section{Capstone.2 Design drawing (Geogebra) Tuesday 14 May}
  \frame
  {
    \frametitle{GQ: How do we write up the design of a consumer item?}
    \framesubtitle{CCSS: MP5 Use appropriate tools strategically: dynamic geometry software \hfill \alert{Capstone.2 Tuesday 14 May}}

    \begin{block}{Project: Design a household lamp fixture}
      \begin{enumerate}
        \item Write up the design and specifications of a household lamp
        \item Keep it simple: leave out the shade and electrical parts
        \item Estimate its volume, weight, and materials cost
        \item Draw in Geogebra, compile in Word: add heading \& title, text, and formulas using Microsoft's equation editor
        \item Email me: Last-Title.pdf, with subject line \& message
        \item Rubric: correct, aesthetics, MLA \& email standards
      \end{enumerate}
    \end{block}
    Task: Draw in Geogebra (spicy: use 3-D). Email .png file to me. Write dimensions in notebook. (draft calculations for homework)\\
    Homework: Problem set handout, test corrections due tomorrow
      }

\section{Capstone.3 Complete draft (MS Word calcs) Wednesday 15 May}
  \frame
  {
    \frametitle{GQ: How do we complete a professional design proposal?}
    \framesubtitle{CCSS: MP5 Use appropriate tools strategically: Microsoft Word wordprocessor and equation editor \hfill \alert{Capstone.3 Wednesday 15 May}}

    \begin{block}{Project: Largely complete write up of lamp design}
      \begin{enumerate}
        \item Header, title, MLA formatting in Microsoft Word. (pdf final)
        \item Simple text description of lamp (optional: sales language)
        \item Design drawing, clear dimensions
        \item Show calculations of volume, weight, and materials cost using Microsoft's equation editor
        \item Email me: Last-Title.pdf, with subject line \& message
        \item Rubric: correct, aesthetics, MLA \& email standards
      \end{enumerate}
    \end{block}
    Notebook check: dimensions and volume, weight, \& cost calcs\\
    Task: Print draft of paper. Should be almost complete.\\
    Homework: Problem set handout
  }

\section{Capstone.4 Complete writeup (peer edit) Thursday 16 May}
  \frame
  {
    \frametitle{GQ: How do we complete a professional design proposal?}
    \framesubtitle{CCSS: MP5 Use appropriate tools strategically: Microsoft Word wordprocessor and equation editor \hfill \alert{Capstone.4 Thursday 16 May}}

    \begin{block}{Project: Complete write up of lamp design}
      \begin{enumerate}
        \item Header, title, MLA formatting in Microsoft Word. (pdf final)
        \item Simple text description of lamp (optional: sales language)
        \item Design drawing, clear dimensions
        \item Show calculations of volume, weight, and materials cost using Microsoft's equation editor
        \item Email me: Last-Title.pdf, with subject line \& message
        \item Rubric: correct, aesthetics, MLA \& email standards
      \end{enumerate}
    \end{block}
    Peer review\\
    Task: Final paper due by 10:00 tonight, to husonbeca@gmail.com\\
    Homework: Problem set handout
  }

\section{Capstone.5 Volume, density, trig quiz Friday 17 May}
  \frame
  {
    \frametitle{GQ: How do we calculate volume \& weight of 3-D shapes?}
    \framesubtitle{CCSS: HSG.GMD.A.3 Use volume formulas to solve problems \hfill \alert{Capstone.5 Friday 17 May}}

    Do Now handout: Volume, density, \& trig problems\\
    Classwork review
    \begin{block}{Assessment: Pop quiz}
      \begin{enumerate}
        \item Sector areas and arc lengths; compound areas
        \item Volume formulas, compound shapes, density problems
        \item Unit conversions, rounding
        \item Trigonometric situations
        \item Solving for a missing input given a formula result
    \end{enumerate}
    \end{block}
    Homework: Cumulative review weekend packet
  }

\end{document}
