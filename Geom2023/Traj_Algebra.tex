% \documentclass[12pt, twoside]{article}
\usepackage[letterpaper, margin=1in, headsep=0.2in]{geometry}
\setlength{\headheight}{0.6in}
%\usepackage[english]{babel}
\usepackage[utf8]{inputenc}
\usepackage{microtype}
\usepackage{amsmath}
\usepackage{amssymb}
%\usepackage{amsfonts}
\usepackage[nomessages]{fp} %\FPeval{\var-name}{2*sin(pi/6)}
\usepackage{siunitx} %units in math. eg 20\milli\meter
\usepackage{yhmath} % for arcs, overparenth command
\usepackage{tikz} %graphics
\usetikzlibrary{quotes, angles, arrows, arrows.meta}
\usepackage{graphicx} %consider setting \graphicspath{{images/}}
\usepackage{parskip} %no paragraph indent
\usepackage{enumitem}
\usepackage{multicol}
\usepackage{venndiagram}

\usepackage{fancyhdr}
\pagestyle{fancy}
\fancyhf{}
\renewcommand{\headrulewidth}{0pt} % disable the underline of the header
\raggedbottom
\hfuzz=2mm %suppresses overfull box warnings

\usepackage{hyperref}

\fancyhead[LE]{\thepage}
\fancyhead[RO]{\thepage \\ Name: \hspace{4cm} \,\\}
\fancyhead[LO]{BECA / Dr. Huson / Geometry\\*  Learning Trajectories\\* 2022-2023}

\begin{document}

\subsubsection*{Algebra Learning Trajectories}
\begin{enumerate}
  \subsubsection*{Solving linear equations in one unknown}
\item In the following two problems, solve for the value of $x$.
  \begin{multicols}{2}
    \begin{enumerate}
      \item   $2x+3=x + 9$ %\vspace{6cm}
      \item   $\frac{1}{2}(11-x)=5$ %\vspace{3cm}
    \end{enumerate}
  \end{multicols}
  %\vspace{3cm}

\item Solve for $x$
\begin{multicols}{2}
  \begin{enumerate}%[itemsep=4cm]
    \item $\frac{1}{3} x-7=-4$
    \item $\frac{3}{4}x =9$
    \item $\frac{1}{2}(x-7)=12$
    \item $\frac{2}{3}(x+7)=x-4$
  \end{enumerate}
\end{multicols} %\vspace{4cm}

\item Solve for the value of $x$.
  \begin{multicols}{2}
    \begin{enumerate}
      \item   $3x-3=x + 7$ %\vspace{6cm}
      \item   $\frac{1}{2}(4x+2)=7$ %\vspace{6cm}
    \end{enumerate}
  \end{multicols}
  %\vspace{5cm}

  \item Solve for the value of $x$.
  \begin{multicols}{2}
    \begin{enumerate}
      \item   $\frac{4}{3}(6x-3)=x + 10$
      \item   $\frac{2}{5}(x-1)+\frac{5}{2}(1-x)=0$
    \end{enumerate}
  \end{multicols}
  %\vspace{6cm}


\subsubsection*{Functions}
\item Given the linear function $f(x)=3x+4$.
\begin{multicols}{2}
  \begin{enumerate}
    \item Find $f(0)$ %\vspace{6cm}
    \item   $f(x)=10$. Find $x$. %\vspace{3cm}
  \end{enumerate}
\end{multicols}
  %\vspace{3cm}

\item Given the linear function $f(x)=2x-6$.
\begin{multicols}{2}
  \begin{enumerate}
    \item   $f(x)=0$. Find $x$. %\vspace{6cm}
    \item Find $f(2)$ %\vspace{6cm}
  \end{enumerate}
\end{multicols}
  %\vspace{6cm}

\item Given the linear function $f(x)=-2x+14$, find $x$.
\begin{multicols}{2}
  \begin{enumerate}
    \item Find $f(4)$
    \item   $f(x)=21$. Find $x$.
  \end{enumerate}
\end{multicols} %\vspace{4cm}


\subsubsection*{Quadratics}
\item Practice these techniques for quadratics ($x^2$)
  \begin{enumerate}%[itemsep=2cm]
    \item Expand $(x+4)(x+3)$
    \item Convert to \emph{standard form} (equal to zero): $x^2+4=4x$
    \item Factor, $x^2+9x+8=0$
  \end{enumerate}
  
\item Given $x^2+9x+8=0$. Factor and find the roots. %\vspace{3cm} 

\item Given $x^2+8x+7=0$. Factor and find the roots. %\vspace{4cm}

\item Given $x^2+6x+5=0$. Factor and find the roots. %\vspace{4cm}

\item Solve for $x$, $x^2+10x+7=2x$


\subsubsection*{Simplifying expressions}
\item Perform each calculation, writing down the full calculator display and then rounding to the \emph{nearest hundredth}.
\begin{multicols}{2}
\begin{enumerate}
  \item $V=\frac{1}{3} \pi (2.4)^2(5.1)$
  \item $P=3.6 + \frac{1}{2} \pi (3.6)$  
\end{enumerate}
\end{multicols}%\vspace{2cm}

\item Solve each equation for the appropriate variable. Do not round. Simplify radicals.
\begin{multicols}{2}
\begin{enumerate}[itemsep=2cm]
  \item $A=\pi r^2=27\pi$
  \item $V=\frac{1}{3}(6.0)^2h=153$  
\end{enumerate}
\end{multicols}%\vspace{5cm}

\item Perform each calculation, writing down the full calculator display and then rounding to the \emph{nearest hundredth}.
\begin{multicols}{2}
\begin{enumerate}
  \item $V=\frac{1}{3} \pi (2.7)^2(1.1)$
  \item $W=5.1 + \frac{1}{2} \pi (7.1)$  
\end{enumerate}
\end{multicols} %\vspace{2cm}

\item Solve each equation for the appropriate variable. Do not round. Simplify radicals.
\begin{multicols}{2}
\begin{enumerate}%[itemsep=2cm]
  \item $A=\pi r^2=18\pi$
  \item $V=\frac{1}{4}(2.2)^2h=12.1$
\end{enumerate}
\end{multicols}%\vspace{5cm}

\item Perform each calculation, writing down the full calculator display and then rounding to the \emph{nearest hundredth}.
  \begin{multicols}{2}
  \begin{enumerate}%[itemsep=4cm]
    \item $A=15.944732$
    \item $W=3.4 \times 9.8 \times 4.3 \times 0.15$
          
    \item $V=\frac{1}{3} \pi (3.4)^2(6.1)$
    \item $P=8.6 + \frac{1}{2} \pi (8.6)$  
    \item $V=199.19711$
    \item $W=\frac{1}{3} (13)  3.3^2 \times 1.175$
    \item $V=\frac{1}{3} \pi (12.4)^2(8.1)$
    \item $P=12 + \frac{1}{4} \pi (12)$ 
  \end{enumerate}
  \end{multicols}%\vspace{2cm}

\item Perform each calculation, writing down the full calculator display and then rounding to the \emph{nearest hundredth}.
  \begin{multicols}{2}
  \begin{enumerate}%[itemsep=4cm]
    \item $A=15.944732$
    \item $W=3.4 \times 9.8 \times 4.3 \times 0.15$
          
    \item $V=\frac{1}{3} \pi (3.4)^2(6.1)$
    \item $P=8.6 + \frac{1}{2} \pi (8.6)$  
    \item $V=199.19711$
    \item $W=\frac{1}{3} (13)  3.3^2 \times 1.175$
    \item $V=\frac{1}{3} \pi (12.4)^2(8.1)$
    \item $P=12 + \frac{1}{4} \pi (12)$ 
  \end{enumerate}
  \end{multicols}%\vspace{2cm}

  
\subsubsection*{Trigonometric evaluation using calculator}
\item Express the result to the nearest thousandth.  \vspace{1cm}
  \begin{multicols}{2}
    \begin{enumerate}
      \item $\sin 35^\circ = $ %\vspace{1cm}
      \item $\tan 70^\circ =$
      \item $\sin 78^\circ = $ %\vspace{1cm}
      \item $\cos 12^\circ =$
    \end{enumerate}
  \end{multicols} %\vspace{0.5cm}


\end{enumerate}
\end{document}