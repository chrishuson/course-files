\documentclass[12pt, twoside]{article}
\usepackage[letterpaper, margin=1in, headsep=0.2in]{geometry}
\setlength{\headheight}{0.6in}
%\usepackage[english]{babel}
\usepackage[utf8]{inputenc}
\usepackage{microtype}
\usepackage{amsmath}
\usepackage{amssymb}
%\usepackage{amsfonts}
\usepackage{siunitx} %units in math. eg 20\milli\meter
\usepackage{yhmath} % for arcs, overparenth command
\usepackage{tikz} %graphics
\usetikzlibrary{quotes, angles}
\usepackage{graphicx} %consider setting \graphicspath{{images/}}
\usepackage{parskip} %no paragraph indent
\usepackage{enumitem}
\usepackage{multicol}
\usepackage{venndiagram}

\usepackage{fancyhdr}
\pagestyle{fancy}
\fancyhf{}
\renewcommand{\headrulewidth}{0pt} % disable the underline of the header
\raggedbottom
\hfuzz=2mm %suppresses overfull box warnings

\usepackage{hyperref}

\fancyhead[LE]{\thepage}
\fancyhead[RO]{\thepage \\ Name: \hspace{4cm} \,\\}
\fancyhead[LO]{BECA / Dr. Huson / Geometry\\*  Unit 1: Segments, length, and area\\* 19 Sept 2022}

\begin{document}

\subsubsection*{4.3 Classwork: Volume of a prism (box)}
\begin{enumerate}
\item Find the volume of a rectangular prism with length 5 cm, width 4 cm, and height 3 cm. \vspace{1cm}

\item Find the volume of a pyramid ($V=\frac{1}{3}Bh$) having a height of 11.3 inches and with a square base having side lengths of 7 inches. Express your result to the \emph{nearest cubic inch}. \vspace{3cm}


\item Find the volume of a sphere with a radius of 30 inches, to the \emph{nearest whole cubic inch}. (The formula for the volume of a \emph{sphere} is $V=\frac{4}{3}\pi r^3$) \vspace{3cm}

\item A waffle cone has a radius of 2 inches and height of 4 inches. 
\begin{enumerate}
  \item Write down the general formula for the volume of a cone. \vspace{1cm}
  \item Find the volume of the waffle cone.
\end{enumerate}  \vspace{3cm}

\item A given sphere has a radius of 6 inches.
\begin{enumerate}
  \item Write down the general formula for the volume of a sphere, using $r$ to represent the radius. \vspace{1cm}
  \item Find the volume of the sphere, to the \emph{nearest whole cubic inch}.
\end{enumerate}  \vspace{3cm}

\item A pyramid with a square base has a volume of 576 cubic inches. Its height is the same as the lengths of the sides of the base. Find the area of its base.

\end{enumerate}
\end{document}