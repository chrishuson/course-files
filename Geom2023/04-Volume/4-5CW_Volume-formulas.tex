\documentclass[12pt, twoside]{article}
\usepackage[letterpaper, margin=1in, headsep=0.2in]{geometry}
\setlength{\headheight}{0.6in}
%\usepackage[english]{babel}
\usepackage[utf8]{inputenc}
\usepackage{microtype}
\usepackage{amsmath}
\usepackage{amssymb}
%\usepackage{amsfonts}
\usepackage{siunitx} %units in math. eg 20\milli\meter
\usepackage{yhmath} % for arcs, overparenth command
\usepackage{tikz} %graphics
\usetikzlibrary{quotes, angles}
\usepackage{graphicx} %consider setting \graphicspath{{images/}}
\usepackage{parskip} %no paragraph indent
\usepackage{enumitem}
\usepackage{multicol}
\usepackage{venndiagram}

\usepackage{fancyhdr}
\pagestyle{fancy}
\fancyhf{}
\renewcommand{\headrulewidth}{0pt} % disable the underline of the header
\raggedbottom
\hfuzz=2mm %suppresses overfull box warnings

\usepackage{hyperref}

\fancyhead[LE]{\thepage}
\fancyhead[RO]{\thepage \\ Name: \hspace{4cm} \,\\}
\fancyhead[LO]{BECA / Dr. Huson / Geometry\\*  Unit 4: Volume and polyhedra \\* 10 November 2022}

\begin{document}

\subsubsection*{4.5 Classwork: Volume of cylinders, cones, pyramids, spheres}
\begin{enumerate}
\item Find the volume of a rectangular prism with length 2 cm, width 5 cm, and height 3 cm. \vspace{1cm}

\item Find the volume of a pyramid ($V=\frac{1}{3}Bh$) having a height of 11.3 inches and with a square base having side lengths of 7 inches. Express your result to the \emph{nearest cubic inch}. \vspace{3cm}


\item Find the volume of a sphere with a radius of 30 inches, to the \emph{nearest whole cubic inch}. (The formula for the volume of a \emph{sphere} is $V=\frac{4}{3}\pi r^3$) \vspace{3cm}

\item A waffle cone has a radius of 2 inches and height of 4 inches. 
\begin{enumerate}
  \item Write down the general formula for the volume of a cone. \vspace{1cm}
  \item Find the volume of the waffle cone.
\end{enumerate}  \vspace{3cm}

\item A given sphere has a radius of 6 inches.
\begin{enumerate}
  \item Write down the general formula for the volume of a sphere, using $r$ to represent the radius. \vspace{1cm}
  \item Find the volume of the sphere, to the \emph{nearest whole cubic inch}.
\end{enumerate}  \vspace{3cm}

\newpage
\item The volume of a sphere is $(121 \frac{1}{2}) \pi$. Find its radius. \vspace{2cm}

\item A pyramid with a square base has a volume of 576 cubic inches. Its height is the same as the lengths of the sides of the base. Find the area of its base.\\[0.5cm] Given the volume formula $V=\frac{1}{3}(s^2)h$ for a pyramid with a square base ($B=s^2$).
\begin{enumerate}[itemsep=0.5cm]
  \item Write down the variable representing the height
  \item Write down the variable representing the length of the base's side
  \item Write an equation relating the two variables in (a) and (b)
  \item Substitute and solve \[V=\frac{1}{3}(s^2)h\]
\end{enumerate} \vspace{2cm}

\item A waffle cone has a radius of 2 inches and height of 4 inches. 
\begin{enumerate}
  \item Write down the general formula for the volume of a cone. \vspace{1cm}
  \item Find the volume of the waffle cone.
\end{enumerate}  \vspace{2cm}

\item A given sphere has a radius of 6 inches.
\begin{enumerate}
  \item Write down the general formula for the volume of a sphere, using $r$ to represent the radius. \vspace{1cm}
  \item Find the volume of the sphere, to the \emph{nearest whole cubic inch}.
\end{enumerate}  \vspace{3cm}

\newpage
\item A triangle has an area of 68 square centimeters. Its height is 16 centimeters. Find the length of its base. \vspace{3cm}

\item The perimeter of a square is 10 inches. Find its area. \vspace{4cm}

\item A pyramid with a square base has a volume of 576 cubic inches. Its height is the same as the lengths of the sides of the base. Find the area of its base.

\item A sphere has a radius of 5 centimeters. \hfill CCSSM.8.G.C.9
\begin{enumerate}
  \item Write down the general formula for the volume of a sphere. \vspace{1cm}
  \item Find the volume of the sphere, rounded to the nearest cubic centimeter.
\end{enumerate}  \vspace{3cm}

\item A cylinder is 12.3 cm tall and has a volume of 966 cubic cm. Find the area of the base of the cylinder. Express your result to the \emph{nearest hundredth of a square centimeter}. \vspace{3cm}

\subsubsection*{Model the situation with an equation. \hfill Do NOT solve!}
\item A large concrete post in the shape of a cylinder has a volume of 250 cubic feet. Its height is 12 feet. Find the radius of the base of the post. \vspace{2cm}

\item A spherical cork fishing net float has a volume of 4000 cubic centimeters. Find its radius. \vspace{2cm}

\item The volume of a cone having a \textbf{diameter} of 10 inches is 200 cubic inches. Find the cone's height.
  

\item A sphere has a radius of 9 centimeters. \hfill CCSSM.8.G.C.9
\begin{enumerate}
  \item Write down the general formula for the volume of a sphere. \vspace{1cm}
  \item Find the volume of the sphere, rounded to the nearest cubic centimeter.
\end{enumerate}  \vspace{3cm}

\item A cylinder is 12.3 cm tall and has a volume of 966 cubic cm. Find the area of the base of the cylinder. Express your result to the \emph{nearest hundredth of a square centimeter}. \vspace{3cm}



\end{enumerate}
\end{document}