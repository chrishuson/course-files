\documentclass[12pt, twoside]{article}
\usepackage[letterpaper, margin=1in, headsep=0.2in]{geometry}
\setlength{\headheight}{0.6in}
%\usepackage[english]{babel}
\usepackage[utf8]{inputenc}
\usepackage{microtype}
\usepackage{amsmath}
\usepackage{amssymb}
%\usepackage{amsfonts}
\usepackage{siunitx} %units in math. eg 20\milli\meter
\usepackage{yhmath} % for arcs, overparenth command
\usepackage{tikz} %graphics
\usetikzlibrary{quotes, angles}
\usepackage{graphicx} %consider setting \graphicspath{{images/}}
\usepackage{parskip} %no paragraph indent
\usepackage{enumitem}
\usepackage{multicol}
\usepackage{venndiagram}

\usepackage{fancyhdr}
\pagestyle{fancy}
\fancyhf{}
\renewcommand{\headrulewidth}{0pt} % disable the underline of the header
\raggedbottom
\hfuzz=2mm %suppresses overfull box warnings

\usepackage{hyperref}

\fancyhead[LE]{\thepage}
\fancyhead[RO]{\thepage \\ Name: \hspace{4cm} \,\\}
\fancyhead[LO]{BECA / Dr. Huson / Geometry\\* Unit 8: Congruence transformations\\* 3 January 2023}

\begin{document}

\subsubsection*{8.1 Classwork: Construction}
\begin{enumerate}
\item Complete the construction of an equilateral triangle including the six steps.
\begin{enumerate}
  \item Given the line segment $\overline{MN}$.
  \bigskip
  \item Construct circle $M$ with radius $\rule{2cm}{0.15mm}$.
  \bigskip
  \item Construct circle $\rule{2cm}{0.15mm}$  with radius $\rule{2cm}{0.15mm}$. \bigskip
  \item Label the intersection $P$ of the two circles.
  \bigskip
  \item Draw line segments $\rule{2cm}{0.15mm}$  and $\rule{2cm}{0.15mm}$
  \bigskip
  \item $\triangle MNP$ is equilateral.
\end{enumerate}
\vspace{7cm}
\begin{center}
\begin{tikzpicture}
  \draw [-, thick] (0,0)--(7,0);
  \draw [fill] (0,0) circle [radius=0.05] node[below]{$M$};
  \draw [fill] (7,0) circle [radius=0.05] node[below]{$N$};
\end{tikzpicture}
\end{center}

\newpage
\item Complete the construction of an equilateral triangle including the six steps.
\begin{enumerate}
  \item Given the line segment $\overline{MN}$.
  \bigskip
  \item %Construct circle $M$ with radius $\rule{2cm}{0.15mm}$.
  \bigskip
  \item %Construct circle $\rule{2cm}{0.15mm}$  with radius $\rule{2cm}{0.15mm}$.
  \bigskip
  \item %Label the intersection $P$ of the two circles.
  \bigskip
  \item %Draw line segments $\rule{2cm}{0.15mm}$  and $\rule{2cm}{0.15mm}$
  \bigskip
  \item $\triangle MNP$ is equilateral.
\end{enumerate}
\vspace{7cm}
\begin{center}
\begin{tikzpicture}
  \draw [-, thick] (0,0)--(7,0);
  \draw [fill] (0,0) circle [radius=0.05] node[below]{$M$};
  \draw [fill] (7,0) circle [radius=0.05] node[below]{$N$};
\end{tikzpicture}
\end{center}

\newpage
\item Complete the construction of an angle bisector including the six steps.
  \begin{enumerate}
    \item Given an angle with vertex $A$.
    \item Construct circle $A$ with arbitrary radius (i.e. the radius does not matter).
    \item Label the intersections $B$ and $C$ of the angle's rays and circle $A$.
    \item Construct circle $B$  with radius $BC$. \bigskip
    \item Construct circle $\rule{2cm}{0.15mm}$  with radius $\rule{2cm}{0.15mm}$. \bigskip
    \item Label $D$, the intersection of circle $B$ and $C$. \bigskip
    \item Draw ray $\rule{2cm}{0.15mm}$.
    \bigskip
    \item Ray $\overrightarrow {AD}$ bisects $\angle A$.
  \end{enumerate}
  \vspace{3cm}
  \begin{center}
  \begin{tikzpicture}
    \draw [<->, thick] (5,6)--(0,0)--(9,0);
    \draw [fill] (0,0) circle [radius=0.05] node[below]{$A$};
    %\draw [fill] (7,0) circle [radius=0.05] node[below]{$N$};
  \end{tikzpicture}
  \end{center}

\newpage
\item Complete the construction of a perpendicular bisector including the six steps.
  \begin{enumerate}
    \item Given the line segment $\overline{PQ}$.
    \bigskip
    \item %Construct circle $P$ with radius $\rule{2cm}{0.15mm}$.
    \bigskip
    \item %Construct circle $\rule{2cm}{0.15mm}$  with radius $\rule{2cm}{0.15mm}$.
    \bigskip
    \item %Label the intersection $P$ of the two circles.
    \bigskip
    \item Draw the line $\rule{2cm}{0.15mm}$.
    \bigskip
    \item The line $\rule{2cm}{0.15mm}$ is the perpendicular bisector of $\overline{PQ}$.
  \end{enumerate}
  \vspace{7cm}
  \begin{center}
  \begin{tikzpicture}
    \draw [-, thick] (0,0)--(6,0);
    \draw [fill] (0,0) circle [radius=0.05] node[below]{$P$};
    \draw [fill] (6,0) circle [radius=0.05] node[below]{$Q$};
  \end{tikzpicture}
  \end{center}


\end{enumerate}
\end{document}