\documentclass[12pt, twoside]{article}
\usepackage[letterpaper, margin=1in, headsep=0.2in]{geometry}
\setlength{\headheight}{0.6in}
%\usepackage[english]{babel}
\usepackage[utf8]{inputenc}
\usepackage{microtype}
\usepackage{amsmath}
\usepackage{amssymb}
%\usepackage{amsfonts}
\usepackage{siunitx} %units in math. eg 20\milli\meter
\usepackage{yhmath} % for arcs, overparenth command
\usepackage{tikz} %graphics
\usetikzlibrary{quotes, angles}
\usepackage{graphicx} %consider setting \graphicspath{{images/}}
\usepackage{parskip} %no paragraph indent
\usepackage{enumitem}
\usepackage{multicol}
\usepackage{venndiagram}

\usepackage{fancyhdr}
\pagestyle{fancy}
\fancyhf{}
\renewcommand{\headrulewidth}{0pt} % disable the underline of the header
\raggedbottom
\hfuzz=2mm %suppresses overfull box warnings

\usepackage{hyperref}

\fancyhead[LE]{\thepage}
\fancyhead[RO]{\thepage \\ Name: \hspace{4cm} \,\\}
\fancyhead[LO]{BECA / Dr. Huson / Geometry\\*  Unit 1: Segments, length, and area\\* 14 Sept 2022}

\begin{document}

\subsubsection*{1.5 Extension: Find an endpoint given the midpoint}
\begin{enumerate}

\item Given $M$ is the midpoint of $\overline{AB}$, with $A=0$ and $M=4$. Find the value of point $B$ and mark and label it on the number line. \par \medskip
  \begin{tikzpicture}
    \draw[<->] (-2.5,0)--(9.5,0);
    \foreach \x in {-2,...,9}
      \draw[shift={(\x,0)}] (0pt,-3pt)--(0pt,3pt) node[below=5pt]{$\x$};
    \draw[fill] (0,0) circle [radius=0.05] node[above]{$A$};
    \draw[fill] (4,0) circle [radius=0.05] node[above]{$M$};
  \end{tikzpicture} \vspace{2cm}  

\item Given collinear points with $Q$ the bisector of $\overline{PR}$, $Q(25)$ and $R(40)$. Find $P$, marking it and labeling it on the number line. \par \medskip
\begin{tikzpicture}[scale=0.3]
  \draw[<->] (4,0)--(46,0);
  \foreach \x in {5, 10,...,45}
    \draw[shift={(\x,0)}] (0pt,-8pt)--(0pt,8pt) node[below=5pt]{$\x$};
  \draw[fill] (25,0) circle [radius=0.2] node[above]{$Q$};
  \draw[fill] (40,0) circle [radius=0.2] node[above]{$R$};
  \end{tikzpicture} \par \vspace{2cm}

\item Given points $S(13)$ and $T(31)$, find the value of $U$ such that $T$ is the midpoint of $\overline{SU}$. Mark $U$ and label it on the number line. \par \medskip
\begin{tikzpicture}[scale=0.2]
  \draw[<->] (-3,0)--(73,0);
  \foreach \x in {0, 10,...,70}
    \draw[shift={(\x,0)}] (0pt,-8pt)--(0pt,8pt) node[below=5pt]{$\x$};
  \draw[fill] (13,0) circle [radius=0.2] node[above]{$S(13)$};
  \draw[fill] (31,0) circle [radius=0.2] node[above]{$T(31)$};
  \end{tikzpicture} \par \vspace{2cm}

\item The point $M(2.3)$ is the midpoint of segment $\overline{AB}$. Given $A(-1.5)$, find the value of $B$. Mark and label it below. \par \smallskip
  \begin{tikzpicture}
    \draw[<->] (-2.5,0)--(8.5,0);
    \foreach \x in {-2,...,8}
      \draw[shift={(\x,0)}] (0pt,-3pt)--(0pt,3pt) node[below=5pt]{$\x$};
    \draw[fill] (-1.5,0) circle [radius=0.05] node[above]{$A(-1.5)$};
    \draw[fill] (2.3,0) circle [radius=0.05] node[above]{$M(2.3)$};
  \end{tikzpicture}

\newpage
\item Point $B$ bisects segment $\overline{AC}$, $AB=\frac{1}{2}x+7$ and $AC=26$.
  Find $x$.
  \begin{center}
    \begin{tikzpicture}[scale=1]
      \draw[thick] (0,0)--(8,0);
      \draw[fill] (0,0) circle [radius=0.05] node[below]{$A$};
      \draw[fill] (4,0) circle [radius=0.05] node[below]{$B$};
      \draw[fill] (8,0) circle [radius=0.05] node[below]{$C$};
      \node at (2,0.7){$\frac{1}{2}x+7$};
      \draw[<->, dashed] (0,-1)--(8,-1);
      \node at (4,-1.5){$26$};
      \draw (1.8,-0.2)--(1.9,0.2);
      \draw (2.1,-0.2)--(2.2,0.2);
      \draw (5.8,-0.2)--(5.9,0.2);
      \draw (6.1,-0.2)--(6.2,0.2);
    \end{tikzpicture}
  \end{center} \vspace{4cm}

\item Given the points $S$ and $T$ trisect the line segment $\overline{RU}$, as shown below. If $RT=7$, find $RU$.\\[1cm]
  \begin{tikzpicture}[scale=0.75]
    \draw[thick] (0,0)--(9,0);
    \draw[fill] (0,0) circle [radius=0.05] node[below]{$R$};
    \draw[fill] (3,0) circle [radius=0.05] node[below]{$S$};
    \draw[fill] (6,0) circle [radius=0.05] node[below]{$T$};
    \draw[fill] (9,0) circle [radius=0.05] node[below]{$U$};
    \draw[thick] (1.4,-0.2)--(1.4,0.2);
    \draw[thick] (1.55,-0.2)--(1.55,0.2);
    \draw[thick] (4.4,-0.2)--(4.4,0.2);
    \draw[thick] (4.55,-0.2)--(4.55,0.2);
    \draw[thick] (7.4,-0.2)--(7.4,0.2);
    \draw[thick] (7.55,-0.2)--(7.55,0.2);
  \end{tikzpicture} \vspace{4cm}

\item The point $Q$ lies on $\overline{AB}$ three quarters of the way from $A$ to $B$. Given $AB=28$.
  \begin{enumerate}
    \item Mark and label the location of $Q$. (measure)
    \item Find ${AQ}$. State an equation for full credit.
  \end{enumerate} \vspace{1cm} 
  \begin{center}
    \begin{tikzpicture}
      \draw[thick] (0,0)--(7,0);
      \draw[fill] (0,0) circle [radius=0.05] node[below]{$A$};
      \draw[fill] (7,0) circle [radius=0.05] node[below]{$B$};
    \end{tikzpicture} 
  \end{center}


\end{enumerate}
\end{document}