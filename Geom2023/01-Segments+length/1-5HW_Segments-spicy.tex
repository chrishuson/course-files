\documentclass[12pt, twoside]{article}
\usepackage[letterpaper, margin=1in, headsep=0.2in]{geometry}
\setlength{\headheight}{0.6in}
%\usepackage[english]{babel}
\usepackage[utf8]{inputenc}
\usepackage{microtype}
\usepackage{amsmath}
\usepackage{amssymb}
%\usepackage{amsfonts}
\usepackage{siunitx} %units in math. eg 20\milli\meter
\usepackage{yhmath} % for arcs, overparenth command
\usepackage{tikz} %graphics
\usetikzlibrary{quotes, angles}
\usepackage{graphicx} %consider setting \graphicspath{{images/}}
\usepackage{parskip} %no paragraph indent
\usepackage{enumitem}
\usepackage{multicol}
\usepackage{venndiagram}

\usepackage{fancyhdr}
\pagestyle{fancy}
\fancyhf{}
\renewcommand{\headrulewidth}{0pt} % disable the underline of the header
\raggedbottom
\hfuzz=2mm %suppresses overfull box warnings

\usepackage{hyperref}

\fancyhead[LE]{\thepage}
\fancyhead[RO]{\thepage \\ Name: \hspace{4cm} \,\\}
\fancyhead[LO]{BECA / Dr. Huson / Geometry\\*  Unit 1: Segments, length, and area\\* 8 Sept 2022}

\begin{document}

\subsubsection*{1.5 Homework: Segments, equilateral and isosceles triangles, perimeter}
\begin{enumerate}

\item The perimeter of the isosceles $\triangle FGH$ is 35 with $\overline{FH} \cong \overline{GH}$. If $FG=x+5$ and $FH=13 \frac{1}{2}$, find $x$.\\[0.25cm]
    Show your work with an equation for full credit.\\[0.5cm]
      \begin{tikzpicture}[scale=0.5]
        \draw [thick](0,0)--(4,0)--(2,6)--(0,0);
        \draw [fill] (0,0) circle [radius=0.05] node[below left]{$F$};
        \draw [fill] (4,0) circle [radius=0.05] node[below right]{$G$};
        \draw [fill] (2,6) circle [radius=0.05] node[above right]{$H$};
        \draw [thick] (0.8,3.1)--(1.2,3); %tick mark
        \draw [thick] (2.8,3)--(3.2,3.1); %tick mark
        \node at (2,0) [below]{$x+5$};
        \node at (0.8,3.4) [left]{$13 \frac{1}{2}$};
      \end{tikzpicture} \vspace{1cm}

\item Given $\overline{PQR}$, $PQ=2x+11$, $QR=x+1$, $PR=x+26$. Find ${x}$.
  \begin{center}
      \begin{tikzpicture}
      \draw [-, thick] (0,0)--(7,0);
      \draw [fill] (0,0) circle [radius=0.05] node[below]{$P$};
      \draw [fill] (5,0) circle [radius=0.05] node[below]{$Q$};
      \draw [fill] (7,0) circle [radius=0.05] node[below]{$R$};
      \node at (2,0) [above]{$2x+11$};
      \node at (6,0) [above]{$x+1$};
      \draw [<->, dashed] (0,-0.7)--(7,-0.7);
      \node at (3.5,-0.7) [below]{$x+26$};
    \end{tikzpicture}
  \end{center}
  \begin{enumerate}
      \item Write down an equation to represent the situation. \vspace{0.5cm}
      \item Solve for $x$. \vspace{1.5cm}
      \item Check your answer. \vspace{2cm}
    \end{enumerate}

\item Given the points $S$ and $T$ trisect the line segment $\overline{RU}$, as shown below. If $RT=7$, find $RU$.\\[1cm]
    \begin{tikzpicture}[scale=0.75]
     \draw [-, thick] (0,0)--(9,0);
     \draw [fill] (0,0) circle [radius=0.05] node[below]{$R$};
     \draw [fill] (3,0) circle [radius=0.05] node[below]{$S$};
     \draw [fill] (6,0) circle [radius=0.05] node[below]{$T$};
     \draw [fill] (9,0) circle [radius=0.05] node[below]{$U$};
     %\node at (1.6,0.3) [above]{$x$};
     %\node at (4.6,0.3) [above]{$x$};
     %\node at (7.6,0.3) [above]{$x$};
     \draw [-, thick] (1.4,-0.2)--(1.4,0.2);
     \draw [-, thick] (1.55,-0.2)--(1.55,0.2);
     \draw [-, thick] (4.4,-0.2)--(4.4,0.2);
     \draw [-, thick] (4.55,-0.2)--(4.55,0.2);
     \draw [-, thick] (7.4,-0.2)--(7.4,0.2);
     \draw [-, thick] (7.55,-0.2)--(7.55,0.2);
     %\draw [<->, dashed] (0,-1.3)--(9,-1.3);
     %\node at (4.5,-1.3) [below]{$15$};
   \end{tikzpicture} \vspace{3cm}

\item Given $S(-1.3)$ and $T(3.3)$, as shown on the number line. \\[0.25cm]
Mark and label the midpoint $M$ that bisects $\overline{ST}$. \\[0.5cm]
  \begin{tikzpicture}
    \draw [<->] (-3.5,0)--(6.5,0);
    \draw [-, thick] (-1.3,0)--(3.3,0);
    \foreach \x in {-3,...,6} %2 leading for diff!=1
      \draw[shift={(\x,0)},color=black] (0pt,-3pt) -- (0pt,3pt) node[below=5pt]  {$\x$};
      \draw [fill] (-1.3,0) circle [radius=0.05] node[above] {$S$};
      \draw [fill] (3.3,0) circle [radius=0.05] node[above] {$T$};
  \end{tikzpicture} \vspace{4cm} 

\item Given $A(-2.1)$ and $M(1.8)$, as shown on the number line. The point $B$ is such that $M$ bisects $\overline{AB}$. \\[0.25cm] 
Find the value of $B$. Mark and label it on the number line.\\[0.5cm]
        \begin{tikzpicture}
          \draw [<->] (-3.5,0)--(8.5,0);
          \draw [-, thick] (-2.1,0)--(1.8,0);
          \foreach \x in {-3,...,8} %2 leading for diff!=1
            \draw[shift={(\x,0)},color=black] (0pt,-3pt) -- (0pt,3pt) node[below=5pt]  {$\x$};
            \draw [fill] (-2.1,0) circle [radius=0.05] node[above] {$A$};
            \draw [fill] (1.8,0) circle [radius=0.05] node[above] {$M$};
        \end{tikzpicture} \vspace{4cm}  

\item The point $Q$ lies on $\overline{AB}$ three quarters of the way from $A$ to $B$. Given $AB=28$.
    \begin{enumerate}
      \item Mark and label the approximate location of $Q$.
      \item Find ${AQ}$. State an equation for full credit.
    \end{enumerate} \vspace{1cm} 
    \begin{center}
      \begin{tikzpicture}
        \draw [-, thick] (0,0)--(8,0);
        \draw [fill] (0,0) circle [radius=0.05] node[below]{$A$};
        \draw [fill] (8,0) circle [radius=0.05] node[below]{$B$};
      \end{tikzpicture} 
    \end{center} \vspace{3cm} 

\end{enumerate}
\end{document}