\documentclass[12pt, twoside]{article}
\usepackage[letterpaper, margin=1in, headsep=0.2in]{geometry}
\setlength{\headheight}{0.6in}
%\usepackage[english]{babel}
\usepackage[utf8]{inputenc}
\usepackage{microtype}
\usepackage{amsmath}
\usepackage{amssymb}
%\usepackage{amsfonts}
\usepackage{siunitx} %units in math. eg 20\milli\meter
\usepackage{yhmath} % for arcs, overparenth command
\usepackage{tikz} %graphics
\usetikzlibrary{quotes, angles}
\usepackage{graphicx} %consider setting \graphicspath{{images/}}
\usepackage{parskip} %no paragraph indent
\usepackage{enumitem}
\usepackage{multicol}
\usepackage{venndiagram}

\usepackage{fancyhdr}
\pagestyle{fancy}
\fancyhf{}
\renewcommand{\headrulewidth}{0pt} % disable the underline of the header
\raggedbottom
\hfuzz=2mm %suppresses overfull box warnings

\usepackage{hyperref}

\fancyhead[LE]{\thepage}
\fancyhead[RO]{\thepage \\ Name: \hspace{4cm} \,\\}
\fancyhead[LO]{BECA / Dr. Huson / Geometry\\*  Unit 1: Segments, length, and area\\* 15 Sept 2022}

\begin{document}

\subsubsection*{1.6 Classwork: Vocabulary checklist (should all be in your notebook)}
\emph{What do they mean and what is the standard notation?}
\begin{enumerate}
\item How do we describe ``undefined terms"?
\begin{itemize}
    \item Point
    \item Line
    \item Plane
\end{itemize}

\item What are the basic geometry terms involving points, lines, and line segments?
\begin{itemize}
    \item Collinear set of points
    \item Non-collinear
    \item Coplanar
    \item Line segment
    \item End point
    \item Distance between two points (on the real number line)
    \item Length/measure of a line segment
    \item Congruent segments (hash marks)
\end{itemize}

\item What are the basic geometry terms involving rays and angles?
\begin{itemize}
    \item Rays
    \item Opposite rays
    \item Adjacent
    \item Midpoint
    \item Bisect (trisect, partition)
    \item Equilateral, isosceles, and scalene triangles
    \item Polygon
    \item Quadrilaterals (rectangle, square, parallelogram, trapezoid, kite, rhombus)
    \item Perimeter
\end{itemize}

\item What are the elements of algebra?
\begin{itemize}
    \item Variable
    \item Coefficient
    \item Term
    \item Numerator and denominator
    \item Distribute multiplication over addition
    \item LCD, Lowest Common Denominator
\end{itemize}
  
\end{enumerate}
\end{document}

\item Angle
\item Straight angle
\item Right angle
\item Obtuse angle
\item Acute angle
\item Interior/exterior of an angle