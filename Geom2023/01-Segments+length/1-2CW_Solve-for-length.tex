\documentclass[12pt, twoside]{article}
\usepackage[letterpaper, margin=1in, headsep=0.2in]{geometry}
\setlength{\headheight}{0.6in}
%\usepackage[english]{babel}
\usepackage[utf8]{inputenc}
\usepackage{microtype}
\usepackage{amsmath}
\usepackage{amssymb}
%\usepackage{amsfonts}
\usepackage{siunitx} %units in math. eg 20\milli\meter
\usepackage{yhmath} % for arcs, overparenth command
\usepackage{tikz} %graphics
\usetikzlibrary{quotes, angles}
\usepackage{graphicx} %consider setting \graphicspath{{images/}}
\usepackage{parskip} %no paragraph indent
\usepackage{enumitem}
\usepackage{multicol}
\usepackage{venndiagram}

\usepackage{fancyhdr}
\pagestyle{fancy}
\fancyhf{}
\renewcommand{\headrulewidth}{0pt} % disable the underline of the header
\raggedbottom
\hfuzz=2mm %suppresses overfull box warnings

\usepackage{hyperref}

\fancyhead[LE]{\thepage}
\fancyhead[RO]{\thepage \\ Name: \hspace{4cm} \,\\}
\fancyhead[LO]{BECA / Dr. Huson / Geometry\\*  Unit 1: Segments, length, and area\\* 9 Sept 2022}

\begin{document}

\subsubsection*{1.2 Classwork: Solve for length}
\begin{enumerate}
\item Given $\overline{ABC}$, $AB=8$, and $BC=4$. Find ${AC}$.\\[1cm]
    \begin{tikzpicture}
      \draw [-, thick] (0,0)--(6,0);
      \draw [fill] (0,0) circle [radius=0.05] node[below]{$A$};
      \draw [fill] (4,0) circle [radius=0.05] node[below]{$B$};
      \draw [fill] (6,0) circle [radius=0.05] node[below]{$C$};
    \end{tikzpicture} \vspace{1cm}
    
\item Given $\overline{RST}$, $RS=5$, and $RT=7 \frac{1}{2}$.
  \begin{enumerate}
  \item Find ${ST}$.\\[0.75cm]
    \begin{tikzpicture}
      \draw [-, thick] (1,0)--(7,0);
      \draw [fill] (1,0) circle [radius=0.05] node[below]{$R$};
      \draw [fill] (5,0) circle [radius=0.05] node[below]{$S$};
      \draw [fill] (7,0) circle [radius=0.05] node[below]{$T$};
    \end{tikzpicture}  \vspace{1cm}
  \item The postulate used in this problem is the \rule{6cm}{0.15mm}.
  \end{enumerate} \vspace{0.5cm}

\item Given $\overline{DEF}$, $DE=x+4$, $EF=x+2$, $DF=14$. Find ${DE}$.
  \begin{enumerate}
  \item Label the diagram with the given values.
  \begin{flushright}
    \begin{tikzpicture}
        \draw [-, thick] (0,0)--(6,0);
        \draw [fill] (0,0) circle [radius=0.05] node[below]{$D$};
        \draw [fill] (3.25,0) circle [radius=0.05] node[below]{$E$};
        \draw [fill] (6,0) circle [radius=0.05] node[below]{$F$};
    \end{tikzpicture}
  \end{flushright} \vspace{0.5cm}
  \item Write an equation: \vspace{1cm}
  \item Solve for $x$
  \vspace{3cm}
  \item Answer the question. \\
  Find $DE$ by substituting for $x$. \vspace{1.5cm}
  \item Check your answer
  \end{enumerate}

\item The points shown are in a straight line, $\overline{XYZ}$. 
\begin{enumerate}
  \item Measure and label the lengths $XY$ and $YZ$ to the nearest centimeter.\\[1.5cm]
    \begin{tikzpicture}
      \draw [-, thick] (1,0)--(7,0);
      \draw [fill] (1,0) circle [radius=0.05] node[below]{$X$};
      \draw [fill] (5,0) circle [radius=0.05] node[below]{$Y$};
      \draw [fill] (7,0) circle [radius=0.05] node[below]{$Z$};
    \end{tikzpicture} \vspace{0.5cm}
  \item Write an equation employing the Segment Addition Postulate.\\ (fill in the blanks with values in centimeters)\\[1cm]
  $XZ=$ \rule{2cm}{0.15mm} $+$ \rule{2cm}{0.15mm} $=$ \rule{2cm}{0.15mm}
\end{enumerate} \vspace{0.5cm}

\item Given $\overline{LMN}$, $LM=3x+1$, $MN=7$, $LN=17$. Find ${x}$.\\[0.15in]
  \begin{tikzpicture}
   \draw [-, thick] (0,0)--(7,0);
   \draw [fill] (0,0) circle [radius=0.05] node[below]{$L$};
   \draw [fill] (4,0) circle [radius=0.05] node[below]{$M$};
   \draw [fill] (7,0) circle [radius=0.05] node[below]{$N$};
   \node at (1.7,0) [above]{$3x+1$};
   \node at (5.5,0) [above]{$7$};
   \draw [<->, dashed] (0,-1)--(7,-1);
   \node at (3.5,-1) [below]{$17$};
 \end{tikzpicture} %\vspace{1cm}
  \begin{enumerate}
  \item Write down an equation to represent the situation. \vspace{0.5cm}
  \item Solve for $x$. \vspace{1.5cm}
  \item Check your answer. \vspace{1.5cm}
  \end{enumerate}


\newpage
\item Given point $B$ is the midpoint of $\overline{AC}$, with $AB=x+2$, $BC=11$. \\[0.3cm]
    First write and equation representing the situation, then find $x$.\\[0.3cm]
      %\begin{center}
        \begin{tikzpicture}
          \draw [fill] (0,0) circle [radius=0.05] node[below]{$A$};
          \draw [-, thick] (0,0)--(7,0);
          \draw [fill] (3.5,0) circle [radius=0.05] node[below]{$B$};
          \draw [fill] (7,0) circle [radius=0.05] node[below]{$C$};
          \node at (1.7,0.25) [above]{$x+2$};
          \node at (5.2,0.25) [above]{$11$};
          %\draw [<->, dashed] (0,-1)--(7,-1);
          %\node at (3.5,-1) [below]{$20$};
        \end{tikzpicture}
      %\end{center} 
      \vspace{1cm}

\item Find the value of each expression.
\begin{multicols}{2}
  \begin{enumerate}
    \item $|11|=$ \bigskip
    \item $|-7|=$
    \item $|-4.75|=$
    \item $|10-7|=$
  \end{enumerate}
\end{multicols} \vspace{0.5cm}

\item Given $\overleftrightarrow{QS}$ as shown on the number line. \\[20pt] % Midpoint
  \begin{tikzpicture}
    \draw [<->] (-4.5,0)--(6.5,0);
    \foreach \x in {-4,...,6} %2 leading for diff!=1
      \draw[shift={(\x,0)},color=black] (0pt,-3pt) -- (0pt,3pt) node[below=5pt]  {$\x$};
      \draw [fill] (2,0) circle [radius=0.05] node[above] {$Q$};
      \draw [fill] (6,0) circle [radius=0.05] node[above] {$S$};
  \end{tikzpicture} \bigskip
  \begin{enumerate}
    \item In the given number line units, what is the distance between $Q$ and $S$? \\[0.5cm]
    $QS=$
    \bigskip
    \item Mark the point $R$, the midpoint of $\overline{QS}$.
  \end{enumerate}\vspace{1cm}

\item Given $\overline{MN}$ with $M(-1)$ and $N(3)$, as shown on the number line. \\[0.25cm]
\begin{tikzpicture}
  \draw [<->] (-3.5,0)--(6.5,0);
  \draw [-, thick] (-1,0)--(3,0);
  \foreach \x in {-3,...,6} %2 leading for diff!=1
    \draw[shift={(\x,0)},color=black] (0pt,-3pt) -- (0pt,3pt) node[below=5pt]  {$\x$};
    \draw [fill] (-1,0) circle [radius=0.05] node[above] {$M$};
    \draw [fill] (3,0) circle [radius=0.05] node[above] {$N$};
\end{tikzpicture} \\ \bigskip
What is the length of the segment $\overline{MN}$? Show your work as an equation.


\end{enumerate}
\end{document}