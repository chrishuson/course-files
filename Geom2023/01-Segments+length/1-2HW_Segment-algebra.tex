\documentclass[12pt, twoside]{article}
\usepackage[letterpaper, margin=1in, headsep=0.2in]{geometry}
\setlength{\headheight}{0.6in}
%\usepackage[english]{babel}
\usepackage[utf8]{inputenc}
\usepackage{microtype}
\usepackage{amsmath}
\usepackage{amssymb}
%\usepackage{amsfonts}
\usepackage{siunitx} %units in math. eg 20\milli\meter
\usepackage{yhmath} % for arcs, overparenth command
\usepackage{tikz} %graphics
\usetikzlibrary{quotes, angles}
\usepackage{graphicx} %consider setting \graphicspath{{images/}}
\usepackage{parskip} %no paragraph indent
\usepackage{enumitem}
\usepackage{multicol}
\usepackage{venndiagram}

\usepackage{fancyhdr}
\pagestyle{fancy}
\fancyhf{}
\renewcommand{\headrulewidth}{0pt} % disable the underline of the header
\raggedbottom
\hfuzz=2mm %suppresses overfull box warnings

\usepackage{hyperref}

\fancyhead[LE]{\thepage}
\fancyhead[RO]{\thepage \\ Name: \hspace{4cm} \,\\}
\fancyhead[LO]{BECA / Dr. Huson / Geometry\\*  Unit 1: Segments, length, and area\\* 9 Sept 2022}

\begin{document}

\subsubsection*{1.2 Homework: Number line and algebra practice}
\begin{enumerate}
\item Given $\overline{PQ}$ as shown on the number line. \par \smallskip
    \begin{tikzpicture}
        \draw [<->] (-3.5,0)--(6.5,0);
        \foreach \x in {-3,...,6}
            \draw[shift={(\x,0)}] (0pt,-3pt)--(0pt,3pt) node[below=5pt]{$\x$};
        \draw [fill] (0,0) circle [radius=0.05] node[above]{$P$};
        \draw [fill] (5,0) circle [radius=0.05] node[above]{$Q$};
        \draw [-, thick] (-0,0)--(5,0);
    \end{tikzpicture} \par \smallskip
    What is the length of the segment $\overline{PQ}$? \par \smallskip
    $PQ=$ \vspace{2cm}

\item Two points $R(-2)$, $S(4)$ are shown on the number line. \par \smallskip
\begin{tikzpicture}
    \draw [<->] (-3.5,0)--(6.5,0);
    \foreach \x in {-3,...,6}
        \draw[shift={(\x,0)}] (0pt,-3pt)--(0pt,3pt) node[below=5pt]{$\x$};
    \draw [fill] (-2,0) circle [radius=0.05] node[above]{$R$};
    \draw [fill] (4,0) circle [radius=0.05] node[above]{$S$};
\end{tikzpicture} \par \smallskip
What is the distance between $R$ and $S$? Show your work as an equation. \vspace{3cm}

\item Measure the segment $\overline{TU}$. Write its length in centimeters (expressed as an equation). \vspace{1cm}
    \begin{center}
    \begin{tikzpicture}
        \draw [-, thick] (0,0)--(12,5);
        \draw [fill] (0,0) circle [radius=0.05] node[above left]{$T$};
        \draw [fill] (12,5) circle [radius=0.05] node[below right]{$U$};
    \end{tikzpicture}
    \end{center} \vspace{1cm}

\item Points that fall on the same straight line are $\rule{5cm}{0.15mm}$.

\newpage
%\emph{A check is required for all algebra solutions}
\item Given $\overline{RST}$, $RS=3 \frac{2}{3}$, and $ST=4 \frac{2}{3}$. Find ${RT}$ (expressed as a fraction, not a decimal). \par \smallskip
\begin{tikzpicture}
\draw [-, thick] (0,0)--(7,0);
\draw [fill] (0,0) circle [radius=0.05] node[below]{$R$};
\draw [fill] (3,0) circle [radius=0.05] node[below]{$S$};
\draw [fill] (7,0) circle [radius=0.05] node[below]{$T$};
\end{tikzpicture} \vspace{2cm}

\item As shown, three collinear points with $LM=2x+3$, $MN=7$, $LN=22$. Find ${x}$. \vspace{1cm}
    \begin{center}
    \begin{tikzpicture}
        \draw [-, thick] (0,0)--(9,0);
        \draw [fill] (0,0) circle [radius=0.05] node[below]{$L$};
        \draw [fill] (6,0) circle [radius=0.05] node[below]{$M$};
        \draw [fill] (9,0) circle [radius=0.05] node[below]{$N$};
        \node at (3,0) [above]{$2x+3$};
        \node at (7.5,0) [above]{$7$};
        \draw [<->, dashed] (0,-1)--(9,-1);
        \node at (4.5,-1) [below]{$22$};
    \end{tikzpicture}
    \end{center}
   %\vspace{1cm}
  \begin{enumerate}
    \item Write down an equation to represent the situation. \vspace{0.5cm}
    \item Solve for $x$. \vspace{2.5cm}
    \item Check your answer. \vspace{2cm}
  \end{enumerate}

\item Two textbooks are stacked up. One is a heavy calculus book, two inches thick. The other is one inch thick, \emph{Topics in Topography}. How tall is the stack of both books? \vspace{3cm}

\item Dr. Huson is 5 foot 7 inches tall. If he stepped up onto a 6 inch box how tall would he be then?

  
\end{enumerate}
\end{document}