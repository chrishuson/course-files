\documentclass[12pt, twoside]{article}
\usepackage[letterpaper, margin=1in, headsep=0.2in]{geometry}
\setlength{\headheight}{0.6in}
%\usepackage[english]{babel}
\usepackage[utf8]{inputenc}
\usepackage{microtype}
\usepackage{amsmath}
\usepackage{amssymb}
%\usepackage{amsfonts}
\usepackage{siunitx} %units in math. eg 20\milli\meter
\usepackage{yhmath} % for arcs, overparenth command
\usepackage{tikz} %graphics
\usetikzlibrary{quotes, angles}
\usepackage{graphicx} %consider setting \graphicspath{{images/}}
\usepackage{parskip} %no paragraph indent
\usepackage{enumitem}
\usepackage{multicol}
\usepackage{venndiagram}

\usepackage{fancyhdr}
\pagestyle{fancy}
\fancyhf{}
\renewcommand{\headrulewidth}{0pt} % disable the underline of the header
\raggedbottom
\hfuzz=2mm %suppresses overfull box warnings

\usepackage{hyperref}

\fancyhead[LE]{\thepage}
\fancyhead[RO]{\thepage \\ Name: \hspace{4cm} \,\\}
\fancyhead[LO]{BECA / Dr. Huson / Geometry\\*  Unit 1: Segments, length, and area\\* 19 Sept 2022}

\begin{document}

\subsubsection*{1.7 Exit Note Quiz: Length and perimeter, geometric notation}
\begin{enumerate}
\item Given $\overleftrightarrow{JK}$ as shown on the number line. \\[20pt] % Midpoint
  \begin{tikzpicture}[scale=0.5]
    \draw [<->] (49,0)--(71,0);
    \foreach \x in {50, 52,...,70} %2 leading for diff!=1
      \draw[shift={(\x,0)},color=black] (0pt,-6pt) -- (0pt,6pt) node[below=5pt]  {$\x$};
      \draw [fill] (54,0) circle [radius=0.1] node[above] {$J$};
      \draw [fill] (68,0) circle [radius=0.1] node[above] {$K$};
  \end{tikzpicture} \\ \bigskip
  What is the midpoint between the points $J$ and $K$? \vspace{4cm}  

\item Given $\overline{RST}$, $S$ bisects $\overline{RT}$, $RS=17x-10$, $ST=13x-2$. Find ${RT}$.\\
  Complete all the steps for full credit.\vspace{7cm}

\item Given $\overline{FGHI}$, $FG=8 \frac{1}{6}$, $GH=12 \frac{1}{3}$, and $HI= 5 \frac{1}{2}$. (diagram not to scale)\\ [0.25cm]
  Find ${FI}$.\\[.5in]
      \begin{tikzpicture}
        \draw [-, thick] (0,0)--(9,0);
        \draw [fill] (0,0) circle [radius=0.05] node[below]{$F$};
        \draw [fill] (3,0) circle [radius=0.05] node[below]{$G$};
        \draw [fill] (7,0) circle [radius=0.05] node[below]{$H$};
        \draw [fill] (9,0) circle [radius=0.05] node[below]{$I$};
      \end{tikzpicture} \vspace{1cm}


\end{enumerate}
\end{document}