\documentclass[12pt, twoside]{article}
\usepackage[letterpaper, margin=1in, headsep=0.2in]{geometry}
\setlength{\headheight}{0.6in}
%\usepackage[english]{babel}
\usepackage[utf8]{inputenc}
\usepackage{microtype}
\usepackage{amsmath}
\usepackage{amssymb}
%\usepackage{amsfonts}
\usepackage{siunitx} %units in math. eg 20\milli\meter
\usepackage{yhmath} % for arcs, overparenth command
\usepackage{tikz} %graphics
\usetikzlibrary{quotes, angles}
\usepackage{graphicx} %consider setting \graphicspath{{images/}}
\usepackage{parskip} %no paragraph indent
\usepackage{enumitem}
\usepackage{multicol}
\usepackage{venndiagram}

\usepackage{fancyhdr}
\pagestyle{fancy}
\fancyhf{}
\renewcommand{\headrulewidth}{0pt} % disable the underline of the header
\raggedbottom
\hfuzz=2mm %suppresses overfull box warnings

\usepackage{hyperref}

\fancyhead[LE]{\thepage}
\fancyhead[RO]{\thepage \\ Name: \hspace{4cm} \,\\}
\fancyhead[LO]{BECA / Dr. Huson / Geometry\\*  Unit 1: Segments, length, and area\\* 11 Sept 2022}

\begin{document}

\subsubsection*{1.4 Classwork: Midpoints and bisectors}
\begin{enumerate}
\item Given point $B$ is the midpoint of $\overline{AC}$, with $AB=x+2$, $BC=11$. \\[0.3cm]
First write and equation representing the situation, then find $x$.\\[0.3cm]
    %\begin{center}
    \begin{tikzpicture}
        \draw [fill] (0,0) circle [radius=0.05] node[below]{$A$};
        \draw [-, thick] (0,0)--(7,0);
        \draw [fill] (3.5,0) circle [radius=0.05] node[below]{$B$};
        \draw [fill] (7,0) circle [radius=0.05] node[below]{$C$};
        \node at (1.7,0.25) [above]{$x+2$};
        \node at (5.2,0.25) [above]{$11$};
        %\draw [<->, dashed] (0,-1)--(7,-1);
        %\node at (3.5,-1) [below]{$20$};
    \end{tikzpicture}
    %\end{center} 
    \vspace{1cm}

\item Do Now: Given $M$ is the midpoint of $\overline{AB}$, $AM=5x+2$, $MB=20$.
\begin{enumerate}
    \item Mark the diagram with the values and tick marks
    \item Write an equation and solve for $x$
    \item Check your result
\end{enumerate}
    \begin{center}
    \begin{tikzpicture}
        \draw [fill] (0,0) circle [radius=0.05] node[below]{$A$};
        \draw [-, thick] (0,0)--(7,0);
        \draw [fill] (3.5,0) circle [radius=0.05] node[below]{$M$};
        \draw [fill] (7,0) circle [radius=0.05] node[below]{$B$};
        %\node at (1.7,0.5) [above]{$x+2$};
        %\node at (5.2,0.5) [above]{$11$};
        %\draw [<->, dashed] (0,-1)--(7,-1);
        %\node at (3.5,-1) [below]{$20$};
    \end{tikzpicture}
    \end{center} \vspace{2cm}

      
\end{enumerate}
\end{document}