\documentclass[12pt, twoside]{article}
\usepackage[letterpaper, margin=1in, headsep=0.2in]{geometry}
\setlength{\headheight}{0.6in}
%\usepackage[english]{babel}
\usepackage[utf8]{inputenc}
\usepackage{microtype}
\usepackage{amsmath}
\usepackage{amssymb}
%\usepackage{amsfonts}
\usepackage{siunitx} %units in math. eg 20\milli\meter
\usepackage{yhmath} % for arcs, overparenth command
\usepackage{tikz} %graphics
\usetikzlibrary{quotes, angles}
\usepackage{graphicx} %consider setting \graphicspath{{images/}}
\usepackage{parskip} %no paragraph indent
\usepackage{enumitem}
\usepackage{multicol}
\usepackage{venndiagram}

\usepackage{fancyhdr}
\pagestyle{fancy}
\fancyhf{}
\renewcommand{\headrulewidth}{0pt} % disable the underline of the header
\raggedbottom
\hfuzz=2mm %suppresses overfull box warnings

\usepackage{hyperref}

\fancyhead[LE]{\thepage}
\fancyhead[RO]{\thepage \\ Name: \hspace{4cm} \,\\}
\fancyhead[LO]{BECA / Dr. Huson / Geometry\\*  Unit 1: Segments, length, and area\\* 16 Sept 2022}

\begin{document}

\subsubsection*{1.7 Extension Quiz: Absolute value, trisection, algebra}
\begin{enumerate}
\item Given $\overleftrightarrow{DG}$ as shown on the number line, with $D=11$ and $G=26$. \\[15pt]
  \begin{tikzpicture}[scale=0.5]
    \draw [<->] (9,0)--(31,0);
    \foreach \x in {10, 12,...,30} %2 leading for diff!=1
      \draw[shift={(\x,0)},color=black] (0pt,-6pt) -- (0pt,6pt) node[below=5pt]  {$\x$};
      \draw [fill] (11,0) circle [radius=0.1] node[above] {$D$};
      \draw [fill] (26,0) circle [radius=0.1] node[above] {$G$};
  \end{tikzpicture} \\[10pt]
  Points $E$ and $F$ trisect $\overline{DG}$. Find the values of $E$ and $F$ and mark and label them on the number line $\overleftrightarrow{DG}$. \vspace{6cm}

\item Given $\overleftrightarrow{RS}$ as shown on the number line, with $R=-2.8$ and $S=4.4$. \\[5pt]
\begin{tikzpicture}
  \draw [<->] (-4.5,0)--(6.5,0);
  \foreach \x in {-4,...,6} %2 leading for diff!=1
    \draw[shift={(\x,0)},color=black] (0pt,-3pt) -- (0pt,3pt) node[below=5pt]  {$\x$};
    \draw [fill] (-2.8,0) circle [radius=0.05] node[above] {$R$};
    \draw [fill] (4.4,0) circle [radius=0.05] node[above] {$S$};
\end{tikzpicture} \\ 
The points $T$ and $U$ trisect $\overline{RS}$. Find their values, and mark and label them on the number line. \vspace{4cm}

\newpage
\item Given $\overline{PQR}$, with $PQ=\frac{1}{2} x+4$, $QR=x+3$, and $PR=2x+5$. Find ${PR}$.\\
  Complete all the steps for full credit. \smallskip
  \vspace{9cm}

\item Given $\overline{ABC}$, $AB=\frac{2}{3}$, and $AC=3 \frac{1}{3}$.\\ [0.5cm]
  Find ${BC}$.\\[1.5cm]
  \begin{tikzpicture}
    \draw [-, thick] (1,0)--(7,0);
    \draw [fill] (1,0) circle [radius=0.05] node[below]{$A$};
    \draw [fill] (2,0) circle [radius=0.05] node[below]{$B$};
    \draw [fill] (7,0) circle [radius=0.05] node[below]{$C$};
  \end{tikzpicture}

\item Given $\overline{PQR}$, with $PQ=4x-4$, $QR=2x+3$, and $PR=5x+9$. Find ${PR}$.\\
Complete all the steps for full credit. \smallskip
\vspace{9cm}

\item Given $\overline{DEF}$, $DF=75$ and $\overline{DE}$ is half the length of $\overline{EF}$. \\ [0.25cm]
Find ${DE}$.\\[.5in]
    \begin{tikzpicture}
      \draw [-, thick] (1,0)--(8,0);
      \draw [fill] (1,0) circle [radius=0.05] node[below]{$D$};
      \draw [fill] (3.4,0) circle [radius=0.05] node[below]{$E$};
      \draw [fill] (8,0) circle [radius=0.05] node[below]{$F$};
    \end{tikzpicture} \vspace{2.5cm}

\item Given $\overleftrightarrow{PQ}$ as shown on the number line. Divide segment $\overline{PQ}$ into five congruent segments by marking and labeling the points $R$, $S$, $T$, and $U$ on the numberline.\\[20pt] % Midpoint
  \begin{tikzpicture}
    \draw [<->] (-4.5,0)--(6.5,0);
    \foreach \x in {-4,...,6} %2 leading for diff!=1
      \draw[shift={(\x,0)},color=black] (0pt,-3pt) -- (0pt,3pt) node[below=5pt]  {$\x$};
      \draw [fill] (-4,0) circle [radius=0.05] node[above] {$P$};
      \draw [fill] (6,0) circle [radius=0.05] node[above] {$Q$};
  \end{tikzpicture}

  \item The perimeter of the isosceles $\triangle FGH$ is 35 with $\overline{FH} \cong \overline{GH}$. If $FG=x+5$ and $FH=13 \frac{1}{2}$, find $x$.\\[0.25cm]
  Show your work with an equation for full credit.\\[0.5cm]
    \begin{tikzpicture}[scale=0.5]
      \draw [thick](0,0)--(4,0)--(2,6)--(0,0);
      \draw [fill] (0,0) circle [radius=0.05] node[below left]{$F$};
      \draw [fill] (4,0) circle [radius=0.05] node[below right]{$G$};
      \draw [fill] (2,6) circle [radius=0.05] node[above right]{$H$};
      \draw [thick] (0.8,3.1)--(1.2,3); %tick mark
      \draw [thick] (2.8,3)--(3.2,3.1); %tick mark
      \node at (2,0) [below]{$x+5$};
      \node at (0.8,3.4) [left]{$13 \frac{1}{2}$};
    \end{tikzpicture} \vspace{1cm}

\item Given $\overline{PQR}$, $PQ=2x+11$, $QR=x+1$, $PR=x+26$. Find ${x}$.
  \begin{center}
    \begin{tikzpicture}
    \draw [-, thick] (0,0)--(7,0);
    \draw [fill] (0,0) circle [radius=0.05] node[below]{$P$};
    \draw [fill] (5,0) circle [radius=0.05] node[below]{$Q$};
    \draw [fill] (7,0) circle [radius=0.05] node[below]{$R$};
    \node at (2,0) [above]{$2x+11$};
    \node at (6,0) [above]{$x+1$};
    \draw [<->, dashed] (0,-0.7)--(7,-0.7);
    \node at (3.5,-0.7) [below]{$x+26$};
  \end{tikzpicture}
  \end{center}
  \begin{enumerate}
    \item Write down an equation to represent the situation. \vspace{0.5cm}
    \item Solve for $x$. \vspace{1.5cm}
    \item Check your answer. \vspace{2cm}
  \end{enumerate}

\item Given $\overline{DEF}$, $DE=3 \frac{1}{3}$, and $EF=1$. Find ${DF}$. \par \bigskip
\begin{tikzpicture}
  \draw [-, thick] (1,0)--(7,0);
  \draw [fill] (1,0) circle [radius=0.05] node[below]{$D$};
  \draw [fill] (5,0) circle [radius=0.05] node[below]{$E$};
  \draw [fill] (7,0) circle [radius=0.05] node[below]{$F$};
\end{tikzpicture} \bigskip

\item Given $P(-2.4)$ and $Q(1.8)$, as shown on the number line. 
  Find the length of the line segment $\overline{PQ}$. 
  \begin{flushright}
    \begin{tikzpicture}
      \draw[<->] (-4.5,0)--(4.5,0);
      \foreach \x in {-4,...,4}
        \draw[shift={(\x,0)}] (0pt,-3pt)--(0pt,3pt) node[below=5pt]{$\x$};
      \draw[fill] (-2.4,0) circle [radius=0.05] node[above]{$P$};
      \draw[fill] (1.8,0) circle [radius=0.05] node[above]{$Q$};
      \draw[thick] (-2.4,0)--(1.8,0);
    \end{tikzpicture}
  \end{flushright} \vspace{1cm}

\item Given $\overline{DEFG}$, $DE=3 \frac{1}{4}$, $EF=6 \frac{1}{4}$, and $FG= 1 \frac{3}{4}$. (diagram not to scale) \par \smallskip
Find ${DG}$, expressed as a fraction, not a decimal. \vspace{1cm}
  \begin{flushleft}
    \begin{tikzpicture}
      \draw[thick] (0,0)--(9,0);
      \draw[fill] (0,0) circle [radius=0.05] node[below]{$D$};
      \draw[fill] (3,0) circle [radius=0.05] node[below]{$E$};
      \draw[fill] (7,0) circle [radius=0.05] node[below]{$F$};
      \draw[fill] (9,0) circle [radius=0.05] node[below]{$G$};
    \end{tikzpicture}
  \end{flushleft}
  
\item Given $\overline{FGHI}$, $FG=8 \frac{1}{6}$, $GH=12 \frac{1}{3}$, and $HI= 5 \frac{1}{2}$. (diagram not to scale) \par \smallskip
Find ${FI}$.\\[.5in]
  \begin{tikzpicture}
    \draw[thick] (0,0)--(9,0);
    \draw[fill] (0,0) circle [radius=0.05] node[below]{$F$};
    \draw[fill] (3,0) circle [radius=0.05] node[below]{$G$};
    \draw[fill] (7,0) circle [radius=0.05] node[below]{$H$};
    \draw[fill] (9,0) circle [radius=0.05] node[below]{$I$};
  \end{tikzpicture} \vspace{1cm}

\item Given $\overleftrightarrow{JK}$ as shown on the number line. \par \smallskip
\begin{tikzpicture}[scale=0.5]
  \draw[<->] (49,0)--(71,0);
  \foreach \x in {50, 52,...,70}
    \draw[shift={(\x,0)}] (0pt,-6pt)--(0pt,6pt) node[below=5pt]{$\x$};
  \draw[fill] (54,0) circle [radius=0.1] node[above] {$J$};
  \draw[fill] (68,0) circle [radius=0.1] node[above] {$K$};
\end{tikzpicture} \par \bigskip
What is the midpoint between the points $J$ and $K$?

\end{enumerate}
\end{document}