\documentclass[12pt, twoside]{article}
\usepackage[letterpaper, margin=1in, headsep=0.2in]{geometry}
\setlength{\headheight}{0.6in}
%\usepackage[english]{babel}
\usepackage[utf8]{inputenc}
\usepackage{microtype}
\usepackage{amsmath}
\usepackage{amssymb}
%\usepackage{amsfonts}
\usepackage{siunitx} %units in math. eg 20\milli\meter
\usepackage{yhmath} % for arcs, overparenth command
\usepackage{tikz} %graphics
\usetikzlibrary{quotes, angles}
\usepackage{graphicx} %consider setting \graphicspath{{images/}}
\usepackage{parskip} %no paragraph indent
\usepackage{enumitem}
\usepackage{multicol}
\usepackage{venndiagram}

\usepackage{fancyhdr}
\pagestyle{fancy}
\fancyhf{}
\renewcommand{\headrulewidth}{0pt} % disable the underline of the header
\raggedbottom
\hfuzz=2mm %suppresses overfull box warnings

\usepackage{hyperref}

\fancyhead[LE]{\thepage}
\fancyhead[RO]{\thepage \\ Name: \hspace{4cm} \,\\}
\fancyhead[LO]{BECA / Dr. Huson / Geometry\\*  Unit 1: Segments, length, and area\\* 19 Sept 2022}

\begin{document}

\subsubsection*{1.7 Extension Quiz: Absolute value, trisection, algebra}
\emph{All algebraic solutions require a check for full credit.}
\begin{enumerate}
\item Given $\overline{DEF}$, $DE=3 \frac{2}{3}$, and $EF=1 \frac{2}{3}$. Find ${DF}$. \par \vspace{1cm}
\begin{tikzpicture}
  \draw[thick] (1,0)--(7,0);
  \draw[fill] (1,0) circle [radius=0.05] node[below]{$D$};
  \draw[fill] (5,0) circle [radius=0.05] node[below]{$E$};
  \draw[fill] (7,0) circle [radius=0.05] node[below]{$F$};
\end{tikzpicture} \bigskip

\item Given $P(-3.4)$ and $Q(1.7)$, as shown on the number line. 
  Find the length of the line segment $\overline{PQ}$. 
  \begin{center}
    \begin{tikzpicture}
      \draw[<->] (-4.5,0)--(4.5,0);
      \foreach \x in {-4,...,4}
        \draw[shift={(\x,0)}] (0pt,-3pt)--(0pt,3pt) node[below=5pt]{$\x$};
      \draw[fill] (-3.4,0) circle [radius=0.05] node[above]{$P(-3.4)$};
      \draw[fill] (1.7,0) circle [radius=0.05] node[above]{$Q(1.7)$};
      \draw[thick] (-2.4,0)--(1.8,0);
    \end{tikzpicture}
  \end{center} \vspace{2cm}

\item Given $x=-3$ simplify each expression.
\begin{multicols}{2}
  \begin{enumerate}[itemsep=1.5cm]
    \item $|x-2|=$
    \item $|-1-x|=$
    \item $|x-1|+|x|=$
    \item $3 \times |-x|+x=$
  \end{enumerate}
\end{multicols} \vspace{1cm}

\item Find all values of $x$ satisfying the equation. (show the two cases for each problem) \bigskip
\begin{multicols}{2} 
  \begin{enumerate}
    \item $|2x|=8$
    \item $|x-2|+2 = 7$
  \end{enumerate}
\end{multicols}

\newpage
\item The segment $\overline{AC}$ is bisected by point $B$, $AB=\frac{1}{2}(3x+13)$ and $AC=22$.
  Find $x$.
  \begin{flushleft}
    \begin{tikzpicture}[scale=1]
      \draw[thick] (0,0)--(8,0);
      \draw[fill] (0,0) circle [radius=0.05] node[below]{$A$};
      \draw[fill] (4,0) circle [radius=0.05] node[below]{$B$};
      \draw[fill] (8,0) circle [radius=0.05] node[below]{$C$};
      \node at (2,0.7){$\frac{1}{2}(3x+13)$};
      \draw[<->, dashed] (0,-1)--(8,-1);
      \node at (4,-1.5){$22$};
      \draw (1.8,-0.2)--(1.9,0.2);
      \draw (2.1,-0.2)--(2.2,0.2);
      \draw (5.8,-0.2)--(5.9,0.2);
      \draw (6.1,-0.2)--(6.2,0.2);
    \end{tikzpicture}
  \end{flushleft} \vspace{4cm}

\item The perimeter of the isosceles $\triangle FGH$ is 35 with $\overline{FH} \cong \overline{GH}$. If $FG=x+5$ and $FH=13 \frac{1}{2}$, find $x$.  \par \smallskip
  \begin{tikzpicture}[scale=0.5]
    \draw[thick](0,0)--(4,0)--(2,6)--(0,0);
    \draw[fill] (0,0) circle [radius=0.05] node[below left]{$F$};
    \draw[fill] (4,0) circle [radius=0.05] node[below right]{$G$};
    \draw[fill] (2,6) circle [radius=0.05] node[above right]{$H$};
    \draw[thick] (0.8,3.1)--(1.2,3); %tick mark
    \draw[thick] (2.8,3)--(3.2,3.1); %tick mark
    \node at (2,0) [below]{$x+5$};
    \node at (0.8,3.4) [left]{$13 \frac{1}{2}$};
  \end{tikzpicture} \vspace{3cm}

\item Given points $K(9)$ and $L(23)$, find the value of $M$ such that $L$ is the bisector of $\overline{KM}$. Mark $M$ and label it with its value on the number line. \par \medskip
\begin{tikzpicture}[scale=0.3]
  \draw[<->] (3,0)--(47,0);
  \foreach \x in {5, 10,...,45}
    \draw[shift={(\x,0)}] (0pt,-8pt)--(0pt,8pt) node[below=5pt]{$\x$};
  \draw[fill] (9,0) circle [radius=0.2] node[above]{$K(9)$};
  \draw[fill] (23,0) circle [radius=0.2] node[above]{$L(23)$};
  \end{tikzpicture}

\newpage
\item The points $X$ and $Y$ trisect the line segment $\overline{WZ}$, as shown below. If $WY=14$, find $WZ$.
\begin{center}
  \begin{tikzpicture}[scale=1]
    \draw[thick] (0,0)--(9,0);
    \draw[fill] (0,0) circle [radius=0.05] node[below]{$W$};
    \draw[fill] (3,0) circle [radius=0.05] node[below]{$X$};
    \draw[fill] (6,0) circle [radius=0.05] node[below]{$Y$};
    \draw[fill] (9,0) circle [radius=0.05] node[below]{$Z$};
    \draw[thick] (1.4,-0.2)--(1.4,0.2);
    \draw[thick] (1.55,-0.2)--(1.55,0.2);
    \draw[thick] (4.4,-0.2)--(4.4,0.2);
    \draw[thick] (4.55,-0.2)--(4.55,0.2);
    \draw[thick] (7.4,-0.2)--(7.4,0.2);
    \draw[thick] (7.55,-0.2)--(7.55,0.2);
  \end{tikzpicture}
\end{center} \vspace{3cm}

\item The perimeter of rectangle $ABCD$ is 70 centimeters and its length is one and a half times its width. Find the rectangle's dimensions.
  \begin{flushleft}
  \begin{tikzpicture}[scale=0.7]
    \draw[thick]
      (0,0)node[below left]{$A$}--
      (6,0)node[below right]{$B$}--
      (6,4)node[above right]{$C$}--
      (0,4)node[above left]{$D$}--cycle;
    \node at (3,-0.8){$1 \frac{1}{2} \ x$ cm};
    \node at (7,2.5){$x$ cm};
    \draw (2.9, -0.2)--(2.9,0.2);
    \draw (2.9, 3.8)--(2.9,4.2);
    \draw (-0.2, 1.9)--(0.2, 1.9);
    \draw (-0.2, 2.1)--(0.2, 2.1);
    \draw (5.8, 1.9)--(6.2, 1.9);
    \draw (5.8, 2.1)--(6.2, 2.1);
    \end{tikzpicture}
  \end{flushleft} \vspace{4cm}

\item Given $\overleftrightarrow{DG}$ as shown on the number line, with $D=10$ and $G=28$. \par \smallskip
  \begin{tikzpicture}[scale=0.5]
    \draw[<->] (7,0)--(31,0);
    \foreach \x in {8, 10,...,30}
      \draw[shift={(\x,0)}] (0pt,-6pt)--(0pt,6pt) node[below=5pt]{$\x$};
    \draw[fill] (10,0) circle [radius=0.1] node[above]{$D$};
    \draw[fill] (28,0) circle [radius=0.1] node[above]{$G$};
  \end{tikzpicture} \par \smallskip
  Points $E$ and $F$ trisect $\overline{DG}$. Find the values of $E$ and $F$ and mark and label them on the number line $\overleftrightarrow{DG}$. \vspace{3cm}

\newpage
\item Given $\overline{PQR}$, $PQ=x-2$, $QR=\frac{1}{3}x$, $PR=\frac{1}{3}(3x+6)$. Find ${x}$.
  \begin{flushleft}
    \begin{tikzpicture}
    \draw[thick] (0,0)--(7,0);
    \draw[fill] (0,0) circle [radius=0.05] node[below]{$P$};
    \draw[fill] (5,0) circle [radius=0.05] node[below]{$Q$};
    \draw[fill] (7,0) circle [radius=0.05] node[below]{$R$};
    \node at (2,0) [above]{$x-2$};
    \node at (6,0) [above]{$\frac{1}{3}x$};
    \draw[<->, dashed] (0,-0.7)--(7,-0.7);
    \node at (3.5,-0.7) [below]{$\frac{1}{3}(3x+6)$};
  \end{tikzpicture}
  \end{flushleft} \vspace{6cm}

\item Given $\overline{DEF}$, $DF=75$ and $\overline{DE}$ is half the length of $\overline{EF}$. Find ${DE}$. \par \bigskip
  \begin{tikzpicture}
    \draw[thick] (0,0)--(9,0);
    \draw[fill] (0,0) circle [radius=0.05] node[below]{$D$};
    \draw[fill] (3,0) circle [radius=0.05] node[below]{$E$};
    \draw[fill] (9,0) circle [radius=0.05] node[below]{$F$};
  \end{tikzpicture} \vspace{6cm}

\subsubsection*{Academic integrity pledge}
This assignment must be completed in one sitting. Use your notes and a calculator. \par \smallskip
I have not received any human help on this assignment. \par \bigskip
Signed: \rule{6cm}{0.15mm} \hfill \rule{6cm}{0.15mm}
\begin{flushright}
  Date, start time - end time
\end{flushright}


\end{enumerate}
\end{document}

