\documentclass[12pt, twoside]{article}
\usepackage[letterpaper, margin=1in, headsep=0.2in]{geometry}
\setlength{\headheight}{0.6in}
%\usepackage[english]{babel}
\usepackage[utf8]{inputenc}
\usepackage{microtype}
\usepackage{amsmath}
\usepackage{amssymb}
%\usepackage{amsfonts}
\usepackage{siunitx} %units in math. eg 20\milli\meter
\usepackage{yhmath} % for arcs, overparenth command
\usepackage{tikz} %graphics
\usetikzlibrary{quotes, angles}
\usepackage{graphicx} %consider setting \graphicspath{{images/}}
\usepackage{parskip} %no paragraph indent
\usepackage{enumitem}
\usepackage{multicol}
\usepackage{venndiagram}

\usepackage{fancyhdr}
\pagestyle{fancy}
\fancyhf{}
\renewcommand{\headrulewidth}{0pt} % disable the underline of the header
\raggedbottom
\hfuzz=2mm %suppresses overfull box warnings

\usepackage{hyperref}

\fancyhead[LE]{\thepage}
\fancyhead[RO]{\thepage \\ Name: \hspace{4cm} \,\\}
\fancyhead[LO]{BECA / Dr. Huson / Geometry\\*  Unit 1: Segments, length, and area\\* 8 Sept 2022}

\begin{document}

\subsubsection*{1.5 Classwork: Equilateral and isosceles triangles, perimeter}
\begin{enumerate}
\item Given $\triangle JKL$ with $\overline{JK} \cong \overline{JL}$. On the diagram mark the congruent line segments with tick marks.
\begin{center}
\begin{tikzpicture}[scale=0.5]
  \draw [thick](0,0)--(9,0)--(4,8)--(0,0);
  \draw [fill] (0,0) circle [radius=0.05] node[below]{$J$};
  \draw [fill] (9,0) circle [radius=0.05] node[below]{$K$};
  \draw [fill] (4,8) circle [radius=0.05] node[above right]{$L$};
\end{tikzpicture}
\end{center}

\item Given isosceles $\triangle XYZ$ with $\overline{XY} \cong \overline{YZ}$. On the diagram mark the congruent line segments with tick marks.
\begin{center}
\begin{tikzpicture}[scale=0.3]
  \draw [thick](0,0)--(9,0)--(4,8)--(0,0);
  \draw [fill] (0,0) circle [radius=0.05] node[below]{$X$};
  \draw [fill] (9,0) circle [radius=0.05] node[below]{$Y$};
  \draw [fill] (4,8) circle [radius=0.05] node[above right]{$Z$};
\end{tikzpicture}
\end{center}

\item Given isosceles $\triangle XYZ$ with $\overline{XY} \cong \overline{XZ}$. On the diagram mark the congruent line segments with tick marks.
\begin{center}
\begin{tikzpicture}[scale=0.3]
  \draw [thick](0,0)--(9,0)--(4,8)--(0,0);
  \draw [fill] (0,0) circle [radius=0.05] node[below]{$X$};
  \draw [fill] (9,0) circle [radius=0.05] node[below]{$Y$};
  \draw [fill] (4,8) circle [radius=0.05] node[above right]{$Z$};
\end{tikzpicture}
\end{center}

\item The perimeter of the isosceles $\triangle FGH$ is $19 \frac{1}{2}$ with $\overline{FH} \cong \overline{GH}$. If $FG=x+2$ and $FH=8 \frac{1}{4}$, find $x$.\\[0.25cm]
    Show your work with an equation.\\[0.5cm]
      \begin{tikzpicture}[scale=0.5]
        \draw [thick](0,0)--(4,0)--(2,6)--(0,0);
        \draw [fill] (0,0) circle [radius=0.05] node[below left]{$F$};
        \draw [fill] (4,0) circle [radius=0.05] node[below right]{$G$};
        \draw [fill] (2,6) circle [radius=0.05] node[above right]{$H$};
        \draw [thick] (0.8,3.1)--(1.2,3); %tick mark
        \draw [thick] (2.8,3)--(3.2,3.1); %tick mark
        \node at (2,0) [below]{$x+2$};
        \node at (0.8,3.4) [left]{$8 \frac{1}{4}$};
      \end{tikzpicture}

\end{enumerate}
\end{document}