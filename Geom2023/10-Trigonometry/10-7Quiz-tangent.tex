\documentclass[12pt, twoside]{article}
\documentclass[12pt, twoside]{article}
\usepackage[letterpaper, margin=1in, headsep=0.2in]{geometry}
\setlength{\headheight}{0.6in}
%\usepackage[english]{babel}
\usepackage[utf8]{inputenc}
\usepackage{microtype}
\usepackage{amsmath}
\usepackage{amssymb}
%\usepackage{amsfonts}
\usepackage{siunitx} %units in math. eg 20\milli\meter
\usepackage{yhmath} % for arcs, overparenth command
\usepackage{tikz} %graphics
\usetikzlibrary{quotes, angles}
\usepackage{graphicx} %consider setting \graphicspath{{images/}}
\usepackage{parskip} %no paragraph indent
\usepackage{enumitem}
\usepackage{multicol}
\usepackage{venndiagram}

\usepackage{fancyhdr}
\pagestyle{fancy}
\fancyhf{}
\renewcommand{\headrulewidth}{0pt} % disable the underline of the header
\raggedbottom
\hfuzz=2mm %suppresses overfull box warnings

\usepackage{hyperref}

\fancyhead[LE]{\thepage}
\fancyhead[RO]{\thepage \\ Name: \hspace{4cm} \,\\}
\fancyhead[LO]{BECA / Dr. Huson / Geometry\\*  Unit 10: Trigonometry \\* 2 May 2023}

\begin{document}

\subsubsection*{10.7 Quiz: The tangent function \hfill CCSS.HSG.SRT.C.8}
You must write an equation before solving it. Figures are not necessarily drawn to scale.
\begin{enumerate}
\item Given right $\triangle ABC$ with $AC=10$, m$\angle A=40^\circ$. Find the value of $BC=x$.
  \begin{flushright}
  \begin{tikzpicture}[scale=0.8]
    \draw[thick] (0,0)node[below]{$A$}--
      (6,0)node[below]{$C$}--
      (6,5)node[above]{$B$}--cycle;
    \draw (6,0)++(-0.6,0)--++(0,0.6)--+(0.6,0);
    \node at (3,0)[below]{$10$};
    \node at (6,2.5)[right]{$x$};
    \draw (1,0) arc (0:40:1)node[pos=0.7,right]{$40^\circ$};
  \end{tikzpicture}
  \end{flushright}

\item The right $\triangle ABC$ has a height of $BC=17$ and m$\angle A=58^\circ$. Find the length of its base $AC=x$.
  \begin{flushright}
  \begin{tikzpicture}[scale=0.7]
    \draw[thick] (0,0)node[below]{$A$}--
      (5,0)node[below]{$C$}--
      (5,8)node[above]{$B$}--cycle;
    \draw (5,0)++(-0.6,0)--++(0,0.6)--+(0.6,0);
    \node at (2.5,0)[below]{$x$};
    \node at (5,4)[right]{$17$};
    \draw (1,0) arc (0:58:1)node[pos=0.7,right]{$58^\circ$};
  \end{tikzpicture}
  \end{flushright}

\item The lengths of the legs of right $\triangle ABC$ are $AC=50$ and $BC=25$. Find m$\angle A=x$.
  \begin{flushright}
  \begin{tikzpicture}[scale=1]
    \draw[thick] (0,0)node[below]{$A$}--
      (6,0)node[below]{$C$}--
      (6,3)node[above]{$B$}--cycle;
    \draw (6,0)++(-0.5,0)--++(0,0.5)--+(0.5,0);
    \node at (3,0)[below]{$50$};
    \node at (6,1.5)[right]{$25$};
    \draw (1,0) arc (0:27:1)node[pos=0.7,right]{$x^\circ$};
  \end{tikzpicture}
  \end{flushright}

\newpage
\item The dimensions of right $\triangle ABC$ are $AC=12$ and $BC=5$. Find length of the hypotenuse $AB=x$.
  \begin{flushright}
  \begin{tikzpicture}[scale=1]
    \draw[thick] (0,0)node[below]{$A$}--
      (6,0)node[below]{$C$}--
      (6,3)node[above]{$B$}--cycle;
    \draw (6,0)++(-0.4,0)--++(0,0.4)--+(0.4,0);
    \node at (3,0)[below]{$12$};
    \node at (6,1.5)[right]{$5$};
    \node at (3.1,2)[right]{$x$};
  \end{tikzpicture}
  \end{flushright}

\item The base of right $\triangle ABC$ is 10.0 and its height is 17.3. Find the length of its hypotenuse $AB$, to the \emph{nearest tenth}.
  \begin{flushright}
  \begin{tikzpicture}[scale=0.7]
    \draw[thick] (0,0)node[below]{$A$}--
      (5,0)node[below]{$C$}--
      (5,8)node[above]{$B$}--cycle;
    \draw (5,0)++(-0.6,0)--++(0,0.6)--+(0.6,0);
    \node at (2.5,0)[below]{$10.0$};
    \node at (5,4)[right]{$17.3$};
  \end{tikzpicture}
  \end{flushright}

\subsubsection*{Find $x$ to the \emph{nearest tenth}.}

\item $\displaystyle \tan 75^\circ = \frac{x}{15}$ \vspace{3cm}
\item $\displaystyle \tan 26^\circ = \frac{4}{x}$ \vspace{4cm}
\item $\displaystyle x = \tan^{-1} (\frac{2}{3.5})$ \vspace{3cm}
\item $\displaystyle \tan x^\circ = \frac{17}{9}$ \vspace{3cm}

\newpage
\item Graph and label $\triangle ABC$ with $A(0,0)$, $B(3,6)$, and $C(3,0)$. Calculate each value:
  \begin{enumerate}[itemsep=1.25cm]  
  \begin{multicols}{2}
        \item $AC=$
        \item $BC=$
        \item Express first as a radical, then approximate with a decimal rounded to two decimal places.\\[0.25cm]
        $AB=$\vspace{2cm}
  \begin{center}
    \begin{tikzpicture}%[scale=.635]
      \draw [help lines] (0,0) grid (7,8);
      \draw [thick, ->] (0,0) -- (7.4,0) node [below right] {$x$};
      \draw [thick, ->] (0,0)--(0,8.4) node [left] {$y$};
    \end{tikzpicture}
  \end{center}
  \end{multicols}%\vspace{0.5cm}
    \item Use a protractor to measure $m\angle BAC= \theta$ in degrees.
    \item The tangent of an angle is the ratio of the side lengths \emph{opposite} over \emph{adjacent} to the angle. Write down the value as a fraction.\\[0.5cm]
      $\tan  \theta=$
    \item Find $m\angle BAC=\theta$ in degrees with a calculator's inverse tangent function.\\ $\displaystyle \theta = \tan^{-1}(\frac{opp}{adj})$
    \item Convert $ \theta$ to radians. ($180^\circ=\pi \, \rm{radians}$)
  \end{enumerate}

\newpage
\subsubsection*{Mastery topic: Calculator use}
  \item Express the result to the nearest thousandth. \vspace{.5cm}
    \begin{multicols}{2}
      \begin{enumerate}
        \item $\tan 22^\circ = $ \vspace{1cm}
        \item $\tan 81^\circ =$
        \item $\tan 15^\circ = $ \vspace{1cm}
        \item $\tan 65^\circ =$
      \end{enumerate}
    \end{multicols} \vspace{1cm}

    \item Round each value to the nearest degree. \vspace{.5cm}
    \begin{multicols}{2}
      \begin{enumerate}
        \item $\tan^{-1} (2) = $ \vspace{1cm}
        \item $\tan^{-1} (0.5) =$
        \item $\tan^{-1} (1) = $ \vspace{1cm}
        \item $\displaystyle \tan^{-1} (\frac{1}{\sqrt{3}}) =$
      \end{enumerate}
    \end{multicols} \vspace{1cm}

\subsubsection*{Mastery topic: Modeling situations with right triangles}
\item A tree casts a shadow 12 feet long. The angle of elevation from the tip of the shadow to the top of the tree is $70^\circ$. To the nearest foot, how tall is the tree?
  \begin{flushright}
  \begin{tikzpicture}[scale=1]
    \draw[thick] (0,0)--
      (3,0)--
      (3,7)--cycle;
    \draw (3,0)++(-0.6,0)--++(0,0.6)--+(0.6,0);
    \node at (1.5,0)[below]{$12$ feet};
    \node at (3,3)[right]{$x$};
    \draw (0.5,0) arc (0:65:0.5)node[pos=0.7,right]{$70^\circ$};
  \end{tikzpicture}
  \end{flushright}

\end{enumerate}
\end{document}
