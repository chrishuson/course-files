\documentclass[12pt, twoside]{article}
\documentclass[12pt, twoside]{article}
\usepackage[letterpaper, margin=1in, headsep=0.2in]{geometry}
\setlength{\headheight}{0.6in}
%\usepackage[english]{babel}
\usepackage[utf8]{inputenc}
\usepackage{microtype}
\usepackage{amsmath}
\usepackage{amssymb}
%\usepackage{amsfonts}
\usepackage{siunitx} %units in math. eg 20\milli\meter
\usepackage{yhmath} % for arcs, overparenth command
\usepackage{tikz} %graphics
\usetikzlibrary{quotes, angles}
\usepackage{graphicx} %consider setting \graphicspath{{images/}}
\usepackage{parskip} %no paragraph indent
\usepackage{enumitem}
\usepackage{multicol}
\usepackage{venndiagram}

\usepackage{fancyhdr}
\pagestyle{fancy}
\fancyhf{}
\renewcommand{\headrulewidth}{0pt} % disable the underline of the header
\raggedbottom
\hfuzz=2mm %suppresses overfull box warnings

\usepackage{hyperref}

\fancyhead[LE]{\thepage}
\fancyhead[RO]{\thepage \\ Name: \hspace{4cm} \,\\}
\fancyhead[LO]{BECA / Dr. Huson / Geometry\\*  Unit 10: Trigonometry \\* 18 April 2023}

\begin{document}

\subsubsection*{10.2 Classwork: Tangent inverse \hfill CCSS.HSG.SRT.C.8}
\begin{enumerate}
\item Graph and label $\triangle ABC$ with $A(0,0)$, $B(3,6)$, and $C(3,0)$. Calculate each value:
  \begin{enumerate}[itemsep=1.25cm]  
  \begin{multicols}{2}
        \item $AC=$
        \item $BC=$
        \item Express first as a radical, then approximate with a decimal rounded to two decimal places.\\[0.25cm]
        $AB=$\vspace{2cm}
  \begin{center}
    \begin{tikzpicture}%[scale=.635]
      \draw [help lines] (0,0) grid (7,8);
      \draw [thick, ->] (0,0) -- (7.4,0) node [below right] {$x$};
      \draw [thick, ->] (0,0)--(0,8.4) node [left] {$y$};
    \end{tikzpicture}
  \end{center}
  \end{multicols}%\vspace{0.5cm}
    \item Use a protractor to measure $m\angle BAC= \theta$ in degrees.
    \item The tangent of an angle is the ratio of the side lengths \emph{opposite} over \emph{adjacent} to the angle. Write down the value as a fraction.\\[0.5cm]
      $\tan  \theta=$
    \item Find $m\angle BAC=\theta$ in degrees with a calculator's inverse tangent function.\\ $\displaystyle \theta = \tan^{-1}(\frac{opp}{adj})$
    \item Convert $ \theta$ to radians. ($180^\circ=\pi \, \rm{radians}$)
  \end{enumerate}

\newpage
\subsubsection*{Mastery topic: Calculator use}
  \item Express the result to the nearest thousandth. \vspace{.5cm}
    \begin{multicols}{2}
      \begin{enumerate}
        \item $\tan 22^\circ = $ \vspace{1cm}
        \item $\tan 81^\circ =$
        \item $\tan 15^\circ = $ \vspace{1cm}
        \item $\tan 65^\circ =$
      \end{enumerate}
    \end{multicols} \vspace{1cm}

    \item Round each value to the nearest degree. \vspace{.5cm}
    \begin{multicols}{2}
      \begin{enumerate}
        \item $\tan^{-1} (2) = $ \vspace{1cm}
        \item $\tan^{-1} (0.5) =$
        \item $\tan^{-1} (1) = $ \vspace{1cm}
        \item $\displaystyle \tan^{-1} (\frac{1}{\sqrt{3}}) =$
      \end{enumerate}
    \end{multicols} \vspace{1cm}

\subsubsection*{Mastery topic: Modeling. Do Not Solve}
  \item Given right $\triangle JKL$ with $\overline{JK} \perp \overline{KL}$, $JK=11$, $m\angle J=18^\circ$. (mark the diagram)\\[0.5cm]
    Let $x$ be the length of the side opposite $\angle J$, $x=KL$. Write an equation expressing $\tan \angle J$ as a ratio of \emph{opposite} over \emph{adjacent}.
      \begin{flushright}
          \begin{tikzpicture}[scale=0.7]
            \draw [thick](-1,0)--(7,0)--(7,3)--cycle;
            \draw [fill] (-1,0) circle [radius=0.05] node[below]{$J$};
            \draw [fill] (7,0) circle [radius=0.05] node[below]{$K$};
            \draw [fill] (7,3) circle [radius=0.05] node[above right]{$L$};
          \end{tikzpicture}
        \end{flushright}

\newpage
  \item Given right $\triangle ABC$ with $m\angle C =90^\circ$, $BC=5$, $m\angle A=38^\circ$. (mark the diagram)\\[0.5cm]
  Let $x$ be the length of the side adjacent to $\angle A$, $x=AC$. Write an equation expressing $\tan \angle A$ as a ratio of \emph{opposite} over \emph{adjacent}.
    \begin{flushright}
        \begin{tikzpicture}[scale=0.7]
          \draw [thick](-1,0)--(7,0)--(7,5)--cycle;
          \draw [fill] (-1,0) circle [radius=0.05] node[below]{$A$};
          \draw [fill] (7,0) circle [radius=0.05] node[below]{$C$};
          \draw [fill] (7,5) circle [radius=0.05] node[above right]{$B$};
        \end{tikzpicture}
      \end{flushright}

  \item Given right $\triangle ABC$ with $m\angle C =90^\circ$, $BC=11$, $AC=17$, and $m\angle A=x^\circ$. (mark the diagram)\\[0.5cm]
  Write an equation expressing $\tan x$ as a ratio of \emph{opposite} over \emph{adjacent}.
    \begin{flushright}
        \begin{tikzpicture}[scale=0.7]
          \draw [thick](-1,0)--(6,0)--(6,5)--cycle;
          \draw [fill] (-1,0) circle [radius=0.05] node[below]{$A$};
          \draw [fill] (6,0) circle [radius=0.05] node[below]{$C$};
          \draw [fill] (6,5) circle [radius=0.05] node[above right]{$B$};
        \end{tikzpicture}
      \end{flushright}
  
  \item Given right $\triangle JKL$ with $\overline{JK} \perp \overline{KL}$, $JK=20$, $m\angle J=11^\circ$. (mark the diagram)\\[0.5cm]
    Let $x$ be the length of the side opposite $\angle J$, $x=KL$. Write an equation expressing $\tan \angle J$ as a ratio of \emph{opposite} over \emph{adjacent}.
      \begin{flushright}
          \begin{tikzpicture}[scale=0.9]
            \draw [thick](-1,0)--(7,0)--(7,2)--cycle;
            \draw [fill] (-1,0) circle [radius=0.05] node[below]{$J$};
            \draw [fill] (7,0) circle [radius=0.05] node[below]{$K$};
            \draw [fill] (7,2) circle [radius=0.05] node[above right]{$L$};
          \end{tikzpicture}
        \end{flushright}
\newpage
\subsubsection*{Mastery topic: Algebraic solution\\[0.5cm]
Use your calculator and solve each equation for $x$, rounding to the nearest tenth.}
\item $\displaystyle \tan 75^\circ = \frac{x}{15}$ \vspace{3cm}
\item $\displaystyle \tan 26^\circ = \frac{4}{x}$ \vspace{4cm}
\item $\displaystyle x = \tan^{-1} (\frac{2}{3.5})$ \vspace{3cm}
\item $\displaystyle \tan x^\circ = \frac{17}{9}$ \vspace{3cm}

\end{enumerate}
\end{document}
