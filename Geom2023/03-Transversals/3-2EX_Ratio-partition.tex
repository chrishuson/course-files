\documentclass[12pt, twoside]{article}
\usepackage[letterpaper, margin=1in, headsep=0.2in]{geometry}
\setlength{\headheight}{0.6in}
%\usepackage[english]{babel}
\usepackage[utf8]{inputenc}
\usepackage{microtype}
\usepackage{amsmath}
\usepackage{amssymb}
%\usepackage{amsfonts}
\usepackage{siunitx} %units in math. eg 20\milli\meter
\usepackage{yhmath} % for arcs, overparenth command
\usepackage{tikz} %graphics
\usetikzlibrary{quotes, angles}
\usepackage{graphicx} %consider setting \graphicspath{{images/}}
\usepackage{parskip} %no paragraph indent
\usepackage{enumitem}
\usepackage{multicol}
\usepackage{venndiagram}

\usepackage{fancyhdr}
\pagestyle{fancy}
\fancyhf{}
\renewcommand{\headrulewidth}{0pt} % disable the underline of the header
\raggedbottom
\hfuzz=2mm %suppresses overfull box warnings

\usepackage{hyperref}

\fancyhead[LE]{\thepage}
\fancyhead[RO]{\thepage \\ Name: \hspace{4cm} \,\\}
\fancyhead[LO]{BECA / Dr. Huson / Geometry\\*  Unit 3: Parallel lines and transversals\\* 18 October 2022}

\begin{document}

\subsubsection*{3.2 Extension: Ratio partition of a line segment}
%The distance formula: $\displaystyle d=\sqrt{(x_2-x_1)^2+(y_2-y_1)^2}$
\begin{enumerate}
\item Do Now: Dr. Huson's commute is from 80th Street to 164th Street. 
    \begin{enumerate}
        \item On what block is he half way? Mark it and label it with the street number.
        \item On the way to work, mark and label the block when he is three-quarters of the way to BECA.
    \end{enumerate}
    \begin{tikzpicture}[scale=1.25]
    \draw [<->] (-4.5,0)--(6.5,0);
    \foreach \x in {-3,-1,...,5} %2 leading for diff!=1
        \draw[shift={(\x,0)},color=black] (0pt,-3pt) -- (0pt,3pt);% node[below=5pt]  {$\x$};
        \draw [fill] (-3,0) circle [radius=0.05] node[above] {home};
        \draw [fill] (5,0) circle [radius=0.05] node[above] {BECA};
        \node at (-3,0) [below]{80};
        \node at (5,0) [below]{164};
    \end{tikzpicture}
    \vspace{3cm}

\item Find each pair of numbers with the given sum.
\begin{enumerate}
    \item Example: Two numbers with a ratio of $3:1$ that sum to 20 are $15:5$.
    \item $2:1$, sum 9 \vspace{1cm}
    \item $1:1$, sum 100 \vspace{1cm}
    \item $2:3$, sum 20 \vspace{1cm}
\end{enumerate}

\item Divide (partition) $\overline{AB}$, $A=-3$ and $B=6$, into three equal parts. Mark and label the dividing points $P$ and $Q$. \\
    \begin{tikzpicture}
    \draw [<->] (-4.5,0)--(7.5,0);
    \foreach \x in {-4,...,7} %2 leading for diff!=1
        \draw[shift={(\x,0)},color=black] (0pt,-3pt) -- (0pt,3pt) node[below=5pt]  {$\x$};
        \draw [fill] (-3,0) circle [radius=0.05] node[above] {$A$};
        \draw [fill] (6,0) circle [radius=0.05] node[above] {$B$};
    \end{tikzpicture}
    \vspace{0.5cm}

\item Partition $\overline{MN}$, $M=-3$ and $N=5$, in the ratio $3:1$ with point $P$. \\
\begin{tikzpicture}
\draw [<->] (-7.5,0)--(7.5,0);
\foreach \x in {-7,...,7} %2 leading for diff!=1
    \draw[shift={(\x,0)},color=black] (0pt,-3pt) -- (0pt,3pt) node[below=5pt]  {$\x$};
    \draw [fill] (-3,0) circle [radius=0.05] node[above] {$M$};
    \draw [fill] (5,0) circle [radius=0.05] node[above] {$N$};
\end{tikzpicture}    

\newpage
    \item In the diagram below $\angle BOC = 8x$ and $\angle DOE = 3x+13$. \hfill CCSSM.8.G.B.5 \\Find $m\angle AOB$.
    \vspace{0.25cm}
    \begin{flushright}
    \begin{tikzpicture}[scale=1.3, rotate=0]
    \draw [<->, thick] (-40:3)--(0,0)--(140:3);
    \draw [<->, thick] (-3,0)--(3,0);
    \draw [->, thick] (0,0)--(0,3);
    \draw (0,0)++(0.3,0)--++(0,0.3)--+(-0.3,0);
    %\draw [fill] (-1,2.5) circle [radius=0.05] node[left ]{$B$};
    \draw [fill] (140:2) circle [radius=0.05] node[below left]{$B$};
    \draw [fill] (-2,0) circle [radius=0.05] node[below]{$A$}; 
    \draw [fill] (0,0) circle [radius=0.05] node[below left]{$O$};
    \draw [fill] (0,2) circle [radius=0.05] node[left]{$C$};
    \draw [fill] (2,0) circle [radius=0.05] node[below]{$D$};
    \draw [fill] (-40:2) circle [radius=0.05] node[below]{$E$};
    \end{tikzpicture}
    \end{flushright}

\newpage
\item The point $B$ is two thirds of the way from $A=-1$ to $C=5$. Find the coordinate of $B$. Mark and label $B$ on the graph of $\overleftrightarrow{AC}$. \\[1cm]
  \begin{tikzpicture}
    \draw [<->] (-4.5,0)--(7.5,0);
    \foreach \x in {-4,...,7} %2 leading for diff!=1
      \draw[shift={(\x,0)},color=black] (0pt,-3pt) -- (0pt,3pt) node[below=5pt]  {$\x$};
      \draw [fill] (-1,0) circle [radius=0.05] node[above] {$A$};
      \draw [fill] (5,0) circle [radius=0.05] node[above] {$C$};
  \end{tikzpicture}
  \vspace{1cm}

\item Point $P$ partitions $\overline{MN}$, $M=-4$ and $N=6$, in the ratio $3:2$. Find the value of point $P$. Mark and label $P$ on the graph. \\[1cm]
\begin{tikzpicture}
\draw [<->] (-7.5,0)--(7.5,0);
\foreach \x in {-7,...,7} %2 leading for diff!=1
    \draw[shift={(\x,0)},color=black] (0pt,-3pt) -- (0pt,3pt) node[below=5pt]  {$\x$};
    \draw [fill] (-4,0) circle [radius=0.05] node[above] {$M$};
    \draw [fill] (6,0) circle [radius=0.05] node[above] {$N$};
\end{tikzpicture}

\item Point $P$ partitions $\overline{MN}$, $M=-5$ and $N=7$, in the ratio $3:1$. Find the value of point $P$. Mark and label $P$ on the graph. \\[1cm]
\begin{tikzpicture}
  \draw [<->] (-7.5,0)--(7.5,0);
  \foreach \x in {-7,...,7} %2 leading for diff!=1
    \draw[shift={(\x,0)},color=black] (0pt,-3pt) -- (0pt,3pt) node[below=5pt]  {$\x$};
    \draw [fill] (-5,0) circle [radius=0.05] node[above] {$M$};
    \draw [fill] (7,0) circle [radius=0.05] node[above] {$N$};
\end{tikzpicture}

\item Point $P$ partitions $\overline{MN}$, $M=-6$ and $N=4$, in the ratio $1:4$. Find the value of point $P$. Mark and label $P$ on the graph. \\[1cm]
\begin{tikzpicture}
  \draw [<->] (-7.5,0)--(7.5,0);
  \foreach \x in {-7,...,7} %2 leading for diff!=1
    \draw[shift={(\x,0)},color=black] (0pt,-3pt) -- (0pt,3pt) node[below=5pt]  {$\x$};
    \draw [fill] (-6,0) circle [radius=0.05] node[above] {$M$};
    \draw [fill] (4,0) circle [radius=0.05] node[above] {$N$};
\end{tikzpicture} \vspace{2cm}

\item In the line segment $\overline{ABC}$, $\overline{AB}$ is twice as long as $\overline{BC}$. $AB=12x-6$ and $AC=15x+9$. Find $BC$.
\vspace{5cm}

\end{enumerate}
\end{document}