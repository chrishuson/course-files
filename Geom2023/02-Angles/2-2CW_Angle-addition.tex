\documentclass[12pt, twoside]{article}
\usepackage[letterpaper, margin=1in, headsep=0.2in]{geometry}
\setlength{\headheight}{0.6in}
%\usepackage[english]{babel}
\usepackage[utf8]{inputenc}
\usepackage{microtype}
\usepackage{amsmath}
\usepackage{amssymb}
%\usepackage{amsfonts}
\usepackage{siunitx} %units in math. eg 20\milli\meter
\usepackage{yhmath} % for arcs, overparenth command
\usepackage{tikz} %graphics
\usetikzlibrary{quotes, angles}
\usepackage{graphicx} %consider setting \graphicspath{{images/}}
\usepackage{parskip} %no paragraph indent
\usepackage{enumitem}
\usepackage{multicol}
\usepackage{venndiagram}

\usepackage{fancyhdr}
\pagestyle{fancy}
\fancyhf{}
\renewcommand{\headrulewidth}{0pt} % disable the underline of the header
\raggedbottom
\hfuzz=2mm %suppresses overfull box warnings

\usepackage{hyperref}

\fancyhead[LE]{\thepage}
\fancyhead[RO]{\thepage \\ Name: \hspace{4cm} \,\\}
\fancyhead[LO]{BECA / Dr. Huson / Geometry\\*  Unit 2: Angles\\* 29 Sept 2022}

\begin{document}

\subsubsection*{2.2 Classwork: Angle addition}
\begin{enumerate}
\item Write down the name of the \emph{three} angles shown in the diagram below and their angle measures, using your protractor. \vspace{1cm}
\begin{center}
\begin{tikzpicture}[scale=2]
  \draw [->, thick] (0,0)--(4,3);
  \draw [->, thick] (0,0)--(5,-.5);
  \draw [->, thick] (0,0)--(-1.2,3);
  \draw [fill] (-1,2.5) circle [radius=0.03] node[left ]{$B$};
  \draw [fill] (2.66666,2) circle [radius=0.03] node[above left ]{$C$};
  \draw [fill] (0,0) circle [radius=0.03] node[left]{$A$};
  \draw [fill] (4,-0.4) circle [radius=0.03] node[above]{$D$};
\end{tikzpicture}
\end{center}
\begin{enumerate}
  \item  \rule{4cm}{0.15mm} \bigskip
  \item  \rule{4cm}{0.15mm} \bigskip
  \item  \rule{4cm}{0.15mm} \bigskip
  \item What do you notice about the angle measures?
\end{enumerate}\vspace{1cm}

\item $m\angle ABD=30^\circ$, $m\angle DBC=45^\circ$. Find $m\angle ABC$.\vspace{0.5cm}
  \begin{flushright}
    \begin{tikzpicture}
      \draw [<->, thick] (45:5)--(0,0)--(6,0);
      \draw [->, thick] (0,0)--(75:4);
      \draw [fill] (75:3) circle [radius=0.05] node[left]{$A$};
      \draw [fill] (45:3) circle [radius=0.05] node[below right]{$D$};
      \draw [fill] (0,0) circle [radius=0.05] node[below]{$B$};
      \draw [fill] (4,0) circle [radius=0.05] node[below]{$C$};
      \node at (1.5,0.5){$45^\circ$};
      \node at (1,1.7){$30^\circ$};
    \end{tikzpicture}
    \end{flushright}

\item Two lines intersect with $m\angle 1=80^\circ$. Find the measures of $\angle 2$, $\angle 3$, and $\angle 4$.
  \begin{flushleft}
  \begin{tikzpicture}[scale=0.8]
    \draw [<->, thick] (0,-1.5)--(10,1.5);
    \draw [<->, thick] (2,3.5)--(7,-3.5);
    \node at (2,.4){$m\angle 1=80^\circ$};
    \node at (6,-.6){3};
    \node at (5,1){2};
    \node at (4,-1){4};
    %\draw [fill] (0,0) circle [radius=0.05] node[below]{$P$};
    %\draw [fill] (6,0) circle [radius=0.05] node[below]{$R$};
    %\draw [fill] (3,0) circle [radius=0.05] node[below]{$Q$};
  \end{tikzpicture}
  \end{flushleft}

\item $\angle POQ$ and $\angle QOR$ are complementary angles. Given $m\angle POQ=51^\circ$, find $m\angle QOR$. \vspace{0.25cm}
  \begin{center}
  \begin{tikzpicture}[scale=1.2, rotate=0]
    \draw [->, thick] (0,0)--(129:4);
    \draw [<->, thick] (-5,0)--(5,0);
    \draw [->, thick] (0,0)--(0,4);
    \draw (0,0)++(0.3,0)--++(0,0.3)--+(-0.3,0);
    \draw [fill] (129:3) circle [radius=0.05] node[below left]{$Q$};
    \draw [fill] (-4,0) circle [radius=0.05] node[below]{$P$}; 
    \draw [fill] (0,0) circle [radius=0.05] node[below right]{$O$};
    \draw [fill] (0,3) circle [radius=0.05] node[right]{$R$};
    \node at (-1,0.4){$51^\circ$};
    \node at (-0.4,1.2){$x^\circ$};
  \end{tikzpicture}
  \end{center}

\newpage
\item Given $m\angle ADB=110^\circ$, $m\angle ADC = 75^\circ$, and $m\angle BDC = 3x+5$. Find $x$.
  \begin{enumerate}
    \begin{multicols}{2}
    \item Label the diagram.
    \item Write an equation.
    \item Solve for $x$. \vspace{3cm}
    \begin{flushright}
    \begin{tikzpicture}[scale=1.2]
      \draw [->, thick] (0,0)--(35:5);
      \draw [->, thick] (0,0)--(4,0);
      \draw [->, thick] (0,0)--(110:4);
      \draw [fill] (110:3) circle [radius=0.05] node[left ]{$A$};
      \draw [fill] (35:4) circle [radius=0.05] node[above left ]{$C$};
      \draw [fill] (0,0) circle [radius=0.05] node[left]{$D$};
      \draw [fill] (3,0) circle [radius=0.05] node[below]{$B$};
    \end{tikzpicture}
    \end{flushright}
  \end{multicols}
  \item Check your answer
  \end{enumerate} \vspace{2cm}


\item Apply the Angle Addition postulate. Write and equation to support your work.
\begin{multicols}{2}
  Given $m\angle ABD = 80^\circ$ and \\[0.25cm] $m\angle DBC = 35^\circ$. \\[0.5cm]
  Find $m \angle ABC$. \\
  \begin{tikzpicture}[scale=1.4]
    \draw [<->, thick]
      (3,0) coordinate (a) node[below left] {$C$}
      -- (0,0) coordinate (b) node[below left] {$B$}
      -- (2,1.5) coordinate (c) node[below right] {$D$}
      pic["$35^\circ$", <->, draw=black, angle eccentricity=1.5, angle radius=1cm]
      {angle=a--b--c};
      \draw [<-, thick]
      (-1,2) coordinate (d) node[right] {$A$}
      -- (0,0) coordinate (e)
      pic["$80^\circ$", <->, draw=black, angle eccentricity=1.25, angle radius=1cm]
      {angle=c--e--d};
  \end{tikzpicture}
\end{multicols}

\item Given the angle measures and situation shown, write an equation and solve for $x$.
  \begin{multicols}{2}
    $m\angle ABD = 2x$ \\[0.25cm]
    $m\angle DBC = 50^\circ$ \\[0.25cm]
    $m \angle ABC = 110^\circ$ \\
    \begin{tikzpicture}[scale=2]
      \draw [<->, thick]
        (-20:2.5) coordinate (a) node[below left] {$C$}
        -- (0,0) coordinate (b) node[below left] {$B$}
        -- (30:3) coordinate (c) node[below right] {$D$}
        pic["$50^\circ$", <->, draw=black, angle eccentricity=1.5, angle radius=1cm]
        {angle=a--b--c};
        \draw [<-, thick]
        (80:2) coordinate (d) node[right] {$A$}
        -- (0,0) coordinate (e)
        pic["$2x$", <->, draw=black, angle eccentricity=1.5, angle radius=1cm]
        {angle=c--e--d};
    \end{tikzpicture}
  \end{multicols}

\item The ray $\overrightarrow{BD}$ makes a $90^\circ$ angle with the line $\overleftrightarrow{ABC}$, and $m\angle DBE = x^\circ$, $m\angle EBC = 25^\circ$. \\[0.5cm] 
Find $x$, writing and equation to support your work.
  \begin{flushright}
    \begin{tikzpicture}[scale=1.3]
      \draw [<->, thick]
        (0:5) coordinate (a) node[below left] {$C$}
        -- (0,0) coordinate (b) node[below] {$B$}
        -- (25:5) coordinate (c) node[below right] {$E$}
        pic["$25$", <->, draw=black, angle eccentricity=2, angle radius=1cm]
        {angle=a--b--c};
        \draw [<-, thick]
        (90:3) coordinate (d) node[right] {$D$}
        -- (0,0) coordinate (e)
        pic["$x^\circ$", <->, draw=black, angle eccentricity=1.5, angle radius=1cm]
        {angle=c--e--d};
        \draw [->, thick] (0,0)--(-180:2) node[below right]{$A$};
        \draw (0,0)++(-0.3,0)--++(0,0.3)--+(0.3,0);
    \end{tikzpicture}
  \end{flushright}

\item Two supplementary angles have measures $m\angle ABD = 5x$ and $m\angle DBC = 125^\circ$. \\[0.5cm] 
Write an equation, then find $x$. \vspace{0.5cm}
  \begin{flushright}
    \begin{tikzpicture}[scale=1]
      \draw [<->, thick]
        (0:5) coordinate (a) node[below left] {$C$}
        -- (0,0) coordinate (b) node[below] {$B$}
        -- (125:3) coordinate (c) node[above right] {$D$}
        pic["$125^\circ$", <->, draw=black, angle eccentricity=1.5, angle radius=1cm]
        {angle=a--b--c};
        \draw [<-, thick]
        (180:4) coordinate (d) node[below] {$A$}
        -- (0,0) coordinate (e)
        pic["$5x$", <->, draw=black, angle eccentricity=1.5, angle radius=1cm]
        {angle=c--e--d};
        %\draw [->, thick] (0,0)--(-180:2) node[below right]{$A$};
        %\draw (0,0)++(-0.3,0)--++(0,0.3)--+(0.3,0);
    \end{tikzpicture}
  \end{flushright}
      
\item Given the perpendicular situation shown, $\overrightarrow{BD} \perp \overleftrightarrow{ABC}$ and angle measures given. \\[0.5cm] 
Find $x$. \vspace{0.5cm}
  \begin{multicols}{2}
    $m\angle DBE = 40^\circ$ \\[0.25cm]
    $m\angle EBC = 3x + 5^\circ$ \\[0.25cm]
    \begin{tikzpicture}[scale=1]
      \draw [<->, thick]
        (0:5) coordinate (a) node[below left] {$C$}
        -- (0,0) coordinate (b) node[below] {$B$}
        -- (50:5) coordinate (c) node[below right] {$E$}
        pic["$3x + 5$", <->, draw=black, angle eccentricity=2, angle radius=1cm]
        {angle=a--b--c};
        \draw [<-, thick]
        (90:4) coordinate (d) node[right] {$D$}
        -- (0,0) coordinate (e)
        pic["$40^\circ$", <->, draw=black, angle eccentricity=1.5, angle radius=1cm]
        {angle=c--e--d};
        \draw [->, thick] (0,0)--(-180:2) node[below right]{$A$};
        \draw (0,0)++(-0.3,0)--++(0,0.3)--+(0.3,0);
    \end{tikzpicture}
  \end{multicols}

\item A linear pair have measures $m\angle ABD = 7x + 16^\circ$ and $m\angle DBC = 5x + 20^\circ$. \\[0.5cm] 
Find $m\angle ABD$. \vspace{0.5cm}
  \begin{flushright}
    \begin{tikzpicture}[scale=1]
      \draw [<->, thick]
        (0:5) coordinate (a) node[below left] {$C$}
        -- (0,0) coordinate (b) node[below] {$B$}
        -- (75:3) coordinate (c) node[below right] {$D$}
        pic["$5x + 20$", <->, draw=black, angle eccentricity=2, angle radius=1cm]
        {angle=a--b--c};
        \draw [<-, thick]
        (180:4) coordinate (d) node[below] {$A$}
        -- (0,0) coordinate (e)
        pic["$7x + 16$", <->, draw=black, angle eccentricity=1.5, angle radius=1cm]
        {angle=c--e--d};
        %\draw [->, thick] (0,0)--(-180:2) node[below right]{$A$};
        %\draw (0,0)++(-0.3,0)--++(0,0.3)--+(0.3,0);
    \end{tikzpicture}
  \end{flushright}

\item Given $\overline{DEFG}$, $DE=3 \frac{1}{4}$, $EF=6 \frac{1}{4}$, and $FG= 1 \frac{3}{4}$. (diagram not to scale)\\ [0.25cm]
  Find ${DG}$, expressed as a fraction, not a decimal.
  \begin{flushright}
      \begin{tikzpicture}
        \draw [-, thick] (0,0)--(9,0);
        \draw [fill] (0,0) circle [radius=0.05] node[below]{$D$};
        \draw [fill] (3,0) circle [radius=0.05] node[below]{$E$};
        \draw [fill] (7,0) circle [radius=0.05] node[below]{$F$};
        \draw [fill] (9,0) circle [radius=0.05] node[below]{$G$};
      \end{tikzpicture}
    \end{flushright}

\item Given $P(-2.4)$ and $Q(1.8)$, as shown on the number line. \\[0.25cm]
  Find the length of the line segment $\overline{PQ}$. State an equation for full credit.
  \begin{center}
    \begin{tikzpicture}
      \draw [<->] (-4.5,0)--(4.5,0);
      \draw [-, thick] (-2.4,0)--(1.8,0);
      \foreach \x in {-4,...,4} %2 leading for diff!=1
        \draw[shift={(\x,0)},color=black] (0pt,-3pt) -- (0pt,3pt) node[below=5pt]  {$\x$};
        \draw [fill] (-2.4,0) circle [radius=0.05] node[above] {$P$};
        \draw [fill] (1.8,0) circle [radius=0.05] node[above] {$Q$};
    \end{tikzpicture}
  \end{center}


\end{enumerate}
\end{document}