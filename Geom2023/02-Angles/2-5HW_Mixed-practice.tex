\documentclass[12pt, twoside]{article}
\usepackage[letterpaper, margin=1in, headsep=0.2in]{geometry}
\setlength{\headheight}{0.6in}
%\usepackage[english]{babel}
\usepackage[utf8]{inputenc}
\usepackage{microtype}
\usepackage{amsmath}
\usepackage{amssymb}
%\usepackage{amsfonts}
\usepackage{siunitx} %units in math. eg 20\milli\meter
\usepackage{yhmath} % for arcs, overparenth command
\usepackage{tikz} %graphics
\usetikzlibrary{quotes, angles}
\usepackage{graphicx} %consider setting \graphicspath{{images/}}
\usepackage{parskip} %no paragraph indent
\usepackage{enumitem}
\usepackage{multicol}
\usepackage{venndiagram}

\usepackage{fancyhdr}
\pagestyle{fancy}
\fancyhf{}
\renewcommand{\headrulewidth}{0pt} % disable the underline of the header
\raggedbottom
\hfuzz=2mm %suppresses overfull box warnings

\usepackage{hyperref}

\fancyhead[LE]{\thepage}
\fancyhead[RO]{\thepage \\ Name: \hspace{4cm} \,\\}
\fancyhead[LO]{BECA / Dr. Huson / Geometry\\*  Unit 2: Angles\\* 4 October 2022}

\begin{document}

\subsubsection*{2.5 Homework: Mixed practice}
\begin{enumerate}
\item The ray $\overrightarrow{BD}$ bisects $\angle ABC$. m$\angle ABD=3x+1$, m$\angle DBC=5x-25$. Find m$\angle ABC$.\vspace{0.5cm}
  \begin{flushright}
    \begin{tikzpicture}
      \draw[<->, thick] (100:4)--(0,0)--(20:5);
      \draw[->, thick] (0,0)--(60:4);
      \draw[fill] (100:3) circle [radius=0.05] node[left]{$A$};
      \draw[fill] (60:3) circle [radius=0.05] node[below right]{$D$};
      \draw[fill] (0,0) circle [radius=0.05] node[below]{$B$};
      \draw[fill] (20:4) circle [radius=0.05] node[below]{$C$};
      \node at (1.5,1){$5x-25$};
      \node at (0.4,2){$3x+1$};
    \end{tikzpicture}
    \end{flushright} \vspace{1cm}

\item Two lines intersect with vertical angles m$\angle 1=2x+20$ and m$\angle 3=3x-5$. Find m$\angle 2$.
  \begin{flushleft}
  \begin{tikzpicture}[scale=0.8, rotate=30]
    \draw[<->, thick] (0,-1.5)--(10,1.5);
    \draw[<->, thick] (2,3.5)--(7,-3.5);
    \node at (1.5,1.3){m$\angle 1=2x+20$};
    \node at (7,-1){m$\angle 3=3x-5$};
    \node at (5,1){2};
    \node at (4,-1){4};
  \end{tikzpicture}
  \end{flushleft}

\item Write the appropriate name for the type of angle depending on its measure in degrees. (acute, right, obtuse, or straight)
    \begin{enumerate}
      \item m$\angle = 90$ : \rule{4cm}{0.15mm} \bigskip
      \item $90 < \text{m}\angle < 180$ : \rule{4cm}{0.15mm} \bigskip
      \item $0< \text{m}\angle < 90$ : \rule{4cm}{0.15mm} \bigskip
      \item m$\angle = 180$ : \rule{4cm}{0.15mm} \bigskip
    \end{enumerate}

\item Identify the true statement(s) given $\angle AOB = 2x$ and $\angle BOC = 5x+20$.
 \begin{multicols}{2}
    \begin{enumerate}
      \item $\angle AOB \cong \angle BOC$\\
      $2x = (5x+20)$
      \item $\angle AOB$, $\angle BOC$ are complementary\\
      $2x + (5x+20)=90^\circ$
      \item $\angle AOB$ and $\angle BOC$ are a linear pair\\
      $2x + (5x+20)=180^\circ$
  \end{enumerate}
  \begin{center}
    \begin{tikzpicture}[scale=0.7, rotate=20]
      \draw[<->, thick] (-45:5)--(0,0)--(135:5);
      \draw[<->, thick] (-5,0)--(5,0);
      \draw[->, thick] (0,0)--(0,4);
      \draw (0,0)++(0.3,0)--++(0,0.3)--+(-0.3,0);
      \draw[fill] (135:3) circle [radius=0.05] node[below left]{$B$};
      \draw[fill] (-4,0) circle [radius=0.05] node[below]{$A$}; 
      \draw[fill] (0,0) circle [radius=0.05] node[below]{$O$};
      \draw[fill] (0,3) circle [radius=0.05] node[left]{$C$};
      \draw[fill] (4,0) circle [radius=0.05] node[below]{$D$};
      \draw[fill] (-45:3) circle [radius=0.05] node[above]{$E$};
      \end{tikzpicture}
  \end{center}
\end{multicols}
Copy the correct equation and solve for $x$. Check your answer. \vspace{2cm}

\item The ray $\overrightarrow{KM}$ bisects $\angle JKL$. Given m$\angle JKM = 4x-20$ and \\m$\angle MKL = 3x+4$. Identify the true statement(s).
 \begin{multicols}{2}
    \begin{enumerate}
      \item $\angle JKM$ and $\angle MKL$ are a linear pair\\
      $(4x-20) + (3x+4)=180^\circ$
      \item $\angle JKM$, $\angle MKL$ are adjacent and\\
      $4x-20 =90^\circ$
      \item $\angle JKM \cong \angle MKL$\\
      $4x-20 = 3x+4$
  \end{enumerate}
  \begin{center}
    \begin{tikzpicture}[scale=0.6, rotate=-10]
      \draw[<->, thick] (160:5)node[below left]{$J$} 
      --(0,0)node[below left]{$K$}
      --(10:5)node[above right]{$L$};
      \draw[->, thick] (0,0)--(90:5)node[below left]{$M$};
    \end{tikzpicture}
  \end{center}
\end{multicols}
Copy the correct equation and find m$\angle JKL$. Check your answer. \vspace{2cm}

\item As shown below, two lines intersect making four angles: $\angle 1$, $\angle 2$, $\angle 3$, and $\angle 4$. Given that m$\angle 1= x+32$ and m$\angle 3=2x-8$, find m$\angle 1$.
  \begin{flushright}
    \begin{tikzpicture}[scale=1, rotate=0]
      \draw[<->, thick] (1,-1)--(8,1);
      \draw[<->, thick] (3,2)--(6,-2);
      \node at (2,.3){m$\angle 1= x+32$};
      \node at (6.5,-.3){m$\angle 3=2x-8$};
      \node at (5,1){2};
      \node at (4,-1){4};
    \end{tikzpicture}
    \end{flushright}

\item An angle bisector is shown below, with $\overrightarrow{PR}$ bisecting $\angle QPS$. Given m$\angle QPR = 6x-12$ and m$\angle QPS = 10x+4$, find m$\angle QPS$.
  \begin{flushright}
  \begin{tikzpicture}[scale=0.6, rotate=30]
    \draw[<->, thick] (230:5)node[left]{$Q$} 
    --(0,0)node[above right]{$P$}
    --(110:6)node[above right]{$S$}--(110:7);
    \draw[->, thick] (0,0)--(170:7)node[below right]{$R$};
  \end{tikzpicture}
  \end{flushright}
  
  \item As shown below, two lines intersect making four angles: $\angle 1$, $\angle 2$, $\angle 3$, and $\angle 4$.
  \begin{multicols}{2}  
    \begin{enumerate}
      \item Name a pair of vertical angles. \vspace{1.5cm}
      \item Given m$\angle 4 = 70^\circ$, write down m$\angle 2$. \vspace{1.5cm}
      \item Find m$\angle 1$. \vspace{2cm}
    \end{enumerate}
    \begin{tikzpicture}[scale=0.7, rotate=-20]
    \draw[<->, thick] (0,-1.5)--(10,1.5);
    \draw[<->, thick] (2,3)--(7,-3);
    \node at (3,.4){4};
    \node at (6,-.6){2};
    \node at (5,1){1};
    \node at (4,-1){3};
  \end{tikzpicture}
  \end{multicols}
  
\newpage
\item Demonstrate your ability to classify angles and use standard terminology.
\begin{enumerate}
  \item Which of the following are true with respect to the angle, m$\angle PQR$?
  \begin{multicols}{2}
    True \hspace{0.25cm} False \hspace{0.25cm} It is a right angle \\[0.5cm]
    True \hspace{0.25cm} False \hspace{0.25cm} It's measure is $180^\circ$\\[0.5cm]
    True \hspace{0.25cm} False \hspace{0.25cm} $\overrightarrow{QP}$ is perpendicular to $ \overrightarrow{QR}$ \\[0.5cm]
    \columnbreak
    \begin{tikzpicture}[scale=0.7, rotate=-20]
      \draw[<->, thick] (4,0)--(0,0)--(0,3);
      \draw (0,0)++(0.3,0)--++(0,0.3)--+(-0.3,0);
      \draw[fill] (0,0) circle [radius=0.05] node[below]{$Q$};
      \draw[fill] (0,2) circle [radius=0.05] node[right]{$P$};
      \draw[fill] (3,0) circle [radius=0.05] node[above]{$R$};
    \end{tikzpicture}
  \end{multicols}
  \item What is the sum of the degree measures of this linear pair, $\angle ABD$ and $\angle CBD$?
  \begin{center}
    \begin{tikzpicture}[scale=.8, rotate=0]
      \draw  [<->, thick] (-3,0)--(3,0);
      \draw[->, thick] (0,0)--(2, 1) node[right]{$D$};
      \draw[fill] (-2,0) circle [radius=0.05] node[below]{$A$};
      \draw[fill] (0,0) circle [radius=0.05] node[below]{$B$};
      \draw[fill] (2,0) circle [radius=0.05] node[below]{$C$};
    \end{tikzpicture}
  \end{center}
  \item The given angle $\angle UVW$ is which of the following: acute, obtuse, or right?
  \begin{center}
    \begin{tikzpicture}[scale=.8]
      \draw  [<->, thick] (-3,0)--(0,0)--(35:3);
      \draw[fill] (-2,0) circle [radius=0.05] node[below]{$U$};
      \draw[fill] (0,0) circle [radius=0.05] node[below]{$V$};
      \draw[fill] (35:2) circle [radius=0.05] node[above left]{$W$};
    \end{tikzpicture}
  \end{center}
  \end{enumerate}
  
\item Apply the Angle Addition postulate. Write and equation to support your work.
  \begin{multicols}{2}
    Given m$\angle ABD = 75^\circ$, m$\angle ABC = 90^\circ$. \\[0.5cm]
    Find $m \angle CBD$. \\
    \begin{tikzpicture}[scale=1.4]
      \draw[<->, thick]
        (0:3) coordinate (a) node[below left] {$C$}
        -- (0,0) coordinate (b) node[below left] {$B$}
        -- (20:3) coordinate (c) node[below right] {$D$}
        pic["$?$", <->, draw=black, angle eccentricity=1.5, angle radius=1cm]
        {angle=a--b--c};
        \draw[<-, thick]
        (90:2) coordinate (d) node[right] {$A$}
        -- (0,0) coordinate (e)
        pic["$75$", <->, draw=black, angle eccentricity=1.5, angle radius=1cm]
        {angle=c--e--d};
      \draw (0,0)++(0.3,0)--++(0,0.3)--+(-0.3,0);
    \end{tikzpicture}
  \end{multicols}

\item A linear pair is formed by two angles, m$\angle RUT = 110^\circ$ and m$\angle SUT = 5x + 20$. \\[0.5cm] 
  Write an equation, then solve for $x$. \vspace{0.5cm}
    \begin{flushright}
      \begin{tikzpicture}[scale=1]
        \draw[<->, thick]
          (0:5) coordinate (a) node[below left] {$S$}
          -- (0,0) coordinate (b) node[below] {$U$}
          -- (65:3) coordinate (c) node[above right] {$T$}
          pic["$5x + 20$", <->, draw=black, angle eccentricity=1.5, angle radius=1.5cm]
          {angle=a--b--c};
          \draw[<-, thick]
          (180:4) coordinate (d) node[below] {$R$}
          -- (0,0) coordinate (e)
          pic["$110^\circ$", <->, draw=black, angle eccentricity=1.5, angle radius=1.5cm]
          {angle=c--e--d};
      \end{tikzpicture}
    \end{flushright}
       
\item Given m$\angle ABD = 4x-6$, m$\angle DBC = 5x+10$, and $m \angle ABC = 130^\circ$, as shown. \\[0.25cm]
  Model the situation with an equation, then solve for $x$. Check your solution for full credit.
  \begin{flushright}
      \begin{tikzpicture}[scale=2, rotate=0]
        \draw[<->, thick]
          (-10:1.5) coordinate (a) node[below left] {$C$}
          -- (0,0) coordinate (b) node[below] {$B$}
          -- (70:2) coordinate (c) node[above right] {$D$}
          pic["$5x+10$", <->, draw=black, angle eccentricity=1.75, angle radius=1cm]
          {angle=a--b--c};
          \draw[<-, thick]
          (120:1.75) coordinate (d) node[below left] {$A$}
          -- (0,0) coordinate (e)
          pic["$4x-6$", <->, draw=black, angle eccentricity=1.5, angle radius=1cm]
          {angle=c--e--d};
      \end{tikzpicture}
    \end{flushright}

\item Given vertical angles, m$\angle APD = 3x-5$, m$\angle BPC = 2x+20$, as shown. \\[0.25cm]
  Find $x$. Check your solution for full credit.
  \begin{flushright}
      \begin{tikzpicture}[scale=1.8, rotate=0]
        \draw[<->, thick]
          (0:2) coordinate (a) node[below left] {$C$}
          -- (0,0) coordinate (b) node[below right] {$P$}
          -- (70:1.5) coordinate (c) node[below right] {$B$}
          pic["$2x+20$", <->, draw=black, angle eccentricity=1.75, angle radius=1cm]
          {angle=a--b--c};
          \draw[<->, thick]
          (180:2) coordinate (d) node[above right] {$A$}
          -- (0,0) coordinate (e)
          -- (250:2) coordinate (f) node[above left] {$D$}
          pic["$3x-5$", <->, draw=black, angle eccentricity=1.75, angle radius=1cm]
          {angle=d--e--f};
      \end{tikzpicture}
    \end{flushright}
   
\item In the diagram shown, $\overrightarrow{BD} \perp \overleftrightarrow{ABC}$ with angle measures marked. 
  Find $x$. \\[0.25cm] 
  Show the check for full credit.\vspace{0.25cm}
    \begin{multicols}{2}
      m$\angle DBE = 7x-1^\circ$ \\[0.25cm]
      m$\angle EBC = 6x^\circ$ \\[0.25cm]
      \begin{tikzpicture}[scale=1]
        \draw[<->, thick]
          (0:5) coordinate (a) node[below left] {$C$}
          -- (0,0) coordinate (b) node[below] {$B$}
          -- (40:5) coordinate (c) node[below right] {$E$}
          pic["$6x$", <->, draw=black, angle eccentricity=1.5, angle radius=1.5cm]
          {angle=a--b--c};
          \draw[<-, thick]
          (90:4) coordinate (d) node[right] {$D$}
          -- (0,0) coordinate (e)
          pic["$7x-1$", <->, draw=black, angle eccentricity=1.5, angle radius=1.5cm]
          {angle=c--e--d};
          \draw[->, thick] (0,0)--(-180:2) node[below right]{$A$};
          \draw (0,0)++(-0.3,0)--++(0,0.3)--+(0.3,0);
      \end{tikzpicture}
    \end{multicols}
  
\item Given $\overleftrightarrow{ABC}$, right angle $\angle DBE$, m$\angle ABE = 4x+12$, and m$\angle CBD = 3x-6$. \\[0.5cm] 
  Find m$\angle CBD$. \vspace{0.5cm}
    \begin{flushright}
      \begin{tikzpicture}[scale=1, rotate=10]
        \draw[<->, thick]
          (-30:5) coordinate (a) node[below] {$C$}
          -- (0,0) coordinate (b) node[below] {$B$}
          -- (3,0) coordinate (c) node[above left] {$D$}
          pic["$3x-6$", <->, draw=black, angle eccentricity=1.5, angle radius=1.5cm]
          {angle=a--b--c};
          \draw[<->, thick]
          (150:4) coordinate (d) node[below] {$A$}
          -- (0,0) -- (0, 3) coordinate (e) node[above right] {$E$}
          pic["$4x+12$", <->, draw=black, angle eccentricity=1.5, angle radius=1.5cm]
          {angle=e--b--d};
          \draw (0,0)++(0.4,0)--++(0,0.4)--+(-0.4,0);
      \end{tikzpicture}
    \end{flushright}
  
\item Ray $\overrightarrow{BF}$ is the angle bisector of $\angle ABC$. Given that the angle measures are m$\angle ABF = 7x+9$ and m$\angle CBF = 9x-13$. \\[0.5cm] 
  Find m$\angle ABC$. \vspace{0.5cm}
    \begin{flushright}
      \begin{tikzpicture}[scale=1, rotate=40]
        \draw[<->, thick]
          (0:5) coordinate (a) node[above left] {$C$}
          -- (0,0) coordinate (b) node[below] {$B$}
          -- (80:3) coordinate (c) node[above right] {$F$}
          pic["$9x-13$", <->, draw=black, angle eccentricity=1.25, angle radius=1.5cm]
          {angle=a--b--c};
          \draw[<-, thick]
          (160:4) coordinate (d) node[above right] {$A$}
          -- (0,0) coordinate (e)
          pic["$7x+9$", <->, draw=black, angle eccentricity=1.5, angle radius=1.5cm]
          {angle=c--e--d};
      \end{tikzpicture}
    \end{flushright}

\item Ray $\overrightarrow{XL}$ is the angle bisector of $\angle KXM$. Given m$\angle JXN = 2x+3$. \\[0.5cm] 
Find $x$.
  \begin{center}
  \begin{tikzpicture}[scale=1, rotate=0]
    \draw[<->, thick] (-135:2)--(0,0)--(45:3);
    \draw[<->, thick] (-4,0)--(3,0);
    \draw[->, thick] (0,0)--(0,3);
    \draw (0,0)++(-0.3,0)--++(0,0.3)--+(0.3,0);
    \draw[fill] (45:2) circle [radius=0.05] node[right]{$L$};
    \draw[fill] (-3,0) circle [radius=0.05] node[above left]{$J$}; 
    \draw[fill] (0,0) circle [radius=0.05] node[below right]{$X$};
    \draw[fill] (0,2) circle [radius=0.05] node[left]{$K$};
    \draw[fill] (2,0) circle [radius=0.05] node[below right]{$M$};
    \draw[fill] (-135:1.5) circle [radius=0.05] node[right]{$N$};
  \end{tikzpicture}
  \end{center}


\end{enumerate}
\end{document}