\documentclass[12pt, twoside]{article}
\usepackage[letterpaper, margin=1in, headsep=0.2in]{geometry}
\setlength{\headheight}{0.6in}
%\usepackage[english]{babel}
\usepackage[utf8]{inputenc}
\usepackage{microtype}
\usepackage{amsmath}
\usepackage{amssymb}
%\usepackage{amsfonts}
\usepackage{siunitx} %units in math. eg 20\milli\meter
\usepackage{yhmath} % for arcs, overparenth command
\usepackage{tikz} %graphics
\usetikzlibrary{quotes, angles}
\usepackage{graphicx} %consider setting \graphicspath{{images/}}
\usepackage{parskip} %no paragraph indent
\usepackage{enumitem}
\usepackage{multicol}
\usepackage{venndiagram}

\usepackage{fancyhdr}
\pagestyle{fancy}
\fancyhf{}
\renewcommand{\headrulewidth}{0pt} % disable the underline of the header
\raggedbottom
\hfuzz=2mm %suppresses overfull box warnings

\usepackage{hyperref}

\fancyhead[LE]{\thepage}
\fancyhead[RO]{\thepage \\ Name: \hspace{4cm} \,\\}
\fancyhead[LO]{BECA / Dr. Huson / Geometry\\*  Unit 2: Angles\\* 12 October 2022}

\begin{document}

\subsubsection*{2.5 Classwork: Isosceles base theorem}
\emph{Diagrams are not necessarily drawn to scale unless otherwise stated.}
\begin{enumerate}
\item Given $\triangle ABC$. $\overline{AC} \cong \overline{BC}$,  m$\angle A=55$. Find m$\angle C$.\\[0.5cm]
\begin{tikzpicture}[scale=0.7]
  \draw[thick](0,0)--(4,0)--(2,6)--(0,0);
  \draw[fill] (0,0) circle [radius=0.05] node[below]{$A$};
  \draw[fill] (4,0) circle [radius=0.05] node[below]{$B$};
  \draw[fill] (2,6) circle [radius=0.05] node[above right]{$C$};
  %\draw[color=blue] (0,0) ++(0.75,0) arc [start angle=0, end angle=70, radius=0.75];
  %\draw[color=blue] (4,0) ++(-0.22, 0.73) arc [start angle=110, end angle=180, radius=0.75];
  \draw[thick] (0.8,3.1)--(1.2,2.9); %tick mark
  \draw[thick] (2.8,2.9)--(3.2,3.1); %tick mark
  %\node [right] at (3.25,2.5){$x+7$};
  %\node [left] at (0.75,2.5){$2x+1$};
\end{tikzpicture}%\vspace{1cm}

\item Given $\triangle DEF$. $\overline{DF} \cong \overline{EF}$,  m$\angle F=72$. Find m$\angle D$.\\[0.5cm]
  \begin{tikzpicture} %[scale=3] Isosceles, parallel marks, congruence marks
    \draw[-, thick] (0,0) node[below]{$D$}--
      (5,0) node[below]{$E$}--(2.5,4) node[above]{$F$}--cycle;
    %\draw[>->, thick] (1.0,1.6)--(1.25,2);
    %\draw[{Bar[]}-{Bar[]}, thick] (4,1.6)--(3.75,2);
    %\draw[color=blue] (0.75,0) arc [start angle=0, end angle=58, radius=0.75];
    %\draw (5,0)-- +(-0.75,0) arc [start angle=180, end angle=122, radius=0.75];
  \end{tikzpicture}%\vspace{1cm}

\item Given the triangle shown with congruent sides marked.  m$\angle 1=110$. Find m$\angle 2$\\
Spicy: Find the measure of the vertex angle. \\[0.5cm]
  \begin{tikzpicture}[scale=0.7]
    \draw[thick](0,0)--(4,0)--(2,6)--(0,0);
    \draw[->, thick](0,0)--(4,0) node[above right]{1}--(7,0);
    \draw[fill] (0,0) circle [radius=0.05] node[above right]{$2$};
    %\draw[fill] (4,0) circle [radius=0.05] node[below]{$B$};
    \draw[fill] (2,6) circle [radius=0.05] node[above right]{$V$};
    %\draw[color=blue] (0,0) ++(0.75,0) arc [start angle=0, end angle=70, radius=0.75];
    %\draw[color=blue] (4,0) ++(-0.22, 0.73) arc [start angle=110, end angle=180, radius=0.75];
    \draw[thick] (0.8,3.1)--(1.2,2.9); %tick mark
    \draw[thick] (2.8,2.9)--(3.2,3.1); %tick mark
    %\node [right] at (3.25,2.5){$x+7$};
    %\node [left] at (0.75,2.5){$2x+1$};
  \end{tikzpicture}

\newpage
\item Given circle with center $Z$ and isosceles $\triangle XYZ$. m$\angle Z=100$. Find m$\angle Y$.\\[1cm]
    %\hspace{1cm} Given the line  $l$ and point $P$.
    \begin{tikzpicture}
      %\draw[-, thick] (-6,0) node[left]{$A$}--(0,0);
      \draw  (0,0) circle [radius=3] node[above]{$Z$};
      \draw[-, thick] (220:3) node[left]{$X$}--(0,0)
        --(320:3) node[right]{$Y$}--cycle;
      \node at (0,-0.6){$100^\circ$};
      \draw[fill] (0,0) circle [radius=0.05];
    \end{tikzpicture}

\item Given circle $O$ with inscribed $\triangle SLO$. m$\angle S=x+17$. Find m$\angle L=2x-18$. Find $x$.\\
For full credit, check your answer.\\[1cm]
    %\hspace{1cm} Given the line  $l$ and point $P$.
    \begin{tikzpicture}
      %\draw[-, thick] (-6,0) node[left]{$A$}--(0,0);
      \draw  (0,0) circle [radius=3] node[left]{$O$};
      \draw[-, thick] (15:3) node[right]{$L$}--(0,0)
        --(97:3) node[above]{$S$}--cycle;
      %\node at (8.5,-0.4){$l$};
      \draw[fill] (0,0) circle [radius=0.05];
    \end{tikzpicture}
    \vspace{1cm}

\item Writing to learn: Why do we write down the theorems that justify each step to solve a problem?

\end{enumerate}
\end{document}