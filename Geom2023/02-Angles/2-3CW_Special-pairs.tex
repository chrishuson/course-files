\documentclass[12pt, twoside]{article}
\usepackage[letterpaper, margin=1in, headsep=0.2in]{geometry}
\setlength{\headheight}{0.6in}
%\usepackage[english]{babel}
\usepackage[utf8]{inputenc}
\usepackage{microtype}
\usepackage{amsmath}
\usepackage{amssymb}
%\usepackage{amsfonts}
\usepackage{siunitx} %units in math. eg 20\milli\meter
\usepackage{yhmath} % for arcs, overparenth command
\usepackage{tikz} %graphics
\usetikzlibrary{quotes, angles}
\usepackage{graphicx} %consider setting \graphicspath{{images/}}
\usepackage{parskip} %no paragraph indent
\usepackage{enumitem}
\usepackage{multicol}
\usepackage{venndiagram}

\usepackage{fancyhdr}
\pagestyle{fancy}
\fancyhf{}
\renewcommand{\headrulewidth}{0pt} % disable the underline of the header
\raggedbottom
\hfuzz=2mm %suppresses overfull box warnings

\usepackage{hyperref}

\fancyhead[LE]{\thepage}
\fancyhead[RO]{\thepage \\ Name: \hspace{4cm} \,\\}
\fancyhead[LO]{BECA / Dr. Huson / Geometry\\*  Unit 2: Angles\\* 30 September 2022}

\begin{document}

\subsubsection*{2.3 Classwork: Special angle pairs}
\begin{enumerate}
\item Given a straight line and a ray, making two angles.
  \begin{enumerate}[itemsep=0.5cm]
    \item Write down the names of the two angles using proper notation.
    \item Using a protractor, measure the two angle in degrees.
    \item Do they sum to $180^\circ$?
  \end{enumerate}
  \begin{center}
  \begin{tikzpicture}
    \draw [<->, thick] (-6,0)--(6,0);
    \draw [thick, ->] (0,0)--(60:5);
    \draw [fill] (0,0) circle [radius=0.05] node[below]{$A$};
    \draw [fill] (-4,0) circle [radius=0.05] node[below]{$B$};
    \draw [fill] (4,0) circle [radius=0.05] node[below]{$D$};
    \draw [fill] (60:4) circle [radius=0.05] node[right]{$C$};
  \end{tikzpicture}
  \end{center}

\item Write down the name of the \emph{three} angles shown in the diagram below and their angle measures, using your protractor.
    \begin{center}
    \begin{tikzpicture}[scale=2]
      \draw [->, thick] (0,0)--(15:4);
      \draw [->, thick] (0,0)--(95:4);
      \draw [->, thick] (0,0)--(130:4);
      \draw [fill] (15:3) circle [radius=0.03] node[below]{$B$};
      \draw [fill] (95:2) circle [radius=0.03] node[right]{$C$};
      \draw [fill] (0,0) circle [radius=0.03] node[left]{$A$};
      \draw [fill] (130:2) circle [radius=0.03] node[left]{$D$};
    \end{tikzpicture}
    \end{center}


\item As shown below, two lines intersect making four angles: $\angle 1$, $\angle 2$, $\angle 3$, and $\angle 4$.
\begin{center}
\begin{tikzpicture}[scale=0.8]
  \draw [<->, thick] (0,-1.5)--(10,1.5);
  \draw [<->, thick] (2,3.5)--(7,-3.5);
  \node at (3,.4){1};
  \node at (6,-.6){3};
  \node at (5,1){2};
  \node at (4,-1){4};
  %\draw [fill] (0,0) circle [radius=0.05] node[below]{$P$};
  %\draw [fill] (6,0) circle [radius=0.05] node[below]{$R$};
  %\draw [fill] (3,0) circle [radius=0.05] node[below]{$Q$};
\end{tikzpicture}
\end{center}
\begin{enumerate}
\item Which angle is opposite $\angle 1$? \rule{4cm}{0.15mm} \bigskip
\item Name an angle that is adjacent to $\angle 4$. \rule{4cm}{0.15mm} \bigskip
\item True or false, $\angle 2$ and $\angle 4$ are vertical angles. \rule{3cm}{0.15mm}
\end{enumerate}

\item Given the situation in the diagram, answer each question. Circle True or False.
  \begin{center}
  \begin{tikzpicture}[scale=1, rotate=20]
    \draw [->, thick] (0,0)--(4,3);
    \draw [<->, thick] (-5,.5)--(5,-.5);
    \draw [->, thick] (0,0)--(-1.2,3);
    \draw [fill] (-1,2.5) circle [radius=0.05] node[below left]{$S$};
    \draw [fill] (2.66666,2) circle [radius=0.05] node[above left ]{$T$};
    \draw [fill] (0,0) circle [radius=0.05] node[below]{$P$};
    \draw [fill] (4,-0.4) circle [radius=0.05] node[above]{$U$};
    \draw [fill] (-4,0.4) circle [radius=0.05] node[above]{$R$};
  \end{tikzpicture}
  \end{center}
  \begin{enumerate}
  \item True or False: $\overrightarrow{RP}$ and $\overrightarrow{UP}$ are opposite rays.\bigskip
  \item True or False: $\angle TPR$ is an obtuse angle.\bigskip
  \item True or False: $\angle RPS$ and $\angle SPU$ are supplementary angles.\bigskip
  \item True or False: $\angle RPS$ and $\angle SPT$ are adjacent angles. \bigskip
  \end{enumerate}

\newpage
\item Find the measure of the angle in degrees and the given segment's length in centimeters.
  \begin{enumerate}
    \item  $m \angle UST = $ \rule{3cm}{0.15mm} \bigskip
    \item  $SU=$ \rule{3cm}{0.15mm} \bigskip
    \item Name a pair of opposite rays: \rule{4cm}{0.15mm} \bigskip
  \end{enumerate}
  \begin{center}
  \begin{tikzpicture}[scale=1.]
    \draw [->, thick] (0,0)--(55:4);
    \draw [<->, thick] (-3,0)--(7,0);
    %\draw [->, thick] (0,0)--(-1.2,3);
    %\draw [fill] (-1,2.5) circle [radius=0.05] node[left ]{$B$};
    \draw [fill] (55:3) circle [radius=0.05] node[above left ]{$U$};
    \draw [fill] (-2,0) circle [radius=0.05] node[below]{$R$};
    \draw [fill] (0,0) circle [radius=0.05] node[below]{$S$};
    \draw [fill] (4,0) circle [radius=0.05] node[above]{$T$};
  \end{tikzpicture}
  \end{center}
  
\item Measure the required angles of the diagram below and answer the questions. \vspace{0.25cm}
  \begin{enumerate}
    \item  $m \angle AOB = $ \rule{2cm}{0.15mm} \hspace{0.5cm} $m \angle BOC = $ \rule{2cm}{0.15mm} \hspace{0.5cm} $m \angle DOE = $ \rule{2cm}{0.15mm} \bigskip
    \item Name an angle that is vertical to $\angle DOE$: \rule{4cm}{0.15mm}
    \bigskip
    \item Name an angle that is complementary to $\angle AOB$: \rule{4cm}{0.15mm} 
  \end{enumerate}
  \vspace{0.5cm}
  \begin{center}
  \begin{tikzpicture}[scale=1.3, rotate=-20]
    \draw [<->, thick] (-55:3)--(0,0)--(125:4);
    \draw [<->, thick] (-5,0)--(5,0);
    \draw [->, thick] (0,0)--(0,4);
    \draw (0,0)++(0.3,0)--++(0,0.3)--+(-0.3,0);
    %\draw [fill] (-1,2.5) circle [radius=0.05] node[left ]{$B$};
    \draw [fill] (125:3) circle [radius=0.05] node[below left]{$B$};
    \draw [fill] (-4,0) circle [radius=0.05] node[below]{$A$}; 
    \draw [fill] (0,0) circle [radius=0.05] node[below left]{$O$};
    \draw [fill] (0,3) circle [radius=0.05] node[left]{$C$};
    \draw [fill] (4,0) circle [radius=0.05] node[below]{$D$};
    \draw [fill] (-55:2) circle [radius=0.05] node[left]{$E$};
  \end{tikzpicture}
  \end{center}

\end{enumerate}
\end{document}