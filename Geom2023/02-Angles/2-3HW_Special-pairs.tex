\documentclass[12pt, twoside]{article}
\usepackage[letterpaper, margin=1in, headsep=0.2in]{geometry}
\setlength{\headheight}{0.6in}
%\usepackage[english]{babel}
\usepackage[utf8]{inputenc}
\usepackage{microtype}
\usepackage{amsmath}
\usepackage{amssymb}
%\usepackage{amsfonts}
\usepackage{siunitx} %units in math. eg 20\milli\meter
\usepackage{yhmath} % for arcs, overparenth command
\usepackage{tikz} %graphics
\usetikzlibrary{quotes, angles}
\usepackage{graphicx} %consider setting \graphicspath{{images/}}
\usepackage{parskip} %no paragraph indent
\usepackage{enumitem}
\usepackage{multicol}
\usepackage{venndiagram}

\usepackage{fancyhdr}
\pagestyle{fancy}
\fancyhf{}
\renewcommand{\headrulewidth}{0pt} % disable the underline of the header
\raggedbottom
\hfuzz=2mm %suppresses overfull box warnings

\usepackage{hyperref}

\fancyhead[LE]{\thepage}
\fancyhead[RO]{\thepage \\ Name: \hspace{4cm} \,\\}
\fancyhead[LO]{BECA / Dr. Huson / Geometry\\*  Unit 2: Angles\\* 30 September 2022}

\begin{document}

\subsubsection*{2.3 Homework: Special angle pairs}
\begin{enumerate}
\item Given the situation in the diagram, answer each question. Circle True or False. \vspace{0.25cm}
  \begin{center}
    \begin{tikzpicture}[scale=0.8, rotate=180]
      \draw [->, thick] (0,0)--(50:5);
      \draw [<->, thick] (-5,.5)--(5,-.5);
      \draw [->, thick] (0,0)--(-1.2,3);
      \draw [fill] (-1,2.5) circle [radius=0.05] node[left ]{$S$};
      \draw [fill] (50:3) circle [radius=0.05] node[above left ]{$T$};
      \draw [fill] (0,0) circle [radius=0.05] node[above right]{$P$};
      \draw [fill] (4,-0.4) circle [radius=0.05] node[below]{$U$};
      \draw [fill] (-4,0.4) circle [radius=0.05] node[below]{$R$};
    \end{tikzpicture}
    \end{center}
  \begin{enumerate}
    \item True or False: $\overrightarrow{RP}$ and $\overrightarrow{UP}$ are opposite rays.\bigskip
    \item True or False: $\angle TPR$ is supplementary to $\angle TPU$.\bigskip
    \item True or False: $\angle RPS$ and $\angle TPS$ are complementary angles. \bigskip
    \item True or False: $\angle RPS$ and $\angle TPU$ are vertical angles. \bigskip
  \end{enumerate}

\item The shape shown below is composed of straight lines and right angles, with some lengths as marked. Find the perimeter of the figure. Show your work.
  \begin{flushright}
  \begin{tikzpicture}[scale=0.5]
    \draw [-, thick] (0,0)--(13,0)--(13,3)--(9,3)--(9,7)--(13,7)--
    (13, 9)--(0,9)--(0,7)--(4,7)--(4,3)--(0,3)--cycle;
    %\draw [fill] (0,0) circle [radius=0.05] node[left]{$A$};
    %\draw [fill] (7,0) circle [radius=0.05] node[right]{$B$};
    %\draw [fill] (7,2) circle [radius=0.05] node[right]{$C$};
    %\draw [fill] (0,2) circle [radius=0.05] node[left]{$D$};
    \node at (4.5, 5){3};
    \node at (2, 2.5){3};
    \node at (8.5, 5){3};
    \node at (11, 2.5){3};
    \node at (6.5, -0.5){10};
    \node at (13.5, 1.5){2};
    \node at (13.5, 8){1};
  \end{tikzpicture}
  \end{flushright} 

\item Given $\overline{DEFG}$, $DE=1 \frac{2}{5}$, $EF=2 \frac{3}{10}$, and $FG= \frac{4}{5}$. (diagram not to scale)\\ [0.25cm]
  Find ${DG}$, expressed as a fraction, not a decimal.
  \begin{flushright}
      \begin{tikzpicture}
        \draw [-, thick] (0,0)--(9,0);
        \draw [fill] (0,0) circle [radius=0.05] node[below]{$D$};
        \draw [fill] (3,0) circle [radius=0.05] node[below]{$E$};
        \draw [fill] (7,0) circle [radius=0.05] node[below]{$F$};
        \draw [fill] (9,0) circle [radius=0.05] node[below]{$G$};
      \end{tikzpicture}
    \end{flushright}

\end{enumerate}
\end{document}