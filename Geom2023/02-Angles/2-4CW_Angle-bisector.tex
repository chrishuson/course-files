\documentclass[12pt, twoside]{article}
\usepackage[letterpaper, margin=1in, headsep=0.2in]{geometry}
\setlength{\headheight}{0.6in}
%\usepackage[english]{babel}
\usepackage[utf8]{inputenc}
\usepackage{microtype}
\usepackage{amsmath}
\usepackage{amssymb}
%\usepackage{amsfonts}
\usepackage{siunitx} %units in math. eg 20\milli\meter
\usepackage{yhmath} % for arcs, overparenth command
\usepackage{tikz} %graphics
\usetikzlibrary{quotes, angles}
\usepackage{graphicx} %consider setting \graphicspath{{images/}}
\usepackage{parskip} %no paragraph indent
\usepackage{enumitem}
\usepackage{multicol}
\usepackage{venndiagram}

\usepackage{fancyhdr}
\pagestyle{fancy}
\fancyhf{}
\renewcommand{\headrulewidth}{0pt} % disable the underline of the header
\raggedbottom
\hfuzz=2mm %suppresses overfull box warnings

\usepackage{hyperref}

\fancyhead[LE]{\thepage}
\fancyhead[RO]{\thepage \\ Name: \hspace{4cm} \,\\}
\fancyhead[LO]{BECA / Dr. Huson / Geometry\\*  Unit 2: Angles\\* 3 October 2022}

\begin{document}

\subsubsection*{2.4 Classwork: Angle bisector}
\begin{enumerate}
\item Given an angle with vertex $A$.
  \begin{enumerate}[itemsep=0.5cm]
    \item Using a protractor, measure angle $A$ in degrees. m$\angle A =$
    \item Draw a ray $\overrightarrow{AB}$ that exactly bisects $\angle A$.
    \item What is the measure of each half angle?
  \end{enumerate}
  \begin{center}
  \begin{tikzpicture}
    \draw[<->, thick] (40:9)--(0,0)--(9,0);
    \draw[fill] (0,0) circle [radius=0.05] node[below]{$A$};
  \end{tikzpicture}
  \end{center}

\item An angle bisector is shown below, with $\overrightarrow{PR}$ bisecting $\angle QPS$. Given m$\angle QPR = 3x-12$ and m$\angle QPS = 5x+4$, find m$\angle QPS$.
  \begin{flushright}
  \begin{tikzpicture}[scale=0.6, rotate=30]
    \draw[<->, thick] (230:5)node[left]{$Q$} 
    --(0,0)node[above right]{$P$}
    --(110:6)node[above right]{$S$}--(110:7);
    \draw[->, thick] (0,0)--(170:7)node[below right]{$R$};
  \end{tikzpicture}
  \end{flushright}

\newpage
\item \textbf{Do Not Solve}. Circle the appropriate equation. Cite a justification on the line.
  \begin{multicols}{2}
    \begin{itemize} 
      \item ``definition of bisector" 
      \item ``linear pairs sum to $180^\circ$" 
      \item ``vertical $\angle$s are $\cong$" 
      \item ``isosceles base angle theorem''
      \item ``$\perp$ rays  with complementary $\angle$s adding to $90^\circ$" 
    \end{itemize}
  \end{multicols}
\begin{enumerate}
\item $\overleftrightarrow{RPU}$ with ray $\overrightarrow{PS}$. \hspace{6cm}
  \begin{tikzpicture}[rotate=10, scale=0.6]
    \draw[<->, thick] (-5,.5)--(3,-.3);
    \draw[->, thick] (0,0)--(30:3) node[below right]{$S$};
    \draw[fill] (0,0) circle [radius=0.05] node[below]{$P$};
    \draw[fill] (2,-0.2) circle [radius=0.05] node[below]{$U$};
    \draw[fill] (-4,0.4) circle [radius=0.05] node[above]{$R$};
  \end{tikzpicture} \par \bigskip
  $\angle RPS \cong \angle SPU$ \hspace{0.25cm} $m \angle RPS + m \angle SPU = 180^\circ$ \hspace{0.25cm} \rule{6cm}{0.15mm}  \vspace{0.25cm}

\item Given m$\angle R=m\angle U =65$, and m$\angle UST=130$. Find m$\angle RSU$.
  \begin{tikzpicture}[scale=0.6]
    \draw[<-, thick] (7,0)--(2,0)--(3.25,3)--(4.5,0);
    \draw[fill] (2,0) circle [radius=0.05] node[below right]{$R$};
    \draw[fill] (4.5,0) circle [radius=0.05] node[below]{$S$};
    \draw[fill] (3.25,3) circle [radius=0.05] node[right]{$U$};
    \draw[fill] (6,0) circle [radius=0.05] node[above]{$T$};
  \end{tikzpicture} \par \bigskip
  $\angle UST \cong \angle RSU$ \hspace{0.5cm} m$\angle UST + \text{m}\angle RSU =  180$ \hspace{0.5cm} \rule{5cm}{0.15mm} \vspace{0.25cm}

\item Given $m \angle 1 = 4x+6$, $m \angle 2 = 6x-32$. Find $m \angle 1$.
\hspace{2cm}
\begin{tikzpicture}[scale=.3, rotate=15]
  \draw[<->, thick] (0,-1.5)--(10,1.5);
  \draw[<->, thick] (2,3.5)--(7,-3.5);
  \node at (3,.4){1};
  \node at (6,-.6){2};
\end{tikzpicture} \par \bigskip
$\angle 1 \cong \angle 2$ \hspace{1cm} m$\angle 1 + \text{m}\angle 2 =  180$ \hspace{0.5cm} \rule{6cm}{0.15mm}
\vspace{0.5cm}

\item Given $\overrightarrow{BA} \perp \overrightarrow{BC}$, $m \angle ABD = 2x-5$, and $m \angle DBC = x-10$.
  \begin{tikzpicture}[scale=0.7]
    \draw[<->, thick] (0,2.5)--(0,0)--(4,0);
    \draw[->, thick] (0,0)--(3.5, 2);
    \draw[-, thin] (0, 0.4)--(0.4, 0.4)--(0.4, 0);
    \draw[fill] (0,0) circle [radius=0.05] node[below]{$B$};
    \draw[fill] (0,2) circle [radius=0.05] node[right]{$A$};
    \draw[fill] (3,0) circle [radius=0.05] node[below]{$C$};
    \draw[fill] (2.625, 1.5) circle [radius=0.05] node[below]{$D$};
  \end{tikzpicture} \par \bigskip
  $\angle ABD \cong \angle DBC$ \hspace{0.5cm} m$\angle ABD + \text{m}\angle DBC =  90$ \hspace{0.5cm} \rule{5cm}{0.15mm}
\end{enumerate}

\newpage
\begin{multicols}{2}
  \begin{itemize} 
    \item ``alternate interior $\angle$s are $\cong$"
    \item ``corresponding $\angle$s of $\parallel$ lines are $\cong$" 
    \item ``same-side interior $\angle$s are supplementary" 
  \end{itemize}
\end{multicols}

\item Given two parallel lines and a transversal, as shown. \hspace{2cm}
\begin{tikzpicture}[scale=0.7]
  \draw[<->, thick] (3,2)--(8,2);
  \draw[<->, thick] (2,0)--(7,0);
  \draw[<->, thick] (4,-1)--(5.5,3);
  \node at (4.5,0.3) [left]{$5$};
  \node at (4.5,0.3) [right]{$6$};
  \node at (4.3,-0.3) [left]{$7$};
  \node at (4.3,-0.3) [right]{$8$};
  \node at (5.2,2) [above left]{$1$};
  \node at (5.2,2) [above right]{$2$};
  \node at (5,2) [below left]{$3$};
  \node at (5,2) [below right]{$4$};
\end{tikzpicture} \\[0.5cm]
$\angle 4 \cong \angle 5$ \hspace{1cm} m$\angle 3 + \text{m}\angle 6 =  180$ \hspace{0.5cm} \rule{6cm}{0.15mm}

\end{enumerate}
\end{document}