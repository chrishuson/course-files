\documentclass[12pt, twoside]{article}
\usepackage[letterpaper, margin=1in, headsep=0.2in]{geometry}
\setlength{\headheight}{0.6in}
%\usepackage[english]{babel}
\usepackage[utf8]{inputenc}
\usepackage{microtype}
\usepackage{amsmath}
\usepackage{amssymb}
%\usepackage{amsfonts}
\usepackage{siunitx} %units in math. eg 20\milli\meter
\usepackage{yhmath} % for arcs, overparenth command
\usepackage{tikz} %graphics
\usetikzlibrary{quotes, angles}
\usepackage{graphicx} %consider setting \graphicspath{{images/}}
\usepackage{parskip} %no paragraph indent
\usepackage{enumitem}
\usepackage{multicol}
\usepackage{venndiagram}

\usepackage{fancyhdr}
\pagestyle{fancy}
\fancyhf{}
\renewcommand{\headrulewidth}{0pt} % disable the underline of the header
\raggedbottom
\hfuzz=2mm %suppresses overfull box warnings

\usepackage{hyperref}

\fancyhead[LE]{\thepage}
\fancyhead[RO]{\thepage \\ Name: \hspace{4cm} \,\\}
\fancyhead[LO]{BECA / Dr. Huson / Geometry\\*  Unit 2: Angles\\* 7 October 2022}

\begin{document}

\subsubsection*{2.3 Classwork: Vertical angles}
\begin{enumerate}
\item As shown below, two lines intersect making four angles: $\angle 1$, $\angle 2$, $\angle 3$, and $\angle 4$.
  \begin{center}
  \begin{tikzpicture}[scale=0.6, rotate=15]
    \draw[<->, thick] (0,-1.5)--(10,1.5);
    \draw[<->, thick] (2,3.5)--(7,-3.5);
    \node at (3,.4){1};
    \node at (6,-.6){3};
    \node at (5,1){2};
    \node at (4,-1){4};
  \end{tikzpicture}
  \end{center}
  \begin{enumerate}
    \item Which angle is opposite $\angle 1$? \rule{4cm}{0.15mm} \bigskip
    \item Name an angle that is adjacent to $\angle 4$. \rule{4cm}{0.15mm} \bigskip
    \item True or false, $\angle 2$ and $\angle 4$ are vertical angles. \rule{3cm}{0.15mm}
  \end{enumerate}

\item Two lines intersect with m$\angle 1=80^\circ$. Find and mark the measures of $\angle 2$, $\angle 3$, and $\angle 4$.
  \begin{flushleft}
  \begin{tikzpicture}[scale=0.6]
    \draw[<->, thick] (0,-1.5)--(10,1.5);
    \draw[<->, thick] (2,3.5)--(7,-3.5);
    \node at (2,.4){m$\angle 1=80^\circ$};
    \node at (6,-.6){3};
    \node at (5,1){2};
    \node at (4,-1){4};
  \end{tikzpicture}
  \end{flushleft}

\item Given the situation in the diagram, answer each question. Circle True or False.
  \begin{center}
  \begin{tikzpicture}[scale=1, rotate=20]
    \draw[->, thick] (0,0)--(37:4);
    \draw[<->, thick] (-2.5,0)--(5,0);
    \draw[->, thick] (0,0)--(-1.2,3);
    \draw[fill] (-1,2.5) circle [radius=0.05] node[below left]{$S$};
    \draw[fill] (2.66666,2) circle [radius=0.05] node[above left ]{$T$};
    \draw[fill] (0,0) circle [radius=0.05] node[below]{$P$};
    \draw[fill] (4,0) circle [radius=0.05] node[above]{$U$};
    \draw[fill] (-2,0) circle [radius=0.05] node[above]{$R$};
  \end{tikzpicture}
  \end{center} \bigskip
  \begin{enumerate}
    \item True or False: $\overrightarrow{RP}$ and $\overrightarrow{UP}$ are opposite rays.
    \item True or False: $\angle TPR$ is an obtuse angle.
    \item True or False: $\angle RPS$ and $\angle SPU$ are supplementary angles.
    \item True or False: $\angle RPS$ and $\angle SPT$ are adjacent angles.
  \end{enumerate}

\newpage
\item Identify the true statements
  \begin{multicols}{2}
    \begin{enumerate}
      \item $\angle 1 \cong \angle 2$
      \item $\angle 2 \cong \angle 4$
      \item m$\angle 1 + \text{m}\angle 4=180^\circ$
      \item m$\angle 2 + \text{m}\angle 3=90^\circ$
    \end{enumerate}
  \begin{center}
  \begin{tikzpicture}[scale=0.5, rotate=15]
    \draw[<->, thick] (0,-1.5)--(10,1.5);
    \draw[<->, thick] (2,3.5)--(7,-3.5);
    \node at (3,.4){1};
    \node at (6,-.6){3};
    \node at (5,1){2};
    \node at (4,-1){4};
  \end{tikzpicture}
  \end{center}
  \end{multicols}

\item Measure the required angles of the diagram below and answer the questions. \vspace{0.25cm}
  \begin{enumerate}
    \item  $m \angle AOB = $ \rule{2cm}{0.15mm} \hspace{0.5cm} $m \angle BOC = $ \rule{2cm}{0.15mm} \hspace{0.5cm} $m \angle DOE = $ \rule{2cm}{0.15mm} \bigskip
    \item Name an angle that is vertical to $\angle DOE$: \rule{4cm}{0.15mm}
    \bigskip
    \item Name an angle that is complementary to $\angle AOB$: \rule{4cm}{0.15mm} 
  \end{enumerate}
  \vspace{0.5cm}
  \begin{center}
  \begin{tikzpicture}[scale=1, rotate=20]
    \draw[<->, thick] (-55:3)--(0,0)--(125:5);
    \draw[<->, thick] (-5,0)--(6,0);
    \draw[->, thick] (0,0)--(0,4);
    \draw (0,0)++(0.3,0)--++(0,0.3)--+(-0.3,0);
    \draw[fill] (125:4) circle [radius=0.05] node[below left]{$B$};
    \draw[fill] (-4,0) circle [radius=0.05] node[below]{$A$}; 
    \draw[fill] (0,0) circle [radius=0.05] node[below left]{$O$};
    \draw[fill] (0,3) circle [radius=0.05] node[left]{$C$};
    \draw[fill] (5,0) circle [radius=0.05] node[below]{$D$};
    \draw[fill] (-55:2) circle [radius=0.05] node[below left]{$E$};
  \end{tikzpicture}
  \end{center}

\item Angles $APC$ and $CPD$ form a linear pair. m$\angle APC = 10x+15$ and m$\angle CPD = 3x-4$. Find m$\angle CPD$. Check your answer for full credit.
  \begin{flushright}
    \begin{tikzpicture}[scale=0.8, rotate=-10]
      \draw[->, thick] (0,0)--(35:5);
      \draw[<->, thick] (-3.5,0)--(4,0);
      \draw[->, thick] (0,0)--(0,3);
      \draw (0,0)++(0.4,0)--++(0,0.4)--+(-0.4,0);
      \draw[fill] (35:4) circle [radius=0.05] node[below right]{$C$};
      \draw[fill] (-3,0) circle [radius=0.05] node[below]{$A$};
      \draw[fill] (0,0) circle [radius=0.05] node[below left]{$P$};
      \draw[fill] (0,2) circle [radius=0.05] node[left]{$B$};
      \draw[fill] (3,0) circle [radius=0.05] node[below]{$D$};
    \end{tikzpicture}
    \end{flushright}


\end{enumerate}
\end{document}