% \documentclass[12pt, twoside]{article}
\usepackage[letterpaper, margin=1in, headsep=0.2in]{geometry}
\setlength{\headheight}{0.6in}
%\usepackage[english]{babel}
\usepackage[utf8]{inputenc}
\usepackage{microtype}
\usepackage{amsmath}
\usepackage{amssymb}
%\usepackage{amsfonts}
\usepackage[nomessages]{fp} %\FPeval{\var-name}{2*sin(pi/6)}
\usepackage{siunitx} %units in math. eg 20\milli\meter
\usepackage{yhmath} % for arcs, overparenth command
\usepackage{tikz} %graphics
\usetikzlibrary{quotes, angles, arrows, arrows.meta}
\usepackage{graphicx} %consider setting \graphicspath{{images/}}
\usepackage{parskip} %no paragraph indent
\usepackage{enumitem}
\usepackage{multicol}
\usepackage{venndiagram}

\usepackage{fancyhdr}
\pagestyle{fancy}
\fancyhf{}
\renewcommand{\headrulewidth}{0pt} % disable the underline of the header
\raggedbottom
\hfuzz=2mm %suppresses overfull box warnings

\usepackage{hyperref}

\fancyhead[LE]{\thepage}
\fancyhead[RO]{\thepage \\ Name: \hspace{4cm} \,\\}
\fancyhead[LO]{BECA / Dr. Huson / Geometry\\*  Unit 2: Angles\\* 28 Sept 2022}

\begin{document}

\subsubsection*{2.1 Classwork: Angle measures}
\begin{enumerate}
\item Given an angle with vertex $A$.
  \begin{enumerate}[itemsep=0.5cm]
    \item Using a protractor, measure angle $A$ in degrees. $m\angle A =$
    \item Draw a ray $\overrightarrow{AB}$ that exactly bisects $\angle A$.
    \item What is the measure of each half angle?
  \end{enumerate}
  \begin{center}
  \begin{tikzpicture}
    \draw [<->, thick] (40:9)--(0,0)--(9,0);
    \draw [fill] (0,0) circle [radius=0.05] node[below]{$A$};
    %\draw [fill] (7,0) circle [radius=0.05] node[below]{$N$};
  \end{tikzpicture}
  \end{center}

\item 
\begin{enumerate}
  \item Write down the name of the angle shown in the diagram below using proper geometric notation.
  \item Find the measure of the angle in degrees with a protractor.
  \item Is it an acute, obtuse, or right angle?
\end{enumerate}
    \begin{flushright}
    \begin{tikzpicture}[scale=2]
      \draw [->, thick] (0,0)--(4,3);
      \draw [->, thick] (0,0)--(5,-1);
      \draw [fill] (2.66666,2) circle [radius=0.025] node[above left ]{$D$};
      \draw [fill] (0,0) circle [radius=0.025] node[above left]{$E$};
      \draw [fill] (4,-0.8) circle [radius=0.025] node[above]{$F$};
    \end{tikzpicture}
    \end{flushright}

\newpage
\subsubsection*{Angle measures using the Babylonian system of $360^\circ$ in a circle}
A full rotation is $360^\circ$ (a full ``turn'').\\[0.5cm]
A half turn (straight line) is $180^\circ$. \\[0.5cm]
$90^\circ$ is a quarter turn or a \emph{right} angle. \\[0.5cm]
\emph{Acute} angles measure less than $90^\circ$. \emph{Obtuse} angles measure more than $90^\circ$. \\[0.5cm]
\emph{Adjacent} angles (``next to'' each other) share a common ray and are external to each other. \vspace{0.2cm}

\item Write down the name of the \emph{three} angles shown in the diagram below and their angle measures, using your protractor. \vspace{1cm}
    \begin{center}
    \begin{tikzpicture}[scale=2]
      \draw [->, thick] (0,0)--(4,3);
      \draw [->, thick] (0,0)--(5,-.5);
      \draw [->, thick] (0,0)--(-1.2,3);
      \draw [fill] (-1,2.5) circle [radius=0.03] node[left ]{$B$};
      \draw [fill] (2.66666,2) circle [radius=0.03] node[above left ]{$C$};
      \draw [fill] (0,0) circle [radius=0.03] node[left]{$A$};
      \draw [fill] (4,-0.4) circle [radius=0.03] node[above]{$D$};
    \end{tikzpicture}
    \end{center}
    \begin{enumerate}
      \item  \rule{4cm}{0.15mm} \bigskip
      \item  \rule{4cm}{0.15mm} \bigskip
      \item  \rule{4cm}{0.15mm} \bigskip
      \item What do you notice about the angle measures?
    \end{enumerate}\vspace{1cm}

\item In your notebook, draw an angle that measures $55^\circ$

\newpage    
\begin{center}
  \begin{tikzpicture}
    \draw [<->, thick] (25:6)--(0,0)--(6,0);
    \draw [->, thick] (0,0)--(75:4);
    \draw [fill] (0,0) circle [radius=0.05] node[below]{$A$};
    %\draw [fill] (7,0) circle [radius=0.05] node[below]{$N$};
  \end{tikzpicture}
  \end{center}

\end{enumerate}
\end{document}