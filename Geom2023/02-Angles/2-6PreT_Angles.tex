\documentclass[12pt, twoside]{article}
\usepackage[letterpaper, margin=1in, headsep=0.2in]{geometry}
\setlength{\headheight}{0.6in}
%\usepackage[english]{babel}
\usepackage[utf8]{inputenc}
\usepackage{microtype}
\usepackage{amsmath}
\usepackage{amssymb}
%\usepackage{amsfonts}
\usepackage{siunitx} %units in math. eg 20\milli\meter
\usepackage{yhmath} % for arcs, overparenth command
\usepackage{tikz} %graphics
\usetikzlibrary{quotes, angles}
\usepackage{graphicx} %consider setting \graphicspath{{images/}}
\usepackage{parskip} %no paragraph indent
\usepackage{enumitem}
\usepackage{multicol}
\usepackage{venndiagram}

\usepackage{fancyhdr}
\pagestyle{fancy}
\fancyhf{}
\renewcommand{\headrulewidth}{0pt} % disable the underline of the header
\raggedbottom
\hfuzz=2mm %suppresses overfull box warnings

\usepackage{hyperref}

\fancyhead[LE]{\thepage}
\fancyhead[RO]{\thepage \\ Name: \hspace{4cm} \,\\}
\fancyhead[LO]{BECA / Dr. Huson / Geometry\\*  Unit 2: Angles\\* 6 October 2022}

\begin{document}

\subsubsection*{2.6 PreTest: Angle measures}
\begin{enumerate}
\item Given the situation in the diagram, answer each question. Circle True or False. 
  \vspace{0.25cm}
      \begin{multicols}{2}
        \begin{enumerate}
          \item T or F: $\overrightarrow{PR}$ and $\overrightarrow{PU}$ are opposite rays.\bigskip
          \item T or F: $\angle TPR$ is an obtuse angle.\bigskip
          \item T or F: $\angle RPS$ and $\angle TPU$ are \\adjacent angles. \bigskip
        \end{enumerate}
      \begin{tikzpicture}[scale=1]
        \draw[->, thick] (0,0)--(50:4);
        \draw[<->, thick] (-3,0)--(3,0);
        \draw[->, thick] (0,0)--(110:3);
        \draw[fill] (110:2.5) circle [radius=0.05] node[left ]{$S$};
        \draw[fill] (50:3) circle [radius=0.05] node[above left ]{$T$};
        \draw[fill] (0,0) circle [radius=0.05] node[below]{$P$};
        \draw[fill] (2,0) circle [radius=0.05] node[above]{$U$};
        \draw[fill] (-2,0) circle [radius=0.05] node[above]{$R$};
      \end{tikzpicture}
      \end{multicols}

\item As shown below, two lines intersect making four angles: $\angle 1$, $\angle 2$, $\angle 3$, and $\angle 4$.
  \begin{center}
  \begin{tikzpicture}[scale=0.7, rotate=15]
    \draw[<->, thick] (0,-1.5)--(10,1.5);
    \draw[<->, thick] (2,3.5)--(7,-3.5);
    \node at (3,.4){1};
    \node at (6,-.6){3};
    \node at (5,1){2};
    \node at (4,-1){4};
  \end{tikzpicture}
  \end{center}
  \begin{enumerate}
    \item Given that m$\angle 1= 75^\circ$, find m$\angle 2=$ \rule{2.5cm}{0.15mm} \bigskip
    \item Find m$\angle 3=$ \rule{2.5cm}{0.15mm} \bigskip
    \item True or false, $\angle 1$ and $\angle 4$ are supplementary angles. \rule{3cm}{0.15mm}
  \end{enumerate}

\item 
  \begin{enumerate}
    \item Given, the diagram below. Name a right angle:  \rule{4cm}{0.15mm}  \bigskip
    \item Name the angle that is opposite to $\angle AOB$: \rule{4cm}{0.15mm}  \bigskip
    \item Name an angle that is supplementary to $\angle COB$: \rule{4cm}{0.15mm}
  \end{enumerate}
  \begin{center}
  \begin{tikzpicture}[scale=1.3, rotate=20]
    \draw[<->, thick] (-25:5)--(0,0)--(155:5);
    \draw[<->, thick] (-5,0)--(5,0);
    \draw[->, thick] (0,0)--(0,3);
    \draw (0,0)++(0.3,0)--++(0,0.3)--+(-0.3,0);
    %\draw[fill] (-1,2.5) circle [radius=0.05] node[left ]{$B$};
    \draw[fill] (155:3) circle [radius=0.05] node[below left]{$B$};
    \draw[fill] (-4,0) circle [radius=0.05] node[below]{$A$}; 
    \draw[fill] (0,0) circle [radius=0.05] node[below left]{$O$};
    \draw[fill] (0,2) circle [radius=0.05] node[left]{$C$};
    \draw[fill] (4,0) circle [radius=0.05] node[below]{$D$};
    \draw[fill] (-25:2) circle [radius=0.05] node[below]{$E$};
  \end{tikzpicture}
  \end{center}

\emph{For full credit on these three problems, start with an equation and check your solution.}
\item As shown below, two lines intersect making four angles: $\angle 1$, $\angle 2$, $\angle 3$, and $\angle 4$. Given that m$\angle 1= x+30$ and m$\angle 3=2x-10$, find m$\angle 1$.
  \begin{flushright}
    \begin{tikzpicture}[scale=1, rotate=0]
      \draw[<->, thick] (1,-1)--(8,1);
      \draw[<->, thick] (2,2.5)--(6,-2);
      \node at (2,.3){m$\angle 1= x+30$};
      \node at (6.5,-.3){m$\angle 3=2x-10$};
      \node at (5,1){2};
      \node at (4,-1){4};
      %\draw[fill] (0,0) circle [radius=0.05] node[below]{$P$};
      %\draw[fill] (6,0) circle [radius=0.05] node[below]{$R$};
      %\draw[fill] (3,0) circle [radius=0.05] node[below]{$Q$};
    \end{tikzpicture}
    \end{flushright}

\item Given m$\angle BAC = 5x-5$ and m$\angle DAC = x$, m$\angle BAD=115^\circ$. Find m$\angle BAC$.
  \begin{flushright}
  \begin{tikzpicture}[scale=1]
    \draw[<->, thick] (115:4)node[left]{$B$} 
    --(0,0)node[below]{$A$}
    --(10:5)node[below]{$D$};
    \draw[->, thick] (0,0)--(30:4)node[below right]{$C$};
    %\draw[fill] (0,0) circle [radius=0.05] node[below]{$A$};
    %\draw[fill] (5,0) circle [radius=0.05] node[below]{$B$};
  \end{tikzpicture}
  \end{flushright} \vspace{1cm}

\item An angle bisector is shown below, with $\overrightarrow{PR}$ bisecting $\angle QPS$. Given m$\angle QPR = 4x+2$ and m$\angle QPS = 10x-20$, find m$\angle QPS$.
    \begin{flushright}
    \begin{tikzpicture}[scale=0.6, rotate=30]
      \draw[<->, thick] (100:7)node[left]{$Q$} 
      --(0,0)node[below]{$P$}
      --(8,0)node[below]{$S$}--(9,0);
      \draw[->, thick] (0,0)--(50:7)node[below right]{$R$};
      %\draw[fill] (0,0) circle [radius=0.05] node[below]{$A$};
      %\draw[fill] (5,0) circle [radius=0.05] node[below]{$B$};
    \end{tikzpicture}
    \end{flushright}

\newpage
\subsubsection*{Do Not Solve! \\
Model the situation with an equation. Circle where it states what to find.}
\vspace{0.5cm}

\item Two lines intersect making four angles: $\angle 1$, $\angle 2$, $\angle 3$, and $\angle 4$. Given that m$\angle 1= 4x+30$ and m$\angle 2=8x-10$, find $x$.
\begin{flushright}
\begin{tikzpicture}[scale=0.5, rotate=-10]
\draw[<->, thick] (0,-1.5)--(10,1.5);
\draw[<->, thick] (2,2)--(7,-2);
\node at (3,.4){1};
\node at (6,-.6){3};
\node at (5,1){2};
\node at (4,-1){4};
\end{tikzpicture}
\end{flushright}

\item Given that m$\angle 2= 5x+30$ and m$\angle 4=7x-10$ as shown in the diagram, find m$\angle 2$.
\begin{flushright}
\begin{tikzpicture}[scale=0.5, rotate=-30]
\draw[<->, thick] (0,-1.5)--(10,1.5);
\draw[<->, thick] (2,2)--(7,-2);
\node at (3,.4){1};
\node at (6,-.6){3};
\node at (5,1){2};
\node at (4,-1){4};
\end{tikzpicture}
\end{flushright}

\item In the diagram below $\angle AOB = 30^\circ$ and $\angle COB = 5x+10$. Find $x$. \vspace{0.25cm}
\begin{flushright}
\begin{tikzpicture}[scale=0.7, rotate=20]
\draw[<->, thick] (-25:5)--(0,0)--(155:5);
\draw[<->, thick] (-5,0)--(5,0);
\draw[->, thick] (0,0)--(0,4);
\draw (0,0)++(0.3,0)--++(0,0.3)--+(-0.3,0);
%\draw[fill] (-1,2.5) circle [radius=0.05] node[left ]{$B$};
\draw[fill] (155:3) circle [radius=0.05] node[below left]{$B$};
\draw[fill] (-4,0) circle [radius=0.05] node[below]{$A$}; 
\draw[fill] (0,0) circle [radius=0.05] node[below]{$O$};
\draw[fill] (0,3) circle [radius=0.05] node[left]{$C$};
\draw[fill] (4,0) circle [radius=0.05] node[below]{$D$};
\draw[fill] (-25:2) circle [radius=0.05] node[below]{$E$};
\end{tikzpicture}
\end{flushright}

\item In the diagram below $\angle DOE = 60^\circ$ and $\angle DOB = 13x-10$. Find $x$. \vspace{0.25cm}
\begin{flushright}
\begin{tikzpicture}[scale=0.7, rotate=-20]
\draw[<->, thick] (-55:3)--(0,0)--(125:4);
\draw[<->, thick] (-5,0)--(5,0);
\draw[->, thick] (0,0)--(0,4);
\draw (0,0)++(0.3,0)--++(0,0.3)--+(-0.3,0);
%\draw[fill] (-1,2.5) circle [radius=0.05] node[left ]{$B$};
\draw[fill] (125:3) circle [radius=0.05] node[below left]{$B$};
\draw[fill] (-4,0) circle [radius=0.05] node[below]{$A$}; 
\draw[fill] (0,0) circle [radius=0.05] node[below left]{$O$};
\draw[fill] (0,3) circle [radius=0.05] node[left]{$C$};
\draw[fill] (4,0) circle [radius=0.05] node[below]{$D$};
\draw[fill] (-55:2) circle [radius=0.05] node[left]{$E$};
\end{tikzpicture}
\end{flushright}


\end{enumerate}
\end{document}