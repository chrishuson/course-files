\documentclass{beamer}
\usepackage{geometry}
\usepackage[english]{babel}
\usepackage[utf8]{inputenc}
\usepackage{amsmath}
\usepackage{amsfonts}
\usepackage{amssymb}
\usepackage{tikz}
\usepackage{graphicx}
\usepackage{venndiagram}

\setlength{\headheight}{26pt}%doesn't seem to fix warning

\usepackage{fancyhdr}
\pagestyle{fancy}
\fancyhf{}

\rhead{\small{7 February 2018}}
\lhead{\small{BECA / Dr. Huson / Mathematics}}

%\vspace{1cm}

\renewcommand{\headrulewidth}{0pt}


\title{Mathematics Class Slides}
\subtitle{Bronx Early College Academy}
\author{Chris Huson}
\date{February 2018}

\begin{document}

\frame{\titlepage}

%\section[Outline]{}
%\frame{\tableofcontents}

\section{12.1 Drui}
\frame
{
  \frametitle{GQ: How do we calculate the area between two curves?}
  \framesubtitle{CCSS: F.IF.B.6 Calculate \& interpret the rate of change of a function}

  \begin{block}{Do Now: Consider the function $f(x)=-x^2+2x+3$}
  \begin{enumerate}
      \item Factor $f$ and state its zeros.
      \item Restate $f$ in vertex form. Write down the vertex as an ordered pair.
      \item Differentiate $f$. Show that the zero of $f^\prime(x)$ is the vertex of $f$.
      \item If $f(x)$ represents the height of a diver over the domain $0 \leq x \leq 3$, interpret $f(0)$ and $f^\prime(0)$
      \item What is the size of the area bounded by $f$, $x=0$, and $y=0$?
  \end{enumerate}
  \end{block}
  Lesson: The area between two functions p. 313\\%*[5pt]
  %Task: Practice Examples 11, 12 p. 310-1\\%*[5pt]
  Assessment: Example \#13 p. 314 \\%*[5pt]
  Homework: Exercises 9K p. 316  
}

\section{11.1 Drui}
\frame
{
  \frametitle{GQ: How do we organize data using sample space diagrams?}
  \framesubtitle{CCSS: HSS.CP.B.6 Probabilities}

  \begin{block}{Do Now}
  \begin{enumerate}
      \item Exercise 3C \#1 page 74.
      \item Problem 3C \#8 p. 75. Start by making a Venn diagram.
      \item Exercise 3D \#2 page 77.
  \end{enumerate}
  Review homework
  \end{block}
  Lesson: Sample space tables, Example 8 (p. 78)\\*[5pt]
  Task: Exercises 3E page 79.\\*[5pt]
  Assessment: When rolling two dice, why aren't all the possible totals equally likely?\\*[5pt]
  Homework: Pretest review handout  
}

\begin{frame}{$\mathrm P(A \cup B) = \mathrm P(A) + \mathrm P(B) - \mathrm P(A \cap B)$}
    \framesubtitle{The addition rule}
    \begin{venndiagram2sets}[tikzoptions={scale=2}]
    \end{venndiagram2sets}
\end{frame}

\begin{frame}{Distributions}
    \framesubtitle{Tables and charts used to summarize a problem situation}
    A \alert{frequency distribution} displays the number of times each event in the sample space occurs, either in tabular or graphical form.\\*[10pt]
    A \alert{probability distribution} shows the same data, normalizing the totals to one.
\end{frame}


\begin{frame}{Technical writing}
    \framesubtitle{Write a short paper answering the query: \\* "How many subsets can be picked from a group of four students?"}
    \begin{enumerate}
        \item Logical, step-by-step explanation, using an example
        \item Precise terminology, succinct: combination, permutation, order (matters), event, sample space, set, subset, with /without replacement, factorial
        \item Notation: algebra symbols, tables, trees, grids
        \item Summary, big-picture, conceptual idea
        \item Audience: student peers
    \end{enumerate}
\end{frame}





\begin{frame}{Combinatorics formulas}
    \alert{Combinations}, when order doesn't matter
	$$_nC_r = \frac{n!}{(n-r)! r!} \qquad \text{''n pick r"}$$
    \alert{Permutations}, when order does matter
	$$_nP_r = \frac{n!}{(n-r)!} $$
\end{frame}

\begin{frame}{Definition of theoretical probability}
    The \alert{theoretical probability} of an event $A$ is $\displaystyle \mathrm P(A) = \frac{n(A)}{n(U)}$\\*[10pt]
    \quad where $n(A)$ is the number of ways an event can occur\\*[5pt]
    \quad and $n(U)$ is the total number of possible outcomes (p. 65)\\*[10pt]
    Theoretically, in $n$ trials, one would expect the event to occur $n \times \mathrm P(A)$ times\\*[10pt]
    Probabilities are between 0 and 1, inclusive. $0 \leq \mathrm P(X) \leq 1$
\end{frame}

\begin{frame}{Empirical (experimental) probability}
    The \alert{relative frequency} of an event can be used as an estimate of its probability. $$\displaystyle \mathrm P(A) = \frac{\text{number of occurrences of event } A}{\text{total number of trials}}$$
    The larger the number of trials the more reliable the estimate of probability.
\end{frame}

\begin{frame}{Independence and mutual exclusivity}
    Two events are \alert{independent} if the occurrence of one does not affect the probability of the other. $$\displaystyle \mathrm P(\text{both }A \text{ and }B \text{ occur}) = \mathrm P(A) \times \mathrm P(B)$$
    Two events are \alert{mutually exclusive} if they never occur together. 
    $$\displaystyle \mathrm P(\text{both }A \text{ and }B \text{ occur}) = 0 \qquad \text{and}$$
    $$\mathrm P(\text{either }A \text{ or }B \text{ occur}) = \mathrm P(A) + \mathrm P(B)$$
\end{frame}

\begin{frame}{Venn diagrams}
    \framesubtitle{For organizing compound events}
    When two events can occur, and perhaps both, or neither.
    \begin{venndiagram2sets}[tikzoptions={scale=1.5}]
    \end{venndiagram2sets}
\end{frame}

\begin{frame}{The union of sets: $A \cup B$}
    That $A$ happens, or $B$ happens, or both
    \begin{venndiagram2sets}[tikzoptions={scale=1.5}]
    \fillA
    \fillB
    \end{venndiagram2sets}
\end{frame}

\begin{frame}{The intersection of sets: $A \cap B$}
    That both $A$ and $B$ happen
    \begin{venndiagram2sets}[tikzoptions={scale=1.5}]
    \fillACapB
    \end{venndiagram2sets}
\end{frame}

\begin{frame}{The addition rule}
    \framesubtitle{That $A$ or $B$ or both occur}
    
    When two events can occur, and perhaps both
    
    \begin{venndiagram2sets}%[labelA={primes}, labelB={evens}, shade =lightgray]%
    %\fillA
    %\fillB
    %\fillACapB
    \end{venndiagram2sets}

    $$P(\text{either }A \text{ or }B \text{ occur}) = P(A) + P(B) - P(\text{both }A \text{ and }B \text{ occur})$$
\end{frame}

\begin{frame}{Vocabulary for probability \& statistics}
    event, experiment, random\\*[5pt]
    probability, P(A), values [0,1]\\*[5pt]
    theoretical, empirical, subjective\\*[5pt]
    sample space, U; frequency, trials\\*[5pt]
    n(U) = number of possibilities\\*[5pt]
    P(A) = n(A)/n(U); expected = n * P
\end{frame}

\end{document}
independent events: P(A&B)=P(A)*P(B) 
dependent events, mutually exclusive
P(A or B)= P(A) + P(B) - P(A&B)
P(A) = n(A)/n(U)