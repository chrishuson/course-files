\documentclass[12pt, twoside]{article}
\usepackage[letterpaper, margin=1in, headsep=0.5in]{geometry}
\usepackage[english]{babel}
\usepackage[utf8]{inputenc}
\usepackage{amsmath}
\usepackage{amsfonts}
\usepackage{amssymb}
\usepackage{tikz}
\usetikzlibrary{quotes, angles}
\usepackage{graphicx}
%\usepackage{pgfplots}
%\pgfplotsset{width=10cm,compat=1.9}
%\usepgfplotslibrary{statistics}
%\usepackage{pgfplotstable}
%\usepackage{tkz-fct}
%\usepackage{venndiagram}

\usepackage{fancyhdr}
\pagestyle{fancy}
\fancyhf{}
\renewcommand{\headrulewidth}{0pt} % disable the underline of the header

\fancyhead[RE]{\thepage}
\fancyhead[RO]{\thepage \\ Name: \hspace{3cm}}
\fancyhead[L]{BECA / Dr. Huson / Geometry 10th Grade\\* Unit 3: Volume and angle bisectors \\ 
7 October 2019}

\begin{document}
  \subsubsection*{3.3 Classwork: Angle bisectors, modeling segments}
  \begin{enumerate}

  \item An angle bisector is shown below, with $\overrightarrow{AC}$ bisecting $\angle BAD$. Given $m\angle BAC = 7x+5$ and $m\angle DAC = 9x-5$, find $m\angle BAD$. (Show check)
    \begin{flushright}
    \begin{tikzpicture}[scale=0.7]
      \draw [<->, thick] (80:7)node[left]{$B$} 
      --(0,0)node[below]{$A$}
      --(6,0)node[below]{$D$}--(7,0);
      \draw [->, thick] (0,0)--(40:7)node[below right]{$C$};
      %\draw [fill] (0,0) circle [radius=0.05] node[below]{$A$};
      %\draw [fill] (5,0) circle [radius=0.05] node[below]{$B$};
    \end{tikzpicture}
    \end{flushright} \vspace{3cm}

    \item An angle bisector is shown below, with $\overrightarrow{PR}$ bisecting $\angle QPS$. Given $m\angle QPR = 4x+2$ and $m\angle QPS = 10x-20$, find $m\angle QPS$.
    \begin{flushright}
    \begin{tikzpicture}[scale=0.6, rotate=30]
      \draw [<->, thick] (100:7)node[left]{$Q$} 
      --(0,0)node[below]{$P$}
      --(8,0)node[below]{$S$}--(9,0);
      \draw [->, thick] (0,0)--(50:7)node[below right]{$R$};
      %\draw [fill] (0,0) circle [radius=0.05] node[below]{$A$};
      %\draw [fill] (5,0) circle [radius=0.05] node[below]{$B$};
    \end{tikzpicture}
    \end{flushright}

\newpage   
  \subsubsection*{Do Not Solve! Make a drawing on the right, an equation to the left, and circle where it states what to find.}
  \vspace{0.5cm}

\item The point $Q$ is the midpoint of $\overline{PR}$, $PQ=11$, and $QR=2x+1$. Find ${x}$.
\vspace{4cm}

\item Given $\overline{PQR}$, with $PQ=3x-7$, $QR=x+3$, and $PR=12$. Find ${x}$.
\vspace{4cm}

\item Given that $Q$ bisects $\overline{PR}$. $PQ=2x-5$, $PR=42$. Find ${x}$.
\vspace{4cm}

\item The points $P$, $Q$, and $R$ are collinear, with $PQ=x+4$ and $PR=27$. $\overline{QR}$ is twice the length of $\overline{PQ}$. Find ${x}$.




\end{enumerate}
\end{document}
