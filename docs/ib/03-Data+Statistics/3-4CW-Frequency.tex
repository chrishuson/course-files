\documentclass[12pt, twoside]{article}
% \documentclass[12pt, twoside]{article}
\usepackage[letterpaper, margin=1in, headsep=0.2in]{geometry}
\setlength{\headheight}{0.6in}
%\usepackage[english]{babel}
\usepackage[utf8]{inputenc}
\usepackage{microtype}
\usepackage{amsmath}
\usepackage{amssymb}
%\usepackage{amsfonts}
\usepackage[nomessages]{fp} %\FPeval{\var-name}{2*sin(pi/6)}
\usepackage{siunitx} %units in math. eg 20\milli\meter
\usepackage{yhmath} % for arcs, overparenth command
\usepackage{tikz} %graphics
\usetikzlibrary{quotes, angles, arrows, arrows.meta}
\usepackage{graphicx} %consider setting \graphicspath{{images/}}
\usepackage{parskip} %no paragraph indent
\usepackage{enumitem}
\usepackage{multicol}
\usepackage{venndiagram}

\usepackage{fancyhdr}
\pagestyle{fancy}
\fancyhf{}
\renewcommand{\headrulewidth}{0pt} % disable the underline of the header
\raggedbottom
\hfuzz=2mm %suppresses overfull box warnings

\usepackage{hyperref}
\usepackage{float}


\title{IB Math AA Test: Sequences, Interest, Regression, Functions, Logs}
\author{Chris Huson}
\date{November 2025}

\fancyhead[LE]{\thepage}
\fancyhead[RO]{\thepage \\ First \& last name: \hspace{2.25cm} \,\\ Grade: \hspace{2.25cm} \,}
%\fancyhead[RO]{First \& last name: \hspace{2.25cm} \,\\ \,}
\fancyhead[LO]{La Scuola d'Italia / Huson / IB Math: Sequences \\* 14 November 2025}

\begin{document}

\subsubsection*{3.4 Classwork: Frequency tables}
\begin{enumerate}[itemsep=0.5cm]

\item A class of students was asked about their preferred type of music.  
The results are summarized in the table.

\begin{center}
\begin{tabular}{|c|c|}
\hline
\textbf{Type of music} & \textbf{Frequency} \\
\hline
Pop        & 7 \\
Rock       & 5 \\
Hip--hop   & 9 \\
Classical  & 3 \\
Other      & 2 \\
\hline
\end{tabular}
\end{center}

\begin{enumerate}[itemsep=0.75cm, label=(\alph*)]
  \item How many students are in the survey altogether?
  \item Which type of music is the mode of this data set?
  \item What fraction of the students chose rock? (Do not simplify.)
  \item Is \emph{type of music} a categorical variable or a quantitative variable? Explain in one short sentence.
\end{enumerate} \vspace{0.5cm}

\item The table shows the number of books each student in a reading club finished last month.

\begin{center}
\begin{tabular}{|c|c|}
\hline
\textbf{Number of books} & \textbf{Frequency} \\
\hline
0 & 2 \\
1 & 4 \\
2 & 7 \\
3 & 5 \\
4 & 2 \\
\hline
\end{tabular}
\end{center}

\begin{enumerate}[itemsep=0.75cm, label=(\alph*)]
  \item How many students read at least 2 books?
  \item How many students are in the reading club?
  \item What is the most common number of books read?
  \item Is \emph{number of books} discrete or continuous? Briefly justify.
\end{enumerate}

\newpage
\item A teacher recorded the test scores (out of 100) for one class and grouped them into intervals.

\begin{center}
\begin{tabular}{|c|c|}
\hline
\textbf{Score interval} & \textbf{Frequency} \\
\hline
50--59 & 1 \\
60--69 & 3 \\
70--79 & 7 \\
80--89 & 9 \\
90--99 & 4 \\
\hline
\end{tabular}
\end{center}

\begin{enumerate}[itemsep=0.75cm, label=(\alph*)]
  \item How many students scored at least 80?
  \item Which interval is the modal class?
  \item Looking at the frequencies, is the distribution roughly symmetric, skewed left, or skewed right? Explain your choice in one or two sentences.
\end{enumerate} \vspace{0.5cm}

\vspace{0.5cm}
\item Forty IB high school students range in age from 15 to 18 years old. The following table shows the frequencies of each age.
        \begin{center}
            \begin{tabular}{|l|r|r|r|r|}
                \hline
                Age (years) & 15 & 16 & 17 & 18\\ 
                \hline 
                Frequency & 5 & $k$ & 15 & 7\\ 
                \hline 
                \end{tabular}      
        \end{center}
        \begin{enumerate}[itemsep=0.8cm]
            \item Calculate the value of $k$. \hfill [1 mark]
            \item Write down the mode. \hfill [1 mark]
            \item Find the value of the range. \hfill [1 marks]
            \item Find the median. \hfill [1 marks]
            \item Find the mean. \hfill [2 marks]
            \item Find the standard deviation. \hfill [2 marks]
        \end{enumerate} \vspace{0.5cm}

\newpage
\item A runner records her pace in terms of distance run ($d$) in miles over time ($t$) in minutes during a 4.5 mile run. She models her pace with a linear regression equation $d=at+b$.
        \begin{center}
        \begin{tabular}{|l|c|c|c|c|c|}
            \hline
            minutes ($t$) & 0 & 8 & 15 & 22 & 30 \\ 
            \hline 
            miles ($d$) & 0 & 1.8 & 2.7 & 3.7 & 4.5 \\ 
            \hline 
            \end{tabular}
        \end{center}
        \begin{enumerate}[itemsep=3.5cm]
            \item Find the values of $a$, $b$, and the correlation $r$. \hfill [3 marks]
            \item Explain what the value of $a$ represents in the context of the situation. \hfill [2 marks]
        \end{enumerate}
 \vspace{2.5cm}

\section*{5.\ Constructing a frequency table from raw data}

\item The following data show the number of goals scored by a soccer team in 15 matches:

\[
0,\ 1,\ 3,\ 2,\ 1,\ 4,\ 2,\ 2,\ 0,\ 1,\ 3,\ 5,\ 2,\ 1,\ 2
\]

\begin{enumerate}[label=(\alph*)]
  \item Complete a frequency table with the possible number of goals from 0 to 5.

\begin{center}
\begin{tabular}{|c|c|}
\hline
\textbf{Goals} & \textbf{Frequency} \\
\hline
0 & \\
1 & \\
2 & \\
3 & \\
4 & \\
5 & \\
\hline
\end{tabular}
\end{center}

  \item From your table, what is the mode of this data set?
  \item Explain in one or two sentences why a frequency table is helpful for describing this data.
\end{enumerate}

\newpage
\item A small survey asked students how many hours they sleep on a school night.  
The results are summarized below.

\begin{center}
\begin{tabular}{|c|c|}
\hline
\textbf{Hours of sleep} & \textbf{Frequency} \\
\hline
4 & 1 \\
5 & 3 \\
6 & 5 \\
7 & 4 \\
8 & 2 \\
\hline
\end{tabular}
\end{center}

\begin{enumerate}[itemsep=1.5cm, label=(\alph*)]
  \item How many students were surveyed?
  \item What is the mode of the data?
  \item What is the median number of hours of sleep? Show how you locate it using the frequencies.
  \item Write an expression for the mean number of hours of sleep, using the values and their frequencies. 
  \item Compare your estimation of the mean to the value given by the calculator.
\end{enumerate}
\end{enumerate}

\end{document}