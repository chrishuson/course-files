\documentclass[12pt, twoside]{article}
% \documentclass[12pt, twoside]{article}
\usepackage[letterpaper, margin=1in, headsep=0.2in]{geometry}
\setlength{\headheight}{0.6in}
%\usepackage[english]{babel}
\usepackage[utf8]{inputenc}
\usepackage{microtype}
\usepackage{amsmath}
\usepackage{amssymb}
%\usepackage{amsfonts}
\usepackage[nomessages]{fp} %\FPeval{\var-name}{2*sin(pi/6)}
\usepackage{siunitx} %units in math. eg 20\milli\meter
\usepackage{yhmath} % for arcs, overparenth command
\usepackage{tikz} %graphics
\usetikzlibrary{quotes, angles, arrows, arrows.meta}
\usepackage{graphicx} %consider setting \graphicspath{{images/}}
\usepackage{parskip} %no paragraph indent
\usepackage{enumitem}
\usepackage{multicol}
\usepackage{venndiagram}

\usepackage{fancyhdr}
\pagestyle{fancy}
\fancyhf{}
\renewcommand{\headrulewidth}{0pt} % disable the underline of the header
\raggedbottom
\hfuzz=2mm %suppresses overfull box warnings

\usepackage{hyperref}
\usepackage{float}


\title{IB Math AA/AI Mixed Practice Draft\\(Sequences, Interest, Regression, Functions, Logs)}
\author{Chris Huson}
\date{November 2025}

\fancyhead[LE]{\thepage}
\fancyhead[RO]{\thepage \\ First \& last name: \hspace{2.25cm} \,\\ Grade: \hspace{2.25cm} \,}
%\fancyhead[RO]{First \& last name: \hspace{2.25cm} \,\\ \,}
\fancyhead[LO]{La Scuola d'Italia / Huson / IB Math: Sequences \\* 5 November 2025}

\begin{document}

\subsubsection*{2.4+5 Review problem sets in LaTeX (Test-Draft3)}
\begin{enumerate}[itemsep=0.1cm]

% --------------------------------------------------
% 1. Music academy regression (table)
% --------------------------------------------------
\item Eight piano students reported their average weekly practice time and their diploma score (out of 150). The data are shown below. \hfill [5 marks]

\begin{center}
\begin{tabular}{c|cccccccc}
\textbf{Practice time $h$ (h)} & 28 & 13 & 45 & 33 & 17 & 29 & 39 & 36 \\
\hline
\textbf{Diploma score $D$}     & 115 & 82 & 120 & 116 & 79 & 101 & 110 & 121
\end{tabular}
\end{center}

The relationship between $h$ and $D$ is modelled by a regression line $D = ah + b$.

\begin{enumerate}[itemsep=0.25cm]
  \item Find the Pearson product–moment correlation coefficient $r$ for these data.
  \item Write down the values of $a$ and $b$ from your GDC regression output.
  \item One of the students says she would have practised 5 more hours per week. Using the model, estimate how her score might have changed.
\end{enumerate}
\vspace{0.25cm}

% --------------------------------------------------
% 2. 5000 m race: age vs time
% --------------------------------------------------
\item A runner collects data to see whether the time to run $5000$ m depends on the runner’s age. For eight male runners he records: \hfill [6 marks]

\begin{center}
\begin{tabular}{c|cccccccc}
\textbf{Age $x$ (years)}     & 18 & 24 & 28 & 36 & 40 & 46 & 52 & 62 \\
\hline
\textbf{Time $t$ (minutes)}  & 29.4 & 29.2 & 31.1 & 33.6 & 32.2 & 33.1 & 35.2 & 40.4
\end{tabular}
\end{center}

(There is also a scatter diagram showing time increasing with age.)

\begin{enumerate}[itemsep=0.25cm]
  \item Find the Pearson correlation coefficient $r$.
  \item A sports science book gives the following guidance:

  \begin{quote}
    $0 \le |r| < 0.4$: weak,\quad
    $0.4 \le |r| < 0.8$: moderate,\quad
    $0.8 \le |r| \le 1$: strong.
  \end{quote}

  Comment on the strength of the correlation for this data.
  \item Write down the regression line of $t$ on $x$ in the form $t = ax + b$.
  \item Estimate the time for a $57$-year-old runner using your regression line.
\end{enumerate}
\vspace{0.25cm}

% --------------------------------------------------
% 3. Mould growth (exponential)
% --------------------------------------------------
\item In an experiment the area of a mould patch is modelled by \hfill [4 marks]
\[
P(t) = A e^{kt},
\]
where $P$ is the area in mm$^2$ and $t$ is the time in days. At $t=0$ the area is $112$ mm$^2$, and after $5$ days the area is $360$ mm$^2$.

\begin{enumerate}[itemsep=0.25cm]
  \item Write down the value of $A$.
  \item Find the value of $k$.
\end{enumerate}
\vspace{0.25cm}

% --------------------------------------------------
% 4. Kite on coordinate axes
% --------------------------------------------------
\item Dilara is designing a kite $ABCD$ on a coordinate plane (1 unit = 10 cm). The points are  \hfill [6 marks]
\[
A(2,0), \quad B(0,4), \quad C(4,6),
\]
and point $D$ lies on the $x$–axis. Segment $AC$ is perpendicular to segment $BD$.

\begin{tikzpicture}[scale=0.6,>=stealth]
  % axes
  \draw[->] (0,0) -- (0,7) node[left] {$y$};
  \draw[->] (0,0) -- (11,0) node[below] {$x$};

  % coordinates (chosen to match the problem)
  \coordinate (B) at (0,4);
  \coordinate (A) at (2,0);
  \coordinate (C) at (4,6);
  \coordinate (D) at (10,0);

  % sides of the figure
  \draw (B) -- (C) -- (D) -- (A) -- cycle;   % outer shape BAD + CD
  \draw (B) -- (A);                           % BA
  \draw (B) -- (D);                           % BD
  \draw (A) -- (C);                           % AC

  % intersection of AC and BD for right-angle mark
%  \path [name path=AC] (A) -- (C);
%  \path [name path=BD] (B) -- (D);
%  \path [name intersections={of=AC and BD, by=E}];

  % right angle at intersection
%  \draw ($(E)!0.15!(B)$) -- ($(E)!0.15!(A)$) -- ($(E)!0.15!(D)$);

  % points
  \fill (A) circle (2pt) node[below] {$A$};
  \fill (B) circle (2pt) node[left] {$B$};
  \fill (C) circle (2pt) node[above] {$C$};
  \fill (D) circle (2pt) node[below] {$D$};

  % note
  \node[anchor=west] at (7,6.5) {\small diagram not to scale};
\end{tikzpicture}

\begin{enumerate}[itemsep=0.25cm]
  \item Find the gradient of the line through $A$ and $C$.
  \item Hence write down the gradient of the line through $B$ and $D$.
  \item Find the equation of line $BD$ in the form $ax + by + d = 0$, where $a,b,d$ are integers.
  \item Write down the $x$–coordinate of $D$.
\end{enumerate}
\vspace{0.25cm}

% --------------------------------------------------
% 5. University applications (geometric + linear)
% --------------------------------------------------
\item A new university records the number of applications in its first two years: \hfill [16 marks]

\begin{center}
\begin{tabular}{c|cc}
Year $n$ & 1 & 2 \\
\hline
Applications $u_n$ & 12\,300 & 12\,669
\end{tabular}
\end{center}

\begin{enumerate}[itemsep=0.25cm]
  \item Calculate the percentage increase in applications from year 1 to year 2.
  \item Assume that the applications follow a geometric sequence $(u_n)$.
    \begin{enumerate}[itemsep=0.25cm]
      \item Write down the common ratio.
      \item Find a formula for $u_n$.
      \item Find the number of applications expected in year $11$, giving your answer to the nearest integer.
    \end{enumerate}
  \item In year 1 there are $10\,380$ places available. The number of places increases by $600$ each year. Let $(v_n)$ be the number of places in year $n$. Write down a formula for $v_n$.
  \item For the first $10$ years every place is filled. Each student who takes a place pays an \$80 acceptance fee. Find the total amount of acceptance fees received in the first $10$ years.
  \item Let $n = k$ be the first year in which the number of places available exceeds the number of applications. Find $k$.
  \item State whether for all $n > k$ the university will have places for all applicants. Justify your answer briefly.
\end{enumerate}
\vspace{0.25cm}

\item Give all numerical answers correct to two decimal places. \hfill [14 marks]

    A person places \$30\,000 in an account on 1 January 2005. The account pays \emph{simple} interest at a fixed annual rate. On 1 January 2007 the balance is \$31\,650.

    \begin{enumerate}
    \item Find the annual simple interest rate.
    \vspace{0.25cm}

    \item A second person also invests \$30\,000 on 1 January 2005, but in an account that pays a nominal annual rate of $2.5\%$ compounded annually. Find the balance after two years.
    \vspace{0.25cm}

    \item Determine the number of complete years from 1 January 2005 until the compound-interest account first has a greater balance than the simple-interest account.
    \vspace{0.25cm}

    \item On 1 January 2007 the first person reinvests $80\%$ of the money from the simple-interest account into a new account paying $3\%$ per year, compounded quarterly.
        \begin{enumerate}
        \item Calculate the amount reinvested on 1 January 2007.
        \vspace{0.25cm}
        \item Find the number of complete years it will take for the balance in this new account to exceed \$30\,000.
        \vspace{0.25cm}
        \end{enumerate}
    \end{enumerate}

    \vspace{0.5cm}

\item In a geometric sequence, the fourth term is 8 times the first term. The sum of the
first 10 terms is 2557.5. Find the 10th term of this sequence. \hfill [6 marks]

\item A population of rare birds, $P_t$, can be modelled by the equation \hfill [8 marks]
    $$P_t = P_0 e^{kt},$$
    where $P_0$ is the initial population and $t$ is measured in decades. After one decade it is estimated that
    $$\frac{P_1}{P_0} = 0.9.$$

    \begin{enumerate}[itemsep=0.25cm]
    \item
    \begin{enumerate}[itemsep=0.25cm]
        \item Find the value of $k$.
        \item Interpret the meaning of the value of $k$ in the context of the population.
    \end{enumerate}

    \item Find the least number of whole years for which
    $$\frac{P_t}{P_0} < 0.75.$$
    \end{enumerate}
    \vspace{0.25cm}

\item The price of a used car depends partly on the distance it has travelled. The following table shows the distance and the price for seven cars on 1 January 2010. \hfill [15 marks]

    \begin{center}
    \begin{tabular}{c|ccccccc}
        \textbf{Distance, $x$ km} & 11500 & 7500 & 13600 & 10800 & 9500 & 12200 & 10400 \\ \hline
        \textbf{Price, $y$ dollars} & 15000 & 21500 & 12000 & 16000 & 19000 & 14500 & 17000
        \end{tabular}
    \end{center}
    \vspace{0.25cm}

    The relationship between $x$ and $y$ can be modelled by the regression equation $y = ax + b$.

    \begin{enumerate}[itemsep=0.25cm]
    \item 
    \begin{enumerate}[itemsep=0.25cm]
        \item Find the correlation coefficient.
        \item Write down the value of $a$ and of $b$.
    \end{enumerate}

    \item On 1 January 2010, Lina buys a car which has travelled $11000$ km. Use the regression equation to estimate the price of Lina's car, giving your answer to the nearest \$100.

    \item The price of a car decreases by $5\%$ each year. Calculate the price of Lina's car after $6$ years.

    \item Lina will sell her car when its price reaches \$10\,000. Find the year when Lina sells her car.
    \end{enumerate}
    \vspace{0.25cm}

% =========================================================
% Problem 2 -- Sequences (Geometric + Arithmetic)
% (in PDF this was numbered 3 with Part A and Part B)
% =========================================================
\item Sequences: geometric and arithmetic \hfill [18 marks]

\textbf{Part A}

    A geometric sequence has first term $1024$ and fourth term $128$.

    \begin{enumerate}
    \item Show that the common ratio is $\dfrac{1}{2}$.
    \vspace{0.25cm}

    \item Find the eleventh term of the sequence.
    \vspace{0.25cm}

    \item Find the sum of the first eight terms.
    \vspace{0.25cm}

    \item Find the smallest number of terms for which the sum of the sequence first exceeds $2047.968$.
    \vspace{0.25cm}
    \end{enumerate}

\textbf{Part B}

    Consider the arithmetic sequence
    \[
    1,\ 4,\ 7,\ 10,\ 13,\ \dots
    \]

    \begin{enumerate}
    \item Find the eleventh term.
    \vspace{0.25cm}

    \item The sum of the first $n$ terms of this sequence is given by
    \[
        S_n = \frac{n(3n-1)}{2}.
    \]
    \begin{enumerate}
        \item Find the sum of the first $100$ terms.
        \vspace{0.25cm}
        \item The sum of the first $n$ terms is $477$.
        \begin{enumerate}
            \item Show that $3n^2 - n - 954 = 0$.
            \vspace{0.25cm}
            \item Hence find the value of $n$. You may use your GDC.
            \vspace{0.25cm}
        \end{enumerate}
    \end{enumerate}
    \end{enumerate}

    \vspace{1cm}

\item The mass $M$ of a decaying substance is measured at one–minute intervals. The points $(t, \ln M)$ are plotted for $0 \le t \le 10$, where $t$ is in minutes. The line of best fit is drawn. This is shown in the diagram. \hfill [6 marks]

\begin{tikzpicture}[xscale=0.7,yscale=1.2]
  % axes
  \draw[->] (0,3.4) -- (11,3.4) node[below] {$t$};
  \draw[->] (0,3.3) -- (0,4.9) node[left] {$\ln M$};

  % y ticks
  \foreach \y/\label in {3.5/{3.5},4/{4},4.5/{4.5}}
    \draw (0,\y) -- (-0.1,\y) node[left] {\label};

  % x ticks
  \foreach \x in {2,4,6,8,10}
    \draw (\x,3.4) -- (\x,3.3) node[below] {\x};

  % line of best fit: ln M = 4.67 - 0.12 t
  \draw[thick] (0,4.67) -- (10,3.47);

  % sample plotted points near the line
  \foreach \t in {0.5,2,3.5,5,6.5,8,9.5}{
    \pgfmathsetmacro{\y}{4.67 - 0.12*\t}
    \fill (\t,\y) circle (1.2pt);
  }
\end{tikzpicture}

The correlation coefficient for this linear model is $r = -0.998$.

\begin{enumerate}[itemsep=0.25cm]
  \item State two words that describe the linear correlation between $\ln M$ and $t$.
  \item The equation of the line of best fit is
  \[
    \ln M = -0.12t + 4.67.
  \]
  Given that $M = a \times b^{t}$, find the value of $b$.
\end{enumerate}
\vspace{0.25cm}

\end{enumerate}
\end{document}

