\documentclass[12pt, twoside]{article}
% \documentclass[12pt, twoside]{article}
\usepackage[letterpaper, margin=1in, headsep=0.2in]{geometry}
\setlength{\headheight}{0.6in}
%\usepackage[english]{babel}
\usepackage[utf8]{inputenc}
\usepackage{microtype}
\usepackage{amsmath}
\usepackage{amssymb}
%\usepackage{amsfonts}
\usepackage[nomessages]{fp} %\FPeval{\var-name}{2*sin(pi/6)}
\usepackage{siunitx} %units in math. eg 20\milli\meter
\usepackage{yhmath} % for arcs, overparenth command
\usepackage{tikz} %graphics
\usetikzlibrary{quotes, angles, arrows, arrows.meta}
\usepackage{graphicx} %consider setting \graphicspath{{images/}}
\usepackage{parskip} %no paragraph indent
\usepackage{enumitem}
\usepackage{multicol}
\usepackage{venndiagram}

\usepackage{fancyhdr}
\pagestyle{fancy}
\fancyhf{}
\renewcommand{\headrulewidth}{0pt} % disable the underline of the header
\raggedbottom
\hfuzz=2mm %suppresses overfull box warnings

\usepackage{hyperref}
\usepackage{float}


\title{IB Math AA/AI Mixed Practice Draft\\(Sequences, Interest, Regression, Functions, Logs)}
\author{Chris Huson}
\date{November 2025}

\fancyhead[LE]{\thepage}
\fancyhead[RO]{\thepage \\ First \& last name: \hspace{2.25cm} \,\\ Grade: \hspace{2.25cm} \,}
%\fancyhead[RO]{First \& last name: \hspace{2.25cm} \,\\ \,}
\fancyhead[LO]{La Scuola d'Italia / Huson / IB Math: Sequences \\* 31 October 2025}

\begin{document}

\subsubsection*{2.3 Classwork: Review; due Monday 3 November}
\begin{enumerate}[itemsep=0.25cm]

%\maketitle

% =========================================================
% PAPER 1 STYLE (no calculator)
% =========================================================
\section*{Paper 1 style (no calculator)}
\item 
% Inspired by: AA_SL_P1_generic_sequences (AA P1 file text not readable)
\textbf{1.} A sequence is defined by
\[
u_1 = 4,\qquad u_{n+1} = 3u_n - 2 \quad (n \ge 1).
\]
\begin{enumerate}
  \item[(a)] Write down $u_2$ and $u_3$.
  \item[(b)] Show that $u_n = 3^{\,n} + 1$ satisfies the recurrence.
  \item[(c)] Hence find $u_6$.
\end{enumerate}

% Inspired by: AA_SL_P2_N21_Q6 (geometric sum) but rewritten for P1
\textbf{2.} A geometric sequence has first term $12$ and common ratio $\dfrac{5}{6}$.
\begin{enumerate}
  \item[(a)] Write down the second and third terms.
  \item[(b)] Find the sum of the first $n$ terms, $S_n$.
  \item[(c)] Find the least value of $n$ such that $S_n > 50$.
\end{enumerate}

% Inspired by: AI_SL_P1_N21_Q3 (exponential model) but with logs made exact
\textbf{3.} A quantity $P$ decreases according to $P(t)=1800 e^{-kt}$.
\begin{enumerate}
  \item[(a)] Given that $P(2)=1500$, find $k$.
  \item[(b)] Find the time when $P(t)=900$.
  \item[(c)] State whether the graph of $P$ is linear, quadratic or exponential.
\end{enumerate}

% Inspired by: your topic list (logs)
\textbf{4.} Solve the following, giving exact values when possible.
\begin{enumerate}
  \item[(a)] $3^{x}=27$.
  \item[(b)] $\log_{5}(2x-1)=2$.
  \item[(c)] $\ln(4) - \ln(x) = \ln(2)$.
\end{enumerate}

% Inspired by: AA_SL_P2_N21_Q2 style (function and sketch) but algebraic
\textbf{5.} Consider the function
\[
f(x) = 2^{x} - 3.
\]
\begin{enumerate}
  \item[(a)] Find $f(0)$ and $f(2)$.
  \item[(b)] Solve $2^{x} - 3 = 5$.
  \item[(c)] Describe the transformation that maps $y=2^{x}$ to $y=f(x)$.
\end{enumerate}

% Inspired by: AI_SL_P2_N21_Q2 part on arithmetic sequence (places) but made no-calc
\textbf{6.} An arithmetic sequence has first term $a_1=950$ and common difference $d=25$.
\begin{enumerate}
  \item[(a)] Write an expression for $a_n$.
  \item[(b)] Find the smallest $n$ such that $a_n \ge 1400$.
  \item[(c)] Find the sum of the first $n$ terms when $n$ is the value from part (b).
\end{enumerate}

% Inspired by: your topic list (quadratic and linear functions)
\textbf{7.} A quadratic function is given by $g(x)=x^2 - 6x + 5$.
\begin{enumerate}
  \item[(a)] Find the coordinates of the vertex.
  \item[(b)] Solve $g(x)=0$.
  \item[(c)] The line $y=mx$ intersects the graph of $g$ at two distinct points. Find the range of $m$ for which this happens.
\end{enumerate}

% =========================================================
% PAPER 2 STYLE (calculator allowed)
% Source tags point to your uploaded 2021 Nov AA/AI SL papers
% =========================================================
\section*{Paper 2 style (GDC allowed)}

% Inspired by: AA_SL_N21_P2_Q1 (regression: practice time vs score)
\textbf{8.} % Inspired by: AA\_SL\_N21\_P2\_Q1
A music teacher records the number of hours $h$ students practise each week and the mark $M$ each student receives on a test. The data for eight students are entered into a GDC.
\begin{enumerate}
  \item[(a)] Use your GDC to find the product–moment correlation coefficient $r$.
  \item[(b)] The regression line of $M$ on $h$ is $M = ah + b$. Write down $a$ and $b$ from your GDC output.
  \item[(c)] A student currently practises $10$ hours per week. Use the regression line to estimate how many marks the student might get if they increase practice to $14$ hours per week.
\end{enumerate}

% Inspired by: AI_SL_N21_P2_Q2 (applications: geometric growth of applications)
\textbf{9.} % Inspired by: AI\_SL\_N21\_P2\_Q2
A school enrolment in year 1 is $1200$ students and increases each year by $3.5\%$.
\begin{enumerate}
  \item[(a)] Show that the number of students in year $n$ can be modelled by $E_n = 1200(1.035)^{n-1}$.
  \item[(b)] Find the number of students in year $8$.
  \item[(c)] A scholarship fund pays \$75 to each enrolled student each year. Find the total amount paid in the first 6 years.
\end{enumerate}

% Inspired by: AI_SL_N21_P2_Q2 second half (linear growth vs geometric growth)
\textbf{10.} % Inspired by: AI\_SL\_N21\_P2\_Q2
The number of seats available at the school in year $n$ is given by $S_n = 1100 + 40(n-1)$.
\begin{enumerate}
  \item[(a)] Write down $S_1$ and $S_{10}$.
  \item[(b)] Find the least value of $n$ such that $S_n \ge E_n$ from question 9.
  \item[(c)] Explain, with reference to the models, whether the school will always have enough seats for all applicants for all future years.
\end{enumerate}

% Inspired by: AA_SL_N21_P2_Q6 (sum to infinity, tail < 0.001)
\textbf{11.} % Inspired by: AA\_SL\_N21\_P2\_Q6
The sum of the first $n$ terms of a geometric sequence is
\[
S_n = 9\left(1 - \left(\frac{4}{5}\right)^n\right).
\]
\begin{enumerate}
  \item[(a)] Find the first term and the common ratio.
  \item[(b)] Find $S_\infty$.
  \item[(c)] Find the least value of $n$ such that $S_\infty - S_n < 0.0008$.
\end{enumerate}

% Inspired by: your topic list (compound interest)
\textbf{12.} An investment of €5000 grows at a nominal annual rate of $2.4\%$ compounded monthly.
\begin{enumerate}
  \item[(a)] Write a formula for the value $V$ after $t$ years.
  \item[(b)] Use your GDC to find the value after 5 years, correct to the nearest euro.
  \item[(c)] Determine the time needed for the investment to reach €6000.
\end{enumerate}

% Inspired by: AA_SL_N21_P2_Q2 (function and graph) but made exponential/log mix
\textbf{13.} % Inspired by: AA\_SL\_N21\_P2\_Q2
Consider the function $f(x)=\ln(x+2) - \dfrac12$ for $x \ge -1$.
\begin{enumerate}
  \item[(a)] Find $f(-1)$ and $f(0)$.
  \item[(b)] Solve $\ln(x+2) - \dfrac12 = 0$.
  \item[(c)] Sketch the graph of $y=f(x)$ on $-1 \le x \le 4$.
\end{enumerate}

% Inspired by: AI_SL_N21_P1_Q1 (correlation) but with a finance flavour
\textbf{14.} % Inspired by: AI\_SL\_N21\_P1\_Q1
A researcher records the age of used cars (years) and their selling price (in thousands of euros). The data are entered into a GDC.
\begin{enumerate}
  \item[(a)] Find the correlation coefficient.
  \item[(b)] Write down the regression equation of price on age.
  \item[(c)] Comment on whether it is sensible to use this regression equation to predict the price of a brand new car.
\end{enumerate}

% Inspired by: your topic list (function operations/composition often paired with expo/log)
\textbf{15.} Define $f(x)=2x+3$ and $g(x)=5e^{0.2x}$.
\begin{enumerate}
  \item[(a)] Find $(f\circ g)(x)$.
  \item[(b)] Find $(g\circ f)(x)$.
  \item[(c)] Solve $(f\circ g)(x)=33$ using your GDC.
\end{enumerate}

% =========================================================
% MARKSCHEME (teacher version)
% =========================================================
\newpage
\section*{Markscheme (outline)}

\textbf{1.} (a) $u_2=3\cdot 4 - 2=10$, $u_3=3\cdot 10 - 2=28$. (b) Substitute: $3(3^{n}+1)-2=3^{n+1}+1$. (c) $u_6=3^{6}+1=729+1=730$.

\textbf{2.} (a) $12, 12\cdot \frac56=10$, etc. (b) $S_n = 12\frac{1-(5/6)^n}{1-5/6} = 72\bigl(1-(5/6)^n\bigr)$. (c) Solve $72(1-(5/6)^n)>50$.

\textbf{3.} (a) $1500=1800e^{-2k}\Rightarrow e^{-2k}=\frac{5}{6}\Rightarrow k=\frac12\ln\frac{6}{5}$. (b) $900=1800e^{-kt}\Rightarrow e^{-kt}=\frac12\Rightarrow t=\frac{\ln 2}{k}$. (c) Exponential.

\textbf{4.} (a) $x=3$. (b) $2x-1=25\Rightarrow x=13$. (c) $\ln 4 - \ln x = \ln 2 \Rightarrow \ln\frac{4}{x} = \ln 2 \Rightarrow \frac{4}{x}=2 \Rightarrow x=2$.

\textbf{5.} (a) $f(0) = -3$, $f(2)=1$. (b) $2^{x}=8\Rightarrow x=3$. (c) Vertical translation down $3$ units.

\textbf{6.} (a) $a_n=950+25(n-1)$. (b) $950+25(n-1)\ge 1400 \Rightarrow 25(n-1)\ge 450 \Rightarrow n-1\ge 18 \Rightarrow n=19$. (c) $S_{19}=\frac{19}{2}(2\cdot 950 + 18\cdot 25)$.

\textbf{7.} (a) Vertex at $(3,-4)$. (b) Roots $x=1,5$. (c) Discriminant of $x^2-6x+5-mx=0$ positive.

\textbf{8.} Use GDC: $r$ close to the one in your source; slope $a>0$; increase of 4 h multiplies slope by 4.

\textbf{9.} (a) Standard geometric growth statement. (b) $E_8=1200(1.035)^7$. (c) Arithmetic sum with 6 terms.

\textbf{10.} (a) $S_1=1100$, $S_{10}=1100+40\cdot 9=1460$. (b) Solve $1100+40(n-1)\ge 1200(1.035)^{n-1}$ numerically. (c) Geometric will eventually outgrow linear.

\textbf{11.} (a) $u_1 = 9(1 - 4/5)=9/5$, $r=4/5$. (b) $S_\infty = \frac{9}{1-4/5}=45$. (c) $45-9(1-(4/5)^n) < 0.0008 \Rightarrow 9(4/5)^n < 0.0008$.

\textbf{12.} (a) $V(t)=5000\left(1+\frac{0.024}{12}\right)^{12t}$. (b), (c) Use solver.

\textbf{13.} (a) $f(-1)=\ln 1 - \tfrac12 = -\tfrac12$, $f(0)=\ln 2 - \tfrac12$. (b) $\ln(x+2)=\tfrac12 \Rightarrow x+2=e^{1/2} \Rightarrow x=e^{1/2}-2$. (c) Standard log sketch.

\textbf{14.} Positive correlation, negative slope (if age vs price with usual data), regression usable only inside data range.

\textbf{15.} (a) $(f\circ g)(x) = 2\bigl(5e^{0.2x}\bigr)+3 = 10e^{0.2x}+3$. (b) $(g\circ f)(x)=5e^{0.2(2x+3)}$. (c) Solve $10e^{0.2x}+3=33$.
       
\end{enumerate}
\end{document}
