\documentclass[12pt, twoside]{article}
% \documentclass[12pt, twoside]{article}
\usepackage[letterpaper, margin=1in, headsep=0.2in]{geometry}
\setlength{\headheight}{0.6in}
%\usepackage[english]{babel}
\usepackage[utf8]{inputenc}
\usepackage{microtype}
\usepackage{amsmath}
\usepackage{amssymb}
%\usepackage{amsfonts}
\usepackage[nomessages]{fp} %\FPeval{\var-name}{2*sin(pi/6)}
\usepackage{siunitx} %units in math. eg 20\milli\meter
\usepackage{yhmath} % for arcs, overparenth command
\usepackage{tikz} %graphics
\usetikzlibrary{quotes, angles, arrows, arrows.meta}
\usepackage{graphicx} %consider setting \graphicspath{{images/}}
\usepackage{parskip} %no paragraph indent
\usepackage{enumitem}
\usepackage{multicol}
\usepackage{venndiagram}

\usepackage{fancyhdr}
\pagestyle{fancy}
\fancyhf{}
\renewcommand{\headrulewidth}{0pt} % disable the underline of the header
\raggedbottom
\hfuzz=2mm %suppresses overfull box warnings

\usepackage{hyperref}
\usepackage{float}


\title{IB Math AA Test: Sequences, Interest, Regression, Functions, Logs}
\author{Chris Huson}
\date{November 2025}

\fancyhead[LE]{\thepage}
\fancyhead[RO]{\thepage \\ First \& last name: \hspace{2.25cm} \,\\ Grade: \hspace{2.25cm} \,}
%\fancyhead[RO]{First \& last name: \hspace{2.25cm} \,\\ \,}
\fancyhead[LO]{La Scuola d'Italia / Huson / IB Math: Sequences \\* 6 November 2025}

\begin{document}

\subsubsection*{2.6 Test: Sequences, Regression, Interest, Functions, Logs}
\begin{enumerate}[itemsep=0.1cm]


\item A geometric sequence has first term $24$ and common ratio $\frac{1}{2}$. \hfill [10 marks]
\begin{enumerate}
  \item Write down the second and third terms.
  \item Write an expression for the sum of the first $n$ terms, $S_n$.
  \item Hence or otherwise find the sum of the first 10 terms of the sequence $S_{10}$.
\vspace{0.5cm}

An arithmetic sequence has a first term $150$ and fourth term $111$.
  \item Find the common difference $d$.
  \item Find the number of the first negative term in the arithmetic sequence, i.e. $n$ such that $u_n < 0$.
\end{enumerate}
\vspace{0.1cm}
    \begin{tikzpicture}
        \draw (0,2) rectangle (15.5,11);
        \draw [dotted] (1,10)--(14,10);
        \draw [dotted] (1,9)--(14,9);
        \draw [dotted] (1,8)--(14,8);
        \draw [dotted] (1,7)--(14,7);
        \draw [dotted] (1,6)--(14,6);
    \end{tikzpicture}
\newpage
% --------------------------------------------------
% 1. Music academy regression (table)
% --------------------------------------------------
\item Six piano students reported their average weekly practice time and their Music class grade (out of 100). The data are shown below. \hfill [7 marks]
  \begin{center}
  \begin{tabular}{|c|cccccc|}
    \hline
  Practice time ($h$) & 8 & 3 & 5 & 3 & 7 & 9 \\
    \hline
  Grade ($G$) & 85 & 70 & 75 & 85 & 90 & 95 \\
    \hline
  \end{tabular}
  \end{center}

  \begin{enumerate}[itemsep=0.1cm]
    \item Find the Pearson product–moment correlation coefficient $r$ for these data.
    \item Based on the following guidance:

    \begin{quote}
      $0 \le |r| < 0.4$: weak,\quad
      $0.4 \le |r| < 0.8$: moderate,\quad
      $0.8 \le |r| \le 1$: strong.
    \end{quote}

    Comment on the strength of the correlation for this data.
    \item The relationship between $h$ and $G$ can be modelled by a regression equation $G = ah + b$. Write down the values of $a$ and $b$.
    \item One of the students says she should have practised more. Based on the data and assuming that she practiced 6 hours per week, estimate what her score would have been.
  \end{enumerate}
  \vspace{0.25cm}
      \begin{tikzpicture}
          \draw (0,2) rectangle (15.5,11);
          \draw [dotted] (1,10)--(14,10);
          \draw [dotted] (1,9)--(14,9);
          \draw [dotted] (1,8)--(14,8);
          \draw [dotted] (1,7)--(14,7);
          \draw [dotted] (1,6)--(14,6);
      \end{tikzpicture}

\newpage

\item Give all numerical answers correct to two decimal places. \hfill [8 marks]

    Sofia invests \$1000 in an annuity on 1 January 2020. The investment earns a fixed amount of \$60 per year.

    \begin{enumerate}
    \item Find the value of her investment on 1 January 2022. \\[0.5cm]
    Rafael also invests \$1000 on 1 January 2020. He deposits his funds in a bank account that pays a nominal annual rate of $5\%$ compounded monthly. 
    \item Find his balance after two years.
    \item Determine the number of complete years from 1 January 2020 until Rafael's account first has a greater balance than Sofia's investment.
  \end{enumerate}

    \vspace{0.5cm}
    \begin{tikzpicture}
        \draw (0,2) rectangle (15.5,11);
        \draw [dotted] (1,10)--(14,10);
        \draw [dotted] (1,9)--(14,9);
        \draw [dotted] (1,8)--(14,8);
        \draw [dotted] (1,7)--(14,7);
        \draw [dotted] (1,6)--(14,6);
    \end{tikzpicture}
\newpage

% --------------------------------------------------
% 3. Bacteria growth (exponential)
% --------------------------------------------------
\item In an experiment the area of a bacteria culture is modelled by \hfill [4 marks]
\[
P(t) = A e^{kt},
\]
where $P$ is the area in mm$^2$ and $t$ is the time in hours. At $t=0$ the area is $80$ mm$^2$, and after $11$ hours the area is $612$ mm$^2$.

\begin{enumerate}[itemsep=0.25cm]
  \item Write down the value of $A$.
  \item Find the value of $k$.
\end{enumerate}
\vspace{0.25cm}
    \begin{tikzpicture}
        \draw (0,2) rectangle (15.5,11);
        \draw [dotted] (1,10)--(14,10);
        \draw [dotted] (1,9)--(14,9);
        \draw [dotted] (1,8)--(14,8);
        \draw [dotted] (1,7)--(14,7);
        \draw [dotted] (1,6)--(14,6);
    \end{tikzpicture}
\newpage

% --------------------------------------------------
% 4. Kite on coordinate axes
% --------------------------------------------------
\item Camilla is designing a kite $ABCD$ on a coordinate plane (1 unit = 10 cm). The points are 
$A(3,0), \; B(0,5), \; C(5,8)$, and point $D$ lies on the $x$–axis. Segment $AC$ is perpendicular to segment $BD$.  \hfill [6 marks]
  \begin{center}
  \begin{tikzpicture}[scale=0.5,>=stealth]
    % axes
    \draw[->] (0,0) -- (0,9) node[left] {$y$};
    \draw[->] (0,0) -- (22,0) node[below] {$x$};

    % coordinates (chosen to match the problem)
    \coordinate (B) at (0,5);
    \coordinate (A) at (3,0);
    \coordinate (C) at (5,8);
    \coordinate (D) at (20,0);

    % sides of the figure
    \draw (B) -- (C) -- (D) -- (A) -- cycle;   % outer shape BAD + CD
    \draw (B) -- (A);                           % BA
    \draw (B) -- (D);                           % BD
    \draw (A) -- (C);                           % AC

    % intersection of AC and BD for right-angle mark
  %  \path [name path=AC] (A) -- (C);
  %  \path [name path=BD] (B) -- (D);
  %  \path [name intersections={of=AC and BD, by=E}];

    % right angle at intersection
  %  \draw ($(E)!0.15!(B)$) -- ($(E)!0.15!(A)$) -- ($(E)!0.15!(D)$);

    % points
    \fill (A) circle (2pt) node[below] {$A$};
    \fill (B) circle (2pt) node[left] {$B$};
    \fill (C) circle (2pt) node[above] {$C$};
    \fill (D) circle (2pt) node[below] {$D$};

    % note
    \node[anchor=west] at (15,6.5) {\small diagram not to scale};
  \end{tikzpicture}
  \end{center}
  \begin{enumerate}
    \item Find the gradient of the line through $A$ and $C$.
    \item Hence write down the gradient of the line through $B$ and $D$.
    \item Find the equation of line $BD$ in the form $ax + by + d = 0$, where $a,b,d$ are integers.
    \item Write down the $x$–coordinate of $D$.
  \end{enumerate}
  \vspace{0.25cm}
      \begin{tikzpicture}
          \draw (0,2) rectangle (15.5,11);
          \draw [dotted] (1,10)--(14,10);
          \draw [dotted] (1,9)--(14,9);
          \draw [dotted] (1,8)--(14,8);
          \draw [dotted] (1,7)--(14,7);
          \draw [dotted] (1,6)--(14,6);
    \end{tikzpicture}

\newpage
\item Solve the following, giving exact values when possible. \hfill [6 marks]
\begin{enumerate}[itemsep=0.1cm]
  \item $3^{x}=243$.
  \item $\log_{5}(2x-1)=2$.
  \item $\ln(4) - \ln(x) = \ln(2)$.
\end{enumerate}
\vspace{0.1cm}
  \vspace{0.25cm}
      \begin{tikzpicture}
          \draw (0,2) rectangle (15.5,11);
          \draw [dotted] (1,10)--(14,10);
          \draw [dotted] (1,9)--(14,9);
          \draw [dotted] (1,8)--(14,8);
          \draw [dotted] (1,7)--(14,7);
          \draw [dotted] (1,6)--(14,6);
    \end{tikzpicture}


\end{enumerate}
\end{document}

