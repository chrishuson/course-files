\documentclass[12pt, twoside]{article}
\documentclass[12pt, twoside]{article}
\usepackage[letterpaper, margin=1in, headsep=0.2in]{geometry}
\setlength{\headheight}{0.6in}
%\usepackage[english]{babel}
\usepackage[utf8]{inputenc}
\usepackage{microtype}
\usepackage{amsmath}
\usepackage{amssymb}
%\usepackage{amsfonts}
\usepackage{siunitx} %units in math. eg 20\milli\meter
\usepackage{yhmath} % for arcs, overparenth command
\usepackage{tikz} %graphics
\usetikzlibrary{quotes, angles}
\usepackage{graphicx} %consider setting \graphicspath{{images/}}
\usepackage{parskip} %no paragraph indent
\usepackage{enumitem}
\usepackage{multicol}
\usepackage{venndiagram}

\usepackage{fancyhdr}
\pagestyle{fancy}
\fancyhf{}
\renewcommand{\headrulewidth}{0pt} % disable the underline of the header
\raggedbottom
\hfuzz=2mm %suppresses overfull box warnings

\usepackage{hyperref}
\usepackage{float}

\title{IB}
\author{Chris Huson}
\date{October 2025}

\fancyhead[LE]{\thepage}
\fancyhead[RO]{\thepage \\ First \& last name: \hspace{2.25cm} \,\\ Grade: \hspace{2.25cm} \,}
%\fancyhead[RO]{First \& last name: \hspace{2.25cm} \,\\ \,}
\fancyhead[LO]{La Scoula d'Italia / Huson / IB Math: Sequences \\* 3 October 2025}

\begin{document}

\subsubsection*{1.3 Homework: Algebra review}
\begin{enumerate}[itemsep=0.5cm]
\item {[Maximum mark: 6]} \\[0.3cm]
    The number of apartments in a housing development has been increasing by a constant amount every year. At the end of the first year the number of apartments was 150, and at the end of the sixth year the number of apartments was 600. \\[0.25cm]
    The number of apartments, $y$, can be determined by the equation $y=mt+n$, where $t$ is the time, in years.
    \begin{enumerate}[itemsep=0.25cm]
        \item Find the value of $m$. \hfill [2]
        \item State what $m$ represents \textbf{in this context}. \hfill [1]
        \item Find the value of $n$. \hfill [2]
        \item State what $n$ represents \textbf{in this context}. \hfill [1]
    \end{enumerate}
    \begin{tikzpicture}
        \draw (0,2) rectangle (15.5,11);
        \draw [dotted] (1,10)--(14,10);
        \draw [dotted] (1,9)--(14,9);
        \draw [dotted] (1,8)--(14,8);
        \draw [dotted] (1,7)--(14,7);
        \draw [dotted] (1,6)--(14,6);
    \end{tikzpicture}

\newpage
\item {[Maximum mark: 6]} \\[0.3cm]
    An iron bar is heated. Its length, $L$, in millimetres can be modelled by a linear function, $L=mT+c$, where $T$ is the temperature measured in degrees Celsius ($^\circ$C). \\[0.25cm]
    At $150^\circ$C the length of the iron bar is 180 mm.
    \begin{enumerate}[itemsep=0.25cm]
        \item Write down an equation that shows this information. \hfill [1]
        \item At $210^\circ$C the length of the iron bar is 181.5 mm. \\[0.25cm]
        Write down an equation that shows this second piece of information. \hfill [1]
        \item Hence, find the length of the iron bar at $40^\circ$C. \hfill [4]
    \end{enumerate}
    \begin{tikzpicture}
        \draw (0,2) rectangle (15.5,11);
        \draw [dotted] (1,10)--(14,10);
        \draw [dotted] (1,9)--(14,9);
        \draw [dotted] (1,8)--(14,8);
        \draw [dotted] (1,7)--(14,7);
        \draw [dotted] (1,6)--(14,6);
    \end{tikzpicture}

\newpage
\item {[Maximum mark: 5]} \\[0.3cm]
    The diagram shows the straight line $L_1$, which intersects the $x$-axis at $A(j, 0)$ and the $y$-axis at $B(0,k)$.
        \begin{center}
            \begin{tikzpicture}[scale=1]
            %\draw [help lines] (0,0) grid (10,8);
            \draw [thick, ->] (-0.5,0) -- (7.4,0) node [below right] {$x$};
            \draw [thick, ->] (0,-0.5)--(0,3.4) node [left] {$y$};
            %\draw [fill] (9,5) circle [radius=0.1];
            \draw [thick, -] (-0.25,3) node [below] {$B$}--(6,-0.1) node [above] {$A$};
            \node at (6,4){\textbf{diagram is not to scale}};
            \end{tikzpicture}
        \end{center}
        The equation of $L_1$ is $\displaystyle y=-\frac{2}{5}x+3$.
        \begin{enumerate}%[itemsep=3cm]
            \item Find the value of \hfill [2]
                \begin{enumerate}
                    \item $j$
                    \item $k$
                \end{enumerate}
            \item The line $L_2$ is perpendicular to $L_1$ and passes through $(4,3)$.
                \begin{enumerate}
                    \item Write down the gradient for the line $L_2$. \hfill [1]
                    \item Hence, write down the equation of $L_2$. Leave your answer in the form \\ $y-a=m(x-b)$. \hfill [2]
                \end{enumerate}
        \end{enumerate}
        \begin{tikzpicture}
            \draw (0,3) rectangle (15.5,11);
            \draw [dotted] (1,10)--(14,10);
            \draw [dotted] (1,9)--(14,9);
            \draw [dotted] (1,8)--(14,8);
            \draw [dotted] (1,7)--(14,7);
        \end{tikzpicture}

\newpage 
\item {[Maximum mark: 6]} \\[0.3cm]
    The diagram shows part of the graph of the quadratic function $f$. 
         \begin{center}
         \begin{tikzpicture}[xscale=1.0, yscale=0.5]
                 %\draw [thin, color=gray, xstep=1.0cm,ystep=1.0cm] (-3.5,-3.5) grid (3.5,3.5);
                 %\draw [thin, color=lightgray,, xstep=0.2cm,ystep=0.2cm] (-5.5,-4.5) grid (5.5,6.5);
                 \foreach \x in {-1, 0,1,2,3, 4, 5}
                     \draw[shift={(\x,0)},color=black] (0pt,-1pt) -- (0pt,3pt) node[below]  {$\quad \x$};
                 %\foreach \y in {-3, -2,-1,0,1,2,3}
                     %\draw[shift={(0,\y)},color=black] (2pt,0pt) -- (-2pt,0pt) node[left]  {$\y$};
                 \draw [thick, ->] (-1.5,0) -- (+5.5,0) node [right] {$x$};
                 \draw [thick, ->] (0,-1.5) -- (0,9.5) node [left] {$y$};
             \draw [thick] plot[domain= -1:5] (\x, {(\x-2)^2 -1});
             %\draw [thick] plot[domain= -1:2] (\x, \x*1/2 -2);
             %\draw [thick] (-2,-2).. controls (0,-1) and (2,0) .. (3,1);
             %\draw [fill] (-3,3) circle[radius=0.1] node[above left]{$A(-3,3)$};
         \end{tikzpicture}
         \end{center}
         The vertex is at $(2, -1)$ and the $x$-intercepts are at 1 and 3.\\[0.25cm]
         The function $f$ can be written in the form $f(x)=(x-h)^2+k$.
         \begin{enumerate}%[itemsep=2cm]
             \item Write down the value of $h$ and $k$. \hfill [2]\\[0.25cm]
             The function can also be written in the form $f(x)=(x-a)(x-b)$.
             \item Write down the value of $a$ and $b$. \hfill [2]
             \item Find the $y$-intercept. \hfill [2]
         \end{enumerate}
         \begin{tikzpicture}
             \draw (0,2) rectangle (15.5,11);
             \draw [dotted] (1,10)--(14,10);
             \draw [dotted] (1,9)--(14,9);
             \draw [dotted] (1,8)--(14,8);
             \draw [dotted] (1,7)--(14,7);
             \draw [dotted] (1,6)--(14,6);
         \end{tikzpicture}

\newpage

       
\end{enumerate}
\end{document}