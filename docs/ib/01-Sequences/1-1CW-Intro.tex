\documentclass[12pt, twoside]{article}
% \documentclass[12pt, twoside]{article}
\usepackage[letterpaper, margin=1in, headsep=0.2in]{geometry}
\setlength{\headheight}{0.6in}
%\usepackage[english]{babel}
\usepackage[utf8]{inputenc}
\usepackage{microtype}
\usepackage{amsmath}
\usepackage{amssymb}
%\usepackage{amsfonts}
\usepackage[nomessages]{fp} %\FPeval{\var-name}{2*sin(pi/6)}
\usepackage{siunitx} %units in math. eg 20\milli\meter
\usepackage{yhmath} % for arcs, overparenth command
\usepackage{tikz} %graphics
\usetikzlibrary{quotes, angles, arrows, arrows.meta}
\usepackage{graphicx} %consider setting \graphicspath{{images/}}
\usepackage{parskip} %no paragraph indent
\usepackage{enumitem}
\usepackage{multicol}
\usepackage{venndiagram}

\usepackage{fancyhdr}
\pagestyle{fancy}
\fancyhf{}
\renewcommand{\headrulewidth}{0pt} % disable the underline of the header
\raggedbottom
\hfuzz=2mm %suppresses overfull box warnings

\usepackage{hyperref}
\usepackage{float}

\title{IB}
\author{Chris Huson}
\date{October 2025}

\fancyhead[LE]{\thepage}
%\fancyhead[RO]{\thepage \\ First \& last name: \hspace{2.25cm} \,\\ Grade: \hspace{2.25cm} \,}
\fancyhead[RO]{First \& last name: \hspace{2.25cm} \,\\ \,}
\fancyhead[LO]{La Scuola d'Italia / Huson / IB Math: Sequences \\* 1 October 2025}

\begin{document}

\subsubsection*{1.1 Classwork: Introduction}
\begin{enumerate}[itemsep=0.5cm]
\item What name do you like to be called? \vspace{0.5cm}
\item What kind of calculator do you have? \vspace{0.5cm}

\item Simplify the expression $\sqrt{18}=$ \vspace{0.5cm}

\item Solve for $x$: $3x^4=48$ \vspace{2cm}
\item Write down as many digits of $\pi$ as you know from memory. \vspace{0.5cm}
\item Given an infinite list of numbers with the following pattern: $1, 1, 2, 3, 5, 8, \ldots$
\begin{enumerate}
    \item Write down the next two numbers in the sequence. \vspace{0.5cm}
    \item Define the rule for the sequence as precisely as you can.
\end{enumerate} \vspace{3cm}
\item Find the sum $1+2+3+4+\ldots + 99+100$. (be clever about it) \vspace{2cm}

       
\end{enumerate}
\end{document}