\documentclass[12pt, twoside]{article}
\documentclass[12pt, twoside]{article}
\usepackage[letterpaper, margin=1in, headsep=0.2in]{geometry}
\setlength{\headheight}{0.6in}
%\usepackage[english]{babel}
\usepackage[utf8]{inputenc}
\usepackage{microtype}
\usepackage{amsmath}
\usepackage{amssymb}
%\usepackage{amsfonts}
\usepackage{siunitx} %units in math. eg 20\milli\meter
\usepackage{yhmath} % for arcs, overparenth command
\usepackage{tikz} %graphics
\usetikzlibrary{quotes, angles}
\usepackage{graphicx} %consider setting \graphicspath{{images/}}
\usepackage{parskip} %no paragraph indent
\usepackage{enumitem}
\usepackage{multicol}
\usepackage{venndiagram}

\usepackage{fancyhdr}
\pagestyle{fancy}
\fancyhf{}
\renewcommand{\headrulewidth}{0pt} % disable the underline of the header
\raggedbottom
\hfuzz=2mm %suppresses overfull box warnings

\usepackage{hyperref}
\usepackage{float}

\title{IB}
\author{Chris Huson}
\date{October 2025}

\fancyhead[LE]{\thepage}
\fancyhead[RO]{\thepage \\ First \& last name: \hspace{2.25cm} \,\\ Grade: \hspace{2.25cm} \,}
%\fancyhead[RO]{First \& last name: \hspace{2.25cm} \,\\ \,}
\fancyhead[LO]{La Scuola d'Italia / Huson / IB Math: Sequences \\* 10 October 2025}

\begin{document}

\subsubsection*{1.7 Quiz - Early Finishers: Sequences and quadratics, Complete on lined paper}
\begin{enumerate}[itemsep=0.5cm]

\item The first term of an arithmetic sequence is 24 and the common difference is 16.
    \begin{enumerate}
        \item Find the value of the 62nd term of the sequence. \hfill [2 marks] \\[0.5cm]
        The first term of a geometric sequence is 8. The 4th term of the geometric sequence is equal to the 13th term of the arithmetic sequence given above. \vspace{0.25cm}
        \item Write down an equation using this information. \hfill [2 marks]
        \item Calculate the common ratio of the geometric sequence. \hfill [2 marks]
    \end{enumerate}

\item The 1st, 5th, and 13th terms of an arithmetic sequence, with common difference $d$, 
$d\ne 0$, are the first three terms of a geometric sequence, with common ratio $r$, $r \ne 1$. Given that the 1st term of both sequences is 12, find the value of $d$ and the value of $r$. \hfill [6 marks]


\item Let $f(x)=2x^2+3x-1$. \hfill [6 marks]
    \begin{enumerate}
        \item Write down the coordinates of the vertex.
        \item Hence or otherwise, express the function in the form $f(x)=2(x-h)^2 +k$.
        \item Solve the equation  $f(x)=0$.
    \end{enumerate}

\item Consider the function $f(x)=x^2-6x-1$.
    \begin{enumerate}
        \item Sketch the graph of $f$, for $-4 \leq x \leq 3$.
        \item This function can also be written in the form $f(x)=(x-p)^2 -10$.\\*
        Write down the value of $p$.
        \item The graph of $g$ is obtained by reflecting the graph of $f$ in the $x$-axis, followed by a translation of $(0, 4)$ (i.e. move the parabola up four).\\* Show that $g(x)=-x^2+6x+5$.
        \item The graphs of $f$ and $g$ intersect at two points.\\*
        Write down the x-coordinates of these two points.
    \end{enumerate} \hfill [8 marks]

       
\end{enumerate}
\end{document}