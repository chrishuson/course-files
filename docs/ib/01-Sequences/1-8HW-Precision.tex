\documentclass[12pt, twoside]{article}
\documentclass[12pt, twoside]{article}
\usepackage[letterpaper, margin=1in, headsep=0.2in]{geometry}
\setlength{\headheight}{0.6in}
%\usepackage[english]{babel}
\usepackage[utf8]{inputenc}
\usepackage{microtype}
\usepackage{amsmath}
\usepackage{amssymb}
%\usepackage{amsfonts}
\usepackage{siunitx} %units in math. eg 20\milli\meter
\usepackage{yhmath} % for arcs, overparenth command
\usepackage{tikz} %graphics
\usetikzlibrary{quotes, angles}
\usepackage{graphicx} %consider setting \graphicspath{{images/}}
\usepackage{parskip} %no paragraph indent
\usepackage{enumitem}
\usepackage{multicol}
\usepackage{venndiagram}

\usepackage{fancyhdr}
\pagestyle{fancy}
\fancyhf{}
\renewcommand{\headrulewidth}{0pt} % disable the underline of the header
\raggedbottom
\hfuzz=2mm %suppresses overfull box warnings

\usepackage{hyperref}
\usepackage{float}

\title{IB}
\author{Chris Huson}
\date{October 2025}

\fancyhead[LE]{\thepage}
\fancyhead[RO]{\thepage \\ First \& last name: \hspace{2.25cm} \,\\ Grade: \hspace{2.25cm} \,}
%\fancyhead[RO]{First \& last name: \hspace{2.25cm} \,\\ \,}
\fancyhead[LO]{La Scoula d'Italia / Huson / IB Math: Sequences \\* 14 October 2025}

\begin{document}

\subsubsection*{1.8 Homework: Precision, Scientific notation, Significant figures}

In IB we answer exactly or rounded to three significant figures. Copy the calculator display followed by three dots, then round: $$\pi = 3.1415926\ldots \approx 3.14$$

\begin{enumerate}[itemsep=0.5cm]
\item Round each value to three sig figs
  \begin{multicols}{2}
    \begin{enumerate}[itemsep=0.5cm]
      \item 2,746,984 \par (population of Rome)
      \item $\sqrt{2}$
      \item 8,804,190 \par (population of New York City)
      \item $e$
    \end{enumerate}
  \end{multicols}

\item Write down the number of significant digits in each value.
  \begin{multicols}{3}
    \begin{enumerate}[itemsep=1cm]
      \item 8
      \item 27
      \item 60
      \item 120
      \item 105.5
      \item 1.7320
    \end{enumerate}
  \end{multicols}

\item Calculate and write as scientific notation, rounding to three sig figs.
  \begin{multicols}{2}
    \begin{enumerate}[itemsep=1cm]
      \item $22.5 \times 14^2-665$
      \item The mean distance of the earth to the moon: 384,400 kilometers.
    \end{enumerate}
  \end{multicols} \vspace{2cm}

  \item The Earth's mass is $5.972 \times 10^{24}$ kg and the moon's mass is $7.348 \times 10^{22}$ kg. What is the ratio of the Earth's mass to the moon's mass? Round to three significant figures.
       
\end{enumerate}
\end{document}