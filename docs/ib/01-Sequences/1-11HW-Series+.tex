
\documentclass[12pt]{article}
\usepackage{amsmath}
\usepackage{amssymb}
\usepackage{geometry}
\geometry{margin=1in}

\begin{document}

\begin{center}
\textbf{BECA / Huson / 11.1 IB Math SL}\\
Name:\underline{\hspace{5cm}}\\[6pt]
4 April 2019\\[12pt]
\textbf{Do Now: Pre-Exam Sequences and Series}
\end{center}

\begin{enumerate}
\item In an arithmetic sequence, the first term is 3 and the second term is 7.
    \begin{enumerate}
        \item Find the common difference. \hfill [2 marks]
        \item Find the tenth term. \hfill [2 marks]
        \item Find the sum of the first ten terms of the sequence. \hfill [2 marks]
    \end{enumerate}

\item The first three terms of an arithmetic sequence are \underline{\hspace{2cm}}.
    \begin{enumerate}
        \item Find the common difference. \hfill [2 marks]
        \item Find the 30th term of the sequence. \hfill [2 marks]
        \item Find the sum of the first 30 terms. \hfill [2 marks]
    \end{enumerate}

\item The first three terms of a geometric sequence are \underline{\hspace{1cm}}, \underline{\hspace{1cm}}, and \underline{\hspace{1cm}}.
    \begin{enumerate}
        \item Find the value of \underline{\hspace{1cm}}. \hfill [2 marks]
        \item Find the value of \underline{\hspace{1cm}}. \hfill [2 marks]
        \item Find the least value of $n$ such that \underline{\hspace{1cm}}. \hfill [3 marks]
    \end{enumerate}

\item The first three terms of a geometric sequence are \underline{\hspace{1cm}}, \underline{\hspace{1cm}}, \underline{\hspace{1cm}}, for \underline{\hspace{1cm}}.
    \begin{enumerate}
        \item Find the common ratio. \hfill [3 marks]
        \item Solve \underline{\hspace{2cm}}. \hfill [5 marks]
    \end{enumerate}

\item Consider a geometric sequence where the first term is 768 and the second term is 576.\\
Find the least value of $n$ such that the $n$th term of the sequence is less than 7. \hfill [6 marks]

\item[\textbf{Homework:}] \textbf{Spicy IB Exam problems}
    \begin{enumerate}
        \item Consider the following sequence of figures.\\
        Figure 1 contains 5 line segments.\\
        Given that Figure $n$ contains 801 line segments, show that \underline{\hspace{1cm}}. \hfill [3 marks]
        \item Find the total number of line segments in the first 200 figures. \hfill [3 marks]
    \end{enumerate}

\item An arithmetic sequence has the first term \underline{\hspace{1cm}} and a common difference \underline{\hspace{1cm}}.\\
The 13th term in the sequence is \underline{\hspace{1cm}}. Find the value of \underline{\hspace{1cm}}. \hfill [6 marks]

\item The first two terms of an infinite geometric sequence, in order, are \underline{\hspace{1cm}}, where \underline{\hspace{1cm}}.
    \begin{enumerate}
        \item Find \underline{\hspace{1cm}}. \hfill [2 marks]
        \item Show that the sum of the infinite sequence is \underline{\hspace{1cm}}. \hfill [2 marks]
        \item The first three terms of an arithmetic sequence, in order, are \underline{\hspace{1cm}}, where \underline{\hspace{1cm}}.\\
        Find \underline{\hspace{1cm}}, giving your answer as an integer. \hfill [4 marks]
        \item Let $S_{12}$ be the sum of the first 12 terms of the arithmetic sequence.\\
        Show that \underline{\hspace{1cm}}. \hfill [2 marks]
        \item Given that $S_{12}$ is equal to half the sum of the infinite geometric sequence, find \underline{\hspace{1cm}}, giving your answer in the form \underline{\hspace{1cm}}, where \underline{\hspace{1cm}}. \hfill [3 marks]
    \end{enumerate}
\end{enumerate}

\end{document}
