\documentclass[12pt, twoside]{article}
% \documentclass[12pt, twoside]{article}
\usepackage[letterpaper, margin=1in, headsep=0.2in]{geometry}
\setlength{\headheight}{0.6in}
%\usepackage[english]{babel}
\usepackage[utf8]{inputenc}
\usepackage{microtype}
\usepackage{amsmath}
\usepackage{amssymb}
%\usepackage{amsfonts}
\usepackage[nomessages]{fp} %\FPeval{\var-name}{2*sin(pi/6)}
\usepackage{siunitx} %units in math. eg 20\milli\meter
\usepackage{yhmath} % for arcs, overparenth command
\usepackage{tikz} %graphics
\usetikzlibrary{quotes, angles, arrows, arrows.meta}
\usepackage{graphicx} %consider setting \graphicspath{{images/}}
\usepackage{parskip} %no paragraph indent
\usepackage{enumitem}
\usepackage{multicol}
\usepackage{venndiagram}

\usepackage{fancyhdr}
\pagestyle{fancy}
\fancyhf{}
\renewcommand{\headrulewidth}{0pt} % disable the underline of the header
\raggedbottom
\hfuzz=2mm %suppresses overfull box warnings

\usepackage{hyperref}
\usepackage{float}

\title{IB}
\author{Chris Huson}
\date{October 2025}

\fancyhead[LE]{\thepage}
\fancyhead[RO]{\thepage \\ First \& last name: \hspace{2.25cm} \,\\ Grade: \hspace{2.25cm} \,}
%\fancyhead[RO]{First \& last name: \hspace{2.25cm} \,\\ \,}
\fancyhead[LO]{La Scuola d'Italia / Huson / IB Math: Sequences \\* 10 October 2025}

\begin{document}

\subsubsection*{1.7 Quiz: Sequences, Open book: notes and calculator allowed}
\begin{enumerate}[itemsep=0.5cm]
\item A sequence is defined as follows: 3, 7, 11, 15, \ldots
    \begin{enumerate}[itemsep=0.25cm]
        \item Write down the first term $u_1$. \hfill [1]
        \item Is the sequence arithmetic, geometric, or neither? \hfill [1]
        \item Find the value of the next term in the sequence. \hfill [1]
        \item Find a general expression for $u_n$, the $n^{th}$ term.  \hfill [3]
    \end{enumerate}
    \begin{tikzpicture}
        \draw (0,2) rectangle (15.5,11);
        \draw [dotted] (1,10)--(14,10);
        \draw [dotted] (1,9)--(14,9);
        \draw [dotted] (1,8)--(14,8);
        \draw [dotted] (1,7)--(14,7);
        \draw [dotted] (1,6)--(14,6);
    \end{tikzpicture}
    
\newpage
\item The first three terms of a geometric sequence are 27, 9, 3, \ldots
    \begin{enumerate}[itemsep=0.25cm]
        \item Find the common ratio $r$. \hfill [2]
        \item Find the next two terms in the sequence. \hfill [2]
        \item Find a general expression for $u_n$, the $n^{th}$ term.  \hfill [2]
    \end{enumerate}
    \begin{tikzpicture}
        \draw (0,2) rectangle (15.5,11);
        \draw [dotted] (1,10)--(14,10);
        \draw [dotted] (1,9)--(14,9);
        \draw [dotted] (1,8)--(14,8);
        \draw [dotted] (1,7)--(14,7);
        \draw [dotted] (1,6)--(14,6);
    \end{tikzpicture}
    
\newpage
\item The fourth term of a geometric sequence $u_4=108$ and the fifth term $u_5=162$.
    \begin{enumerate}[itemsep=0.25cm]
        \item Find the common ratio $r$. \hfill [1]
        \item Find the first term in the sequence. \hfill [3]
        \item Hence, find a general expression for $u_n$, the $n^{th}$ term.  \hfill [2]
    \end{enumerate}
    \begin{tikzpicture}
        \draw (0,2) rectangle (15.5,11);
        \draw [dotted] (1,10)--(14,10);
        \draw [dotted] (1,9)--(14,9);
        \draw [dotted] (1,8)--(14,8);
        \draw [dotted] (1,7)--(14,7);
        \draw [dotted] (1,6)--(14,6);
    \end{tikzpicture}
    
\newpage
\item In an arithmetic sequence $u_5=38$ and $u_{13}=86$.
    \begin{enumerate}[itemsep=0.25cm]
        \item Find the common difference. \hfill [2]
        \item Find $u_1$, the first term of the sequence.  \hfill [2]
        \item Find the largest term in the sequence that is less than 200. \hfill [2]
    \end{enumerate}
    \begin{tikzpicture}
        \draw (0,2) rectangle (15.5,11);
        \draw [dotted] (1,10)--(14,10);
        \draw [dotted] (1,9)--(14,9);
        \draw [dotted] (1,8)--(14,8);
        \draw [dotted] (1,7)--(14,7);
        \draw [dotted] (1,6)--(14,6);
    \end{tikzpicture}
    
\newpage
\subsubsection*{Challenge: Linear equations and quadratic functions}
\item The diagram shows the straight line $L_1$, which intersects the $x$-axis at $A(k, 0)$ and the $y$-axis at $B(0,3)$.
        \begin{center}
            \begin{tikzpicture}[scale=1]
            %\draw [help lines] (0,0) grid (10,8);
            \draw [thick, ->] (-0.5,0) -- (7.4,0) node [below right] {$x$};
            \draw [thick, ->] (0,-0.5)--(0,3.4) node [left] {$y$};
            %\draw [fill] (9,5) circle [radius=0.1];
            \draw [thick, -] (-0.25,3) node [below] {$B$}--(6,-0.1) node [above] {$A$};
            \node at (6,4){\textbf{diagram is not to scale}};
            \end{tikzpicture}
        \end{center}
        The gradient of $L_1$ is $-\frac{3}{4}$.
        \begin{enumerate}%[itemsep=3cm]
            \item Write down the equation of the line $L_1$. \hfill [1]
            \item Find the value of $k$. \hfill [2]
            \item The line $L_2$ is perpendicular to $L_1$ and passes through $(2,1)$.
                \begin{enumerate}
                    \item Write down the gradient of the line $L_2$. \hfill [1]
                    \item Hence, write down the equation of $L_2$. Leave your answer in the form \\ $y-a=m(x-b)$. \hfill [2]
                \end{enumerate}
        \end{enumerate}
        \begin{tikzpicture}
            \draw (0,3) rectangle (15,11);
            \draw [dotted] (1,10)--(14,10);
            \draw [dotted] (1,9)--(14,9);
            \draw [dotted] (1,8)--(14,8);
            \draw [dotted] (1,7)--(14,7);
        \end{tikzpicture}

\newpage
  \item Let $f$ be a quadratic function. Part of the graph of $f$ is shown below.\\*
  The vertex is at $P(4,-1)$ and the $y$-intercept is at $Q(0, 3)$.\\*

    \begin{figure}[!htbp]
    \begin{center}
    \begin{tikzpicture}

        %grid
        %\draw [thin, color=lightgray,, xstep=1.0cm,ystep=1.0cm] (-5.5,-5.5) grid (5.5,5.5);
        %\draw [thin, color=lightgray,, xstep=0.2cm,ystep=0.2cm] (-5.5,-1.5) grid (5.5,16.5);

        \foreach \x in {-2, -1,1,2,3,4,5,6}
        \draw[shift={(\x,0)},color=black] (0pt,-3pt) -- (0pt,3pt) node[below]  {$\x$};

        \foreach \y in {-1,1,2,3,4,5,6}
        \draw[shift={(0,\y)},color=black] (2pt,0pt) -- (-2pt,0pt) node[left]  {$\y$};

        \draw [thick, ->] (-2.5,0) -- (+6.5,0) node [right] {$x$};
        \draw [thick, ->] (0,-1.5) -- (0,6.5) node [left] {$y$};

        \draw (4,-1) circle[radius=2pt] node [below] {$P$};
        \fill (4,-1) circle[radius=2pt];
        \draw (0,3) circle[radius=2pt] node [right] {$Q$};
        \fill (0,3) circle[radius=2pt];

        \draw [<->] plot[domain= -0.75:7] (\x, .25*\x*\x -2*\x +3);

    \end{tikzpicture}
    \end{center}
    \end{figure}

    \begin{enumerate}
        \item The function $f$ can be written in the form $f(x)=a(x-h)^2 +k$. \\*
        Write down the value of $h$ and of $k$.
        \item Find $a$.
        \item Find the zeros of the function $f$, such that $f(x)=0$.
    \end{enumerate}
    \begin{tikzpicture}
        \draw (0,2) rectangle (15.5,11);
        \draw [dotted] (1,10)--(14,10);
        \draw [dotted] (1,9)--(14,9);
        \draw [dotted] (1,8)--(14,8);
        \draw [dotted] (1,7)--(14,7);
        \draw [dotted] (1,6)--(14,6);
    \end{tikzpicture}
 


       
\end{enumerate}
\end{document}