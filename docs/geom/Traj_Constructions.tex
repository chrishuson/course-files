\documentclass[12pt, twoside]{article}
\usepackage[letterpaper, margin=1in, headsep=0.2in]{geometry}
\setlength{\headheight}{0.6in}
%\usepackage[english]{babel}
\usepackage[utf8]{inputenc}
\usepackage{microtype}
\usepackage{amsmath}
\usepackage{amssymb}
%\usepackage{amsfonts}
\usepackage{siunitx} %units in math. eg 20\milli\meter
\usepackage{yhmath} % for arcs, overparenth command
\usepackage{tikz} %graphics
\usetikzlibrary{quotes, angles}
\usepackage{graphicx} %consider setting \graphicspath{{images/}}
\usepackage{parskip} %no paragraph indent
\usepackage{enumitem}
\usepackage{multicol}
\usepackage{venndiagram}

\usepackage{fancyhdr}
\pagestyle{fancy}
\fancyhf{}
\renewcommand{\headrulewidth}{0pt} % disable the underline of the header
\raggedbottom
\hfuzz=2mm %suppresses overfull box warnings

\usepackage{hyperref}

\fancyhead[LE]{\thepage}
\fancyhead[RO]{\thepage \\ Name: \hspace{4cm} \,\\}
\fancyhead[LO]{BECA / Dr. Huson / Geometry\\*  Learning Trajectories\\* 2022-2023}

\begin{document}

\subsubsection*{Learning Trajectory: Classical constructions}
\begin{enumerate}
  \item Elementary, single constuctions
  \begin{enumerate}
    \item Equilateral Triangle
    \item Duplicate a line segment
    \item Perpendicular (bisector, through a point on/off the line)
    \item Bisect an angle
    \item Duplicate an angle
    \end{enumerate}
  \item Triangle centers (perpendicular, bisectors, altitudes, medians)
  \item Hexagon and square inscribed in a circle.
  %\item Triangle midline
  \end{enumerate}

\begin{enumerate}
\subsubsection*{Equilateral triangle}
  \item Construct an equilateral triangle having one side on $\overrightarrow{T}$ with each leg congruent to $\overline{AB}$.
  [Leave all construction marks.]\\
    \vspace{5cm}
    \begin{center}
    \begin{tikzpicture}
      \draw [-, thick] (0,3)--(3,0);
      \draw [->, thick] (4,-3)--(11,-3);
      \draw [fill] (4,-3) circle [radius=0.05] node[above left]{$T$};
      \draw [fill] (0,3) circle [radius=0.05] node[above left]{$A$};
      %\node at (8.5,-0.4){$l$};
      \draw [fill] (3,0) circle [radius=0.05] node[below right]{$B$};
    \end{tikzpicture}
    \end{center}
    \vspace{2cm}

\newpage
\subsubsection*{Perpendicular (bisector, through a point on/off the line)}
  \item Construct a perpendicular bisector the given line segment $\overline{AB}$. Label the midpoint of $\overline{AB}$ as $M$.  [Leave all construction marks.]\\
    \vspace{2cm}
    \begin{center}
    \begin{tikzpicture}
      \draw [-, thick] (0,3)--(3,0);
      \draw [fill] (0,3) circle [radius=0.05] node[above left]{$A$};
      %\node at (8.5,-0.4){$l$};
      \draw [fill] (3,0) circle [radius=0.05] node[below right]{$B$};
    \end{tikzpicture}
    \end{center}
    \vspace{2cm}

\subsubsection*{Angle bisector}
  \item Construct an angle bisector the given angle $A$.  [Leave all construction marks.]\\
      \vspace{2cm}
      \begin{center}
      \begin{tikzpicture}
        \draw [<->, thick] (-2,5)--(0,0)--(7,0);
        \draw [fill] (0,0) circle [radius=0.05] node[below]{$A$};
        %\draw [fill] (7,0) circle [radius=0.05] node[below]{$N$};
      \end{tikzpicture}
      \end{center}

\newpage
\subsubsection*{Triangle centers}

  \item Construct a perpendicular to $\overline{AB}$ through $C$.\\
    %\hspace{1cm} Given the line  $l$ and point $P$.
    \vspace{1cm}
    \begin{center}
    \begin{tikzpicture}
      \draw [<->, thick] (0,0)--(11,0)--(7,3)--cycle;
      \draw [fill] (0,0) circle [radius=0.05] node[left]{$A$};
      \draw [fill] (11,0) circle [radius=0.05] node[right]{$B$};
      \draw [fill] (7,3) circle [radius=0.05] node[above right]{$C$};
    \end{tikzpicture}
    \end{center}
    \vspace{4cm}

  \item Construct the midpoint $M$ of $\overline{AC}$ by using the perpendicular bisector construction. Draw $\overline{BM}$, a \emph{median} of $\triangle ABC$.\\
  Spicy: Construct the other two medians, and hence, the centroid.
    %\vspace{1cm}
    \begin{center}
    \begin{tikzpicture}
      \draw [<->, thick] (0,0) node[left]{$A$}--
      (6.5,0) node[right]{$B$}
      --(6,4) node[above right]{$C$}
      --cycle;
    \end{tikzpicture}
  \end{center} %\vspace{1.5cm}

  \item Using  a  compass  and  straightedge,  construct  the  median  to  side $\overline{AC}$ in $\triangle ABC$ below.\\ (Leave all construction marks.)
    %\vspace{1cm}
    \begin{center}
    \begin{tikzpicture}
      \draw [<->, thick]
        (0,0) node[left]{$A$}--
        (10,-2) node[right]{$B$}--
        (4,-5) node[below]{$C$}
        --cycle;
    \end{tikzpicture}
    \end{center}

\newpage
    \item In the circle below, $\overline{AB}$ is a chord. Using a compass and straightedge, construct a perpendicular bisector of $\overline{AB}$, and hence, a diameter of the circle. [Leave all construction marks.]\\[4cm]
    \begin{center}
    \begin{tikzpicture}[scale=.635]
      %\draw [help lines] (-4,-4) grid (4,4);
      \draw [thick, -] (-5,0) node[left]{$A$} -- (0,5) node[above]{$B$};
      %\draw [thick, ->] (0,-2.2)--(0,10.4) node [left] {$y$};
      \draw (0,0) circle [radius=5]; %node[below]{$C$};
      %\draw [fill] (0,0) circle [radius=0.05];
    \end{tikzpicture}
    \end{center}

\newpage
  \item Spicy: Given $\angle ABC$, construct duplicate $\angle CDE$. (Leave all construction marks.)
    \begin{center}
    \begin{tikzpicture}
      \draw [<->, thick] (2,2)--(0,0)--(8,0);
      \draw [->, thick, dashed] (4,0)--(6,2) node[below right]{$E$};
      \draw [fill] (1.5,1.5) circle [radius=0.05] node[above left]{$A$};
      \draw [fill] (0,0) circle [radius=0.05] node[above left]{$B$};
      \draw [fill] (4,0) circle [radius=0.05] node[above left]{$D$};
      \draw [fill] (7,0) circle [radius=0.05] node[above right]{$C$};
    \end{tikzpicture}
    \end{center}

\newpage
\subsubsection*{Triangle congruence ($\triangle \cong$)}

  \item Function notation: $A \rightarrow A'$ is pronounced ``A gets mapped to A prime," or ``A corresponds to A prime."

  \item Given $\triangle ABC$, duplicate $\triangle ABC$ by duplicating each side. (``side-side-side" or ``SSS")
  \begin{enumerate}
    \item Construct $\overrightarrow{A'}$.
    \item Circle $A'$ with radius $AB$.
    \item Intersection $B'$.
    \item Circle $A'$ with radius $AC$.
    \item Circle $B'$ with radius $BC$.
    \item Intersection $C'$.
    \item $\triangle ABC \cong \triangle A'B'C'$ by the SSS $\triangle \cong$ Postulate.
  \end{enumerate}
  \begin{tikzpicture}[scale=0.6]
    \draw [thick] (0,0)--(5.5,0)--(3,4)--cycle;
    \draw [fill] (0,0) circle [radius=0.05] node[left]{$A$};
    \draw [fill] (5.5,0) circle [radius=0.05] node[right]{$B$};
    \draw [fill] (3,4) circle [radius=0.05] node[above]{$C$};
    \draw[thick,red] ([shift=(-10:5.5)]0,0) arc (-10:10:5.5);
    \draw[thick,red] ([shift=(45:5)]0,0) arc (45:65:5);
    \draw[thick,red] ([shift=(110:4.72)]5.5,0) arc (110:130:5.5);
  \end{tikzpicture}
  \begin{tikzpicture}[scale=0.6]
    \draw [dashed, thick] (0,0)--(5.5,0)--(3,4)--cycle;
    \draw [->, thick] (0,0)--(7.5,0);
    \draw [fill] (0,0) circle [radius=0.05] node[left]{$A'$};
    \draw [fill] (5.5,0) circle [radius=0.05] node[below right]{$B'$};
    \draw [fill] (3,4) circle [radius=0.05] node[above]{$C'$};
    \draw[thick,red] ([shift=(-10:5.5)]0,0) arc (-10:10:5.5);
    \draw[thick,red] ([shift=(45:5)]0,0) arc (45:65:5);
    \draw[thick,red] ([shift=(110:4.72)]5.5,0) arc (110:130:5.5);
  \end{tikzpicture}

\item The Side-side-side triangle congruence postulate (SSS $\triangle \cong$).\\
$\triangle ABC \cong \triangle A'B'C'$ iff $\overline{AB} \cong \overline{A'B'}, \overline{BC} \cong \overline{B'C'}, \text{ and } \overline{AC} \cong \overline{A'C'}$


\item Duplicate a given angle.\\[0.5cm]
  \hspace{1cm} Construct an angle with vertex $R$ and one leg the ray $\overrightarrow{R}$, congruent to $\angle A$. Show all construction marks.\\[0.5cm]
    Spicy: List the steps\\
    \vspace{1cm}
    \begin{center}
    \begin{tikzpicture}
      \draw [<->, thick] (3,6)--(0,0)--(9,0);
      \draw [fill] (0,0) circle [radius=0.05] node[below]{$A$};
      \draw [->, thick] (2,-8)--(10,-8);
      \draw [fill] (2,-8) circle [radius=0.05] node[below]{$R$};
    \end{tikzpicture}
    \end{center}
\newpage

\item Spicy: Construct the perpendicular bisectors of the legs of a triangle and, hence, the circumcenter.\\
  %\hspace{1cm} Given the line  $l$ and point $P$.
  \vspace{3cm}
  \begin{center}
  \begin{tikzpicture}
    \draw [<->, thick] (0,0)--(9,0)--(7,11)--cycle;
    %\draw [fill] (2,3) circle [radius=0.05] node[right]{$P$};
    %\node at (8.5,-0.4){$l$};
    %\draw [fill] (6,0) circle [radius=0.05] node[below]{$Q$};
  \end{tikzpicture}
  \end{center}

\item Construct a perpendicular bisector the given line segment $\overline{AB}$. Label the midpoint of $\overline{AB}$ as $M$.  [Leave all construction marks.]\\
  \vspace{2cm}
  \begin{center}
  \begin{tikzpicture}
    \draw [-, thick] (0,3)--(3,0);
    \draw [fill] (0,3) circle [radius=0.05] node[above left]{$A$};
    %\node at (8.5,-0.4){$l$};
    \draw [fill] (3,0) circle [radius=0.05] node[below right]{$B$};
  \end{tikzpicture}
  \end{center}
  \vspace{3cm}

\item Spicy: Construct a perpendicular to $\overline{AB}$ though $C$.\\
Hint: Start with a circle centered on $C$.
  %\hspace{1cm} Given the line  $l$ and point $P$.
  \vspace{4cm}
  \begin{center}
  \begin{tikzpicture}
    \draw [-, thick] (0,0)--(11,0);
    \draw [fill] (0,0) circle [radius=0.05] node[left]{$A$};
    \draw [fill] (11,0) circle [radius=0.05] node[right]{$B$};
    \draw [fill] (7,0) circle [radius=0.05] node[above right]{$C$};
  \end{tikzpicture}
\end{center} %\vspace{3cm}

\newpage
\subsubsection*{Construct a triangle's circumcenter}

\item Construct a perpendicular bisector of each of the legs of the triangle. Show their intersection, the circumcenter.\\[0.2cm]
Hint: Circles should be centered at the triangle vertices, but should only be sufficiently large to intersect the other circles.
  %\hspace{1cm} Given the line  $l$ and point $P$.
  \vspace{3cm}
  \begin{center}
  \begin{tikzpicture}
    \draw [<->, thick] (0,0)--(9,0)--(7,11)--cycle;
    %\draw [fill] (2,3) circle [radius=0.05] node[right]{$P$};
    %\node at (8.5,-0.4){$l$};
    %\draw [fill] (6,0) circle [radius=0.05] node[below]{$Q$};
  \end{tikzpicture}
  \end{center}

\newpage
\subsubsection*{Construct a triangle's centroid}
\item Bisect each leg of the triangle using only a compass and straightedge. Mark each midpoint, and draw a line (a \emph{median}) connecting it to the opposite vertex. Show the medians' intersection, the centroid.\\[0.2cm]
Hint: Circles should be centered at the triangle vertices, but should only be sufficiently large to intersect the other circles.
    %\hspace{1cm} Given the line  $l$ and point $P$.
    \vspace{3cm}
    \begin{center}
    \begin{tikzpicture}
      \draw [<->, thick] (0,0)--(9,0)--(7,11)--cycle;
      %\draw [fill] (2,3) circle [radius=0.05] node[right]{$P$};
      %\node at (8.5,-0.4){$l$};
      %\draw [fill] (6,0) circle [radius=0.05] node[below]{$Q$};
    \end{tikzpicture}
    \end{center}


\item Construct a perpendicular to $\overline{AB}$ though $C$.\\
  Hint: Start with a circle centered on $C$ that intersects $\overleftrightarrow{AB}$ in two places.
    %\hspace{1cm} Given the line  $l$ and point $P$.
    \vspace{2cm}
    \begin{center}
    \begin{tikzpicture}
      \draw [-, thick] (4,0)--(8,0);
      \draw [<->, dashed] (0,0)--(11,0);
      \draw [fill] (4,0) circle [radius=0.05] node[below]{$A$};
      \draw [fill] (8,0) circle [radius=0.05] node[below]{$B$};
      \draw [fill] (7,3) circle [radius=0.05] node[above right]{$C$};
    \end{tikzpicture}
  \end{center} \vspace{2cm}

\subsubsection*{Construct a triangle's orthocenter}
\item Construct a perpendicular to each of the leg of the triangle from the opposite vertex. Show their intersection, the orthocenter. Hint: you may extend the triangle sides as has been done for you on one side.%\\[0.2cm]
    %\hspace{1cm} Given the line  $l$ and point $P$.
    \vspace{3cm}
    \begin{center}
    \begin{tikzpicture}[scale=0.8]
      \draw [<->, thick] (0,0)--(9,0)--(5.5,5)--cycle;
      \draw [<->, dashed] (-1,0)--(11,0);

      %\draw [fill] (2,3) circle [radius=0.05] node[right]{$P$};
      %\node at (8.5,-0.4){$l$};
      %\draw [fill] (6,0) circle [radius=0.05] node[below]{$Q$};
    \end{tikzpicture}
    \end{center}

\newpage
\subsubsection*{Spicy: Construct a hexagon inscribed in a circle}
\item Construct an equilateral triangle on $\overline{AB}$ by drawing a circle centered on $A$. Continue with a second equilateral triangle on  $\overline{AC}$ by drawing a circle centered on $C$. Work around the circle $B$ four more times to construct the hexagon.
    %\hspace{1cm} Given the line  $l$ and point $P$.
    \vspace{3cm}
    \begin{center}
    \begin{tikzpicture}
      \draw [-, thick] (-6,0) node[left]{$A$}--(0,0);
      \draw  (0,0) circle [radius=6] node[right]{$B$};
      \draw [-, dashed] (120:6) node[above left]{$C$}--(0,0);
      %\node at (8.5,-0.4){$l$};
      %\draw [fill] (6,0) circle [radius=0.05] node[below]{$Q$};
    \end{tikzpicture}
    \end{center}


  \end{enumerate}
\end{document}
