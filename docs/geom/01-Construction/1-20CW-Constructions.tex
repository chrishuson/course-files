\documentclass[12pt, twoside]{article}
\documentclass[12pt, twoside]{article}
\usepackage[letterpaper, margin=1in, headsep=0.2in]{geometry}
\setlength{\headheight}{0.6in}
%\usepackage[english]{babel}
\usepackage[utf8]{inputenc}
\usepackage{microtype}
\usepackage{amsmath}
\usepackage{amssymb}
%\usepackage{amsfonts}
\usepackage{siunitx} %units in math. eg 20\milli\meter
\usepackage{yhmath} % for arcs, overparenth command
\usepackage{tikz} %graphics
\usetikzlibrary{quotes, angles}
\usepackage{graphicx} %consider setting \graphicspath{{images/}}
\usepackage{parskip} %no paragraph indent
\usepackage{enumitem}
\usepackage{multicol}
\usepackage{venndiagram}

\usepackage{fancyhdr}
\pagestyle{fancy}
\fancyhf{}
\renewcommand{\headrulewidth}{0pt} % disable the underline of the header
\raggedbottom
\hfuzz=2mm %suppresses overfull box warnings

\usepackage{hyperref}
\usepackage{float}

\fancyhead[LE]{\thepage}
\fancyhead[RO]{\thepage \\ First and last name: \hspace{2.5cm} \,\\ Section: \hspace{2.5cm} \,}
\fancyhead[LO]{BECA/Huson/Geometry: Construction \\* 2 October 2024}

\begin{document}
\subsubsection*{1.20 Classwork: Constructions using only a compass \& straightedge}
\begin{enumerate}[itemsep=0.5cm]
\item Construct an angle bisector of the given angle.
  \vspace{3cm}
  \begin{center}
  \begin{tikzpicture}
    \draw [<->, thick] (-7,5)--(0,0)--(3,5);
    %\draw [fill] (0,0) circle [radius=0.05] node[below]{$A$};
  \end{tikzpicture}
  \end{center} \vspace{1cm}

\item Construct a perpendicular bisector of $\overline{PQ}$.  
\vspace{4cm}
\begin{center}
\begin{tikzpicture}
  \draw [-, thick] (0,0)--(5,2);
  \draw [fill] (0,0) circle [radius=0.05] node[below]{$P$};
  \draw [fill] (5,2) circle [radius=0.05] node[below]{$Q$};
\end{tikzpicture}
\end{center} 
\vspace{3cm}

\newpage
\item Construct an equilateral triangle with one side $\overline{AB}$.  
\vspace{5cm}
\begin{center}
\begin{tikzpicture}
  \draw [-, thick] (0,0)--(5,-1);
  \draw [fill] (0,0) circle [radius=0.05] node[below]{$A$};
  \draw [fill] (5,-1) circle [radius=0.05] node[below]{$B$};
\end{tikzpicture}
\end{center}

\item Construct a perpendicular to line $l$ through the point $P$.  
    \vspace{4cm}
    \begin{center}
    \begin{tikzpicture}
        \draw [<->, thick] (0,0)--(10,6) node [below right]{$l$};
        \draw [fill] (4,2.4) circle [radius=0.05] node[below right]{$P$};
    \end{tikzpicture}
    \end{center}


\end{enumerate}
\end{document}