\documentclass[12pt, twoside]{article}
\documentclass[12pt, twoside]{article}
\usepackage[letterpaper, margin=1in, headsep=0.2in]{geometry}
\setlength{\headheight}{0.6in}
%\usepackage[english]{babel}
\usepackage[utf8]{inputenc}
\usepackage{microtype}
\usepackage{amsmath}
\usepackage{amssymb}
%\usepackage{amsfonts}
\usepackage{siunitx} %units in math. eg 20\milli\meter
\usepackage{yhmath} % for arcs, overparenth command
\usepackage{tikz} %graphics
\usetikzlibrary{quotes, angles}
\usepackage{graphicx} %consider setting \graphicspath{{images/}}
\usepackage{parskip} %no paragraph indent
\usepackage{enumitem}
\usepackage{multicol}
\usepackage{venndiagram}

\usepackage{fancyhdr}
\pagestyle{fancy}
\fancyhf{}
\renewcommand{\headrulewidth}{0pt} % disable the underline of the header
\raggedbottom
\hfuzz=2mm %suppresses overfull box warnings

\usepackage{hyperref}
\usepackage{float}

\title{Algebra 2}
\author{Chris Huson}
\date{September 2024}

\fancyhead[RO]{\\ First and last name: \hspace{2.5cm} \,\\ Section: \hspace{2.5cm} \,}
\fancyhead[LO]{BECA/Huson/Geometry: Construction \\* 16 September 2024}

\begin{document}
\subsubsection*{1.7 Do Now: Powers, radicals, constructions}
\begin{enumerate}[itemsep=0.5cm]

\item Memorize the single digit powers. \hfill \emph{3.OA.7 Fluently multiply and divide within 100}
    \begin{multicols}{2}
        \begin{enumerate}[itemsep=0.5cm]
            \item $3^2 =$
            \item $6^2 =$
            \item $5^2 =$
            \item $9^2 =$
            \item $4^2 =$
            \item $2^3 =$
        \end{enumerate}
    \end{multicols}

\item Memorize the square roots of whole numbers through 100 and cubes through five.
    \begin{multicols}{2}
        \begin{enumerate}[itemsep=0.5cm]
            \item $\sqrt{9} =$
            \item $\sqrt{47} =$
            \item $\sqrt{64} =$
            \item $\sqrt{36} =$
            \item $\sqrt[3]{8} =$
            \item $\sqrt[3]{27} =$
          \end{enumerate}
    \end{multicols} \vspace{0.25cm}


\item Perform each calculation, write down the full calculator display and then round to the \emph{nearest hundredth}.
    \begin{multicols}{2}
    \begin{enumerate}[itemsep=1.5cm]
      \item $A=15.944732$
      \item $W=3.4 \times 9.8 \times 4.3 \times 0.15$
      \item $V=\frac{1}{3} \pi (3.4)^2(6.1)$
      \item $V=199.19711$
    \end{enumerate}
    \end{multicols} \vspace{1cm}

\item Simplify each expression by ``collecting like terms''
\begin{enumerate}[itemsep=2cm]
    \begin{multicols}{2}
      \item $2x+4-x+11$
      \item $5y-4-7y+y$
      \item $14+5\pi-2\pi+4$
      \item $2a-7a+3\sqrt{5}+\sqrt{5}$
    \end{multicols}
    \end{enumerate}

\newpage
\subsubsection*{Constructions: Use only a compass and straightedge}

\item Construct an equilateral triangle with one side $\overline{AB}$.  [Leave all construction marks.]
\vspace{5cm}
\begin{center}
\begin{tikzpicture}
  \draw [-, thick] (0,0)--(5,0);
  \draw [fill] (0,0) circle [radius=0.05] node[left]{$A$};
  \draw [fill] (5,0) circle [radius=0.05] node[right]{$B$};
\end{tikzpicture}
\end{center} \vspace{2cm}

\item Construct a perpendicular bisector the given line segment $\overline{PQ}$. Label the midpoint of $\overline{PQ}$ as $M$. Mark the right angle with a small square and hash marks on the two congruent segments.
    \vspace{3cm}
    \begin{center}
    \begin{tikzpicture}
        \draw [-, thick] (0,3)--(3,0);
        \draw [fill] (0,3) circle [radius=0.05] node[above left]{$P$};
        %\node at (8.5,-0.4){$l$};
        \draw [fill] (3,0) circle [radius=0.05] node[below right]{$Q$};
    \end{tikzpicture}
    \end{center}
    \vspace{2cm}

\newpage
  \item Construct an angle bisector the given angle $A$.  [Leave all construction marks.]
      \vspace{4cm}
      \begin{center}
      \begin{tikzpicture}
        \draw [<->, thick] (3,6)--(0,0)--(8,0);
        \draw [fill] (0,0) circle [radius=0.05] node[below]{$A$};
        %\draw [fill] (7,0) circle [radius=0.05] node[below]{$N$};
      \end{tikzpicture}
      \end{center}

\newpage
\subsubsection*{Spicy: Construct a hexagon inscribed in a circle}
\item Construct an equilateral triangle on $\overline{AB}$ by drawing a circle centered on $A$. Continue with a second equilateral triangle on  $\overline{AC}$ by drawing a circle centered on $C$. Work around the circle $B$ four more times to construct the hexagon.
    %\hspace{1cm} Given the line  $l$ and point $P$.
    \vspace{3cm}
    \begin{center}
    \begin{tikzpicture}
      \draw [-, thick] (-6,0) node[left]{$A$}--(0,0);
      \draw  (0,0) circle [radius=6] node[right]{$B$};
      \draw [-, dashed] (120:6) node[above left]{$C$}--(0,0);
      %\node at (8.5,-0.4){$l$};
      %\draw [fill] (6,0) circle [radius=0.05] node[below]{$Q$};
    \end{tikzpicture}
    \end{center}

\end{enumerate}
\end{document}
