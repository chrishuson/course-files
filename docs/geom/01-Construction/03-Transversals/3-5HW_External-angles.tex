\documentclass[12pt, twoside]{article}
\usepackage[letterpaper, margin=1in, headsep=0.2in]{geometry}
\setlength{\headheight}{0.6in}
%\usepackage[english]{babel}
\usepackage[utf8]{inputenc}
\usepackage{microtype}
\usepackage{amsmath}
\usepackage{amssymb}
%\usepackage{amsfonts}
\usepackage{siunitx} %units in math. eg 20\milli\meter
\usepackage{yhmath} % for arcs, overparenth command
\usepackage{tikz} %graphics
\usetikzlibrary{quotes, angles}
\usepackage{graphicx} %consider setting \graphicspath{{images/}}
\usepackage{parskip} %no paragraph indent
\usepackage{enumitem}
\usepackage{multicol}
\usepackage{venndiagram}

\usepackage{fancyhdr}
\pagestyle{fancy}
\fancyhf{}
\renewcommand{\headrulewidth}{0pt} % disable the underline of the header
\raggedbottom
\hfuzz=2mm %suppresses overfull box warnings

\usepackage{hyperref}

\fancyhead[LE]{\thepage}
\fancyhead[RO]{\thepage \\ Name: \hspace{4cm} \,\\}
\fancyhead[LO]{BECA / Dr. Huson / Geometry\\*  Unit 3: Parallel lines and transversals\\* 24 October 2022}

\begin{document}

\subsubsection*{3.5 Homework: External angles of triangles}
\begin{enumerate}
\item Given $\triangle RSU$. If $m\angle UST=155^\circ$ and $m\angle R=60^\circ$, find $m\angle U$.
  \begin{flushright}
  \begin{tikzpicture}[scale=0.8]
    %\draw [->, thick] (0,0)--(5,5);
  \draw [<-, thick] (8,0)--
    (7,0) node[below]{$T$}--
    (0,0) node[below]{$R$}--
    (2.5,3) node[above]{$U$}--
    (5,0) node[below]{$S$};
  \end{tikzpicture}
  \end{flushright} \vspace{2cm}

\item Given  $\triangle EFG$ with $\overline{EF}$ extended to $A$. If $m\angle F=44^\circ$ and $m\angle G=92^\circ$, find $m\angle AEG$.
  \begin{center}
    \begin{tikzpicture}%[scale=0.7]
      \draw [thick](0,0)node[below]{$A$}--
        (2,0)node[below]{$E$}--
        (8,0)node[below]{$F$}--
        (4,3)node[above]{$G$} --(2,0);
    \end{tikzpicture}
  \end{center} \vspace{3cm}

\item The measures in degrees of the three angles of a triangle are $x$, $\frac{1}{2}x$, and $\frac{3}{2}x$. Find the measures of the triangle's angles. \vspace{4cm}

\newpage
\item Given $\triangle RSU$. If $m\angle UST=x$ and $m\angle R=x-80$, and $m\angle U=x-50$. Find $x$.
  \begin{flushright}
  \begin{tikzpicture}[scale=0.8]
    %\draw [->, thick] (0,0)--(5,5);
  \draw [<-, thick] (8,0)--
    (7,0) node[below]{$T$}--
    (0,0) node[below]{$R$}--
    (2.5,3) node[above]{$U$}--
    (5,0) node[below]{$S$};
  \end{tikzpicture}
  \end{flushright} \vspace{2cm}
  
\item Given isosceles $\triangle LMN$ with $\overline{LM} \cong \overline{NM}$. If $m\angle L=2x+20$ and $m\angle N=3x+5$, find $m\angle M$.
\begin{flushright}
\begin{tikzpicture}[scale=0.8]
  %\draw [->, thick] (0,0)--(5,5);
  \draw [-, thick] (0,0) node[below]{$L$}--
    (2.5,3) node[above]{$M$}--
    (5,0) node[below]{$N$}--cycle;
\end{tikzpicture}
\end{flushright} \vspace{3cm}

\item Given $\triangle RSU$. If $m\angle UST=x+50$, $m\angle R=x-20$, and $m\angle U=x+10$, find $m\angle R$.
  \begin{flushright}
  \begin{tikzpicture}[scale=0.8]
    %\draw [->, thick] (0,0)--(5,5);
    \draw [<-, thick] (8,0)--
      (7,0) node[below]{$T$}--
      (0,0) node[below]{$R$}--
      (2.5,3) node[above]{$U$}--
      (5,0) node[below]{$S$};
  \end{tikzpicture}
  \end{flushright} \vspace{2cm}

\newpage
\item In  $\triangle ABC$ shown below, side $\overline{AC}$ is extended to point $D$ with \\ $m\angle DAB=(11x+12)^\circ$, $m\angle C=(3x+3)^\circ$, and $m\angle B=(9x+2)^\circ$. \\[0.25cm]
Find $m\angle BAC$.
\begin{flushright}
    \begin{tikzpicture}
      \draw [thick](0,0)node[below]{$D$}--
        (1.8,0)node[below]{$A$}--
        (9,0)node[below]{$C$}--
        (4,3)node[above]{$B$} --(2,0);
        \node at (2.2,0)[above left]{$(11x+12)^\circ$};
        \node at (8.2,0)[above left]{$(3x+3)^\circ$};
        \node at (4.4,2.4)[below]{$(9x-2)^\circ$};
    \end{tikzpicture}
  \end{flushright} \vspace{1cm}

\item Given isosceles $\triangle RSU$ with $\overline{US} \cong \overline{RS}$. If $m\angle UST=150$ find $m\angle U$.
  \begin{flushright}
  \begin{tikzpicture}[scale=0.8]
    %\draw [->, thick] (0,0)--(5,5);
    \draw [<-, thick] (8,0)--
      (7,0) node[below]{$T$}--
      (0.5,0) node[below]{$R$}--
      (2,3) node[above]{$U$}--
      (5,0) node[below]{$S$};
  \end{tikzpicture}
  \end{flushright}

\item In  $\triangle ABC$ shown below, $m\angle A=(10x)^\circ$, $m\angle B=(16x-5)^\circ$, and $m\angle C=(2x+3)^\circ$. \\[0.25cm] 
  Find $m\angle A$. (show the check for full credit)
  \begin{flushright}
      \begin{tikzpicture}
        \draw [thick]
          (2,0)node[below]{$A$}--
          (9,0)node[below]{$C$}--
          (4.5,3)node[above]{$B$} --(2,0);
          \node at (3.3,0)[above]{$(10x)^\circ$};
          \node at (8.2,0)[above left]{$(2x+3)^\circ$};
          \node at (4.9,2.3)[below]{$(16x-5)^\circ$};
      \end{tikzpicture}
    \end{flushright}

\item In  $\triangle ABC$ shown below, $m\angle A=(10x)^\circ$, $m\angle B=(16x-5)^\circ$, and $m\angle C=(2x+3)^\circ$. \\[0.25cm] 
Find $m\angle A$. (show the check for full credit)
\begin{flushright}
    \begin{tikzpicture}
      \draw [thick]
        (2,0)node[below]{$A$}--
        (9,0)node[below]{$C$}--
        (4.5,3)node[above]{$B$} --(2,0);
        \node at (3.3,0)[above]{$(10x)^\circ$};
        \node at (8.2,0)[above left]{$(2x+3)^\circ$};
        \node at (4.9,2.3)[below]{$(16x-5)^\circ$};
    \end{tikzpicture}
  \end{flushright}

\item In  $\triangle ABC$ shown below, side $\overline{AC}$ is extended to point $D$ with $m\angle DAB=(6x-16)^\circ$, $m\angle C=(x+4)^\circ$, and $m\angle B=(4x+3)^\circ$. \\[0.25cm]
Find $m\angle BAC$.
\begin{flushright}
    \begin{tikzpicture}[scale=0.9]
      \draw [thick](0,0)node[below]{$D$}--
        (1.8,0)node[below]{$A$}--
        (9,0)node[below]{$C$}--
        (4,3)node[above]{$B$} --(2,0);
        \node at (2.2,0)[above left]{$(6x-16)^\circ$};
        \node at (8.2,0)[above left]{$(x+4)^\circ$};
        \node at (4.4,2.4)[below]{$(4x+3)^\circ$};
    \end{tikzpicture}
  \end{flushright} \vspace{1cm}

\item In $\triangle ABC$ shown below, $m\angle A=(5x+21)^\circ$, $m\angle B=(13x+4)^\circ$, and $m\angle C=(2x+15)^\circ$.\\[0.5cm] What is $m\angle A$?
\begin{center}
  \begin{tikzpicture}
    \draw [thick]
      (2,0)node[below]{$A$}--
      (9,0)node[below]{$C$}--
      (4,3)node[above]{$B$} --(2,0);
      \node at (3.3,0)[above]{$(5x+21)^\circ$};
      \node at (8.2,0)[above left]{$(2x+15)^\circ$};
      \node at (4.4,2.3)[below]{$(13x+4)^\circ$};
  \end{tikzpicture}
\end{center}


\end{enumerate}
\end{document}