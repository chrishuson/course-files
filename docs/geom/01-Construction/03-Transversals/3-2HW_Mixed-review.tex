% \documentclass[12pt, twoside]{article}
\usepackage[letterpaper, margin=1in, headsep=0.2in]{geometry}
\setlength{\headheight}{0.6in}
%\usepackage[english]{babel}
\usepackage[utf8]{inputenc}
\usepackage{microtype}
\usepackage{amsmath}
\usepackage{amssymb}
%\usepackage{amsfonts}
\usepackage[nomessages]{fp} %\FPeval{\var-name}{2*sin(pi/6)}
\usepackage{siunitx} %units in math. eg 20\milli\meter
\usepackage{yhmath} % for arcs, overparenth command
\usepackage{tikz} %graphics
\usetikzlibrary{quotes, angles, arrows, arrows.meta}
\usepackage{graphicx} %consider setting \graphicspath{{images/}}
\usepackage{parskip} %no paragraph indent
\usepackage{enumitem}
\usepackage{multicol}
\usepackage{venndiagram}

\usepackage{fancyhdr}
\pagestyle{fancy}
\fancyhf{}
\renewcommand{\headrulewidth}{0pt} % disable the underline of the header
\raggedbottom
\hfuzz=2mm %suppresses overfull box warnings

\usepackage{hyperref}

\fancyhead[LE]{\thepage}
\fancyhead[RO]{\thepage \\ Name: \hspace{4cm} \,\\}
\fancyhead[LO]{BECA / Dr. Huson / Geometry\\*  Unit 3: Parallel lines and transversals\\* 18 October 2022}

\begin{document}

\subsubsection*{3.2 Homework: Mixed review}
\begin{enumerate}
\item Demonstrate your ability to classify angles and use standard terminology.
\begin{enumerate}
\item Which of the following are true with respect to the angle, m$\angle PQR$?
  \begin{multicols}{2}
    True \hspace{0.25cm} False \hspace{0.25cm} It is a right angle \par \bigskip
    True \hspace{0.25cm} False \hspace{0.25cm} It's measure is $180^\circ$\par \bigskip
    True \hspace{0.25cm} False \hspace{0.25cm} $\overrightarrow{QP}$ is perpendicular to $ \overrightarrow{QR}$ \par \bigskip
    \columnbreak
    \begin{tikzpicture}[scale=0.7, rotate=-20]
      \draw[<->, thick] (4,0)--(0,0)--(0,3);
      \draw (0,0)++(0.3,0)--++(0,0.3)--+(-0.3,0);
      \draw[fill] (0,0) circle [radius=0.05] node[below]{$Q$};
      \draw[fill] (0,2) circle [radius=0.05] node[right]{$P$};
      \draw[fill] (3,0) circle [radius=0.05] node[above]{$R$};
    \end{tikzpicture}
  \end{multicols}
  \item What is the sum of the degree measures of this linear pair, $\angle ABD$ and $\angle CBD$?
  \begin{center}
    \begin{tikzpicture}[scale=.8, rotate=0]
      \draw  [<->, thick] (-3,0)--(3,0);
      \draw[->, thick] (0,0)--(2, 1) node[right]{$D$};
      \draw[fill] (-2,0) circle [radius=0.05] node[below]{$A$};
      \draw[fill] (0,0) circle [radius=0.05] node[below]{$B$};
      \draw[fill] (2,0) circle [radius=0.05] node[below]{$C$};
    \end{tikzpicture}
  \end{center}
  \item The given angle $\angle UVW$ is which of the following: acute, obtuse, or right?
  \begin{center}
    \begin{tikzpicture}[scale=.8]
      \draw  [<->, thick] (-3,0)--(0,0)--(35:3);
      \draw[fill] (-2,0) circle [radius=0.05] node[below]{$U$};
      \draw[fill] (0,0) circle [radius=0.05] node[below]{$V$};
      \draw[fill] (35:2) circle [radius=0.05] node[above left]{$W$};
    \end{tikzpicture}
  \end{center}
  \end{enumerate}


\item A linear pair is formed by two angles, m$\angle RUT = 110^\circ$ and m$\angle SUT = 5x + 20$. \par \bigskip 
  Write an equation, then solve for $x$. \vspace{0.5cm}
    \begin{flushright}
      \begin{tikzpicture}[scale=1]
        \draw[<->, thick]
          (0:3) coordinate (a) node[below left] {$S$}
          -- (0,0) coordinate (b) node[below] {$U$}
          -- (65:3) coordinate (c) node[above right] {$T$}
          pic["$5x + 20$", <->, draw=black, angle eccentricity=1.5, angle radius=1.5cm]
          {angle=a--b--c};
          \draw[<-, thick]
          (180:3) coordinate (d) node[below] {$R$}
          -- (0,0) coordinate (e)
          pic["$110^\circ$", <->, draw=black, angle eccentricity=1.5, angle radius=1.5cm]
          {angle=c--e--d};
      \end{tikzpicture}
    \end{flushright}

\newpage
\item Given m$\angle ABD = 4x-6$, m$\angle DBC = 5x+10$, and $m \angle ABC = 130^\circ$, as shown. \par \medskip
  Model the situation with an equation, then solve for $x$. Check your solution for full credit.
  \begin{flushright}
      \begin{tikzpicture}[scale=2, rotate=0]
        \draw[<->, thick]
          (-10:1.5) coordinate (a) node[below left] {$C$}
          -- (0,0) coordinate (b) node[below] {$B$}
          -- (70:2) coordinate (c) node[above right] {$D$}
          pic["$5x+10$", <->, draw=black, angle eccentricity=1.75, angle radius=1cm]
          {angle=a--b--c};
          \draw[<-, thick]
          (120:1.75) coordinate (d) node[below left] {$A$}
          -- (0,0) coordinate (e)
          pic["$4x-6$", <->, draw=black, angle eccentricity=1.5, angle radius=1cm]
          {angle=c--e--d};
      \end{tikzpicture}
    \end{flushright}

\item Given vertical angles, m$\angle APD = 4x+7$, m$\angle BPC = 6x-17$, as shown. \par \medskip
  Find $x$. Check your solution for full credit.
  \begin{flushright}
    \begin{tikzpicture}[scale=1.6, rotate=-35]
      \draw[<->, thick] 
        (0:-2)node[above right]{$A$}--
        (0:2)node[below left]{$C$};
      \draw[<->, thick] 
        (60:-2)node[above left] {$D$}--node[below]{$P$}
        (60:2)node[below right]{$B$};
      \node at (30:1.2){$6x-17$};
      \node at (30:-1.2){$4x+7$};
    \end{tikzpicture} 
    \end{flushright}
   
\item In the diagram shown, $\overrightarrow{BD} \perp \overleftrightarrow{ABC}$ with m$\angle DBE = 7x-1^\circ$ and m$\angle EBC = 6x^\circ$. 
  Find $x$. Show the check for full credit. \par \medskip
      \begin{tikzpicture}[scale=1]
        \draw[<->, thick]
          (0:5) coordinate (a) node[below left] {$C$}
          -- (0,0) coordinate (b) node[below] {$B$}
          -- (40:5) coordinate (c) node[below right] {$E$}
          pic["$6x$", <->, draw=black, angle eccentricity=1.5, angle radius=1.5cm]
          {angle=a--b--c};
          \draw[<-, thick]
          (90:4) coordinate (d) node[right] {$D$}
          -- (0,0) coordinate (e)
          pic["$7x-1$", <->, draw=black, angle eccentricity=1.5, angle radius=1.5cm]
          {angle=c--e--d};
          \draw[->, thick] (0,0)--(-180:2) node[below right]{$A$};
          \draw (0,0)++(-0.3,0)--++(0,0.3)--+(0.3,0);
      \end{tikzpicture}


\end{enumerate}
\end{document}