\documentclass[12pt, twoside]{article}
\usepackage[letterpaper, margin=1in, headsep=0.2in]{geometry}
\setlength{\headheight}{0.6in}
%\usepackage[english]{babel}
\usepackage[utf8]{inputenc}
\usepackage{microtype}
\usepackage{amsmath}
\usepackage{amssymb}
%\usepackage{amsfonts}
\usepackage{siunitx} %units in math. eg 20\milli\meter
\usepackage{yhmath} % for arcs, overparenth command
\usepackage{tikz} %graphics
\usetikzlibrary{quotes, angles}
\usepackage{graphicx} %consider setting \graphicspath{{images/}}
\usepackage{parskip} %no paragraph indent
\usepackage{enumitem}
\usepackage{multicol}
\usepackage{venndiagram}

\usepackage{fancyhdr}
\pagestyle{fancy}
\fancyhf{}
\renewcommand{\headrulewidth}{0pt} % disable the underline of the header
\raggedbottom
\hfuzz=2mm %suppresses overfull box warnings

\usepackage{hyperref}

\fancyhead[LE]{\thepage}
\fancyhead[RO]{\thepage \\ Name: \hspace{4cm} \,\\}
\fancyhead[LO]{BECA / Dr. Huson / Geometry\\*  Unit 2: Angles\\* 3 October 2022}

\begin{document}
\subsubsection*{2.1 Homework: Length and area test review}
\begin{enumerate}[itemsep=1.2cm]
\item Find the distance between $P$ and $Q$. 
  \begin{center}
    \begin{tikzpicture}
      \draw[<->] (-4.5,0)--(5.5,0);
      \foreach \x in {-4,...,5}
        \draw[shift={(\x,0)},color=black] (0pt,-3pt) -- (0pt,3pt) node[below=5pt]  {$\x$};
        \draw[fill] (-2,0) circle [radius=0.05] node[above] {$P(-2)$};
        \draw[fill] (4,0) circle [radius=0.05] node[above] {$Q(4)$};
    \end{tikzpicture}
  \end{center}

\item Find $RS$, given $R=0.7$ and $S=5.3$.
\begin{flushleft}
  \begin{tikzpicture}
    \draw[<->] (-1.5,0)--(6.5,0);
    \foreach \x in {-1,...,6}
      \draw[shift={(\x,0)},color=black] (0pt,-3pt) -- (0pt,3pt) node[below=5pt]  {$\x$};
      \draw[fill] (0.7,0) circle [radius=0.05] node[above] {$R$};
      \draw[fill] (5.3,0) circle [radius=0.05] node[above] {$S$};
  \end{tikzpicture}
\end{flushleft}

\item Find the area of $\triangle ABC$. The altitude $h$ of the triangle is $7$ centimeters and the base $AB=10$ cm. (diagram not to scale) \par \medskip
  \begin{tikzpicture}[scale=1.]
    \draw[thick]
      (2,0)node[below]{$A$}--
      (8,0)node[below]{$B$}--
      (4,3)node[above]{$C$} --(2,0);
  \draw[dashed] (4,0)--(4,3);
  \draw (4,0)++(0.3,0)--++(0,0.3)--+(-0.3,0);
  \node at (4,1.2)[right]{$h=7$};
  \node at (5,0)[below]{$10$ cm};
  \end{tikzpicture}

\item Solve each equation for $x$ then check your result.
\begin{multicols}{2}
  \begin{enumerate}
    \item $(3x+4) + (x-2)=22$
    \item $(6x-21) + (2x-3)=5x$
  \end{enumerate}
\end{multicols}

\newpage
\subsubsection*{Do Not Solve! Complete the diagram of the situation, model with an equation to the right, and circle where it states what to find.}
\item The point $Q$ is on the segment $\overline{PR}$ with $PQ=2x$, $QR=11$, and $PR=21$. Find $x$. \vspace{0.5cm}
  \begin{flushleft}
    \begin{tikzpicture}
    \draw[-, thick] (0,0)--(7,0);
    \draw[fill] (0,0) circle [radius=0.05];
    \draw[fill] (7,0) circle [radius=0.05];
  \end{tikzpicture}
  \end{flushleft}

\item The point $Q$ is the midpoint of $\overline{PR}$, $PQ=11$, and $QR=2x+1$. Find $x$. \vspace{0.5cm}
  \begin{flushleft}
    \begin{tikzpicture}
    \draw[-, thick] (0,0)--(7,0);
    \draw[fill] (0,0) circle [radius=0.05];
    \draw[fill] (7,0) circle [radius=0.05];
  \end{tikzpicture}
  \end{flushleft}

\item Given $\overline{PQR}$, with $PQ=3x-7$, $QR=x+3$, and $PR=12$. Find $PQ$. \vspace{0.5cm}
  \begin{flushleft}
    \begin{tikzpicture}
    \draw[-, thick] (0,0)--(7,0);
    \draw[fill] (0,0) circle [radius=0.05];
    \draw[fill] (7,0) circle [radius=0.05];
  \end{tikzpicture}
  \end{flushleft}

\item Given that $Q$ bisects $\overline{PR}$. $PQ=2x-5$, $PR=42$. Find $x$. \vspace{0.5cm}
  \begin{flushleft}
    \begin{tikzpicture}
    \draw[-, thick] (0,0)--(7,0);
    \draw[fill] (0,0) circle [radius=0.05];
    \draw[fill] (7,0) circle [radius=0.05];
  \end{tikzpicture}
  \end{flushleft}

\item Given collinear points $P$, $Q$, $R$, and $S$. Also, $PQ=3x$ and $PS=72$. Furthermore, $\overline{PQ} \cong \overline{QR} \cong \overline{RS}$. Find ${x}$. \vspace{0.5cm}
  \begin{flushleft}
    \begin{tikzpicture}
    \draw[-, thick] (0,0)--(7,0);
    \draw[fill] (0,0) circle [radius=0.05];
    \draw[fill] (7,0) circle [radius=0.05];
  \end{tikzpicture}
  \end{flushleft}

\item The points $P$, $Q$, and $R$ are collinear, with $PQ=x+4$ and $PR=27$. $\overline{QR}$ is twice the length of $\overline{PQ}$. Find $QR$. \vspace{0.5cm}
  \begin{flushleft}
    \begin{tikzpicture}
    \draw[-, thick] (0,0)--(7,0);
    \draw[fill] (0,0) circle [radius=0.05];
    \draw[fill] (7,0) circle [radius=0.05];
  \end{tikzpicture}
  \end{flushleft}


\end{enumerate}
\end{document}