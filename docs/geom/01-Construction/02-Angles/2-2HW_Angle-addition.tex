\documentclass[12pt, twoside]{article}
\usepackage[letterpaper, margin=1in, headsep=0.2in]{geometry}
\setlength{\headheight}{0.6in}
%\usepackage[english]{babel}
\usepackage[utf8]{inputenc}
\usepackage{microtype}
\usepackage{amsmath}
\usepackage{amssymb}
%\usepackage{amsfonts}
\usepackage{siunitx} %units in math. eg 20\milli\meter
\usepackage{yhmath} % for arcs, overparenth command
\usepackage{tikz} %graphics
\usetikzlibrary{quotes, angles}
\usepackage{graphicx} %consider setting \graphicspath{{images/}}
\usepackage{parskip} %no paragraph indent
\usepackage{enumitem}
\usepackage{multicol}
\usepackage{venndiagram}

\usepackage{fancyhdr}
\pagestyle{fancy}
\fancyhf{}
\renewcommand{\headrulewidth}{0pt} % disable the underline of the header
\raggedbottom
\hfuzz=2mm %suppresses overfull box warnings

\usepackage{hyperref}

\fancyhead[LE]{\thepage}
\fancyhead[RO]{\thepage \\ Name: \hspace{4cm} \,\\}
\fancyhead[LO]{BECA / Dr. Huson / Geometry\\*  Unit 2: Angles\\* 4 October 2022}

\begin{document}

\subsubsection*{2.2 Homework: Angle addition}
\begin{enumerate}
\item The size of an angle is its ``measure,'' which can be from $0^\circ$ to $360^\circ$
\begin{enumerate}
  \item Write down the name of this angle. Start with an angle symbol $\angle$. \\
  \hspace{1cm}
    \begin{tikzpicture}[scale=.8]
      \draw  [<->, thick] (0,2)--(2,0)--(4,1);
      \draw[fill] (1,1) circle [radius=0.05] node[below left]{$D$};
      \draw[fill] (2,0) circle [radius=0.05] node[below]{$E$};
      \draw[fill] (3,0.5) circle [radius=0.05] node[below]{$F$};
    \end{tikzpicture}
  \item What is the degree measure made by these two opposite rays, $\overrightarrow{BA}$ and $\overrightarrow{BC}$? \\[0.25cm]
  \hspace{5cm}
    \begin{tikzpicture}[scale=.8]
      \draw  [<->, thick] (0,2)--(4,0);
      \draw[fill] (1,1.5) circle [radius=0.05] node[below]{$A$};
      \draw[fill] (2,1) circle [radius=0.05] node[below]{$B$};
      \draw[fill] (3,0.5) circle [radius=0.05] node[below]{$C$};
    \end{tikzpicture}
    \item What is the degree measure of the angle, m$\angle XYZ$? \\[0.25cm]
    \begin{tikzpicture}[scale=0.7, rotate=20]
      \draw[<->, thick] (5,0)--(0,0)--(0,3);
      \draw (0,0)++(0.3,0)--++(0,0.3)--+(-0.3,0);
      \draw[fill] (0,0) circle [radius=0.05] node[below]{$Y$};
      \draw[fill] (0,2) circle [radius=0.05] node[right]{$X$};
      \draw[fill] (4,0) circle [radius=0.05] node[below]{$Z$};
    \end{tikzpicture}
  \end{enumerate}

\item Given the diagram, answer each using proper notation, including the angle symbol $\angle$. \par \vspace{0.3cm}
  \begin{enumerate}
    \item Name the ray opposite to $\overrightarrow{OR}$: \rule{4cm}{0.15mm} \bigskip
    \item What is the measure of $\angle POM$?  \rule{4cm}{0.15mm} \bigskip
    \item Name a right angle: \rule{4cm}{0.15mm} \bigskip
    \item Name the angle adjacent to $\angle QOR$: \rule{4cm}{0.15mm}
    \item Spicy: Are $\angle NOP$ and $\angle QOR$ complementary, supplementary, or neither?
  \end{enumerate}
  \begin{center}
  \begin{tikzpicture}[scale=1, rotate=20]
    \draw[<->, thick] (-55:3)--(0,0)--(125:4);
    \draw[<->, thick] (-5,0)--(3,0);
    \draw[->, thick] (0,0)--(0,3);
    \draw (0,0)++(0.3,0)--++(0,0.3)--+(-0.3,0);
    \draw[fill] (125:3) circle [radius=0.05] node[below left]{$N$};
    \draw[fill] (-4,0) circle [radius=0.05] node[below]{$M$}; 
    \draw[fill] (0,0) circle [radius=0.05] node[below left]{$O$};
    \draw[fill] (0,2) circle [radius=0.05] node[right]{$P$};
    \draw[fill] (2,0) circle [radius=0.05] node[below]{$Q$};
    \draw[fill] (-55:2) circle [radius=0.05] node[below left]{$R$};
  \end{tikzpicture}
  \end{center}

\newpage
\subsubsection*{Angle addition situations}
\item Apply the Angle Addition postulate. Write an equation to support your work.
  \begin{multicols}{2}
    Given m$\angle ABD = 50^\circ$, m$\angle ABC = 90^\circ$. \\[0.5cm]
    Find $m \angle DBC$. \\
    \begin{tikzpicture}[scale=1.4]
      \draw[<->, thick]
        (0:3) coordinate (a) node[below left] {$C$}
        -- (0,0) coordinate (b) node[below left] {$B$}
        -- (35:3) coordinate (c) node[below right] {$D$}
        pic["$?$", <->, draw=black, angle eccentricity=1.5, angle radius=1cm]
        {angle=a--b--c};
        \draw[<-, thick]
        (90:2) coordinate (d) node[right] {$A$}
        -- (0,0) coordinate (e)
        pic["$\; 50^\circ$", <->, draw=black, angle eccentricity=1.25, angle radius=1cm]
        {angle=c--e--d};
      \draw (0,0)++(0.3,0)--++(0,0.3)--+(-0.3,0);
    \end{tikzpicture}
  \end{multicols}

\item Given the angle measures and situation shown, write an equation and solve for $x$.
  \begin{multicols}{2}
    m$\angle ABD = 2x$ \\[0.25cm]
    m$\angle DBC = 2x + 15^\circ$ \\[0.25cm]
    $m \angle ABC = 115^\circ$ \\
    \begin{tikzpicture}[scale=2]
      \draw[<->, thick]
        (-20:2.5) coordinate (a) node[below left] {$C$}
        -- (0,0) coordinate (b) node[below left] {$B$}
        -- (30:3) coordinate (c) node[below right] {$D$}
        pic["\hspace{1cm}$2x + 15^\circ$", <->, draw=black, angle eccentricity=1.5, angle radius=1cm]
        {angle=a--b--c};
        \draw[<-, thick]
        (80:2) coordinate (d) node[right] {$A$}
        -- (0,0) coordinate (e)
        pic["$2x$", <->, draw=black, angle eccentricity=1.5, angle radius=1cm]
        {angle=c--e--d};
    \end{tikzpicture}
  \end{multicols}

\item The ray $\overrightarrow{BD}$ makes a $90^\circ$ angle with the line $\overleftrightarrow{ABC}$. Given the complementary angles' measures are m$\angle DBE = 3x + 20^\circ$ and m$\angle EBC = 25^\circ$. \\[0.25cm] 
  Find $x$, writing an equation to support your work.
    \begin{flushright}
      \begin{tikzpicture}[scale=1.3]
        \draw[<->, thick]
          (0:5) coordinate (a) node[below left] {$C$}
          -- (0,0) coordinate (b) node[below] {$B$}
          -- (35:5) coordinate (c) node[below right] {$E$}
          pic["$25$", <->, draw=black, angle eccentricity=2, angle radius=1cm]
          {angle=a--b--c};
          \draw[<-, thick]
          (90:3) coordinate (d) node[right] {$D$}
          -- (0,0) coordinate (e)
          pic["\hspace{0.3cm}$3x + 20^\circ$", <->, draw=black, angle eccentricity=1.6, angle radius=1cm]
          {angle=c--e--d};
          \draw[->, thick] (0,0)--(-180:2) node[below right]{$A$};
          \draw (0,0)++(-0.3,0)--++(0,0.3)--+(0.3,0);
      \end{tikzpicture}
    \end{flushright}

\end{enumerate}
\end{document}