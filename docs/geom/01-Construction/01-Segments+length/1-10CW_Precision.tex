% \documentclass[12pt, twoside]{article}
\usepackage[letterpaper, margin=1in, headsep=0.2in]{geometry}
\setlength{\headheight}{0.6in}
%\usepackage[english]{babel}
\usepackage[utf8]{inputenc}
\usepackage{microtype}
\usepackage{amsmath}
\usepackage{amssymb}
%\usepackage{amsfonts}
\usepackage[nomessages]{fp} %\FPeval{\var-name}{2*sin(pi/6)}
\usepackage{siunitx} %units in math. eg 20\milli\meter
\usepackage{yhmath} % for arcs, overparenth command
\usepackage{tikz} %graphics
\usetikzlibrary{quotes, angles, arrows, arrows.meta}
\usepackage{graphicx} %consider setting \graphicspath{{images/}}
\usepackage{parskip} %no paragraph indent
\usepackage{enumitem}
\usepackage{multicol}
\usepackage{venndiagram}

\usepackage{fancyhdr}
\pagestyle{fancy}
\fancyhf{}
\renewcommand{\headrulewidth}{0pt} % disable the underline of the header
\raggedbottom
\hfuzz=2mm %suppresses overfull box warnings

\usepackage{hyperref}

\fancyhead[LE]{\thepage}
\fancyhead[RO]{\thepage \\ Name: \hspace{4cm} \,\\}
\fancyhead[LO]{BECA / Dr. Huson / Geometry\\*  Unit 1: Segments, length, and area\\* 23 Sept 2022}

\begin{document}

\subsubsection*{1.10 Classwork: Precision and percent error}
\emph{Write formula for percent error in your notebook}
$$\epsilon = \left|\frac{v_A-v_E}{v_E}\right| \times 100\%$$
\begin{enumerate}
\item Round each value to the \emph{nearest thousandth}.
  \begin{multicols}{2}
    \begin{enumerate}
      \item $e=2.7182818...$ \par (Euler's number)
      \item $\pi$
      \item $\phi = 1.618033989...$ \par (the golden ratio)
      \item $\sqrt{3}$
    \end{enumerate}
  \end{multicols} \bigskip 

\item Round each value to the nearest hundred thousand.
  \begin{multicols}{2}
    \begin{enumerate}
      \item 1,694,251 \par \bigskip (population of the Manhattan)
      \item 2,405,464 \par \bigskip (population of Queens)
    \end{enumerate}
  \end{multicols}

\item Find the percent error for each approximation.
  \begin{multicols}{2}
    \begin{enumerate}[itemsep=4cm]
      \item $\displaystyle \pi \approx \frac{355}{113}$ (\href{https://en.wikipedia.org/wiki/Zu_Chongzhi}{Zu's ratio})
      \item $365 \text{ days} \approx 52 \text{ weeks}$
      \item $2^{10} 
      \approx 1000$ (kilobyte)
      \item $1 \text{ gallon} \approx 4 \text{ liters}$ \par (use conversion table's value)    
    \end{enumerate}
  \end{multicols}

\newpage
\item Convert each measure. Show the multiplication by the appropriate conversion factor (fraction), including units. \par \smallskip
  Example: Approximate the number of weeks in $T=2$ years.
  $$T = 2 \,\mathrm{years} \times \displaystyle \frac{52 \,\mathrm{weeks}}{1 \,\mathrm{year}} = 104 \,\mathrm{weeks}$$  \medskip
  \begin{enumerate}[itemsep=2cm]
    \item Find the length in yards of a quarter-mile track.
    \item Find the number of liters in a 15 gallon gas tank.
  \end{enumerate} \vspace{2cm}

\item Find the number of hours in 4 weeks. (multiply by two conversion factors, weeks to days, then days to hours) \vspace{2cm}

\item Find the area of the equilateral triangle two ways and quantify the error.
  \begin{multicols}{2}
  \begin{enumerate}[itemsep=2cm]
    \item Use the exact height of the triangle, $h=\sqrt{3}$.
    \item Assume the height is the same as the base, $h=2$.
    \item Calculate the percent error.
  \end{enumerate}
  \columnbreak
  \begin{flushright}
      \begin{tikzpicture}[scale=1.3]
    \draw[thick] (0,0)--(3,0)--(60:3)--cycle;
      \draw (1.4,-0.2)--(1.4,0.2);
      \draw (1.6,-0.2)--(1.6,0.2);
      \draw (53:1.4)--(68:1.4);
      \draw (53:1.6)--(67:1.6);
      \draw (28:2.4)--(28:2.8);
      \draw (32:2.4)--(32:2.8);
    \node at (1.5, -0.5){$b=2$};
    \draw[<->,dashed] (3.3,0)--(3.3,1.5*1.732);
    \node at (4,1.5){$h=\sqrt{3}$};
  \end{tikzpicture}
  \end{flushright}
\end{multicols}
  

\end{enumerate}
\end{document}