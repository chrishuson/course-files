\documentclass[12pt, twoside]{article}
\usepackage[letterpaper, margin=1in, headsep=0.2in]{geometry}
\setlength{\headheight}{0.6in}
%\usepackage[english]{babel}
\usepackage[utf8]{inputenc}
\usepackage{microtype}
\usepackage{amsmath}
\usepackage{amssymb}
%\usepackage{amsfonts}
\usepackage{siunitx} %units in math. eg 20\milli\meter
\usepackage{yhmath} % for arcs, overparenth command
\usepackage{tikz} %graphics
\usetikzlibrary{quotes, angles}
\usepackage{graphicx} %consider setting \graphicspath{{images/}}
\usepackage{parskip} %no paragraph indent
\usepackage{enumitem}
\usepackage{multicol}
\usepackage{venndiagram}

\usepackage{fancyhdr}
\pagestyle{fancy}
\fancyhf{}
\renewcommand{\headrulewidth}{0pt} % disable the underline of the header
\raggedbottom
\hfuzz=2mm %suppresses overfull box warnings

\usepackage{hyperref}

\fancyhead[LE]{\thepage}
\fancyhead[RO]{\thepage \\ Name: \hspace{4cm} \,\\}
\fancyhead[LO]{BECA / Dr. Huson / Geometry\\*  Unit 1: Segments, length, and area\\* 9 Sept 2022}

\begin{document}

\subsubsection*{1.2 Extension: Algebra with fractional coefficients}
\emph{A check is required for all algebra solutions}
\begin{enumerate}
\item Solve each equation for $x$.
  \begin{multicols}{2}
    \begin{enumerate}
      \item $\frac{1}{2}x=8$
      \item $\frac{1}{2}(x+5)=8$
    \end{enumerate}
  \end{multicols} \vspace{4cm}

\item As shown, three collinear points with $TU=2x+2$, $UV=x$, $TV=4x$. Find $TV$. \medskip
\begin{flushleft}
\begin{tikzpicture}
    \draw[-, thick] (0,0)--(9,0);
    \draw[fill] (0,0) circle [radius=0.05] node[below]{$T$};
    \draw[fill] (7,0) circle [radius=0.05] node[below]{$U$};
    \draw[fill] (9,0) circle [radius=0.05] node[below]{$V$};
    \node at (3,0) [above]{$2x+2$};
    \node at (8,0) [above]{$x$};
    \draw[<->, dashed] (0,-1)--(9,-1);
    \node at (4.5,-1) [below]{$4x$};
\end{tikzpicture}
\end{flushleft} \vspace{4cm}

\item As shown, three collinear points with $AB=\frac{1}{3}x+6$, $BC=x$, $AC=26$. Find $x$. \medskip
\begin{flushleft}
\begin{tikzpicture}
    \draw[-, thick] (0,0)--(9,0);
    \draw[fill] (0,0) circle [radius=0.05] node[below]{$A$};
    \draw[fill] (4,0) circle [radius=0.05] node[below]{$B$};
    \draw[fill] (9,0) circle [radius=0.05] node[below]{$C$};
    \node at (2,0) [above]{$\frac{1}{3}x+6$};
    \node at (6.5,0) [above]{$x$};
    \draw[<->, dashed] (0,-1)--(9,-1);
    \node at (4.5,-1) [below]{$26$};
\end{tikzpicture}
\end{flushleft} \vspace{5cm}

\newpage
\item Given $\overline{DEFG}$, $DE=3 \frac{1}{2}$, $EF=7 \frac{1}{2}$, and $FG= 2 \frac{1}{2}$. \par \medskip
Find ${DG}$, expressed as a fraction, not a decimal. \hfill  \emph{diagram not to scale} \medskip
\begin{flushleft}
    \begin{tikzpicture}
    \draw [-, thick] (0,0)--(8,0);
    \draw [fill] (0,0) circle [radius=0.05] node[below]{$D$};
    \draw [fill] (2.5,0) circle [radius=0.05] node[below]{$E$};
    \draw [fill] (6.5,0) circle [radius=0.05] node[below]{$F$};
    \draw [fill] (8,0) circle [radius=0.05] node[below]{$G$};
    \end{tikzpicture}
\end{flushleft} \vspace{2cm}

\item Find the value of each expression. (learn more by \emph{not} using a calculator)
  \begin{multicols}{2}
    \begin{enumerate}[itemsep=1cm]
      \item $|2-13|=$
      \item $|10+(-3)|=$
      \item $|4-(-2)|=$
      \item $|-5-(-7)|=$
    \end{enumerate}
  \end{multicols} \bigskip

\item Circle true or false for each statement. \bigskip
  \begin{itemize}[label={\textbf{T \; F \;}}, itemsep=0.5cm]
    \item There are two solutions for the equation $|x|=9$, $x=9$ and $x=-9$.
    \item If $x$ is negative, then $|x|$ must be positive.
    \item If $x$ is positive, then $|-x|$ is negative.
    \item The value of $|x|+|3|$ is always positive.
  \end{itemize} \bigskip

\item Rewrite the equation $|x+4|=7$ two ways (positive and negative 7). Then solve both equations to find all values of $x$ that satisfy $|x+4|=7$. (show the check for each solution)
  \begin{multicols}{2} 
    \begin{enumerate}
      \item positive
      \item negative
    \end{enumerate}
  \end{multicols}


\end{enumerate}
\end{document}