\documentclass[12pt, twoside]{article}
\usepackage[letterpaper, margin=1in, headsep=0.2in]{geometry}
\setlength{\headheight}{0.6in}
%\usepackage[english]{babel}
\usepackage[utf8]{inputenc}
\usepackage{microtype}
\usepackage{amsmath}
\usepackage{amssymb}
%\usepackage{amsfonts}
\usepackage{siunitx} %units in math. eg 20\milli\meter
\usepackage{yhmath} % for arcs, overparenth command
\usepackage{tikz} %graphics
\usetikzlibrary{quotes, angles}
\usepackage{graphicx} %consider setting \graphicspath{{images/}}
\usepackage{parskip} %no paragraph indent
\usepackage{enumitem}
\usepackage{multicol}
\usepackage{venndiagram}

\usepackage{fancyhdr}
\pagestyle{fancy}
\fancyhf{}
\renewcommand{\headrulewidth}{0pt} % disable the underline of the header
\raggedbottom
\hfuzz=2mm %suppresses overfull box warnings

\usepackage{hyperref}

\fancyhead[LE]{\thepage}
\fancyhead[RO]{\thepage \\ Name: \hspace{4cm} \,\\}
\fancyhead[LO]{BECA / Dr. Huson / Geometry\\*  Unit 1: Segments, length, and area\\* 15 Sept 2022}

\begin{document}

\subsubsection*{1.5 Homework: Polygons, perimeter}
\begin{enumerate}
\item A polygon with three sides is called a $\rule{4cm}{0.15mm}$.

\item Given isosceles $\triangle XYZ$ with $\overline{XY} \cong \overline{YZ}$. On the diagram mark the congruent line segments with tick marks.
\begin{center}
  \begin{tikzpicture}[scale=1]
    \draw[thick] (0,0)node[below left]{$X$}--
      (4,0) node[below]{$Y$}--
      (65:3.7) node[above]{$Z$}--cycle;
  \end{tikzpicture}
  \end{center} \bigskip

\item Given the rectangle $ABCD$ shown below.
  \begin{enumerate}
    \item Measure and mark the length and width of the rectangle in centimeters.
    \item Calculate its perimeter $P$. (show your work as an equation)
  \end{enumerate} \vspace{1cm}
  \begin{flushleft}
  \begin{tikzpicture}
    \draw [-, thick] (0,0)--(9,0)--(9,3)--(0,3)--cycle;
    \draw [fill] (0,0) circle [radius=0.05] node[left]{$A$};
    \draw [fill] (9,0) circle [radius=0.05] node[right]{$B$};
    \draw [fill] (9,3) circle [radius=0.05] node[right]{$C$};
    \draw [fill] (0,3) circle [radius=0.05] node[left]{$D$};
  \end{tikzpicture}
  \end{flushleft}

\item The perimeter of the isosceles $\triangle FGH$ is $21$ with $\overline{FH} \cong \overline{GH}$, $FG=x+2$, $FH=8$. Fill in the blanks then solve for $x$.
  \begin{flushright}
    $P = 8 + \rule{1cm}{0.15mm} +(x+2) = \rule{2cm}{0.15mm}$
  \end{flushright}
  \begin{tikzpicture}[scale=0.5]
    \draw [thick](0,0)--(4,0)--(2,6)--(0,0);
    \draw [fill] (0,0) circle [radius=0.05] node[below left]{$F$};
    \draw [fill] (4,0) circle [radius=0.05] node[below right]{$G$};
    \draw [fill] (2,6) circle [radius=0.05] node[above]{$H$};
    \draw [thick] (0.8,3.1)--(1.2,3); %tick mark
    \draw [thick] (2.8,3)--(3.2,3.1); %tick mark
    \node at (2,0) [below]{$x+2$};
    \node at (0.8,3.4) [left]{$8$};
  \end{tikzpicture}

\newpage
\item A square has four sides of equal length. Given $ABCD$ with $AB=3x$ and $BC=x+10$. Find the square's perimeter. (hint: first find $x$)
  \begin{flushleft}
  \begin{tikzpicture}[scale=0.7]
    \draw[thick]
      (0,0)node[below left]{$A$}--
      (6,0)node[below right]{$B$}--
      (6,6)node[above right]{$C$}--
      (0,6)node[above left]{$D$}--cycle;
    \node at (3,-0.8){$3x$};
    \node at (7,3.5){$x+10$};
    \draw (2.9, -0.2)--(2.9,0.2);
    \draw (2.9, 5.8)--(2.9,6.2);
    \draw (-0.2, 3)--(0.2, 3);
    \draw (5.8, 3)--(6.2, 3);
    \end{tikzpicture}
  \end{flushleft} \bigskip

\item Given $\overline{ABC}$, $AB=3.8$, and $BC=1.7$. Find ${AC}$. \par \bigskip
  \begin{tikzpicture}
    \draw [-, thick] (1,0)--(7,0);
    \draw [fill] (1,0) circle [radius=0.05] node[below]{$A$};
    \draw [fill] (5,0) circle [radius=0.05] node[below]{$B$};
    \draw [fill] (7,0) circle [radius=0.05] node[below]{$C$};
  \end{tikzpicture} \vspace{1cm}

\item Use symbols to write the name of each geometric figure.
\begin{enumerate}
  \begin{multicols}{3}
  \item
      \begin{tikzpicture}
      \draw [->, thick] (0,0)--(-3,1.5);
      \draw [fill] (0,0) circle [radius=0.05] node[below]{$B$};
      \draw [fill] (-2,1) circle [radius=0.05] node[below]{$A$};
      \end{tikzpicture} \bigskip
  \item \hspace{1cm}
      \begin{tikzpicture}[rotate=10]
      \draw [<->, thick] (1,0)--(1,3);
      \draw [fill] (1,0.5) circle [radius=0.05] node[right]{$C$};
      \draw [fill] (1,2.5) circle [radius=0.05] node[right]{$D$};
      \end{tikzpicture} \bigskip
  \item
    \begin{tikzpicture}[rotate=-40]
      \draw [-, thick] (1,0)--(0,2);
      \draw [fill] (1,0) circle [radius=0.05] node[below]{$E$};
      \draw [fill] (0,2) circle [radius=0.05] node[left]{$F$};
    \end{tikzpicture}
  \end{multicols}
  \end{enumerate} \vspace{1cm}

\item Given $Q$ bisects $\overline{PR}$, with $PQ=3x-12$, $QR=2x$. Find ${PR}$. \vspace{1cm}
\begin{center}
  \begin{tikzpicture}
    \draw [-, thick] (0,0)--(7,0);
    \draw [fill] (0,0) circle [radius=0.05] node[below]{$P$};
    \draw [fill] (3.5,0) circle [radius=0.05] node[below]{$Q$};
    \draw [fill] (7,0) circle [radius=0.05] node[below]{$R$};
  \end{tikzpicture}
\end{center} \vspace{1cm}
      
\end{enumerate}
\end{document}