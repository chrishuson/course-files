\documentclass[12pt, twoside]{article}
% \documentclass[12pt, twoside]{article}
\usepackage[letterpaper, margin=1in, headsep=0.2in]{geometry}
\setlength{\headheight}{0.6in}
%\usepackage[english]{babel}
\usepackage[utf8]{inputenc}
\usepackage{microtype}
\usepackage{amsmath}
\usepackage{amssymb}
%\usepackage{amsfonts}
\usepackage[nomessages]{fp} %\FPeval{\var-name}{2*sin(pi/6)}
\usepackage{siunitx} %units in math. eg 20\milli\meter
\usepackage{yhmath} % for arcs, overparenth command
\usepackage{tikz} %graphics
\usetikzlibrary{quotes, angles, arrows, arrows.meta}
\usepackage{graphicx} %consider setting \graphicspath{{images/}}
\usepackage{parskip} %no paragraph indent
\usepackage{enumitem}
\usepackage{multicol}
\usepackage{venndiagram}

\usepackage{fancyhdr}
\pagestyle{fancy}
\fancyhf{}
\renewcommand{\headrulewidth}{0pt} % disable the underline of the header
\raggedbottom
\hfuzz=2mm %suppresses overfull box warnings

\usepackage{hyperref}
\usepackage{float}

\fancyhead[LE]{\thepage}
\fancyhead[RO]{\thepage \\ First and last name: \hspace{2.5cm} \,\\ Section: \hspace{2.5cm} \,}
\fancyhead[LO]{BECA/Huson/Geometry: Congruence \\* 17 October 2024}

\begin{document}
\subsubsection*{2.3 Classwork: Segment lengths}
\begin{enumerate}[itemsep=0.5cm]
\item Use symbols to write the names of objects in the given figure.
\begin{multicols}{2}
  \begin{tikzpicture}
  \draw [-, thick] (0,0)--(4,3);
  \draw [fill] (0,0) circle [radius=0.05] node[below]{$A$};
  \draw [fill] (4,3) circle [radius=0.05] node[below right]{$B$};
\end{tikzpicture} \par
\columnbreak
\begin{enumerate}[itemsep=1cm]
  \item The two endpoints
  \item The name of the line segment
  \item Measure the segment. Write its length in centimeters (expressed as an equation).
\end{enumerate}
\end{multicols} \vspace{0.25cm}

\item A(n) $\rule{4cm}{0.15mm}$ is a portion of a straight line that includes two points and all of the points between the two points.

\item Given $\overline{PQ}$ as shown on the number line. \par \smallskip
  \begin{tikzpicture}
    \draw [<->] (-4.5,0)--(6.5,0);
    \foreach \x in {-4,...,6} %2 leading for diff!=1
      \draw[shift={(\x,0)},color=black] (0pt,-3pt) -- (0pt,3pt) node[below=5pt]  {$\x$};
      \draw [fill] (2,0) circle [radius=0.05] node[above] {$P$};
      \draw [fill] (6,0) circle [radius=0.05] node[above] {$Q$};
  \end{tikzpicture} \par \smallskip
  What is the distance between $P$ and $Q$? \par \smallskip
    $PQ=$ \vspace{1cm}
  
\item Two points $M(-1)$, $N(3)$ and the segment $\overline{MN}$ are shown on the number line. \par \smallskip
\begin{tikzpicture}
  \draw [<->] (-3.5,0)--(6.5,0);
  \draw [-, thick] (-1,0)--(3,0);
  \foreach \x in {-3,...,6} %2 leading for diff!=1
    \draw[shift={(\x,0)},color=black] (0pt,-3pt) -- (0pt,3pt) node[below=5pt]  {$\x$};
    \draw [fill] (-1,0) circle [radius=0.05] node[above] {$M$};
    \draw [fill] (3,0) circle [radius=0.05] node[above] {$N$};
\end{tikzpicture} \par \smallskip
What is the length of the segment $\overline{MN}$? Show your work as an equation. \vspace{2cm}

\item Can a length be a negative number? Can it be zero?

\newpage
\item Points $A=4$ and $B=16$ are shown below. Find ${AB}$. \par \smallskip
  \begin{tikzpicture}[scale=0.5]
    \draw [<->] (-1,0)--(21,0);
    \foreach \x in {0, 2,...,20} %2 leading for diff!=1
      \draw[shift={(\x,0)},color=black] (0pt,-6pt) -- (0pt,6pt) node[below=5pt]  {$\x$};
      \draw [fill] (4,0) circle [radius=0.1] node[above] {$A$};
      \draw [fill] (16,0) circle [radius=0.1] node[above] {$B$};
  \end{tikzpicture} \vspace{2cm}

\item Given $\overline{FG}$ as shown. What is the distance on the number line between the points? \par \smallskip
    \begin{tikzpicture}
      \draw [<->] (-2.5,0)--(6.5,0);
      \foreach \x in {-2,...,6} %2 leading for diff!=1
        \draw[shift={(\x,0)},color=black] (0pt,-3pt) -- (0pt,3pt) node[below=5pt]  {$\x$};
        \draw [thick] (1.25,0)--(5.5,0);
        \draw [fill] (1.25,0) circle [radius=0.05] node[above] {$F(1.25)$};
        \draw [fill] (5.5,0) circle [radius=0.05] node[above] {$G(5.50)$};
    \end{tikzpicture} \vspace{2cm}
  
\item Given $J(-3.0)$ and $K(2.5)$, as shown on the number line. 
Find the length of the line segment $\overline{JK}$. 
\begin{center}
  \begin{tikzpicture}
    \draw[<->] (-4.5,0)--(4.5,0);
    \foreach \x in {-4,...,4}
      \draw[shift={(\x,0)}] (0pt,-3pt)--(0pt,3pt) node[below=5pt]{$\x$};
    \draw[fill] (-3.0,0) circle [radius=0.05] node[above] {$J$};
    \draw[fill] (2.5,0) circle [radius=0.05] node[above] {$K$};
    \draw[-, thick] (-3,0)--(2.5,0);
  \end{tikzpicture}
\end{center} \vspace{3cm}

\item Terry is 63 inches tall and Steven is 68 inches tall. State who is taller and by how much. \vspace{2cm}

\item Dr. Huson bicycles from 80th Street to 164th Street (straight north). How many blocks is that?
  

\newpage
\item Given $\overline{RST}$, $RS=5$, and $RT=7 \frac{1}{2}$. Find ${ST}$.\par  \vspace{0.5cm}
  \begin{tikzpicture}
    \draw [-, thick] (1,0)--(7,0);
    \draw [fill] (1,0) circle [radius=0.05] node[below]{$R$};
    \draw [fill] (5,0) circle [radius=0.05] node[below]{$S$};
    \draw [fill] (7,0) circle [radius=0.05] node[below]{$T$};
  \end{tikzpicture}  \vspace{2cm}


\item Given $\overline{DEF}$, $DE=x+4$, $EF=x+2$, $DF=14$. Find ${DE}$.
  \begin{enumerate}
  \item Label the diagram with the given values.
  \begin{flushright}
    \begin{tikzpicture}
        \draw [-, thick] (0,0)--(6,0);
        \draw [fill] (0,0) circle [radius=0.05] node[below]{$D$};
        \draw [fill] (3.25,0) circle [radius=0.05] node[below]{$E$};
        \draw [fill] (6,0) circle [radius=0.05] node[below]{$F$};
    \end{tikzpicture}
  \end{flushright} \vspace{0.5cm}
  \item Write an equation: \vspace{0.5cm}
  \item Solve for $x$
  \vspace{3cm}
  \item Answer the question. \par \smallskip
  Find $DE$ by substituting for $x$. \vspace{1.5cm}
  \item Check your answer
  \end{enumerate} \vspace{1cm}

\item Points $P=14$ and $Q=47$ are shown below. Find ${PQ}$. \par \smallskip
  \begin{tikzpicture}[scale=0.22]
    \draw [<->] (-2,0)--(62,0);
    \foreach \x in {0, 5,...,60}
      \draw[shift={(\x,0)}] (0pt,-16pt)--(0pt,16pt)node[below=5pt]  {$\x$};
      \draw [fill] (14,0) circle [radius=0.2] node[above] {$P$};
      \draw [fill] (47,0) circle [radius=0.2] node[above] {$Q$};
  \end{tikzpicture} \vspace{2cm}

\newpage
\subsubsection*{2.3 Extension: Absolute value}
\item Write down the distance of each point from the origin. Use absolute value notation. \par \bigskip
  \begin{tikzpicture}
    \draw [<->] (-6.5,0)--(6.5,0);
    \foreach \x in {-6,...,6}
      \draw[shift={(\x,0)}] (0pt,-3pt)--(0pt,3pt) node[below=5pt]{$\x$};
      \draw[fill] (0,0) circle [radius=0.05] node[above]{origin};
      \draw[fill] (3.7,0) circle [radius=0.05] node[above]{$B(3.7)$};
      \draw[fill] (-4,0) circle [radius=0.05] node[above]{$A(-4)$};
  \end{tikzpicture} \bigskip
  \begin{multicols}{2}
    \begin{enumerate}
      \item[A.] $|-4|=$
      \item[B.] $\rule{4cm}{0.15mm}$
    \end{enumerate}
  \end{multicols}
  
\item Find the value of each expression.
  \begin{multicols}{2}
    \begin{enumerate}[itemsep=1cm]
      \item $|-3|=$
      \item $|5|=$
      \item $|-2.75|=$
      \item $|11-3|=$
      \item $|3-11|=$
      \item $|5+(-7)|=$
    \end{enumerate}
  \end{multicols}

\item Circle true or false for each statement. \bigskip
  \begin{itemize}[label={\textbf{T \; F \;}}, itemsep=0.5cm]
    \item The absolute value of any number must be postive or zero.
    \item In the equation $|x|=4$ the value of $x$ could be positive 4.
    \item If $x=-5$ then $|x|=5$.
    \item The following equation is never true for any $x$: $|x|=-10$.
  \end{itemize} \bigskip

\item Given that $x=-5$, find the value of each expression. \bigskip
  \begin{multicols}{2} 
    \begin{enumerate}[itemsep=1cm]
      \item $|x+2|=$
      \item $|-x|=$
      \item $|2x|=$
      \item $|6-x|=$
    \end{enumerate}
  \end{multicols}


\end{enumerate}
\end{document}


\item Given $\overline{LMN}$, $LM=3x+1$, $MN=7$, $LN=17$. Find ${x}$.\\[0.15in]
  \begin{tikzpicture}
   \draw [-, thick] (0,0)--(7,0);
   \draw [fill] (0,0) circle [radius=0.05] node[below]{$L$};
   \draw [fill] (4,0) circle [radius=0.05] node[below]{$M$};
   \draw [fill] (7,0) circle [radius=0.05] node[below]{$N$};
   \node at (1.7,0) [above]{$3x+1$};
   \node at (5.5,0) [above]{$7$};
   \draw [<->, dashed] (0,-1)--(7,-1);
   \node at (3.5,-1) [below]{$17$};
 \end{tikzpicture} %\vspace{1cm}
  \begin{enumerate}
  \item Write down an equation to represent the situation. \vspace{0.5cm}
  \item Solve for $x$. \vspace{1.5cm}
  \item Check your answer. \vspace{1.5cm}
  \end{enumerate}

