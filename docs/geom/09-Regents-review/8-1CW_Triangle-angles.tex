% \documentclass[12pt, twoside]{article}
\usepackage[letterpaper, margin=1in, headsep=0.2in]{geometry}
\setlength{\headheight}{0.6in}
%\usepackage[english]{babel}
\usepackage[utf8]{inputenc}
\usepackage{microtype}
\usepackage{amsmath}
\usepackage{amssymb}
%\usepackage{amsfonts}
\usepackage[nomessages]{fp} %\FPeval{\var-name}{2*sin(pi/6)}
\usepackage{siunitx} %units in math. eg 20\milli\meter
\usepackage{yhmath} % for arcs, overparenth command
\usepackage{tikz} %graphics
\usetikzlibrary{quotes, angles, arrows, arrows.meta}
\usepackage{graphicx} %consider setting \graphicspath{{images/}}
\usepackage{parskip} %no paragraph indent
\usepackage{enumitem}
\usepackage{multicol}
\usepackage{venndiagram}

\usepackage{fancyhdr}
\pagestyle{fancy}
\fancyhf{}
\renewcommand{\headrulewidth}{0pt} % disable the underline of the header
\raggedbottom
\hfuzz=2mm %suppresses overfull box warnings

\usepackage{hyperref}

\fancyhead[LE]{\thepage}
\fancyhead[RO]{\thepage \\ Name: \hspace{4cm} \,\\}
\fancyhead[LO]{BECA / Dr. Huson / Geometry\\*  Unit 8: Year-to-date Regents review\\* 13 February 2023}

\begin{document}

\subsubsection*{8.1 Classwork: External angles}
\begin{enumerate}
\item A triangle has two angles measuring $70^\circ$ and $60^\circ$ respectively. Find the measure of the third angle. \vspace{2cm}

\item Given  $\triangle LMN$ with $m\angle L=2x+20$, $m\angle N=3x-5$, and $m\angle M=x+15$. Find $x$.
  \begin{flushright}
  \begin{tikzpicture}[scale=0.8]
    %\draw [->, thick] (0,0)--(5,5);
    \draw [-, thick] (0,0) node[below]{$L$}--
      (2.5,3) node[above]{$M$}--
      (5,0) node[below]{$N$}--cycle;
  \end{tikzpicture}
  \end{flushright} \vspace{2cm}

\item The measures in degrees of the three angles of a triangle are $2x$, $x+10$, and $3x-40$. Find $x$. \vspace{4cm}

\item As shown below, triangle $ABC$ has $m\angle A = 52^\circ$ and $m\angle B = 48^\circ$. Find the measure of the external angle $\angle BCD = x$.
  \begin{flushleft}
    \begin{tikzpicture}[scale=0.7]
    \draw [thick]
    (10,0)node[below]{$D$}--
    (2,0)node[below]{$A$}--
    (4,4)node[above]{$B$}--
    (7,0)node[below]{$C$};
    \draw [fill] (10,0) circle [radius=0.07];
    %\draw [thick](2,0)node[below]{$E$}--(4,4);
    \node at (2.8,0.4){$52^\circ$};
    \node at (7.3,0.3){$x$};
    \node at (4.1,3.1){$48^\circ$};
  \end{tikzpicture}
  \end{flushleft}
  
\newpage
\item Given $\triangle ABC$ with $\overrightarrow{ACD}$.
  \begin{center}
    \begin{tikzpicture}[scale=0.5]
    \draw [thick, <-]
    (10,0)node[below]{$D$}--
    (0,0)node[below]{$A$}--
    (4,4)node[above]{$B$}--
    (7,0)node[below]{$C$};
    \node at (0.9,0.4){1};
    \node at (6.2,0.4){3};
    \node at (7.2,0.4){4};
    \node at (4,3.3){2};
  \end{tikzpicture}
  \end{center}
  Which equation is always true?
  \begin{multicols}{2}
  \begin{enumerate}
    \item $m\angle 3 = m\angle 1 + m\angle 2$
    \item $m\angle 3 = m\angle 1 - m\angle 2$ 
    \item $m\angle 4 = m\angle 1 + m\angle 2$
    \item $m\angle 4 = m\angle 3 - m\angle 2$
  \end{enumerate}
  \end{multicols}

\item In $\triangle ABC$ shown below, side $\overline{AC}$ is extended to point $D$ with $m\angle DAB=(180-2x)^\circ$, $m\angle C=(x-10)^\circ$, and $m\angle B=(3x+10)^\circ$. Solve for $x$.
  \begin{flushleft}
    \begin{tikzpicture}[scale=0.8]
      \draw [thick](-0.5,0)node[below]{$D$}--
        (1.8,0)node[below]{$A$}--
        (9,0)node[below left]{$C$}--
        (4,3)node[above]{$B$} --(2,0);
        \node at (2.3,0)[above left]{$(180-2x)^\circ$};
        \node at (8.1,0)[above left]{$(x-10)^\circ$};
        \node at (4.5,2.2)[below]{$(3x+10)^\circ$};
    \end{tikzpicture}
  \end{flushleft} \vspace{2cm}

\item A regular hexagon is rotated about its center. Which degree measure will carry the regular hexagon onto itself? 
\begin{multicols}{2}
  \begin{enumerate}
    \item $45^\circ$
    \item $90^\circ$
    \item $120^\circ$
    \item $135^\circ$
  \end{enumerate}
\end{multicols}

\item What is the smallest non-zero angle of rotation about its center that would map the octagon onto itself?
\begin{center}
    \begin{tikzpicture}[rotate=22.5, scale=1]
      \draw [thick]
      (0:2)--
      (45:2)--
      (90:2)--
      (135:2)--
      (180:2)--
      (225:2)--
      (270:2)--
      (315:2)--cycle;
    \end{tikzpicture}
  \end{center}




\end{enumerate}
\end{document}