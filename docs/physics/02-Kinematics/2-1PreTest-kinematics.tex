\documentclass[12pt]{article}
\usepackage[margin=1in]{geometry}
\usepackage{amsmath,amssymb}
\usepackage{siunitx}
\sisetup{per-mode=symbol}

\setlength{\parindent}{0pt}
\setlength{\parskip}{4pt}

\begin{document}

\begin{center}
{\Large Physics 9 Homework: Measurement, Units, and Intro to Motion}\\[4pt]
Name:\ \rule{2.5in}{0.4pt} \hfill Date:\ \rule{1.5in}{0.4pt}
\end{center}

\hrule
\vspace{0.5cm}

% ----------------------------------------------------------
\textbf{1. Rounding to 3 significant figures}

a) 0.003297 m \hfill Answer: \rule{1.5in}{0.4pt}

b) 93.085 kg \hfill Answer: \rule{1.5in}{0.4pt}

c) 5.9998 s \hfill Answer: \rule{1.5in}{0.4pt}

d) 12{,}450 N \hfill Answer: \rule{1.5in}{0.4pt}

\vspace{0.6cm}

% ----------------------------------------------------------
\textbf{2. Scientific notation (write each in the form $a\times10^n$ with $1\le a<10$)}

a) 0.000\,45 \hfill \rule{2in}{0.4pt}

b) 38{,}200 \hfill \rule{2in}{0.4pt}

c) 6.07 $\times 10^{-4}$ (write in standard/long form) \hfill \rule{2in}{0.4pt}

d) 1.002 $\times 10^{5}$ (write in standard/long form) \hfill \rule{2in}{0.4pt}

\vspace{0.6cm}

% ----------------------------------------------------------
\textbf{3. Operations in scientific notation}

Carry out the operation and give the answer to \emph{3 significant figures}.

a) $(3.2\times10^{3}) + (7.5\times10^{2}) =$ \rule{2.5in}{0.4pt}

b) $(6.40\times10^{-2}) \times (2.5\times10^{3}) =$ \rule{2.5in}{0.4pt}

c) $\dfrac{4.5\times10^{5}}{9.0\times10^{2}} =$ \rule{2.5in}{0.4pt}

\vspace{0.6cm}

% ----------------------------------------------------------
\textbf{4. Percent uncertainty / percent precision}

A length is measured as $12.4 \text{ cm} \pm 0.2 \text{ cm}$.

a) What is the range of possible values? \hfill \rule{2.5in}{0.4pt}

b) What is the percent uncertainty? (show setup) \vspace{1cm}

A mass is quoted as $0.815 \text{ kg}$ and the instrument is reliable to $\pm 0.005 \text{ kg}$.  
c) Percent uncertainty for the mass = \rule{2.5in}{0.4pt}

\vspace{0.6cm}

% ----------------------------------------------------------
\textbf{5. Unit conversions: length}

Show your factor(s). Keep 3 sig figs.

a) Convert $2.35 \text{ m}$ to cm. \hfill \rule{2.5in}{0.4pt}

b) Convert $7.20 \text{ km}$ to m. \hfill \rule{2.5in}{0.4pt}

c) Convert $18.0 \text{ in}$ to cm. (Use $1~\text{in} = 2.54~\text{cm}$ exactly.) \hfill \rule{2.5in}{0.4pt}

\vspace{0.6cm}

% ----------------------------------------------------------
\textbf{6. Unit conversions: mass}

a) Convert $0.650 \text{ kg}$ to g. \hfill \rule{2.5in}{0.4pt}

b) Convert $4.20 \times 10^{3} \text{ g}$ to kg. \hfill \rule{2.5in}{0.4pt}

c) A backpack has a mass of $15.5 \text{ lb}$. Using $1~\text{lb} = 0.454~\text{kg}$, find its mass in kg to 3 sig figs. \hfill \rule{2.5in}{0.4pt}

\vspace{0.6cm}

% ----------------------------------------------------------
\textbf{7. Unit conversions: speed}

a) Convert $55.0~\text{mi/h}$ to m/s. (Use $1~\text{mi} = 1609~\text{m}$ and $1~\text{h} = 3600~\text{s}$.) \vspace{1cm}

b) A runner moves at $4.20~\text{m/s}$. What is this speed in km/h? \hfill \rule{2.5in}{0.4pt}

c) A car moves at $22.5~\text{m/s}$. How many seconds does it take to travel $1.00~\text{km}$ at this speed? \hfill \rule{2.5in}{0.4pt}

\vspace{0.6cm}

% ----------------------------------------------------------
\textbf{8. Order-of-magnitude quick check} (circle one)

1) The mass of a textbook is closest to  
A) $10^{-3}$ kg \quad B) $10^{-1}$ kg \quad C) $10^{0}$ kg \quad D) $10^{2}$ kg

2) The length of a classroom is closest to  
A) $10^{-2}$ m \quad B) $10^{0}$ m \quad C) $10^{1}$ m \quad D) $10^{3}$ m

3) The time for light to cross a classroom ($\sim 10$ m) is closest to  
A) $10^{-8}$ s \quad B) $10^{-6}$ s \quad C) $10^{-3}$ s \quad D) $10^{0}$ s

\vspace{0.6cm}

% ----------------------------------------------------------
\textbf{9. One-dimensional displacement}

A student walks along a straight hallway. Take forward (to the right) as $+$.  
He starts at $x_0 = \SI{2.0}{m}$.

a) He walks to $x = \SI{7.5}{m}$. What is his displacement $\Delta x$? \hfill \rule{2in}{0.4pt}

b) From $x = \SI{7.5}{m}$ he walks back to $x = \SI{4.0}{m}$. What is his displacement for this part? \hfill \rule{2in}{0.4pt}

c) What is his \emph{total} displacement from the start at \SI{2.0}{m} to the end at \SI{4.0}{m}? \hfill \rule{2in}{0.4pt}

\vspace{0.6cm}

% ----------------------------------------------------------
\textbf{10. Average velocity from position--time}

A cart is at $x_1 = \SI{1.2}{m}$ at $t_1 = \SI{0.0}{s}$ and later is at $x_2 = \SI{5.8}{m}$ at $t_2 = \SI{3.0}{s}$.

a) Find the displacement. \hfill \rule{2.5in}{0.4pt}

b) Find the average velocity $v_\text{avg}$ during this interval. \hfill \rule{2.5in}{0.4pt}

c) Write the units for $v_\text{avg}$ clearly. \hfill \rule{1.5in}{0.4pt}

\vspace{0.6cm}

% ----------------------------------------------------------
\textbf{11. Motion with a direction change}

A cyclist rides east at constant speed. She is at $x = \SI{0}{m}$ at $t = \SI{0}{s}$. After \SI{12}{s} she is at $x = \SI{30}{m}$. She then turns around and rides west and is at $x = \SI{18}{m}$ at $t = \SI{20}{s}$.

a) Displacement from $t=0$ to $t=12~\text{s}$: \rule{2in}{0.4pt}

b) Average velocity from $t=0$ to $t=12~\text{s}$: \rule{2in}{0.4pt}

c) Displacement from $t=12~\text{s}$ to $t=20~\text{s}$: \rule{2in}{0.4pt}

d) Total displacement from $t=0$ to $t=20~\text{s}$: \rule{2in}{0.4pt}

e) Total distance traveled from $t=0$ to $t=20~\text{s}$: \rule{2in}{0.4pt}

\vspace{0.6cm}

% ----------------------------------------------------------
\textbf{12. Simple ``one-dimensional vector'' additions}

Treat right/east as $+$ and left/west as $-$.

a) $\Delta x_1 = +\SI{12}{m}$ and $\Delta x_2 = -\SI{5}{m}$. Net displacement = \rule{1.5in}{0.4pt}

b) A car drives $+0.85~\text{km}$, then $+1.40~\text{km}$, then $-0.50~\text{km}$. Net displacement (in km) = \rule{1.5in}{0.4pt}

c) Explain in one sentence why distance traveled is not always equal to displacement. \vspace{1cm}

\vspace{0.4cm}

% ----------------------------------------------------------
\textbf{13. Short application, like the textbook but with new numbers}

A car travels at $88~\text{km/h}$ along a straight road. The driver looks down at the radio for $1.6~\text{s}$.

a) Convert $88~\text{km/h}$ to m/s. \hfill \rule{2.5in}{0.4pt}

b) How far does the car travel in $1.6~\text{s}$ at this speed? Give 3 sig figs. \hfill \rule{2.5in}{0.4pt}

c) Why is it important to keep track of units in every step? \vspace{1cm}

\vspace{0.6cm}

% ----------------------------------------------------------
\textbf{14. Intro acceleration (very simple)}

A cart starts from rest and speeds up in a straight line with a constant acceleration of $0.80~\text{m/s}^2$.

a) What is its speed after $5.0~\text{s}$? (Use $v = v_0 + at$.) \hfill \rule{2.5in}{0.4pt}

b) In words, what does the number $0.80~\text{m/s}^2$ tell you about the motion? \vspace{1cm}

c) If the cart’s displacement in that time was $\SI{10.0}{m}$, what was its average velocity? \hfill \rule{2.5in}{0.4pt}

\vspace{0.6cm}

% ----------------------------------------------------------
\textbf{15. Mixed practice (convert + compute)}

A small drone flies north at $12.0~\text{m/s}$ for $45.0~\text{s}$.

a) Convert $12.0~\text{m/s}$ to km/h. \hfill \rule{2.5in}{0.4pt}

b) How far (in m) does it travel in $45.0~\text{s}$? \hfill \rule{2.5in}{0.4pt}

c) If we choose south as the negative direction, what sign should we give to the displacement you found in part (b)? Explain. \vspace{1cm}

% ----------------------------------------------------------
\end{document}