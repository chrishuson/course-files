% context: physics homework formatting rules
% source: ./latex_format_style.md

\documentclass[12pt, twoside]{article}
% \documentclass[12pt, twoside]{article}
\usepackage[letterpaper, margin=1in, headsep=0.2in]{geometry}
\setlength{\headheight}{0.6in}
%\usepackage[english]{babel}
\usepackage[utf8]{inputenc}
\usepackage{microtype}
\usepackage{amsmath}
\usepackage{amssymb}
%\usepackage{amsfonts}
\usepackage[nomessages]{fp} %\FPeval{\var-name}{2*sin(pi/6)}
\usepackage{siunitx} %units in math. eg 20\milli\meter
\usepackage{yhmath} % for arcs, overparenth command
\usepackage{tikz} %graphics
\usetikzlibrary{quotes, angles, arrows, arrows.meta}
\usepackage{graphicx} %consider setting \graphicspath{{images/}}
\usepackage{parskip} %no paragraph indent
\usepackage{enumitem}
\usepackage{multicol}
\usepackage{venndiagram}

\usepackage{fancyhdr}
\pagestyle{fancy}
\fancyhf{}
\renewcommand{\headrulewidth}{0pt} % disable the underline of the header
\raggedbottom
\hfuzz=2mm %suppresses overfull box warnings

\usepackage{hyperref}
\usepackage{float}

\title{IB}
\author{Chris Huson}
\date{November 2025}

\fancyhead[LE]{\thepage}
\fancyhead[RO]{\thepage \\ First \& last name: \hspace{2.25cm} \,\\ Grade: \hspace{2.25cm} \,}
\fancyhead[LO]{La Scuola d'Italia / Huson / Physics \\* 11 November 2025}

\begin{document}

\subsubsection*{2.2 Test: Precision, Scientific Notation, Vectors, Kinematics Intro}

\begin{enumerate}[itemsep=0.5cm]
\item Round each value to three significant figures.
  \begin{multicols}{2}
  \begin{enumerate}[itemsep=0.5cm]
    \item 0.004872
    \item 76.438
    \item 9.1245
    \item 24,670
  \end{enumerate}
  \end{multicols}
  \vspace{0.5cm}

\item Write each number in proper scientific notation ($a \times 10^k$ with $1 \le a < 10$, $k$ an integer).
  \begin{multicols}{2}
  \begin{enumerate}[itemsep=0.5cm]
    \item 0.000735
    \item 45,800
    \item 905,000,000
    \item 0.00710
  \end{enumerate}
  \end{multicols}
  \vspace{0.25cm}

\item Express in standard (long) form.
  \begin{multicols}{2}
  \begin{enumerate}[itemsep=0.5cm]
    \item $4.91\times10^{-3}$
    \item $8.35\times10^{4}$
  \end{enumerate}
  \end{multicols}
  \vspace{0.5cm}

\item Perform each operation and give the answer to 3 significant figures in scientific notation.
  \begin{multicols}{2}
  \begin{enumerate}[itemsep=1.5cm]
    \item $(2.85\times10^{3}) + (6.30\times10^{4})$
    \item $(4.50\times10^{-2}) \times (3.15\times10^{6})$
    \item $\dfrac{5.67\times10^{5}}{7.09\times10^{2}}$
  \end{enumerate}
  \end{multicols}
  \vspace{1cm}

\item A board’s length is measured as $8.6 \pm 0.3~\text{cm}$.
  \begin{enumerate}[itemsep=0.5cm]
    \item What is the range of possible values?
    \item What is the percent uncertainty? (show work)
  \end{enumerate}
  \vspace{0.5cm}

\newpage
\item A mass is measured as $2.240~\text{kg}$ with an uncertainty of $\pm 0.035~\text{kg}$.
  \begin{enumerate}[itemsep=0.5cm]
    \item Calculate the percent uncertainty.
    \item Which measurement (the board or this mass) is more precise? Explain briefly.
  \end{enumerate}
  \vspace{0.5cm}

\subsubsection*{Unit Conversions}

\item Convert each value. Show one line of work using unit factors.
  \begin{multicols}{2}
  \begin{enumerate}[itemsep=1.5cm]
    \item $3.42~\text{meters}$ to cm
    \item $5.85~\text{km}$ to meters
    \item $22.0~\text{in}$ to cm  \hfill (1 in = 2.54 cm)
    \item $0.780~\text{kg}$ to grams
    \item $3.65\times10^{3}~\text{grams}$ to kg
    \item $12.8~\text{lb}$ to kg  \hfill (1 lb = 0.454 kg)
  \end{enumerate}
  \end{multicols}
  \vspace{0.5cm}

\item Convert each speed.
  \begin{enumerate}[itemsep=1.8cm]
    \item $62.0~\text{miles/hour}$ to meters/second \hfill (1 mile = 1609 meters)
    \item $3.80~\text{m/s}$ to km/hour
    \item A car moves at $25.0~\text{m/s}$. How long does it take to travel $2~\text{kilometers}$?
  \end{enumerate}
  \vspace{0.5cm}

\newpage
\subsubsection*{Order of Magnitude Estimation}

\item Choose the best order of magnitude.
  \begin{enumerate}[itemsep=0.5cm]
    \item Mass of a smartphone: \quad A) $10^{-3}$ kg \quad B) $10^{-1}$ kg \quad C) $10^{1}$ kg \quad D) $10^{3}$ kg
    \item A pen's length in centimeters: \quad A) $10^{-2}$ cm \quad B) $10^{0}$ cm \quad C) $10^{1}$ cm \quad D) $10^{3}$ cm
  \end{enumerate}
  \vspace{0.5cm}

\subsubsection*{One-Dimensional Motion and Vectors}

\item A student walks down a hallway (forward is $+$). Start $x_0 = 1.5$ meters.
  \begin{enumerate}[itemsep=0.75cm]
    \item To $x = 9.0$ m, find displacement $\Delta x$.
    \item From $x=9.0$ m back to $x=3.5$ m, find $\Delta x$.
    \item Total displacement from start to finish.
  \end{enumerate}
  \vspace{0.5cm}

\item A cart moves from $x_1 = 0.8$ m at $t_1=0$ to $x_2 = 6.0$ m at $t_2 = 2.5$ seconds.
  \begin{enumerate}[itemsep=0.75cm]
    \item Find displacement.
    \item Find average velocity $v_\text{avg}$.
  \end{enumerate}
  \vspace{0.5cm}

\item A cyclist rides east, then west. Assume east is $+$.
  \begin{enumerate}[itemsep=1cm]
    \item From $x=0$ m to $x=30$ kilometers in 2.0 hours. Find the average velocity.
    \item Then back to $x=20$ km taking another hour. Find the average velocity for this leg.
    \item Find the total displacement and total distance traveled.
  \end{enumerate}
  \vspace{0.5cm}

\newpage
\item One-dimensional vector additions: find the total displacement.
  \begin{multicols}{2}
  \begin{enumerate}[itemsep=1cm]
    \item $\Delta x_1 = +15$ m, $\Delta x_2 = -8$ m
    \item $+1.10$ km, $+0.75$ km, $-0.40$ km
  \end{enumerate}
  \end{multicols}
  \vspace{0.5cm}

\item A car travels at $96~\text{km/h}$. The driver looks away for $2.0~\text{seconds}$.
  \begin{enumerate}[itemsep=1cm]
    \item Convert $96~\text{km/h}$ to m/s.
    \item How far does the car move in meters over the $2$ seconds?
  \end{enumerate}
  \vspace{0.5cm}

\item A cart starts from rest and accelerates at $1.5~\text{m/s}^2$.
  \begin{enumerate}[itemsep=1.5cm]
    \item Find its speed after $4.0~\text{seconds}$.
    \item Given that the displacement is $12.0$ meters over the four seconds, find average velocity.
  \end{enumerate}
  \vspace{1.5cm}

\item A drone flies north at $15~\text{m/s}$ for $5~\text{minutes}$.
  \begin{enumerate}[itemsep=1.5cm]
    \item Convert $15~\text{m/s}$ to km/h.
    \item Find distance traveled.
    \item If north is positive, what sign would you assign to this displacement?
  \end{enumerate}
  \vspace{0.5cm}

\end{enumerate}

\end{document}