\documentclass[12pt, twoside]{article}
\documentclass[12pt, twoside]{article}
\usepackage[letterpaper, margin=1in, headsep=0.2in]{geometry}
\setlength{\headheight}{0.6in}
%\usepackage[english]{babel}
\usepackage[utf8]{inputenc}
\usepackage{microtype}
\usepackage{amsmath}
\usepackage{amssymb}
%\usepackage{amsfonts}
\usepackage{siunitx} %units in math. eg 20\milli\meter
\usepackage{yhmath} % for arcs, overparenth command
\usepackage{tikz} %graphics
\usetikzlibrary{quotes, angles}
\usepackage{graphicx} %consider setting \graphicspath{{images/}}
\usepackage{parskip} %no paragraph indent
\usepackage{enumitem}
\usepackage{multicol}
\usepackage{venndiagram}

\usepackage{fancyhdr}
\pagestyle{fancy}
\fancyhf{}
\renewcommand{\headrulewidth}{0pt} % disable the underline of the header
\raggedbottom
\hfuzz=2mm %suppresses overfull box warnings

\usepackage{hyperref}
\usepackage{float}

\title{Algebra 2}
\author{Chris Huson}
\date{March 2024}

\fancyhead[RO]{\\ Name: \hspace{1.5cm} \,\\}
\fancyhead[LO]{BECA/Huson/Algebra 2: Complex Numbers \& Rational Exponents \\* 12 March 2024}

\begin{document}

\subsubsection*{3.20 PreTest: Solving quadratics, complex numbers, radicals and exponents}
\emph{Do Not Use a Calculator}\hfill A2.REI.4 Solve quadratic equations
\begin{enumerate}

\item Given the function $f(x)=(x-3)^2$. How many solutions are there to $f(x)=0$? Mark and label it on the graph.
\begin{center}
\begin{tikzpicture}[xscale=0.7, yscale=0.7]
    \draw [thick, ->] (-1.2,0) -- (5,0) node [below] {$x$};
    \draw [thick, ->] (0,-1.2)--(0,5.5) node [left] {$y$};
    \foreach \x in {3} \draw (\x cm,5pt) -- (\x cm,-5pt)node[below]{$3$};
    %\foreach \y in {-6,...,7} \draw (5pt,\y cm) -- (-5pt,\y cm);
    \draw [<->, thick, smooth, domain=1:5] plot (\x, {(\x-3)^2});
\end{tikzpicture}
\end{center}

\item How many solutions are there to $(x-3)^2=4$? Mark and label them on the graph.
\begin{center}
\begin{tikzpicture}[xscale=0.7, yscale=0.7]
    \draw [thick, ->] (-1.2,0) -- (6,0) node [below] {$x$};
    \draw [thick, ->] (0,-1.2)--(0,6.5) node [left] {$y$};
    \foreach \x in {1,3,5} \draw (\x cm,5pt) -- (\x cm,-5pt)node[below]{$\x$};
    \foreach \y in {4} \draw (5pt,\y cm) -- (-5pt,\y cm)node[above left]{$y=4$};
    \draw [<->, thick, smooth, domain=0.5:5.5] plot (\x, {(\x-3)^2});
    \draw [<->, thick, smooth, domain=-2:6] plot (\x, {4});
\end{tikzpicture}
\end{center}

\item How many, if any, solutions are there to $(x-3)^2=-1$? Mark and label it on the graph.
\begin{center}
\begin{tikzpicture}[xscale=0.7, yscale=0.7]
\draw [thick, ->] (-2.2,0) -- (6,0) node [below] {$x$};
\draw [thick, ->] (0,-2.2)--(0,6.5) node [left] {$y$};
\foreach \x in {1,3,5} \draw (\x cm,5pt) -- (\x cm,-5pt)node[below]{$\x$};
\foreach \y in {-1} \draw (5pt,\y cm) -- (-5pt,\y cm);
\draw [<->, thick, smooth, domain=0.5:5.5] plot (\x, {(\x-3)^2});
\draw [<->, thick, smooth, domain=-1:6] plot (\x, {-1})node[below left]{$y=-1$};
\end{tikzpicture}
\end{center}

\newpage
\item Given the quadratic equation, complete the square to determine the number of solutions: $$x^2 + 6x + 7 = 0$$
    \begin{enumerate}
        \item Find $\displaystyle \frac{b}{2}=$
        \item Find $\displaystyle \left( \frac{b}{2} \right)^2=$

        \item Rewrite the equation, adding or subtracting to both sides to complete the square. \vspace{1.5cm}
        \item How many solutions does the equation have? \vspace{0.5cm}
    \end{enumerate}

\item \hspace{5cm} $x^2 + 12x + 42 = 0$
    \begin{enumerate}
        \item Find $\displaystyle \frac{b}{2}=$
        \item Find $\displaystyle \left( \frac{b}{2} \right)^2=$

        \item Rewrite the equation, adding or subtracting to both sides to complete the square. \vspace{1.5cm}
        \item How many solutions does the equation have? \vspace{0.5cm}
    \end{enumerate}

\item \hspace{5cm} $x^2 + 14x + 49 = 0$
    \begin{enumerate}
        \item Find $\displaystyle \frac{b}{2}=$
        \item Find $\displaystyle \left( \frac{b}{2} \right)^2=$

        \item Rewrite the equation, adding or subtracting to both sides to complete the square. \vspace{1.5cm}
        \item How many solutions does the equation have? \vspace{0.5cm}
    \end{enumerate}

\newpage 
\item Square both sides of the equation and solve for $x$.
    \begin{multicols}{2}
    \begin{enumerate}[itemsep=0.5cm]
        \item  $\sqrt{x+9}=4$
        \item Check your solution.
    \end{enumerate}
    \end{multicols} \vspace{3cm}

\item Cube both sides of the equation and solve for $x$.
    \begin{multicols}{2}
    \begin{enumerate}[itemsep=0.5cm]
        \item  $\sqrt[3]{x-3}=3$
        \item Check your solution.
    \end{enumerate}
    \end{multicols} \vspace{4cm}

\item Solve for $x$ and check.
    \begin{multicols}{2}
    \begin{enumerate}[itemsep=0.5cm]
        \item  $\sqrt{2x+1} - 7 = -2$
        \item Check your solution.
    \end{enumerate}
    \end{multicols} \vspace{3cm}

\end{enumerate}
\end{document}