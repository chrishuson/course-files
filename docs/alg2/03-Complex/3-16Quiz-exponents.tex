\documentclass[12pt, twoside]{article}
\documentclass[12pt, twoside]{article}
\usepackage[letterpaper, margin=1in, headsep=0.2in]{geometry}
\setlength{\headheight}{0.6in}
%\usepackage[english]{babel}
\usepackage[utf8]{inputenc}
\usepackage{microtype}
\usepackage{amsmath}
\usepackage{amssymb}
%\usepackage{amsfonts}
\usepackage{siunitx} %units in math. eg 20\milli\meter
\usepackage{yhmath} % for arcs, overparenth command
\usepackage{tikz} %graphics
\usetikzlibrary{quotes, angles}
\usepackage{graphicx} %consider setting \graphicspath{{images/}}
\usepackage{parskip} %no paragraph indent
\usepackage{enumitem}
\usepackage{multicol}
\usepackage{venndiagram}

\usepackage{fancyhdr}
\pagestyle{fancy}
\fancyhf{}
\renewcommand{\headrulewidth}{0pt} % disable the underline of the header
\raggedbottom
\hfuzz=2mm %suppresses overfull box warnings

\usepackage{hyperref}
\usepackage{float}

\title{Algebra 2}
\author{Chris Huson}
\date{December 2024}

\fancyhead[LE]{\thepage}
\fancyhead[RO]{\thepage \\ First and last name: \hspace{2.5cm} \,\\ Section: \hspace{2.5cm} \,}
\fancyhead[LO]{BECA / Huson / Precalculus: Exponents \\* 6 December 2024}

\begin{document}

\subsubsection*{3.16 Quiz: Rational exponents and complex numbers \hfill A2.A.APR.6}
\begin{enumerate}[itemsep=0.5cm]

%\subsubsection*{A2-A.SSE.3c Apply the properties of exponents}

\item Square both sides of the equation and solve for $x$.
    \begin{multicols}{2}
    \begin{enumerate}[itemsep=0.5cm]
        \item  $\sqrt{x+7}=6$
        \item Check your solution.
    \end{enumerate}
    \end{multicols} \vspace{3cm}

\item Solve for $x$ and check.
    \begin{multicols}{2}
    \begin{enumerate}[itemsep=0.5cm]
        \item  $\sqrt{4x+5} + 5 = 14$
        \item Check your solution.
    \end{enumerate}
    \end{multicols} \vspace{3cm}

\item Write each expression in the form $a+bi$ with $a,b$ real numbers. \\[0.25cm]
Given  $s = 7 - 3i $ and $t = -2 + i$.
    \begin{enumerate}[itemsep=1.5cm]
        \item $s+t =$
        \item $s-t =$
        \item $st =$
    \end{enumerate} \vspace{3cm}

\newpage
\subsubsection*{HSN.RN.2 Expressions with radicals and rational exponents}
\item Simplify each radical expression.
    \begin{multicols}{2}
    \begin{enumerate}[itemsep=0.5cm]
        \item $\sqrt{49}=$
        \item $\sqrt{32}=$
        \item $\sqrt{-45}=$
        \item $\displaystyle \frac{\sqrt{-12}}{\sqrt{3}}=$
    \end{enumerate}
    \end{multicols} \vspace{1cm}
    
\item Simplify each expression.
    \begin{multicols}{2}
    \begin{enumerate}[itemsep=0.5cm]
        \item $\displaystyle 8^{\frac{2}{3}} =$
        \item $\left( \sqrt{\frac{4}{9}} \right)^{3} =$
    \end{enumerate}
    \end{multicols} \vspace{2cm}

    
\item Rewrite each expression to a fractional exponent in simplest terms.
    \begin{multicols}{2}
      \begin{enumerate}[itemsep=1cm]
          \item $\sqrt[3]{3} =$
          \item $\displaystyle \frac{1}{\sqrt[2]{3}}=$
          \item $\sqrt[4]{x^3} =$
          \item $\displaystyle \frac{1}{(\sqrt[4]{x})^2}=$
      \end{enumerate}
      \end{multicols} \vspace{1cm}
  
\item Rewrite each expression with fractional exponent as a radical.
    \begin{multicols}{2}
      \begin{enumerate}[itemsep=1cm]
        \item $\displaystyle 3^{\frac{1}{2}}=$
        \item $\displaystyle 3^{-\frac{2}{3}}=$
        \item $\displaystyle x^{\frac{1}{3}}=$
        \item $\displaystyle x^{-\frac{2}{3}}=$
      \end{enumerate}
      \end{multicols}

\newpage


%\subsubsection*{6.EE.b Reason about and solve one-variable equations and inequalities}

\item The expression $\displaystyle 2 - \frac{x - 1}{x + 2}$ is equivalent to 
\begin{multicols}{2}
    \begin{enumerate}
    \item $\displaystyle 1 - \frac{3}{x + 2}$
    \item $\displaystyle 1 + \frac{3}{x + 2}$ 
    \item $\displaystyle 1 - \frac{1}{x + 2}$
    \item $\displaystyle 1 + \frac{1}{x + 2}$ 
    \end{enumerate} 
\end{multicols} \vspace{3cm}

\item Find all of the values of $x$ that make the equation true. 
$$\frac{3}{x-4} = \frac{x-5}{x}$$ \vspace{4cm}


\item Select all of the solutions to $(x-4)^2=7$. \hfill (HSN.CN.2 Complex numbers)
    \begin{multicols}{2}
    \begin{enumerate}
        \item $x= 4+7i$
        \item $x= 4-7i$
        \item $x= 4 - \sqrt{7}$
        \item $x= 4-7 = -3$
        \item $x= 4+7 = 11$
        \item $x= 4 + \sqrt{7}$
    \end{enumerate}
    \end{multicols}

\end{enumerate}
\end{document}
