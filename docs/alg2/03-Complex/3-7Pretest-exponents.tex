\documentclass[12pt, twoside]{article}
% \documentclass[12pt, twoside]{article}
\usepackage[letterpaper, margin=1in, headsep=0.2in]{geometry}
\setlength{\headheight}{0.6in}
%\usepackage[english]{babel}
\usepackage[utf8]{inputenc}
\usepackage{microtype}
\usepackage{amsmath}
\usepackage{amssymb}
%\usepackage{amsfonts}
\usepackage[nomessages]{fp} %\FPeval{\var-name}{2*sin(pi/6)}
\usepackage{siunitx} %units in math. eg 20\milli\meter
\usepackage{yhmath} % for arcs, overparenth command
\usepackage{tikz} %graphics
\usetikzlibrary{quotes, angles, arrows, arrows.meta}
\usepackage{graphicx} %consider setting \graphicspath{{images/}}
\usepackage{parskip} %no paragraph indent
\usepackage{enumitem}
\usepackage{multicol}
\usepackage{venndiagram}

\usepackage{fancyhdr}
\pagestyle{fancy}
\fancyhf{}
\renewcommand{\headrulewidth}{0pt} % disable the underline of the header
\raggedbottom
\hfuzz=2mm %suppresses overfull box warnings

\usepackage{hyperref}
\usepackage{float}

\title{Algebra 2}
\author{Chris Huson}
\date{February 2024}

\fancyhead[RO]{\\ Name: \hspace{1.5cm} \,\\}
\fancyhead[LO]{BECA/Huson/Algebra 2: Complex Numbers \& Rational Exponents \\* 7 February 2024}

\begin{document}

\subsubsection*{3.7 Pretest: Working with exponents \hfill A.SSE.3c Exponent properties}
\emph{Do Not Use a Calculator}
\begin{enumerate}
\item Select all of the solutions to $x^2=16$.
    \begin{multicols}{2}
    \begin{enumerate}
        \item $x=4$
        \item $x=-4$
        \item $x=8$
        \item $x=-8$
        \item $x=16$
        \item $x=-16$
    \end{enumerate}
    \end{multicols}

    \item Find the value of each variable that makes the equation true.
    \begin{multicols}{2}
    \begin{enumerate}[itemsep=0.5cm]
        \item $5^2 \cdot 5^3 = 5^a \qquad a=$
        \item $\displaystyle \frac{3^7}{3^6} = 3^b \hspace{1.4cm}  b=$
        \item $7^c=1 \hspace{1.6cm} c=$
        \item $(4^3)^5 = 4^d \qquad d=$
        \item $\displaystyle 2^e = \frac{1}{2} \hspace{1.5cm}  e=$
        \item $3^4 \cdot f^4 = 15^4 \quad f=$
    \end{enumerate}
    \end{multicols} \vspace{0.5cm}

\item Evaluate each expression.
    \begin{multicols}{2}
    \begin{enumerate}[itemsep=0.5cm]
        \item $\displaystyle \frac{1}{4} \cdot 24$
        \item $\displaystyle \frac{3}{2} \cdot 10$
        \item $\displaystyle \frac{3}{5} \cdot 8 \cdot \frac{5}{3}$
        \item $\displaystyle \frac{2}{3} \cdot \frac{5}{2} \cdot 9$
    \end{enumerate}
    \end{multicols} \vspace{0.5cm}

\item $p = 3x +1$ and $q = 2x-5$. \hfill (AI-A.APR.1 Add, subtract, \& multiply polynomials) \\[0.5cm]
For each expression, write an equivalent expression and simplify.
    \begin{enumerate}[itemsep=0.5cm]
        \item $p+q$
        \item $p-q$
        \item $pq$
    \end{enumerate}

\newpage
\subsubsection*{A2-F.BF.2 Write arithmetic and geometric sequences with recursive formulas}
\item Given the geometric sequence beginning $a_1 = 2$, $a_2 = 1$, $a_3 = \frac{1}{2}$, $a_4 = \frac{1}{4}, \ldots$      
    \begin{enumerate}[itemsep=2cm]
        \item Write a recursive definition of the sequence.
        \item Write a formula expression of the sum of the first 10 terms of the sequence. (You do not need to calculate the sum's value.)
    \end{enumerate}
    \vspace{2cm}

\item Given the function $f(x)=(2x+5)(x+7)(x-1)$. \hfill (AII-F.IF.7c Graph polynomials)
    \begin{center}
    \begin{tikzpicture}[xscale=0.7, yscale=0.7]
        \draw [thick, ->] (-7.2,0) -- (7.5,0) node [above] {$x$};
        \draw [thick, ->] (0,-6.2)--(0,7.5) node [right] {$y$};
        %\foreach \x in {-7,...,7} \draw (\x cm,5pt) -- (\x cm,-5pt);
    \end{tikzpicture}
    \end{center}
    \begin{enumerate}
        \item Sketch a graph of the function.
        \item Mark and label all $x$-intercepts of the graph.
        \item Calculate the function's $y$-intercept and mark it on the graph.
    \end{enumerate}


\end{enumerate}
\end{document}