\documentclass[12pt, twoside]{article}
\documentclass[12pt, twoside]{article}
\usepackage[letterpaper, margin=1in, headsep=0.2in]{geometry}
\setlength{\headheight}{0.6in}
%\usepackage[english]{babel}
\usepackage[utf8]{inputenc}
\usepackage{microtype}
\usepackage{amsmath}
\usepackage{amssymb}
%\usepackage{amsfonts}
\usepackage{siunitx} %units in math. eg 20\milli\meter
\usepackage{yhmath} % for arcs, overparenth command
\usepackage{tikz} %graphics
\usetikzlibrary{quotes, angles}
\usepackage{graphicx} %consider setting \graphicspath{{images/}}
\usepackage{parskip} %no paragraph indent
\usepackage{enumitem}
\usepackage{multicol}
\usepackage{venndiagram}

\usepackage{fancyhdr}
\pagestyle{fancy}
\fancyhf{}
\renewcommand{\headrulewidth}{0pt} % disable the underline of the header
\raggedbottom
\hfuzz=2mm %suppresses overfull box warnings

\usepackage{hyperref}
\usepackage{float}

\title{Algebra 2}
\author{Chris Huson}
\date{September 2024}

\fancyhead[RO]{\\ First and last name: \hspace{2.5cm} \,\\ Section: \hspace{2.5cm} \,}
\fancyhead[LO]{BECA/Huson/Precalculus: Sequences \\* 5 September 2024}

\begin{document}
\begin{itemize}
    \item[$\square$] I brought a math notebook to class today.
    \item[$\square$] I brought a math folder for handouts.
\end{itemize}


\subsubsection*{1.1 Do Now: Multiplication tables \hfill Mental math - no calculators}
\begin{enumerate}[itemsep=0.5cm]

\item 3.OA.7 Fluently multiply and divide within 100
    \begin{multicols}{2}
    \begin{enumerate}[itemsep=0.5cm]
        \item $5 \times 6 =$
        \item $7 \times 3 =$
        \item $9 \times 8 =$
        \item $3 \times 4 =$
        \item $8 \times 7 =$
        \item $4 \times 6 =$
    \end{enumerate}
    \end{multicols}

\item 3.OA.7 Use the relationship between multiplication and division, know from memory all products of two one-digit numbers.
    \begin{multicols}{2}
    \begin{enumerate}[itemsep=0.5cm]
        \item $20 \div 4 =$
        \item $48 \div 8 =$
        \item $32 \div 4 =$
        \item $45 \div 9 =$
        \item $18 \div 6 =$
        \item $42 \div 7 =$
    \end{enumerate}
    \end{multicols}

\item Convert between fractions and percentages.
\begin{multicols}{2}
\begin{enumerate}[itemsep=0.5cm]
    \item $\frac{1}{4}=$
    \item $\frac{1}{3}=$
    \item $\frac{5}{4}=$
    \item $50\% =$
    \item $75\% =$
    \item $66 \frac{2}{3}\% =$
\end{enumerate}
\end{multicols}

\item Simplify the expression by combining like terms.
    \begin{multicols}{2}
    \begin{enumerate}[itemsep=0.5cm]
        \item $3x+2x=$
        \item $4y-2y=$
        \item $5x-3x+2x=$
        \item $-7y+3y-2y=$
        \item $3x^2+2x^2=$
        \item $-4y^2+2y^2=$
    \end{enumerate}
    \end{multicols}

\newpage

\item Perform the operations and simplify the expression.
\begin{multicols}{2}
\begin{enumerate}[itemsep=0.5cm]
    \item $\displaystyle \frac{1}{4} + \frac{1}{4} =$
    \item $\displaystyle \frac{3}{10} + \frac{2}{5} =$
    \item $\displaystyle \frac{2}{3} + \frac{1}{3} =$
    \item $\displaystyle \frac{1}{2} - \frac{1}{6} =$
    \item $\displaystyle \frac{3}{4} - \frac{1}{8} =$
    \item $\displaystyle \frac{1}{2} - \frac{1}{4} =$
\end{enumerate}
\end{multicols}


\item Round to the accuracy stated.
\begin{multicols}{2}
\begin{enumerate}[itemsep=0.75cm]
    \item nearest hundredth: $0.125$
    \item nearest tenth: $5.7111$
    \item nearest thousandth: $11.54795$
    \item nearest tenth: $9.9505$
    \item nearest hundredth: $\pi$
    \item nearest hundredth: $\sqrt{2}$
\end{enumerate}
\end{multicols} \vspace{0.5cm}

\item 6.EE.A.1 Evaluate numerical expressions involving whole-number exponents.
    \begin{multicols}{2}
    \begin{enumerate}[itemsep=0.5cm]
        \item $3^2=$
        \item $7^2=$
        \item $9^2=$
        \item $2^3=$
        \item $3^3=$
        \item $4^3=$
    \end{enumerate}
    \end{multicols}

\item 8.EE.A.2 Evaluate square roots of small perfect squares and cube roots of small perfect cubes.
    \begin{multicols}{2}
    \begin{enumerate}[itemsep=0.5cm]
        \item $\sqrt{25}=$
        \item $\sqrt{64}=$
        \item $\sqrt{4}=$
        \item $\sqrt{36}=$
        \item $\sqrt[3]{1}=$
        \item $\sqrt[3]{125}=$
    \end{enumerate}
    \end{multicols}

\end{enumerate}
\end{document}

3.OA.7
Fluently multiply and divide within 100, using such as the relationship between multiplication and division. By the end of Grade 3, know from memory all products of two one-digit numbers.
