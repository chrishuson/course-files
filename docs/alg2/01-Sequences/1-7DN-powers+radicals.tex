\documentclass[12pt, twoside]{article}
\documentclass[12pt, twoside]{article}
\usepackage[letterpaper, margin=1in, headsep=0.2in]{geometry}
\setlength{\headheight}{0.6in}
%\usepackage[english]{babel}
\usepackage[utf8]{inputenc}
\usepackage{microtype}
\usepackage{amsmath}
\usepackage{amssymb}
%\usepackage{amsfonts}
\usepackage{siunitx} %units in math. eg 20\milli\meter
\usepackage{yhmath} % for arcs, overparenth command
\usepackage{tikz} %graphics
\usetikzlibrary{quotes, angles}
\usepackage{graphicx} %consider setting \graphicspath{{images/}}
\usepackage{parskip} %no paragraph indent
\usepackage{enumitem}
\usepackage{multicol}
\usepackage{venndiagram}

\usepackage{fancyhdr}
\pagestyle{fancy}
\fancyhf{}
\renewcommand{\headrulewidth}{0pt} % disable the underline of the header
\raggedbottom
\hfuzz=2mm %suppresses overfull box warnings

\usepackage{hyperref}
\usepackage{float}

\title{Algebra 2}
\author{Chris Huson}
\date{September 2024}

\fancyhead[RO]{\\ First and last name: \hspace{2.5cm} \,\\ Section: \hspace{2.5cm} \,}
\fancyhead[LO]{BECA/Huson/Precalculus: Sequences \\* 16 September 2024}

\begin{document}
\subsubsection*{1.8 Do Now: Powers and radicals, sequences}
\begin{enumerate}[itemsep=0.5cm]

\item Memorize the single digit powers. \hfill \emph{3.OA.7 Fluently multiply and divide within 100}
    \begin{multicols}{2}
        \begin{enumerate}[itemsep=0.5cm]
            \item $3^2 =$
            \item $6^2 =$
            \item $5^2 =$
            \item $9^2 =$
            \item $4^2 =$
            \item $2^3 =$
        \end{enumerate}
    \end{multicols}

\item Memorize the square roots of whole numbers through 100 and cubes through five.
    \begin{multicols}{2}
        \begin{enumerate}[itemsep=0.5cm]
            \item $\sqrt{9} =$
            \item $\sqrt{47} =$
            \item $\sqrt{64} =$
            \item $\sqrt{36} =$
            \item $\sqrt[3]{8} =$
            \item $\sqrt[3]{27} =$
          \end{enumerate}
    \end{multicols} \vspace{0.25cm}


\item Perform each calculation, write down the full calculator display and then round to the \emph{nearest hundredth}.
    \begin{multicols}{2}
    \begin{enumerate}[itemsep=1.5cm]
      \item $A=15.944732$
      \item $W=3.4 \times 9.8 \times 4.3 \times 0.15$
      \item $V=\frac{1}{3} \pi (3.4)^2(6.1)$
      \item $V=199.19711$
    \end{enumerate}
    \end{multicols} \vspace{1cm}

\item Simplify each expression by ``collecting like terms''
\begin{enumerate}[itemsep=2cm]
    \begin{multicols}{2}
      \item $2x+4-x+11$
      \item $5y-4-7y+y$
      \item $14+5\pi-2\pi+4$
      \item $2a-7a+3\sqrt{5}+\sqrt{5}$
    \end{multicols}
    \end{enumerate}

\newpage
\item Circle whether the sequence is arithmetic, geometric, or neither.
  \begin{enumerate}[itemsep=0.5cm]
    \item $2, 4, 6, 8, \dots$ \hspace{1.6cm} arithmetic, geometric, neither
    \item $1, 2, 4, 7, 11, \dots$ \hspace{1cm} arithmetic, geometric, neither
    \item $3, 6, 12, 24, \dots$ \hspace{1.2cm} arithmetic, geometric, neither
    \item $13, 10, 7, 4, 1, \dots$ \hspace{0.9cm} arithmetic, geometric, neither
    \item $\displaystyle \frac{1}{2}, \frac{1}{4}, \frac{1}{8}, \dots$ \hspace{1.8cm} arithmetic, geometric, neither
  \end{enumerate}


\item Write a recursive formula for the sequence $5, 10, 15, 20, \dots$ \vspace{3cm}
\item Write a recursive formula for the sequence $3, 9, 27, 81, \dots$ \vspace{3cm}

\item A metal sculpture is made from welded steel rods. The first rod is 3 feet long. Each successive rod is 80\% of the length of the previous rod. Indicate whether each formula correctly defines the length $L(n)$ of the $n$th rod by circling True or False.
  \begin{enumerate}[label=(\alph*)]
    \item T \; F \qquad $L(n) = 3(0.8)^n$
    \item T \; F \qquad $L(n) = 3(0.8)^{n-1}$
    \item T \; F \qquad $L(n) = 3-0.20n$
    \item T \; F \qquad $L(1) = 3$, $L(n) = L(n-1)(0.8) \text{ for } n \ge 2$
  \end{enumerate}


\end{enumerate}
\end{document}
