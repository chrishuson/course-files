\documentclass[12pt, twoside]{article}
% \documentclass[12pt, twoside]{article}
\usepackage[letterpaper, margin=1in, headsep=0.2in]{geometry}
\setlength{\headheight}{0.6in}
%\usepackage[english]{babel}
\usepackage[utf8]{inputenc}
\usepackage{microtype}
\usepackage{amsmath}
\usepackage{amssymb}
%\usepackage{amsfonts}
\usepackage[nomessages]{fp} %\FPeval{\var-name}{2*sin(pi/6)}
\usepackage{siunitx} %units in math. eg 20\milli\meter
\usepackage{yhmath} % for arcs, overparenth command
\usepackage{tikz} %graphics
\usetikzlibrary{quotes, angles, arrows, arrows.meta}
\usepackage{graphicx} %consider setting \graphicspath{{images/}}
\usepackage{parskip} %no paragraph indent
\usepackage{enumitem}
\usepackage{multicol}
\usepackage{venndiagram}

\usepackage{fancyhdr}
\pagestyle{fancy}
\fancyhf{}
\renewcommand{\headrulewidth}{0pt} % disable the underline of the header
\raggedbottom
\hfuzz=2mm %suppresses overfull box warnings

\usepackage{hyperref}
\usepackage{float}

\title{Algebra 2}
\author{Chris Huson}
\date{September 2024}

\fancyhead[RO]{\\ First and last name: \hspace{2.5cm} \,\\ Section: \hspace{2.5cm} \,}
\fancyhead[LO]{BECA/Huson/Precalculus: Sequences \\* 12 September 2024}

\begin{document}
\subsubsection*{1.7 Quiz challenge problems \hfill \textnormal{\textit{HSF-IF.A.3 Recognize sequences, define them recursively}}}

\begin{enumerate}[itemsep=1.5cm]
    \item Given the arithmetic sequence $f(n)$ whose first two terms are 4 and 9.
    \begin{enumerate}[itemsep=1.5cm]
        \item Write down $f(2)$
        \item Write down the value of the common difference $d$
        \item Find $f(3)$
        \item Write an equation relating $f(5)$ and $f(6)$
    \end{enumerate}

    \item Given the geometric sequence $g(n)$ whose first term is 3 with a growth rate of $r=2$.
    \begin{enumerate}[itemsep=1.5cm]
        \item Find the second term $g(2)$.
        \item State the value of the first term using function notation in an equation.
        \item Define $g$ recursively using function notation. (There should be two equations) \vspace{1cm}
        \item Write down the value of $\displaystyle \frac{g(7)}{g(6)}$.
    \end{enumerate}

\newpage
    \item A sequence is defined recursively as 
    \begin{align*}
        f(1) &= 2 \\
        f(n) &= f(n-1) \times 5
    \end{align*}
    \begin{enumerate}[itemsep=0.75cm]
        \item Is the sequence arithmetic, geometric, or neither?
        \item Find the value of $f(3)$.
    \end{enumerate}

    \item Given an arithmetic sequence $f(n)$ whose first term is 11 and third term 17.
    \begin{enumerate}[itemsep=1.5cm]
        \item Using $d$ for the common difference and $x=f(2)$ for the second term, write an equation relating the values of the first two terms. (you may use $x$ or $f(2)$)
        \item Write an equation relating the second and third terms.
        \item Solve the system of equations to find $d$ and $x$.
    \end{enumerate} \vspace{1cm}

    \item Given an arithmetic sequence $47, x, 183, \dots$, find $x$. \vspace{1cm}

    \item Given a geometric sequence $\displaystyle \frac{2}{5}, x, \frac{18}{125}, \dots$, find $x$.
\end{enumerate}

\end{document}