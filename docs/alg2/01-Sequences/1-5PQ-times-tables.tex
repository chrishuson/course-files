\documentclass[12pt, twoside]{article}
\documentclass[12pt, twoside]{article}
\usepackage[letterpaper, margin=1in, headsep=0.2in]{geometry}
\setlength{\headheight}{0.6in}
%\usepackage[english]{babel}
\usepackage[utf8]{inputenc}
\usepackage{microtype}
\usepackage{amsmath}
\usepackage{amssymb}
%\usepackage{amsfonts}
\usepackage{siunitx} %units in math. eg 20\milli\meter
\usepackage{yhmath} % for arcs, overparenth command
\usepackage{tikz} %graphics
\usetikzlibrary{quotes, angles}
\usepackage{graphicx} %consider setting \graphicspath{{images/}}
\usepackage{parskip} %no paragraph indent
\usepackage{enumitem}
\usepackage{multicol}
\usepackage{venndiagram}

\usepackage{fancyhdr}
\pagestyle{fancy}
\fancyhf{}
\renewcommand{\headrulewidth}{0pt} % disable the underline of the header
\raggedbottom
\hfuzz=2mm %suppresses overfull box warnings

\usepackage{hyperref}
\usepackage{float}

\title{Algebra 2}
\author{Chris Huson}
\date{September 2024}

\fancyhead[RO]{\\ First and last name: \hspace{2.5cm} \,\\ Section: \hspace{2.5cm} \,}
\fancyhead[LO]{BECA/Huson/Precalculus: Sequences \\* 10 September 2024}

\begin{document}
\subsubsection*{1.5 Pre-Quiz: Multiplication tables \hfill Mental math - no calculators}
\begin{enumerate}[itemsep=0.5cm]

\item Perform the calculation. \hfill \emph{3.OA.7 Fluently multiply and divide within 100}
    \begin{multicols}{2}
        \begin{enumerate}[itemsep=0.5cm]
            \item $2 \times 3 =$
            \item $4 \times 5 =$
            \item $7 \times 9 =$
            \item $1 \times 8 =$
            \item $6 \times 2 =$
            \item $9 \times 4 =$
        \end{enumerate}
    \end{multicols}

\item Find each quotient. \hfill \emph{3.OA.7 Use the relationship between multiplication and division}
    \begin{multicols}{2}
        \begin{enumerate}[itemsep=0.5cm]
            \item $16 \div 2 =$
            \item $36 \div 6 =$
            \item $24 \div 3 =$
            \item $72 \div 8 =$
            \item $18 \div 9 =$
            \item $30 \div 5 =$
        \end{enumerate}
    \end{multicols}

\item Convert between fractions and percentages.
\begin{multicols}{2}
\begin{enumerate}[itemsep=0.5cm]
    \item $\frac{1}{2}=$
    \item $\frac{2}{5}=$
    \item $\frac{2}{3}=$
    \item $20\% =$
    \item $25\% =$
    \item $33 \frac{1}{3}\% =$
\end{enumerate}
\end{multicols}

\item Simplify the expression by combining like terms.
    \begin{multicols}{2}
    \begin{enumerate}[itemsep=0.5cm]
        \item $1x + 5x =$
        \item $14y - 7y =$
        \item $9z - 8z + z =$
        \item $-3y + 9y - 6y =$
        \item $4x^2 - 6x^2 =$
        \item $7y^2 + 12y^2 =$
    \end{enumerate}
    \end{multicols}

\newpage
\item Use the function $f(x) = -2x-8$ to answer the questions. \hfill \emph{F.IF.4 Interpret functions}
\begin{enumerate}[itemsep=1cm]
    \item What is $f(0)$?
    \item Find $f(\frac{1}{2})$
    \item Find $f(-3)$
    \item What is $x$ when $f(x) = -10$?
\end{enumerate}\vspace{2cm}

\item Fill in the blanks to continue the patterns.\vspace{0.5cm}
\begin{multicols}{2}
\begin{enumerate}[itemsep=1cm]
    \item $1, 3, 5, \rule{1cm}{0.15mm} \;, \rule{1cm}{0.15mm}$
    \item $2, 4, 8, \rule{1cm}{0.15mm} \;, \rule{1cm}{0.15mm}$
    \item $81, 27, 9, \rule{1cm}{0.15mm} \;, \rule{1cm}{0.15mm}$
    \item $-2, -5, -8, \rule{1cm}{0.15mm} \;, \rule{1cm}{0.15mm}$
\end{enumerate}
\end{multicols}

\item Here are three patterns with their first 5 terms listed. For each pattern, describe a way to produce each new term from the previous term.
\begin{enumerate}[itemsep=1cm]
    \item Pattern A: $4, 7, 10, 13, 16, \dots$
    \item Pattern B: $16, 8, 4, 2, 1, \dots$
    \item Pattern C: $3, -6, 12, -24, 48, \dots$
\end{enumerate}

\item Beginning with the first term of zero, write down the first 5 terms of an arithmetic sequence with a constant difference of 5.

\end{enumerate}
\end{document}

3.OA.7
Fluently multiply and divide within 100, using such as the relationship between multiplication and division. By the end of Grade 3, know from memory all products of two one-digit numbers.
