\documentclass[12pt, twoside]{article}
\documentclass[12pt, twoside]{article}
\usepackage[letterpaper, margin=1in, headsep=0.2in]{geometry}
\setlength{\headheight}{0.6in}
%\usepackage[english]{babel}
\usepackage[utf8]{inputenc}
\usepackage{microtype}
\usepackage{amsmath}
\usepackage{amssymb}
%\usepackage{amsfonts}
\usepackage{siunitx} %units in math. eg 20\milli\meter
\usepackage{yhmath} % for arcs, overparenth command
\usepackage{tikz} %graphics
\usetikzlibrary{quotes, angles}
\usepackage{graphicx} %consider setting \graphicspath{{images/}}
\usepackage{parskip} %no paragraph indent
\usepackage{enumitem}
\usepackage{multicol}
\usepackage{venndiagram}

\usepackage{fancyhdr}
\pagestyle{fancy}
\fancyhf{}
\renewcommand{\headrulewidth}{0pt} % disable the underline of the header
\raggedbottom
\hfuzz=2mm %suppresses overfull box warnings

\usepackage{hyperref}
\usepackage{float}

\title{Algebra 2}
\author{Chris Huson}
\date{January 2025}

\fancyhead[LE]{\thepage}
\fancyhead[RO]{\thepage \\ First and last name: \hspace{2.5cm} \,\\ Section: \hspace{2.5cm} \,}
\fancyhead[LO]{BECA / Huson / Precalculus: 3. Complex numbers \\* 16 January 2025}

\begin{document}

\subsubsection*{4.9 Test: Cumulative year-to-date standards}
\begin{enumerate}[itemsep=0.5cm]
\subsubsection*{A1-APR.1 Perform operations with polynomials}
\item Find the difference \(f(x) - g(x)\) as a polynomial in standard form, given:
\[
f(x) = x^4 - 3x^3 - 3x^2 - 2x + 5 \quad \text{and} \quad g(x) = 2x^4 - x^3 + 2x + 5.
\] \vspace{3cm}

\item Select each correct equation.
\begin{multicols}{2}
    \begin{enumerate}
    \item $x^2 + 14 = x^2 + 7^2$
    \item $x^2 + 49 = (x-7)(x+7)$
    \item $x^2 - 49 = (x-7)(x+7)$
    \item $x^2 + 14x + 49 = (x-7)^2$
    \item $x^2 - 14x + 49 = (x+7)^2$
    \item $x^2 - 14x + 49 = (x-7)^2$    
    \item \(x^3 + y^3 = (x + y)(x^2 - xy + y^2)\)
    \item \(x^3 + y^3 = (x - y)(x^2 - xy + y^2)\)
    \end{enumerate}
\end{multicols}

\subsubsection*{A2-A.APR.3 Identify zeros of polynomials given suitable factorizations}
\item Write down the solutions to the equation $(x - 7)(4x + 3)(x - 2) = 0$. \vspace{2cm} 

\subsubsection*{A2-A.REI.2 Solve rational and radical equations, identify extraneous solutions}
\item Square both sides of the equation and solve for $x$.
    \begin{multicols}{2}
    \begin{enumerate}[itemsep=0.5cm]
        \item  $\sqrt{x + 9}=4$
        \item Check your solution.
    \end{enumerate}
    \end{multicols} \vspace{3cm}

\item Solve for $x$ and check.
    \begin{multicols}{2}
    \begin{enumerate}[itemsep=0.5cm]
        \item  $\sqrt{5x+16} + 5 = 14$
        \item Check your solution.
    \end{enumerate}
    \end{multicols} \vspace{3cm}

\item Solve for $x$. $\displaystyle \frac{8}{x+3} = \frac{x+1}{x}$ \vspace{5cm}


\subsubsection*{A2-F.BF.2 Write arithmetic and geometric sequences with recursive formulas}
\item Write a recursive definition of the sequence $a_1 = 0.25$, $a_2 = 0.75$, $a_3 = 1.25$, $a_4 = 1.75, \ldots$ \vspace{1cm}

\item Write a recursive definition of the geometric sequence $b$. \\[0.5cm]
\renewcommand{\arraystretch}{1.5}
\begin{tabular}{|c|c|}
\hline
$n$ & $b_n$ \\
\hline
$1$ & $-1$ \\
$2$ & $5$ \\
$3$ & $-25$ \\
\hline
\end{tabular} \vspace{1cm}

\subsubsection*{A2.N.CN.2 Apply the properties of complex numbers}
\item Write each expression in the form $a+bi$ with $a,b$ real numbers. \\[0.25cm]
Given  $s = 2 - 5i $ and $t = 9 - 3i$.
    \begin{enumerate}[itemsep=1.5cm]
        \item $s+t =$
        \item $s-t =$
        \item $st =$
    \end{enumerate} \vspace{3cm}
    
\item Simplify each expression, using complex numbers as necessary.
    \begin{multicols}{2}
    \begin{enumerate}[itemsep=0.5cm]
        \item $\sqrt{-49}=$
        \item $\displaystyle \frac{1}{2} \sqrt{-12}=$
    \end{enumerate}
    \end{multicols} \vspace{1cm}

\item Does the equation $x^2 + 3x + 7 = 0$ have real or imaginary solutions? Justify your answer.

\newpage
\subsubsection*{A2.HSN.RN.2 Expressions with radicals and rational exponents}
\item Simplify each radical expression, using complex numbers as necessary.
    \begin{multicols}{2}
    \begin{enumerate}[itemsep=0.5cm]
        \item $\sqrt{64}=$
        \item $\sqrt{27}=$
        \item $\sqrt{-9}=$
        \item $\displaystyle \frac{\sqrt{-50}}{\sqrt{2}}=$
    \end{enumerate}
    \end{multicols} \vspace{1cm}
    
\item Simplify each expression.
    \begin{multicols}{2}
    \begin{enumerate}[itemsep=0.5cm]
        \item $\displaystyle 125^{\frac{2}{3}} =$
        \item $\left( \sqrt[3]{\frac{8}{27}} \right)^{2} =$
    \end{enumerate}
    \end{multicols} \vspace{2cm}

    
\item Rewrite each expression as a fractional exponent in simplest terms. $x>0$
    \begin{multicols}{2}
      \begin{enumerate}[itemsep=1cm]
          \item $\sqrt[3]{7} =$
          \item $\displaystyle \frac{1}{\sqrt[3]{5}}=$
          \item $\sqrt[2]{x^4} =$
          \item $\displaystyle \frac{1}{(\sqrt[3]{x})^2}=$
      \end{enumerate}
      \end{multicols} \vspace{1cm}
  
\item Rewrite each expression with fractional exponent as a radical.
    \begin{multicols}{2}
      \begin{enumerate}[itemsep=1cm]
        \item $\displaystyle 5^{\frac{1}{4}}=$
        \item $\displaystyle 5^{-\frac{1}{3}}=$
        \item $\displaystyle x^{\frac{2}{5}}=$
        \item $\displaystyle x^{-\frac{1}{3}}=$
      \end{enumerate}
      \end{multicols}
       
\end{enumerate}
\end{document}