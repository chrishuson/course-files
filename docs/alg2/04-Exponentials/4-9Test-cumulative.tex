\documentclass[12pt, twoside]{article}
\documentclass[12pt, twoside]{article}
\usepackage[letterpaper, margin=1in, headsep=0.2in]{geometry}
\setlength{\headheight}{0.6in}
%\usepackage[english]{babel}
\usepackage[utf8]{inputenc}
\usepackage{microtype}
\usepackage{amsmath}
\usepackage{amssymb}
%\usepackage{amsfonts}
\usepackage{siunitx} %units in math. eg 20\milli\meter
\usepackage{yhmath} % for arcs, overparenth command
\usepackage{tikz} %graphics
\usetikzlibrary{quotes, angles}
\usepackage{graphicx} %consider setting \graphicspath{{images/}}
\usepackage{parskip} %no paragraph indent
\usepackage{enumitem}
\usepackage{multicol}
\usepackage{venndiagram}

\usepackage{fancyhdr}
\pagestyle{fancy}
\fancyhf{}
\renewcommand{\headrulewidth}{0pt} % disable the underline of the header
\raggedbottom
\hfuzz=2mm %suppresses overfull box warnings

\usepackage{hyperref}
\usepackage{float}

\title{Algebra 2}
\author{Chris Huson}
\date{January 2025}

\fancyhead[LE]{\thepage}
\fancyhead[RO]{\thepage \\ First \& last name: \hspace{2.25cm} \,\\ Section: \hspace{2.25cm} \,}
\fancyhead[LO]{BECA / Huson / Precalculus: Exponential functions \\* 16 January 2025}

\begin{document}

\subsubsection*{4.9 Test: Cumulative year-to-date standards}
\begin{enumerate}[itemsep=0.5cm]

\item Simplify to standard form. \hfill \emph{A.APR.1 Perform operations with polynomials} \\[0.25cm]
$(3x^3 - 3x^2 - 2) - (- 2x^3 + 2x^2 - 3x - 5)$ \vspace{2cm}

\item Select each correct equation.
\begin{multicols}{2}
    \begin{enumerate}
    \item $x^2 - 12x - 36 = (x-6)(x+6)$
    \item $x^2 - 12x + 36 = (x-6)^2$
    \item $x^2 + 12x + 36 = (x+6)^2$
    \item $x^2 + 36 = (x-6)(x+6)$
    \item \(x^3 + y^3 = (x + y)(x^2 - xy + y^2)\)
    \item \(x^3 - y^3 = (x - y)(x^2 + xy + y^2)\)
    \end{enumerate}
\end{multicols}

\item Write down the solutions to $x(x + 1)(2x - 3) = 0$. \hfill \emph{A.APR.3 Find zeros of polynomials}
\vspace{2cm} 

\item Solve: $\displaystyle \frac{3}{x} = x-2$ \hfill \emph{A.REI.2 Solve rational and radical equations} \vspace{3cm} 

\item Solve for $x$ and check.
    \begin{multicols}{2}
    \begin{enumerate}[itemsep=0.5cm]
        \item  $\sqrt{x+25} + 11 = 15$
        \item Check your solution.
    \end{enumerate}
    \end{multicols}

\newpage
\item Write a recursive definition of the sequence \hfill \emph{F.BF.2 Sequences} \\[0.25cm]
$a_1 = 1$, $a_2 = 3$, $a_3 = 9$, $a_4 = 27, \ldots$ \vspace{2.5cm}

\item Simplify to the form $a+bi$ with $a,b$ real numbers. \hfill \emph{N.CN.2 Complex numbers}
    \begin{multicols}{2}
        \begin{enumerate}[itemsep=1.5cm]
            \item $(3 - 4i) - (2 + 8i)$
            \item $(2 - i)(5 - 3i)=$
        \end{enumerate}
    \end{multicols}  \vspace{5cm}

\item Simplify each expression, using imaginary numbers as necessary.
    \begin{multicols}{2}
    \begin{enumerate}[itemsep=0.5cm]
        \item $\sqrt{-49}=$
        \item $\displaystyle \frac{1}{2} \sqrt{-12}=$
    \end{enumerate}
    \end{multicols} \vspace{1cm}
  
\item Rewrite each expression as a radical. \hfill \emph{N.RN.2 Radicals and rational exponents} \vspace{0.25cm}
    \begin{multicols}{2}
      \begin{enumerate}[itemsep=1cm]
        \item $\displaystyle 4^{\frac{1}{3}}=$
        \item $\displaystyle x^{-\frac{3}{2}}=$
      \end{enumerate}
      \end{multicols} \vspace{1cm}
      
\item Rewrite each expression as a fractional exponent. $x>0$  \vspace{0.25cm}
    \begin{multicols}{2}
      \begin{enumerate}[itemsep=1cm]
          \item $\sqrt{5} =$
          \item $\sqrt[3]{x^2} =$
      \end{enumerate}
      \end{multicols}

       
\end{enumerate}
\end{document}