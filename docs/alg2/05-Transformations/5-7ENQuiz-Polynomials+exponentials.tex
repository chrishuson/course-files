\documentclass[12pt, twoside]{article}
% \documentclass[12pt, twoside]{article}
\usepackage[letterpaper, margin=1in, headsep=0.2in]{geometry}
\setlength{\headheight}{0.6in}
%\usepackage[english]{babel}
\usepackage[utf8]{inputenc}
\usepackage{microtype}
\usepackage{amsmath}
\usepackage{amssymb}
%\usepackage{amsfonts}
\usepackage[nomessages]{fp} %\FPeval{\var-name}{2*sin(pi/6)}
\usepackage{siunitx} %units in math. eg 20\milli\meter
\usepackage{yhmath} % for arcs, overparenth command
\usepackage{tikz} %graphics
\usetikzlibrary{quotes, angles, arrows, arrows.meta}
\usepackage{graphicx} %consider setting \graphicspath{{images/}}
\usepackage{parskip} %no paragraph indent
\usepackage{enumitem}
\usepackage{multicol}
\usepackage{venndiagram}

\usepackage{fancyhdr}
\pagestyle{fancy}
\fancyhf{}
\renewcommand{\headrulewidth}{0pt} % disable the underline of the header
\raggedbottom
\hfuzz=2mm %suppresses overfull box warnings

\usepackage{hyperref}
\usepackage{float}

\title{Algebra 2}
\author{Chris Huson}
\date{January 2025}

%\fancyhead[LE]{\thepage}
\fancyhead[R]{First \& last name: \hspace{2.25cm} \,\\ Section: \hspace{2.25cm} \,}
\fancyhead[L]{BECA / Huson / Precalculus: Transformations \\* 5 March 2025}

\begin{document}

\subsubsection*{5.7a Exit Note Quiz: Operations on polynomials, write exponential functions}
\begin{enumerate}

\item Simplify to standard form. \hfill \emph{A.APR.1 Perform operations with polynomials} \\[0.25cm]
$(5x^3 + 3x^2 - 3x - 6) - (x^3 - x - 5)$ \vspace{4cm}

\item Given $A = 2x^2+3$ and $B = 5x-4$, simplify $A - 2B$. \vspace{5cm}


\; \hfill \emph{F.LE.2.ii Construct an exponential function given a description}
\item A sample of radioactive material has a half-life of 12 days. Initially there are 150 milligrams of the material. Write an exponential function $A(t)$ to model the amount of material remaining in milligrams after $t$ days. \vspace{2cm}

\item A study of rodents in a certain city district finds that their population in thousands can be represented by the function $R(t) = 12(1.035)^t$ where $t$ is the number of weeks after April 1st. 
\begin{enumerate}[itemsep=1.5cm]
    \item According to the model, how many rodents are there on April 1st?
    \item Express the weekly growth rate as a percentage.
\end{enumerate}  \vspace{2cm}

\end{enumerate}


\newpage

\subsubsection*{5.7b Exit Note Quiz: Operations on polynomials, write exponential functions}
\begin{enumerate}

\item Simplify to standard form. \hfill \emph{A.APR.1 Perform operations with polynomials} \\[0.25cm]
$(6x^3 + 5x^2 - 2x - 9) - (x^3 - 3x - 2)$ \vspace{4cm}

\item Given $A = 4x^2+1$ and $B = 4x-3$, simplify $A - 2B$. \vspace{5cm}


\; \hfill \emph{F.LE.2.ii Construct an exponential function given a description}
\item A sample of radioactive material has a half-life of 16 days. Initially there are 350 milligrams of the material. Write an exponential function $A(t)$ to model the amount of material remaining in milligrams after $t$ days. \vspace{2cm}

\item A study of rodents in a certain city district finds that their population in thousands can be represented by the function $R(t) = 19(1.065)^t$ where $t$ is the number of weeks after April 1st. 
\begin{enumerate}[itemsep=1.5cm]
    \item According to the model, how many rodents are there on April 1st?
    \item Express the weekly growth rate as a percentage.
\end{enumerate}  \vspace{2cm}

\end{enumerate}


\newpage

\subsubsection*{5.7c Exit Note Quiz: Operations on polynomials, write exponential functions}
\begin{enumerate}

\item Simplify to standard form. \hfill \emph{A.APR.1 Perform operations with polynomials} \\[0.25cm]
$(4x^3 + 2x^2 - 7x - 5) - (x^3 - 2x - 1)$ \vspace{4cm}

\item Given $A = 3x^2+5$ and $B = x-5$, simplify $A - 2B$. \vspace{5cm}


\; \hfill \emph{F.LE.2.ii Construct an exponential function given a description}
\item A sample of radioactive material has a half-life of 23 days. Initially there are 125 milligrams of the material. Write an exponential function $A(t)$ to model the amount of material remaining in milligrams after $t$ days. \vspace{2cm}

\item A study of rodents in a certain city district finds that their population in thousands can be represented by the function $R(t) = 19(1.045)^t$ where $t$ is the number of weeks after April 1st. 
\begin{enumerate}[itemsep=1.5cm]
    \item According to the model, how many rodents are there on April 1st?
    \item Express the weekly growth rate as a percentage.
\end{enumerate}  \vspace{2cm}

\end{enumerate}

\end{document}