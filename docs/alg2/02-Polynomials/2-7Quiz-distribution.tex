\documentclass[12pt, twoside]{article}
\documentclass[12pt, twoside]{article}
\usepackage[letterpaper, margin=1in, headsep=0.2in]{geometry}
\setlength{\headheight}{0.6in}
%\usepackage[english]{babel}
\usepackage[utf8]{inputenc}
\usepackage{microtype}
\usepackage{amsmath}
\usepackage{amssymb}
%\usepackage{amsfonts}
\usepackage{siunitx} %units in math. eg 20\milli\meter
\usepackage{yhmath} % for arcs, overparenth command
\usepackage{tikz} %graphics
\usetikzlibrary{quotes, angles}
\usepackage{graphicx} %consider setting \graphicspath{{images/}}
\usepackage{parskip} %no paragraph indent
\usepackage{enumitem}
\usepackage{multicol}
\usepackage{venndiagram}

\usepackage{fancyhdr}
\pagestyle{fancy}
\fancyhf{}
\renewcommand{\headrulewidth}{0pt} % disable the underline of the header
\raggedbottom
\hfuzz=2mm %suppresses overfull box warnings

\usepackage{hyperref}
\usepackage{float}

\title{Algebra 2}
\author{Chris Huson}
\date{October 2023}

\fancyhead[LE]{\thepage}
\fancyhead[RO]{\thepage \\ Name: \hspace{4cm} \,\\}
\fancyhead[LO]{BECA / Huson / Algebra 2: Polynomials \\* 27 October 2023}

\begin{document}

\subsubsection*{2.7 Quiz: Operations on polynomials}
\begin{enumerate}
    \item Evaluate each polynomial for the given value of $x$.
    \begin{multicols}{2}
        \begin{enumerate}[itemsep=1cm]
            \item $f(x)=-x^3+12x^2-x+4$, $x=1$ \\[0.25cm] 
            $f(1) = $ \vspace{2cm}
            \item $g(x)=2x^3+11x^2-3x+15$, $x=-1$ \\[0.25cm] 
            $g(-8) = $ \vspace{2cm}
        \end{enumerate}
        \end{multicols}
    
    \item The polynomial function $A$, shown below, is used to model the value of an investment account. Three deposits were made which earned interest annually.  $$A(x)=200x^5+300x^4+150x^3$$ 
    \begin{enumerate}[itemsep=1cm]
        \item How much was the first deposit, and how long ago was it made? \vspace{1cm}
        \item If the polynomial is evaluated for $x = 1.04$, what interest rate would that represent \emph{as a percentage}?
        \item Find the value of $A(1.04)$ to the \emph{nearest cent}. \vspace{2cm}
    \end{enumerate}

\subsubsection*{A1-A.APR.1 Add, subtract, and multiply polynomials}
\item Write a recursive formula for each sequence. Use subscript notation.
    \begin{multicols}{2}
    \begin{enumerate}
        \item $3, -6, 12, -24, 48, \dots$
        \item $\displaystyle \frac{3}{4}, \frac{5}{4}, \frac{7}{4}, \frac{9}{4},  \dots$ 
    \end{enumerate}
    \end{multicols}

\newpage
\subsubsection*{A1-A.APR.1 Add, subtract, and multiply polynomials}

\item Find the sum in standard form $(x^3-4x^2+2x+16)+(5x^3-2x^2-3x-12)$ \vspace{2cm}

\item Find the difference $f(x)-g(x)$ as a polynomial in standard form, given \\[0.25cm]
    $f(x)=x^4+2x^3-x-9$ and $g(x)=2x^3+x^2-3x-11$. \vspace{4cm}

\item Multiply the two polynomials $f(x)=3x-2$ and $g(x)=x^2-5x+4$. First complete the grid and then collect terms to find the product as a polynomial in standard form. \\[0.25cm]
\begin{tabular}{|p{1cm}|p{3cm}|p{3cm}|p{3cm}|}
    \hline
     & $x^2$ & $-5x$ & $4$ \\
    \hline
    $3x$ &  & & \\[0.5cm]
    \hline
    $-2$ &  & & \\[0.5cm]
    \hline
\end{tabular} \vspace{4cm}

\item Select all of the expressions that are equivalent to $x^2-7x+12$.
    \begin{multicols}{2}
    \begin{enumerate}
        \item $(x-2)(x-6)$
        \item $(x-6)(x-2)$ 
        \item $(x+4)(x+3)$ 
        \item $(x-3)(x-4)$ 
        \item $(x-4)(x+3)$
        \item $(x+3)(x+4)$ 
        \item $(x-4)(x-3)$
        \item $x^2+7x-12$
    \end{enumerate} 
    \end{multicols}
    \vspace{0.25cm}

\newpage
\item Select all solutions to the equation $(x-3)(2x+1)=0$.
    \begin{multicols}{2}
    \begin{enumerate}
        \item $x=-\frac{1}{2}$
        \item $x=3$
        \item $x=-2$
        \item $x=-0.5$
        \item $x=-3$
        \item $x=\frac{1}{2}$
    \end{enumerate}
    \end{multicols}
    \vspace{0.25cm}

\item Here is the graph of a quadratic function. Which of the following could be its equation?
    \begin{center}
    \begin{tikzpicture}[xscale=0.7, yscale=0.2]
        \draw [thick, ->] (-5.2,0) -- (5.4,0) node [above] {$x$};
        \draw [thick, ->] (0,-14.2)--(0,9.5) node [right] {$y$};
        \foreach \x in {-4,-3,...,4} \draw (\x cm,10pt) -- (\x cm,-10pt) node[below] {$\x$};
        %\foreach \y in {-8,-4,4, 8} \draw (2pt,\y cm) -- (-2pt,\y cm) node[left] {$\y$};
        %\fill (-1,0) circle[radius=0.1] node[above left]{$j$};
        %\fill (3,0) circle[radius=0.1] node[above right]{$k$};
        \draw [thick, <->,smooth,samples=20,domain=-5:4] plot(\x,\x*\x+\x-12);
    \end{tikzpicture}
    \end{center}
    \begin{multicols}{2}
    \begin{enumerate}
        \item $y=(x+3)(x-4)$
        \item $y=(x-3)(x+4)$
        \item $y=(x+3)(x+4)$
        \item $y=(x-3)(x-4)$
    \end{enumerate}
    \end{multicols}

\item Find all of the solutions to the equation $x(x+5)(2x-9)(x-13)=0$. 

\newpage


\end{enumerate}
\end{document}