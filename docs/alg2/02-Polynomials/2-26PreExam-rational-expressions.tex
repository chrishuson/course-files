\documentclass[12pt, twoside]{article}
\documentclass[12pt, twoside]{article}
\usepackage[letterpaper, margin=1in, headsep=0.2in]{geometry}
\setlength{\headheight}{0.6in}
%\usepackage[english]{babel}
\usepackage[utf8]{inputenc}
\usepackage{microtype}
\usepackage{amsmath}
\usepackage{amssymb}
%\usepackage{amsfonts}
\usepackage{siunitx} %units in math. eg 20\milli\meter
\usepackage{yhmath} % for arcs, overparenth command
\usepackage{tikz} %graphics
\usetikzlibrary{quotes, angles}
\usepackage{graphicx} %consider setting \graphicspath{{images/}}
\usepackage{parskip} %no paragraph indent
\usepackage{enumitem}
\usepackage{multicol}
\usepackage{venndiagram}

\usepackage{fancyhdr}
\pagestyle{fancy}
\fancyhf{}
\renewcommand{\headrulewidth}{0pt} % disable the underline of the header
\raggedbottom
\hfuzz=2mm %suppresses overfull box warnings

\usepackage{hyperref}
\usepackage{float}

\title{Algebra 2}
\author{Chris Huson}
\date{January 2024}

\fancyhead[RO]{\\ Name: \hspace{3cm} \,\\}
\fancyhead[LO]{BECA / Huson / Algebra 2: Polynomial \& Rational Functions \\* 9 January 2024}

\begin{document}

\subsubsection*{2.27 Homework: Rational expressions exam review}
\begin{enumerate}
\item Use polynomial long division to find an expression of the form $ax+b+\frac{c}{x+d}$ with $a,b,c,d$ integers that is equivalent to $\displaystyle \frac{3x^3 + 19x^2 + 15x}{x^2 + 4x}
$ for $x \neq -4 \text{ or } 0$.
\vspace{8cm}


\subsubsection*{A2-F.BF.2 Write arithmetic and geometric sequences with recursive formulas}
\item Write a recursive definition of the sequence $a_1 = 2$, $a_2 = 6$, $a_3 = 18$, $a_4 = 54, \ldots$ \vspace{2cm}

\item A geometric sequence begins $5, 10, 20, \ldots$.
\begin{enumerate}[itemsep=0.5cm]
    \item Write the first six terms of the sequence.
    \item Find the common ratio $r$. 
    \item Find the sum of the first six terms of the sequence. \vspace{1cm}
    \item Find the sum of the first 20 terms of the sequence.
\end{enumerate}

\newpage
\item  Find all values of $x$ that make the equation true.
$$\frac{x-3}{x}=\frac{2}{x-6}$$ \vspace{4cm}

\item Given the rational function $\displaystyle r(x)= -2 + \frac{x-1}{x+2}$. 
    \begin{enumerate}[itemsep=0.25cm]
        \item Sketch a graph of the function.
        \item Mark the vertical asymptote as dotted line and label it with its equation.
        \item Explain why the asymptote is located there.
    \end{enumerate}
    \begin{center}
    \begin{tikzpicture}[xscale=0.7, yscale=0.7]
        \draw [thick, ->] (-8.2,0) -- (8.5,0) node [above] {$x$};
        \draw [thick, ->] (0,-8.2)--(0,8.5) node [right] {$y$};
        \foreach \x in {-8,-6,-4,-2,2,4,6,8} \draw (\x cm,5pt)--(\x cm,-5pt) node at (\x,-0.5){\x};
        \foreach \y in {-8,-6,-4,-2,2,4,6,8} \draw (5pt,\y cm)--(-5pt,\y cm) node at (-0.5,\y){\y};
    \end{tikzpicture}
    \end{center}


\end{enumerate}
\end{document}