\documentclass[12pt, twoside]{article}
\usepackage{amsmath}
\usepackage{amssymb}
\usepackage{enumerate}
\usepackage{tikz}
\usepackage{multicol}
\title{Algebra 2 Regents Exam}

\begin{document}

\maketitle

\section*{Part I}

\begin{enumerate}%[itemsep=0.5cm]


% January 2020 Regents - Problem 1
\item The expression \( \sqrt[3]{64x^8} \) is equivalent to
\begin{enumerate}
    \item \( 4x^{\frac{8}{3}} \)
    \item \( 4x^3 \)
    \item \( 4x^{\frac{2}{3}} \)
    \item \( 4x^{\frac{11}{3}} \)
\end{enumerate}

% August 2019 Regents - Problem 2
\item The expression \( \sqrt[4]{81x^8y^{12}} \) is equivalent to
\begin{enumerate}
    \item \( 3x^2y^3 \)
    \item \( 3x^2y \)
    \item \( 9x^4y^3 \)
    \item \( 9x^2y^6 \)
\end{enumerate}

% January 2020 Regents - Problem 25
\item For \( n \) and \( p > 0 \), is the expression \( \sqrt[3]{\frac{n^5 p^2}{n^2 p^4}} \) equivalent to
\begin{enumerate}
    \item \( \sqrt[3]{\frac{n^3}{p^2}} \)
    \item \( \frac{n}{p} \)
    \item \( n^{\frac{1}{3}} p^{-1} \)
    \item \( n^{\frac{1}{3}} p^{\frac{1}{3}} \)
\end{enumerate}


% June 2019 Regents - Problem 5
\item The expression \( \frac{\sqrt{50}}{\sqrt{2}} \) is equivalent to
\begin{enumerate}
    \item \( 25 \)
    \item \( 5 \)
    \item \( 5\sqrt{2} \)
    \item \( \sqrt{25} \)
\end{enumerate}

% January 2019 Regents - Problem 3
\item Which expression is equivalent to \( \sqrt[5]{x^3} \)?
\begin{enumerate}
    \item \( x^{\frac{3}{5}} \)
    \item \( x^{\frac{5}{3}} \)
    \item \( \sqrt[5]{x^8} \)
    \item \( \sqrt[3]{x^5} \)
\end{enumerate}



\end{enumerate}
\end{document}
