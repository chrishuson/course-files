\documentclass[12pt, twoside]{article}
% \documentclass[12pt, twoside]{article}
\usepackage[letterpaper, margin=1in, headsep=0.2in]{geometry}
\setlength{\headheight}{0.6in}
%\usepackage[english]{babel}
\usepackage[utf8]{inputenc}
\usepackage{microtype}
\usepackage{amsmath}
\usepackage{amssymb}
%\usepackage{amsfonts}
\usepackage[nomessages]{fp} %\FPeval{\var-name}{2*sin(pi/6)}
\usepackage{siunitx} %units in math. eg 20\milli\meter
\usepackage{yhmath} % for arcs, overparenth command
\usepackage{tikz} %graphics
\usetikzlibrary{quotes, angles, arrows, arrows.meta}
\usepackage{graphicx} %consider setting \graphicspath{{images/}}
\usepackage{parskip} %no paragraph indent
\usepackage{enumitem}
\usepackage{multicol}
\usepackage{venndiagram}

\usepackage{fancyhdr}
\pagestyle{fancy}
\fancyhf{}
\renewcommand{\headrulewidth}{0pt} % disable the underline of the header
\raggedbottom
\hfuzz=2mm %suppresses overfull box warnings

\usepackage{hyperref}
\usepackage{float}

\title{Algebra 2}
\author{Chris Huson}
\date{December 2023}

\fancyhead[LE]{\thepage}
\fancyhead[RO]{\thepage \\ Name: \hspace{4cm} \,\\}
\fancyhead[LO]{BECA / Huson / Algebra 2: Polynomials Jan 2023 Regents \\* 23 December 2023}

\begin{document}

\subsubsection*{Regents problems: Polynomials}
\begin{enumerate}[itemsep=0.5cm]
\item Which expression is equivalent to $(x + 2)^2 - 5(x + 2) + 6$?
\begin{enumerate}
    \item $x(x + 1)$
    \item $(x - 3)(x + 2)$
    \item $(x - 4)(x + 3)$
    \item $(x - 6)(x + 1)$
\end{enumerate}
    
\item To the \emph{nearest tenth}, the solution to the equation $4300e^{0.07x} -123 = 5000$ is
\begin{enumerate}
    \item 1.1
    \item 2.5
    \item 6.3
    \item 68.5
\end{enumerate}

\item The value of an automobile $t$ years after it was purchased is given by the function $V = 38000(0.84)^t$. Which statement is true?
\begin{enumerate}
    \item The value of the car increases 84\% each year.
    \item The value of the car decreases 84\% each year.
    \item The value of the car increases 16\% each year.
    \item The value of the car decreases 16\% each year.
\end{enumerate}

\item Which function represents exponential decay?
\begin{enumerate}
    \item $\displaystyle p(x) = \left(\frac{1}{4}\right)^x$
    \item $q(x) = 1.8^{-x}$
    \item $r(x) = 2.3^{2x}$
    \item $s(x) = 4^{\frac{x}{2}}$
\end{enumerate}

\item The expression $\displaystyle \frac{x^4 - 5x^2 + 4x + 14}{x+2}$ is equivalent to
\begin{enumerate}
    \item $\displaystyle x^3 - 2x^2 - x + 6 - \frac{2}{x + 2}$
    \item $\displaystyle x^3 - 5x + 4 - \frac{14}{x + 2}$
    \item $\displaystyle x^3 + 2x^2 - x + 2 + \frac{18}{x + 2}$
    \item $\displaystyle x^3 + 2x^2 - 9x + 22 - \frac{30}{x + 2}$
\end{enumerate}

\item The sum of the first 20 terms of the series \(2 - 6 + 18 - 54 + \ldots\) is
\begin{enumerate}
    \item $-610$
    \item $-59$
    \item 1,743,392,200
    \item 2,324,522,934
\end{enumerate}

\item If \(f(x) = 2x^4 - x^3 - 16x + 8\), then \(f\left(\frac{1}{2}\right)\)
\begin{enumerate}
    \item equals 0 and \(2x + 1\) is a factor of \(f(x)\)
    \item equals 0 and \(2x - 1\) is a factor of \(f(x)\)
    \item does not equal 0 and \(2x + 1\) is not a factor of \(f(x)\)
    \item does not equal 0 and \(2x - 1\) is a factor of \(f(x)\)
\end{enumerate}

\item If \((6 - ki)^2 = 27 - 36i\), the value of \(k\) is
\begin{enumerate}
    \item \(-36\)
    \item \(-3\)
    \item \(3\)
    \item \(6\)
\end{enumerate}

\item What is the solution set of the equation \(\displaystyle \frac{x+2}{x} + \frac{x}{3} = \frac{2x^2+6}{3x}\)?
\begin{enumerate}
    \item \(\{-3\}\)
    \item \(\{-3, 0\}\)
    \item \(\{3\}\)
    \item \(\{0, 3\}\)
\end{enumerate}

\item How many real solutions exist for the system of equations below?
\begin{align*}
    y &= \frac{1}{4} x - 8 \\
    y &= \frac{1}{2} x^2 + 2x
\end{align*}
\begin{enumerate}
    \item 1
    \item 2
    \item 3
    \item 0
\end{enumerate}

\item Which equation represents a polynomial identity?
\begin{enumerate}
    \item \(x^3 + y^3 = (x + y)^3\)
    \item \(x^3 + y^3 = (x + y)(x^2 - xy + y^2)\)
    \item \(x^3 + y^3 = (x + y)(x^2 - xy - y^2)\)
    \item \(x^3 + y^3 = (x - y)(x^2 + xy + y^2)\)
\end{enumerate}

\item Given \(x > 0\), the expression \(\frac{1}{x^2 - 1}\) can be rewritten as
\begin{enumerate}
    \item \(\frac{3}{x} - 1\)
    \item \(\frac{2}{10x^3}\)
    \item \(\frac{10}{x^3}\)
    \item \(\frac{3}{x^{10}}\)
\end{enumerate}

\item Given $x > 0$, the expression $\frac{1}{\sqrt[3]{x^2} - 1}$ can be rewritten as
\begin{enumerate}
    \item $\frac{1}{\sqrt[3]{x} - 1}$
    \item $\frac{1}{\sqrt[3]{x} + 1}$
    \item $\frac{1}{\sqrt{x} - 1}$
    \item $\frac{1}{\sqrt{x} + 1}$
\end{enumerate}


\end{enumerate}
\end{document}