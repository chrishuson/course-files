\documentclass[12pt, twoside]{article}
\documentclass[12pt, twoside]{article}
\usepackage[letterpaper, margin=1in, headsep=0.2in]{geometry}
\setlength{\headheight}{0.6in}
%\usepackage[english]{babel}
\usepackage[utf8]{inputenc}
\usepackage{microtype}
\usepackage{amsmath}
\usepackage{amssymb}
%\usepackage{amsfonts}
\usepackage{siunitx} %units in math. eg 20\milli\meter
\usepackage{yhmath} % for arcs, overparenth command
\usepackage{tikz} %graphics
\usetikzlibrary{quotes, angles}
\usepackage{graphicx} %consider setting \graphicspath{{images/}}
\usepackage{parskip} %no paragraph indent
\usepackage{enumitem}
\usepackage{multicol}
\usepackage{venndiagram}

\usepackage{fancyhdr}
\pagestyle{fancy}
\fancyhf{}
\renewcommand{\headrulewidth}{0pt} % disable the underline of the header
\raggedbottom
\hfuzz=2mm %suppresses overfull box warnings

\usepackage{hyperref}
\usepackage{float}

\title{Algebra 2}
\author{Chris Huson}
\date{December 2023}

\fancyhead[LE]{\thepage}
\fancyhead[RO]{\thepage \\ Name: \hspace{3cm} \,\\}
\fancyhead[LO]{BECA / Huson / Algebra 2: Exponentials Jan 2023 Regents \\* 4 April 2024}

\begin{document}

\subsubsection*{Regents problems: Polynomials}
\begin{enumerate}[itemsep=0.5cm]
\item To the \emph{nearest tenth}, the solution to the equation $4300e^{0.07x} -123 = 5000$ is
\begin{enumerate}
    \item 1.1
    \item 2.5
    \item 6.3
    \item 68.5
\end{enumerate}

\item The value of an automobile $t$ years after it was purchased is given by the function $V = 38000(0.84)^t$. Which statement is true?
\begin{enumerate}
    \item The value of the car increases 84\% each year.
    \item The value of the car decreases 84\% each year.
    \item The value of the car increases 16\% each year.
    \item The value of the car decreases 16\% each year.
\end{enumerate}

\item Which function represents exponential decay?
\begin{enumerate}
    \item $\displaystyle p(x) = \left(\frac{1}{4}\right)^x$
    \item $q(x) = 1.8^{-x}$
    \item $r(x) = 2.3^{2x}$
    \item $s(x) = 4^{\frac{x}{2}}$
\end{enumerate}

\item For which approximate value(s) of $x$ will $\log(x+5) = |x-1|-3$?
\begin{enumerate}
    \item $5,1$
    \item $-2.41, 0.41$
    \item $-2.41, 5$
    \item $5$, only
\end{enumerate}

\item Mia has a student loan that is in deferment, meaning that she does
not need to make payments right now. The balance of her loan
account during her deferment can be represented by the function
$f(x) = 35,000 (1.0325)^x$, where $x$ is the number of years since the
deferment began. If the bank decides to calculate her balance showing
a monthly growth rate, an approximately equivalent function would be
\begin{enumerate}
    \item $f(x) = 35,000 (1.0027)^{12x}$
    \item $\displaystyle f(x) = 35,000 (1.0027)^{\frac{x}{12}}$
    \item $f(x) = 35,000 (1.0325)^{12x}$
    \item $\displaystyle f(x) = 35,000 (1.0325)^{\frac{x}{12}}$
\end{enumerate}

\end{enumerate}
\end{document}