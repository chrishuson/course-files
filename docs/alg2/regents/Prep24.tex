\documentclass[12pt, twoside]{article}
% \documentclass[12pt, twoside]{article}
\usepackage[letterpaper, margin=1in, headsep=0.2in]{geometry}
\setlength{\headheight}{0.6in}
%\usepackage[english]{babel}
\usepackage[utf8]{inputenc}
\usepackage{microtype}
\usepackage{amsmath}
\usepackage{amssymb}
%\usepackage{amsfonts}
\usepackage[nomessages]{fp} %\FPeval{\var-name}{2*sin(pi/6)}
\usepackage{siunitx} %units in math. eg 20\milli\meter
\usepackage{yhmath} % for arcs, overparenth command
\usepackage{tikz} %graphics
\usetikzlibrary{quotes, angles, arrows, arrows.meta}
\usepackage{graphicx} %consider setting \graphicspath{{images/}}
\usepackage{parskip} %no paragraph indent
\usepackage{enumitem}
\usepackage{multicol}
\usepackage{venndiagram}

\usepackage{fancyhdr}
\pagestyle{fancy}
\fancyhf{}
\renewcommand{\headrulewidth}{0pt} % disable the underline of the header
\raggedbottom
\hfuzz=2mm %suppresses overfull box warnings

\usepackage{hyperref}
\usepackage{float}

\title{Algebra 2}
\author{Chris Huson}
\date{June 2024}

\fancyhead[LE]{\thepage}
\fancyhead[RO]{\thepage \\ Name: \hspace{1.5cm} \,\\}
\fancyhead[LO]{BECA/Huson/Algebra 2: Regents Preparation \\* 6 June 2024}

\begin{document}
\subsubsection*{Prep \#24 Exponential regression}
Casio instructions: \url{https://www.youtube.com/watch?v=jcX93Grn8o8} \\[0.25cm]
Menu \textgreater{} Stat \textgreater{} enter data in List 1 and List 2 \\ \textgreater{} Calc (F2) \textgreater{} REG (F3) \textgreater{} (F6) \textgreater{} EXP (F2) \textgreater{} ab\textasciicircum{}X (F2)
\begin{enumerate}[itemsep=0.5cm]
\item Consider the data in the table below.
    \begin{center}
    \begin{tabular}{|p{1cm}|p{1cm}|p{1cm}|p{1cm}|p{1cm}|p{1cm}|p{1cm}|}
        \hline
        $x$ & 1 & 2 & 3 & 4 & 5 & 6 \\
        \hline
        $y$ & 3.9 & 6 & 11 & 18.1 & 28 & 40.3 \\[0.25cm]
        \hline
    \end{tabular}
    \end{center}
    State an exponential regression equation to model these data, rounding all values to the \emph{nearest thousandth}.  \vspace{2cm} %Regents Jan2023
    % y = 2.459(1.616)^x

\item A cup of coffee is left out on a countertop to cool. The table below represents the temperature, $F(t)$, in degrees Fahrenheit, of the coffee after it is left out for $t$ minutes.
    \begin{center}
    \begin{tabular}{|p{1cm}|p{1cm}|p{1cm}|p{1cm}|p{1cm}|p{1cm}|p{1cm}|}
        \hline
        $t$ & 0 & 5 & 10 & 15 & 20 & 25 \\
        \hline
        $F(t)$ & 180 & 144 & 120 & 104 & 93.3 & 86.2 \\[0.25cm]
        \hline
    \end{tabular}
    \end{center}
    Based on these data, write an exponential regression equation, $F(t)$, to model the temperature of the coffee. Round all values to the \emph{nearest thousandth}.  \vspace{2cm} %Regents Jun2022
    % F(t) = 169.136 (0.971)^t

\item Kelly-Ann has \$20,000 to invest. She puts half of the money into an account that grows at an annual rate of 0.9\%. %compounded monthly
At the same time, she puts the other half of the money into an account that grows continuously at an annual rate of 0.8\%. Write an equation that represents the value of Kelly-Ann's investments after $t$ years.
    
\newpage
\item An investment of \$5000 grows at an annual rate of 3.5\% compounded monthly. 
$$P(t)=5000(1+\frac{0.035}{12})^{12t}$$
\begin{enumerate}
    \item Find the investment value after 10 years. \vspace{2cm}
    \item Determine the time for the investment value to double, to the \emph{nearest month}. \vspace{3cm}
\end{enumerate}

\item The table below gives air pressures in kPa at selected altitudes above sea level measured in kilometers.
\begin{center}
    \begin{tabular}{|p{1cm}|p{4cm}|p{1cm}|p{1cm}|p{1cm}|p{1cm}|p{1cm}|p{1cm}|}
        \hline
        $x$ & Altitude (km) & 0 & 1 & 2 & 3 & 4 & 5 \\
        \hline
        $y$ & Air Pressure (kPa) & 101 & 90 & 79 & 70 & 62 & 54 \\[0.25cm]
        \hline
    \end{tabular}
    \end{center}
    Write an exponential regression equation that models these data rounding all values to the \emph{nearest thousandth}. \\[2cm]
    Use this equation to algebraically determine the altitude, to the \emph{nearest hundredth} of a kilometer, when the air pressure is 29 kPa. %Regents Jan2020

\newpage
\item Expand and simplify each complex expression.
\begin{multicols}{2}
\begin{enumerate}
    \item $6xi^3(-4xi+5)$
    \item $(x+3i)^2-(2x-3i)^2$
\end{enumerate}
\end{multicols} \vspace{4cm}

\item Solve each equation. Express the answer in $a+bi$ form.
\begin{multicols}{2}
\begin{enumerate}[itemsep=0.5cm]
    \item $2x^2+5x+8=0$
    \item $5x^2-2x+13=9$ %Regents Aug2019
\end{enumerate}
\end{multicols} \vspace{4cm}

\item Find the solution set of each equation (round to the \emph{nearest tenth}).
\begin{multicols}{2}
    \begin{enumerate}
        \item $\displaystyle \frac{2}{3x+1} = \frac{1}{x} - \frac{6x}{3x+1}$
        \item $\displaystyle \frac{1}{1-x^2} = - |3x-2|+5$
    \end{enumerate}
\end{multicols}

\newpage
\item Over the set of integers, factor completely $x^4-5x^2+4$. \vspace{3cm} %Regents Jan2023

\item Determine which expressions are equivalent to $\displaystyle \frac{x^4-5x^2+4x+14}{x+2}$. \\[0.25cm]
    (hint: substitute $x=0$ and $x=1$)
    \begin{multicols}{2}
    \begin{enumerate}[itemsep=2cm]
        \item $\displaystyle x^3-2x^2-x+6+\frac{2}{x+2}$
        \item $\displaystyle x^3-5x+4-\frac{14}{x+2}$
        \item $\displaystyle x^3+2x^2-x+2+\frac{18}{x+2}$
        \item $\displaystyle x^3+2x^2-9x+22-\frac{30}{x+2}$
    \end{enumerate}
    \end{multicols} \vspace{2cm}

\item Given $a > 0$, solve the equation $a^{x+1} = \sqrt[3]{a^2}$ for $x$ algebraically. %Regents Jan2023
\vspace{2cm}

\newpage
\item Write a recursive formula for the sequence 1.45, 2.05, 2.55, 3.05, $\ldots$ \vspace{3cm}


\item Given the sequence beginning  $4$, 2, $1$, $\frac{1}{2}$, $\ldots$, find the sum of the first 7 terms, rounded to the \emph{nearest hundredth}. \vspace{3cm}

\item The first two terms of an arithmetic sequence are shown in the table. Complete the table and write a recursive definition for the sequence.
\begin{center}
\begin{tabular}{|p{1cm}|p{1cm}|p{1cm}|p{1cm}|p{1cm}|p{1cm}|}
    \hline
    $n$ & 1 & 2 & 3 & 4 & 5 \\
    \hline
    $a_n$ & 3 & 9 & & & \\[0.25cm]
    \hline
\end{tabular}
\end{center}

\item When a ball bounces, the heights of consecutive bounces form a geometric sequence. The height of the first bounce is 121 centimeters and the height of the third bounce is 64 centimeters. To the \emph{nearest centimeter}, what is the height of the fifth bounce? %Regents Jan2019

\item Rowan is training to run in a race. He runs 15 miles in the first week, and each week following, he runs 3\% more than the week before. Using a geometric series formula, find the total number of miles Rowan runs over the first ten weeks of training, rounded to the \emph{nearest thousandth}. %Regents Jan2019

\newpage
\item Given $P(A) = \frac{1}{3}$ and $P(B) = \frac{5}{12}$, where $A$ and $B$ are independent events. Determine $P(A \cap B)$. \vspace{3cm}

\item The set of data in the table below shows the results of a survey on the number of messages that people of different ages text on their cell phones each month.
\begin{center}
    \begin{tabular}{|c|c|c|c|}
        \hline
        Age Group & 0-10 & 11-50 & Over 50 \\
        \hline
        15-18 & 4 & 37 & 68 \\[0.25cm]
        \hline
        19-22 & 6 & 25 & 87 \\[0.25cm]
        \hline
        23-60 & 25 & 47 & 157 \\[0.25cm]
        \hline
    \end{tabular}
\end{center}
If a person from this survey is selected at random, what is the probability that the person texts over 50 messages per month given that the person is between the ages of 23 and 60?  \vspace{3cm}

\item The scores on a collegiate mathematics readiness assessment are approximately normally distributed with a mean of 680 and a standard deviation of 120.\\[0.5cm]
Determine the percentage of scores between 690 and 900, to the \emph{nearest percent}.\vspace{3cm}




\end{enumerate}
\end{document}