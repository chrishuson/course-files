\documentclass[12pt, twoside]{article}
\documentclass[12pt, twoside]{article}
\usepackage[letterpaper, margin=1in, headsep=0.2in]{geometry}
\setlength{\headheight}{0.6in}
%\usepackage[english]{babel}
\usepackage[utf8]{inputenc}
\usepackage{microtype}
\usepackage{amsmath}
\usepackage{amssymb}
%\usepackage{amsfonts}
\usepackage{siunitx} %units in math. eg 20\milli\meter
\usepackage{yhmath} % for arcs, overparenth command
\usepackage{tikz} %graphics
\usetikzlibrary{quotes, angles}
\usepackage{graphicx} %consider setting \graphicspath{{images/}}
\usepackage{parskip} %no paragraph indent
\usepackage{enumitem}
\usepackage{multicol}
\usepackage{venndiagram}

\usepackage{fancyhdr}
\pagestyle{fancy}
\fancyhf{}
\renewcommand{\headrulewidth}{0pt} % disable the underline of the header
\raggedbottom
\hfuzz=2mm %suppresses overfull box warnings

\usepackage{hyperref}
\usepackage{float}

\title{Algebra 2}
\author{Chris Huson}
\date{December 2023}

\fancyhead[LE]{\thepage}
\fancyhead[RO]{\thepage \\ Name: \hspace{4cm} \,\\}
\fancyhead[LO]{BECA / Huson / Algebra 2: Polynomials Jan 2023 Regents \\* 5 May 2025}

\begin{document}

\subsubsection*{Regents problems: Exponents}
\begin{enumerate}[itemsep=1.5cm]
\item Given $x > 0$, the expression $\displaystyle \frac{x^{\frac{1}{5}}}{x^{\frac{1}{2}}}$ can be rewritten as %January 2023 Regents
\begin{enumerate}
    \item $\sqrt[3]{x}$
    \item $-\sqrt[10]{x^3}$
    \item $\displaystyle \frac{1}{\sqrt[10]{x^3}}$
    \item $\sqrt[3]{x^{10}}$
\end{enumerate}

\item Given $x > 0$, the expression $\displaystyle \frac{1}{\sqrt[3]{x^2} - 1}$ can be rewritten as %January 2023 Regents
\begin{enumerate}
    \item $\displaystyle \frac{1}{\sqrt[3]{x} - 1}$
    \item $\displaystyle \frac{1}{\sqrt[3]{x} + 1}$
    \item $\displaystyle \frac{1}{\sqrt{x} - 1}$
    \item $\displaystyle \frac{1}{\sqrt{x} + 1}$
\end{enumerate}

\item Given $a > 0$, solve the equation $a^{x+1} = \sqrt[3]{a^2}$ for $x$ algebraically. %2 points January 2023 Regents

\newpage
\item Solve the equation $\sqrt{49-10x} +5 = 2x$ algebraically. %4 points January 2023 Regents


\end{enumerate}
\end{document}