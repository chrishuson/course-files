\documentclass[12pt, twoside]{article}
% \documentclass[12pt, twoside]{article}
\usepackage[letterpaper, margin=1in, headsep=0.2in]{geometry}
\setlength{\headheight}{0.6in}
%\usepackage[english]{babel}
\usepackage[utf8]{inputenc}
\usepackage{microtype}
\usepackage{amsmath}
\usepackage{amssymb}
%\usepackage{amsfonts}
\usepackage[nomessages]{fp} %\FPeval{\var-name}{2*sin(pi/6)}
\usepackage{siunitx} %units in math. eg 20\milli\meter
\usepackage{yhmath} % for arcs, overparenth command
\usepackage{tikz} %graphics
\usetikzlibrary{quotes, angles, arrows, arrows.meta}
\usepackage{graphicx} %consider setting \graphicspath{{images/}}
\usepackage{parskip} %no paragraph indent
\usepackage{enumitem}
\usepackage{multicol}
\usepackage{venndiagram}

\usepackage{fancyhdr}
\pagestyle{fancy}
\fancyhf{}
\renewcommand{\headrulewidth}{0pt} % disable the underline of the header
\raggedbottom
\hfuzz=2mm %suppresses overfull box warnings

\usepackage{hyperref}
\usepackage{float}

\title{Algebra 2}
\author{Chris Huson}
\date{December 2023}

\fancyhead[LE]{\thepage}
\fancyhead[RO]{\thepage \\ Name: \hspace{4cm} \,\\}
\fancyhead[LO]{BECA / Huson / Algebra 2: Polynomials Jan 2023 Regents \\* 6 May 2025}

\begin{document}

\subsubsection*{Quiz: Regents problems \hfill \emph{N.RN.2 Radicals and rational exponents}}
\begin{enumerate}[itemsep=1.5cm]

\item Rewrite each expression as a radical, simplify.  \vspace{0.25cm}
\begin{multicols}{2}
  \begin{enumerate}[itemsep=1cm]
    \item $\displaystyle 7^{\frac{1}{2}}=$
    \item $\displaystyle (8x)^{-\frac{2}{3}}=$
  \end{enumerate}
  \end{multicols} \vspace{1cm}
  
\item Rewrite each expression as a fractional exponent. $x>0$  \vspace{0.25cm}
\begin{multicols}{2}
  \begin{enumerate}[itemsep=1cm]
      \item $\sqrt[3]{5} =$
      \item $\sqrt[2]{x^3} =$
  \end{enumerate}
  \end{multicols}

\item Given $x > 0$, the expression $\displaystyle \frac{x^{\frac{1}{2}}}{x^{\frac{1}{5}}}$ can be rewritten as %January 2023 Regents
\begin{enumerate}
    \item $\sqrt[3]{x}$
    \item $\sqrt[10]{x^3}$
    \item $\displaystyle \frac{1}{\sqrt[10]{x^3}}$
    \item $\sqrt[3]{x^{10}}$
\end{enumerate}

\item Given $a > 0$, solve the equation $a^{x+2} = \sqrt[2]{a^3}$ for $x$ algebraically. %2 points January 2023 Regents

\newpage
\item Solve the equation $\sqrt{x^2+5x} - 5 = x$ algebraically. %4 points January 2023 Regents


\end{enumerate}
\end{document}