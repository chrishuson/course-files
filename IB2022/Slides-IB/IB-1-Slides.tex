\documentclass{beamer}
\usepackage{geometry}
\usepackage[english]{babel}
\usepackage[utf8]{inputenc}
\usepackage{amsmath}
\usepackage{amsfonts}
\usepackage{amssymb}
\usepackage{tikz}
\usepackage{graphicx}
\usepackage{venndiagram}

%\usepackage{pgfplots}
%\pgfplotsset{width=10cm,compat=1.9}
%\usepackage{pgfplotstable}

\setlength{\headheight}{26pt}%doesn't seem to fix warning

\usepackage{fancyhdr}
\pagestyle{fancy}
\fancyhf{}

%\rhead{\small{5 September 2018}}
\lhead{\small{BECA / Dr. Huson / 11.1 IB Math Unit 1}}

\renewcommand{\headrulewidth}{0pt}

\title{Mathematics Class Slides}
\subtitle{Bronx Early College Academy}
\author{Chris Huson}
\date{13-17 September 2021}

\begin{document}
\frame{\titlepage}
\section[Outline]{}
\frame{\tableofcontents}


\section{1.1 1st day of Geometry, Segment addition, 13 Sept}
\frame
{
  \frametitle{Learning Target: I can measure and diagram my world}
  \framesubtitle{CCSS: HSG.CO.A.1 Know precise geometric definitions \hfill \alert{1.1 Monday 13 Sept}}

  Welcome back to school
  \begin{block}{Do Now: Measurement}
  \begin{enumerate}
      \item Notebook first page: Name / Course / Instructor
      \item Diagram people closest to you and their distance
      \item Early finishers: Calculate diagonal distances
  \end{enumerate}
  \end{block}
  Supply list: Composition book, looseleaf, pencils \& pens, \\*
  compass and ruler; Optional: calculator, folder \\[0.25cm]
  Lesson: Linear functions, slope, solving; vertical line test p 4-6 \\[0.25cm]
  Homework: Diagram your bedroom (with measurements), or another room
}

%Prepare copies of formula sheets

  \section{1.2 Function domain and range}
  \frame
  {
    \frametitle{Learning Target: I can apply domain and range}
    \framesubtitle{CCSS: HSF.IF.C.7 Analyze functions \hfill \alert{1.2 Tuesday 14 Sept}}

    \begin{block}{Do Now: In your notebook}
    \begin{enumerate}
      \item Solve for $x$: \\
      \hspace{1cm} $x-7=11$ \hspace{1cm}  $2(x-5) \geq 4$
      \item What is the slope of the line $y=3x-2$?
      \item $f(x) = x^2 - 3$. Find $f(1)$
    \end{enumerate}
    \end{block}
    Lesson: Domain, range, function review pp 204-8\\[5pt]
    Groupwork: Investigation 1 pp 206-8\\[5pt]
    Homework: Skills Check p 205
  }

  \section{1.5 Problem sets working with functions}
  \frame
  {
    \frametitle{Learning Target: I can employ the language of functions}
    \framesubtitle{CCSS: HSF.IF.C.7 Analyze functions \hfill \alert{1.5 Monday 20 Sept}}

    \begin{block}{Do Now: In your notebook}
    \begin{enumerate}
      \item Solve for $x$: \\
      \hspace{1cm} $2x-9=3$ \hspace{1cm}  $3(x-3) \leq 12$
      \item What is the slope of the line $y=2x-5$?
      \item $f(x) = x^2 +6$. Find $f(2)$
    \end{enumerate}
    \end{block}
    Lesson: Independent and dependent variables\\[5pt]
    Linear equations and function review pp 204-8\\[5pt]
    Groupwork: Exercises 5C pp 220-221\\[5pt]
  }

  \section{1.6 Problem sets working with functions}
  \frame
  {
    \frametitle{Learning Target: I can use functions to model situations}
    \framesubtitle{CCSS: HSF.IF.C.7 Analyze functions \hfill \alert{1.6 Tuesday 21 Sept}}

    \begin{block}{Do Now: Pyramid lifting routine problem 
      (\href{https://www.bodybuilding.com/content/build-muscle-and-strength-with-pyramid-training.html}{Bill Geiger})}
      \begin{center}
          Set 1: 135 lbs, 15 reps\\
          Set 2: 185 lbs, 12 reps\\
          Set 3: 205 lbs, 10 reps\\
          Set 4: 225 lbs, 8 reps\\
          Set 5: 245 lbs, 6 reps\\
          Set 6: 265 lbs, 4 reps
    \end{center}
    \end{block}
    \begin{enumerate}
      \item On the third set, when $x=3$, how much weight is lifted?
      \item On which set is the weight 245 pounds?
      \item Interpret the ordered pair $(2,185)$ in this context.
      \item Does the weight increase by a constant amount with each set?
    \end{enumerate}
    Prequiz handout; 
    Function review pp 204-220
  }

  \section{1.7 Do Now Quiz functions}
  \frame
  {
    \frametitle{Learning Target: I can use functions to model situations}
    \framesubtitle{CCSS: HSF.IF.C.7 Analyze functions \hfill \alert{1.7 Wednesday 22 Sept}}

    \begin{block}{Do Now Quiz}

    \end{block}
    \begin{enumerate}
      \item On the third set, when $x=3$, how much weight is lifted?
      \item On which set is the weight 245 pounds?
      \item Interpret the ordered pair $(2,185)$ in this context.
      \item Does the weight increase by a constant amount with each set?
    \end{enumerate}
    Review simplifying radicals, solving equations with fractions\\
    Function review pp 204-220 \\
    \alert{Test Friday on functions}
  }

  \section{1.8 PreTest review functions}
  \frame
  {
    \frametitle{Learning Target: I can use functions to model situations}
    \framesubtitle{CCSS: HSF.IF.C.7 Analyze functions \hfill \alert{1.8 Thursday 23 Sept}}

    \begin{block}{Do Now: Algebra warmup problems}
      Given the linear function $f(x)=-2x+12$
      \begin{enumerate}
        \item Find $f(0)$ \hspace{3cm}
        2. $f(x)=0$. Find $x$.
    \end{enumerate}
    \end{block}\vspace{3cm}
    Function review pp 204-220.
    \alert{Test tomorrow on functions}
  }

  \section{1.9 Linear models}
  \frame
  {
    \frametitle{Learning Target: I can use linear equations to model situations}
    \framesubtitle{CCSS: HSF.IF.C.7 Analyze functions \hfill \alert{1.9 Monday 27 Sept}}

    \begin{block}{Do Now: Investigation 5 page 221}
        Answer questions 1, 2, and 3 (including the table on page 222)
    \end{block}\vspace{3cm}
    Function test makeup: Sabrina, Qwaa, Sthefani.\\[0.5cm]
    Groupwork: problems 5D page 225-6
  }

  \section{1.9 Linear models}
  \frame
  {
    \frametitle{I can use linear equations to model situations}
    \framesubtitle{Investigation 5 page 221 \hfill \alert{1.9 Monday 27 Sept}}

    \begin{block}{Linear functions:}\vspace{0.5cm}
       $f(x)=2x+1$\\[1cm]
       $g(x)=-3x+2$\\[1cm]
       $h(x)=3$
    \end{block}\vspace{1.5cm}
  }

  \section{1.10 Linear models}
  \frame
  {
    \frametitle{Learning Target: I can use linear equations to model situations}
    \framesubtitle{CCSS: HSF.IF.C.7 Analyze functions \hfill \alert{1.10 Tuesday 28 Sept}}

    \begin{block}{Do Now: Example 6 page 222}
        Compare the two linear models (d) and (e). (formulas page 222)
        \begin{enumerate}
          \item Which has the greater rate of change?
          \item Which has the higher initial value?
        \end{enumerate}
    \end{block}\vspace{0.5cm}
    Function test makeup: Sthefani.\\[0.5cm]
    Lesson: Calculating rate of change (slope or gradient)\\[0.25cm]
    Variables and parameters
    Groupwork: problems 5D page 225-6
  }

  \section{1.11 Linear models}
  \frame
  {
    \frametitle{Learning Target: I can use linear equations to model situations}
    \framesubtitle{CCSS: HSF.IF.C.7 Analyze functions \hfill \alert{1.11 Wednesday 29 Sept}}

    \begin{block}{Do Now: Calculate your mastery score Functions}
        Let $x$ be the number of points correct on \#1-8
        \begin{enumerate}
          \item $\displaystyle f(x)=\frac{x}{10} + 0.33$
          \item $max(1,min(4,f(x)))$
        \end{enumerate}
    \end{block}\vspace{0.5cm}
    Function test review, test corrections\\[0.5cm]
    Lesson: Calculating rate of change (slope or gradient)\\[0.25cm]
    Variables and parameters
    Groupwork: problems 5D page 225-6
  }

  \frame
  {
    \frametitle{Functions mastery score (problems  \#1-8)}

    \begin{block}{Let $x$ be the number of points}
      
      \begin{enumerate}
        \item $\displaystyle f(x)=\frac{x}{10} + 0.33$
        \item $max(1,min(4,f(x)))$
        \item Example, 25 points $\displaystyle f(25)=\frac{25}{10} + 0.33=2.8$
      \end{enumerate}
    \end{block}
    IB test scoring, points:
    \begin{enumerate}
      \item ``A1'' - correct/Accurate value
      \item ``M1'' - proper Method used
      \item  ``R1'' - good Reasoning
      \item  ``N1'' - No work, but partial credit
      \item  ``ft'' - correct, but Following Through on previous errors
    \end{enumerate}

  }

  \section{1.3 Precision and significant figures, 3 Oct}
  \frame
  {
    \frametitle{Learning Target: What is the appropriate precision for a calculation?}
    \framesubtitle{CCSS: MP5 Attend to precision \hfill \alert{2.1 Monday 4 Oct}}

    \begin{block}{Do Now: Textbook chapter warmup, use looseleaf paper}
    \begin{enumerate}
        \item Skills check \#1-3 p. 3
    \end{enumerate}
    \end{block}
    Lesson: Rounding, significant figures, error bars pp. 1-5\\
    Exercise 1A, \#1-2, p. 5
    \\[0.5cm]
    Homework: Calculation and rounding practice
  }

  \section{1.4 Error bounds, 4 Oct}
  \frame
  {
    \frametitle{Learning Target: How do we measure the bounds of errors?}
    \framesubtitle{CCSS: MP5 attend to precision \hfill \alert{2.2 Tuesday 4 Oct}}

    \begin{block}{Do Now: Calculator practice}
    \begin{enumerate}
        \item Chapter review \#1 p. 39
        \item Pay careful attention to saving calculator values, rather than copying to paper and reentering.
        \item Check your answers in back of book, p. 766
    \end{enumerate}
    \end{block}
    Lesson: Bounds and errors pp. 6-8\\ \bigskip
    Practice exercises 1B p. 8-9\\
    Homework: Function substitution, domain and range
  }

  \section{1.5 Exponents \& scientific notation, 5 Oct}
  \frame
  {
    \frametitle{Learning Target: How do we write very large or small numbers?}
    \framesubtitle{CCSS: MP5 attend to precision \hfill \alert{2.3 Wednesday 5 Oct}}

    \begin{block}{Do Now: Precision practice}
    \begin{enumerate}
        \item Practice exercises 1B p. 8-9
        \item Pay careful attention to saving calculator values, rather than copying to paper and reentering.
        \item Check your answers in back of book, p. 765
    \end{enumerate}
    \end{block}
    Lesson: Exponents \& scientific notation pp. 9-12\\ \smallskip
    Note exponent rules top of page 11\\ \smallskip
    Homework: Practice exercises 1C p. 12-13
  }

  \section{1.9 Deltamath: scientific notation, trig 6 Oct}
  \frame
  {
    \frametitle{GQ: How do we practice the law of sines?}
    \framesubtitle{CCSS: MP5 attend to precision \hfill \alert{2.4 Thursday 6 Oct}}

    \begin{block}{Deltamath practice: scientific notation, trig}
      \begin{enumerate}
        \item Laptops, login with Teacher ID \alert{546068}
        \item Do Deltamath sections in order \\
        Practice comes first, then new topics
        \item Work extra problems on the skills you need to practice
    \end{enumerate}
    \end{block}
    New material: The sine formula for the area of a triangle page 22\\
    Radian / degree conversion; law of cosines\\ \smallskip
    Homework: Complete Deltamath problems, 10:00PM deadline
  }


  \section{1.17 review, bounds, 7 Oct}
  \frame
  {
    \frametitle{GQ: How do we calculate the bounds around a value?}
    \framesubtitle{CCSS: MP5 attend to precision \hfill \alert{2.5 Friday 7 Oct}}

    \begin{block}{Do Now Quiz: \href{https://www.smartbmicalculator.com/}{Calculate Body Mass Index} (link)}
      BMI is a measure of a healthy personal weight, $\displaystyle BMI = \frac{w}{h^2}$ \\ \smallskip
      $w$ is a person's weight in kilograms and $h$ is height in meters
      \begin{enumerate} 
          \item Given a height of 170 cm and weight of 77 kg, find the BMI
          \item These measurements are not exact. Assuming the height is between 169-171 cm and weight 76-78 kg, find the bounds of the BMI.
       \end{enumerate}
      \end{block}
    Lesson: Solid geometry, Chapter Summary\\ \smallskip
    Homework: Chapter review 11-17 p. 39-40 (revisit problems)
  }

  \frame
  {
    \frametitle{GQ: How do we calculate the bounds around a value?}
    \framesubtitle{CCSS: MP5 attend to precision}

    \begin{block}{Solution to Do Now Quiz: Calculate Body Mass Index 
      \alert{(7)}}
      \begin{enumerate} 
          \item $\displaystyle BMI = \frac{77}{1.70^2}$ 
          \hspace{3cm} \alert{M1 A1 allow $170^2$}\\ \smallskip
            \hspace{0.8cm} $=26.64359 \dots$ \\
          \hspace{0.8cm} $\approx 26.6$ 
            \hspace{3.2cm} \alert{A1 (N2))}
          \item Lowerbound: $\displaystyle BMI = \frac{76}{1.71^2}$ 
          \hspace{0.8cm} \alert{M1 A1} \\ \smallskip
          \hspace{0.8cm} $=25.9909 \dots$ \\
          \hspace{0.8cm} $\approx 26.0$  \\ \smallskip
          Upperbound: $\displaystyle BMI = \frac{78}{1.69^2}$ \\ \smallskip
          \hspace{0.8cm} $=27.30996 \dots$ \\
          \hspace{0.8cm} $\approx 27.3$  \hspace{3cm} \alert{A1 A1 (N3)} \\
          \hspace{4cm} \alert{award M1 A1 f.t. for 26.6, 26.7}
       \end{enumerate}
      \end{block}
    Proper header with full name, date, and title (5 percentage points)
    
  }

  \frame
  {
    \frametitle{GQ: How do we calculate the bounds around a value?}
    \framesubtitle{CCSS: MP5 attend to precision}

    \begin{block}{Quiz Corrections: Calculate Body Mass Index 
      \alert{required}}
      \begin{itemize} 
          \item Proper header with full name, date, and title
          \item Work downward in single column on left, in pen 
            \\(you can add notes and diagrams on the right)
          \item Skip a line and number the problem
          \item Label to the left of equals sign (e.g. $BMI=$)
          \item Show substitution step
          \item Write the full calculator display (with ellipse)
          \item Show the rounded value, 3 sig-figs (exact value is also ok)
       \end{itemize}
      \end{block}
    Copy this checklist into your notebook
    
  }

\end{document}
