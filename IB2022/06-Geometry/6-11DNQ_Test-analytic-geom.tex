\documentclass[12pt, twoside]{article}
\usepackage[letterpaper, margin=1in, head=30pt, headsep=0.1in]{geometry}
\usepackage[english]{babel}
\usepackage[utf8]{inputenc}
\usepackage{amsmath}
\usepackage{amsfonts}
\usepackage{amssymb}
\usepackage{tikz}
%\usetikzlibrary{quotes, angles}

\usepackage{graphicx}
\usepackage{enumitem}
\usepackage{multicol}

\newif\ifmeta
\metatrue %print standards and topics tags

\title{Regents Geometry}
\author{Chris Huson}
\date{October 2021}

\usepackage{fancyhdr}
\pagestyle{fancy}
\fancyhf{}
\renewcommand{\headrulewidth}{0pt} % disable the underline of the header
\raggedbottom


\fancyhead[LE]{\thepage}
\fancyhead[RO]{\thepage \\ Name: \hspace{4cm} \,\\}
\fancyhead[LO]{BECA / Dr. Huson / IB Math 6 Geometry}

\begin{document}
\subsubsection*{6.11 Do Now Quiz: Parallel and perpendicular lines}
\begin{enumerate}
\item Write down the slope perpendicular to the given slope. \vspace{0.5cm}
  \begin{enumerate}
    \begin{multicols}{2}
    \item   $m= \frac{5}{2} \hspace{1cm} m_{\perp} = $ \vspace{1cm}
    \item   $m= -\frac{1}{2} \hspace{1cm} m_{\perp} = $
    \item   $m= -\frac{7}{3} \hspace{1cm} m_{\perp} = $ \vspace{1cm}
    \item   $m=  5 \hspace{1cm} m_{\perp} = $
    \end{multicols}
  \end{enumerate} \vspace{0.5cm}

\item The line $l$ has the equation $y=\frac{4}{3}x-11$. To each line below, circle whether $l$ is parallel, perpendicular, or neither.
  \begin{enumerate}
    \item parallel \quad perpendicular \quad neither \qquad $y=-\frac{4}{3}x+11$
    \vspace{0.25cm}
    \item parallel \quad perpendicular \quad neither \qquad $y=-\frac{3}{4}x+4$
    \vspace{0.25cm}
    \item parallel \quad perpendicular \quad neither \qquad $3x+4y=12$
    \vspace{1.5cm}
    \item parallel \quad perpendicular \quad neither \qquad $4x-3y=6$
    \vspace{1.7cm}
  \end{enumerate}

In the following problems, use the point-slope formula: $y-y_A=m (x-x_A)$
  \item What is the equation of a line through the point $A(-5,7)$ and parallel to the line $y=2x-12$?  \vspace{1.5cm}

  \item What is an equation of the perpendicular bisector of $\overline{QR}$ with $Q(-2,1)$ and $R(6,5)$? %\vspace{5cm}

\end{enumerate}
\end{document}
