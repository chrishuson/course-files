\documentclass[12pt, twoside]{article}
\usepackage[letterpaper, margin=1in, headsep=0.5in]{geometry}
\usepackage[english]{babel}
\usepackage[utf8]{inputenc}
\usepackage{amsmath}
\usepackage{amsfonts}
\usepackage{amssymb}
\usepackage{tikz}
\usetikzlibrary{quotes, angles}
\usepackage{graphicx}
\usepackage{enumitem}
\usepackage{multicol}
\usepackage{hyperref}

\newif\ifmeta
\metatrue %print standards and topics tags

\title{IB Mathematics}
\author{Chris Huson}
\date{September 2021}

\usepackage{fancyhdr}
\pagestyle{fancy}
\fancyhf{}
\renewcommand{\headrulewidth}{0pt} % disable the underline of the header
\raggedbottom


\fancyhead[LE]{\thepage}
\fancyhead[RO]{\thepage \\ Name: \hspace{4cm} \,\\}
\fancyhead[LO]{BECA / IB Math 01-Linear functions\\* 1 October 2021}

\begin{document}

\subsubsection*{1.12: I can model with direct variation}
Equations of a straight line: $f(x)=mx+c$, $ax+by+d=0$, $(y-y_1)=m(x-x_1)$
\begin{enumerate}
\item A linear function is such that $f(1)=5$ and $f(5)=1$.
\begin{enumerate}[itemsep=1.5cm]
  \item Name two of the function's points as ordered pairs.
  \item Find the gradient (slope) for the function $f$
  \item Substitute the slope and one point into the formula $f(x)=mx+c$
  \item Solve for the $y$-intercept
  \item Find $f(-3)$
\end{enumerate} \vspace{1cm}

\item A runner begins training and her times are gradually improving with each week. Four weeks into her program she can run three miles in 22 minutes. After eight weeks it only takes her $20 \frac{1}{2}$ minutes.\begin{enumerate}
  \item Model her running progress with a linear model.\vspace{3cm}
  \item If her goal is to beat 20 minutes, use your model to predict when she will achieve it.
\end{enumerate}

\newpage
\item Given the direct variation (and also a linear function) $f(x)=2x$. \hfill [6]
\begin{enumerate}
  \begin{multicols}{2}
    \item Find $f(3)$
    \item   $f(x)=10$. Find $x$.
  \end{multicols}\vspace{2cm}
  \begin{multicols}{2}
      \item Plot the answers to the first two parts, (a) and (b), as points on the grid and label them as ordered pairs. 
      \item Draw a straight line through the points to represent the function.
      \item What is the constant of proportionality? \vspace{2cm}
      \begin{center}
      \begin{tikzpicture}[xscale=0.7, yscale=0.4]
        %\draw [help lines] (-3,-2) grid (4,6);
        \draw [thick, ->] (-1.2,0) -- (7.4,0) node [below right] {$x$};
        \draw [thick, ->] (0,-1.2)--(0,13.5) node [right] {$y$};
        \foreach \x in {-1, 1,2, ..., 7} \draw (\x cm,4pt) -- (\x cm,-4pt) node[anchor=north] {$\x$};
        \foreach \y in {2, 4, ..., 12} \draw (2pt,\y cm) -- (-2pt,\y cm) node[anchor=east] {$\y$};
      \end{tikzpicture}
      \end{center}
    \end{multicols}
\end{enumerate}

\item The gasoline used by a car is the function of the distance driven in miles, as shown in the table.
\begin{center}
  \begin{tabular}{|l|r|r|r|r|r|r|}
    \hline
    Distance (miles) & 10 & 20 & 40 & 50 & 200 & 500\\ 
    \hline 
    Gas (gallons) & 0.5 & 1 & 2 & 2.5 & 10 & 25\\ 
    \hline 
  \end{tabular}
\end{center}
\begin{enumerate}[itemsep=1cm]
  \item Is gas usage a linear function of distance driven? Explain.
  \item Is it a direct variation?
  \item What is the gradient? 
  \item What is the gas mileage in terms of miles per gallon?
  \item Discuss which is the independent and dependent variables.
\end{enumerate}

\end{enumerate}
\end{document}