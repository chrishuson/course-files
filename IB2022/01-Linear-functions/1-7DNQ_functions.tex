\documentclass[12pt, twoside]{article}
\usepackage[letterpaper, margin=1in, headsep=0.5in]{geometry}
\usepackage[english]{babel}
\usepackage[utf8]{inputenc}
\usepackage{amsmath}
\usepackage{amsfonts}
\usepackage{amssymb}
\usepackage{tikz}
\usetikzlibrary{quotes, angles}
\usepackage{graphicx}
\usepackage{enumitem}
\usepackage{multicol}
\usepackage{hyperref}

\newif\ifmeta
\metatrue %print standards and topics tags

\title{IB Mathematics}
\author{Chris Huson}
\date{September 2021}

\usepackage{fancyhdr}
\pagestyle{fancy}
\fancyhf{}
\renewcommand{\headrulewidth}{0pt} % disable the underline of the header
\raggedbottom


\fancyhead[LE]{\thepage}
\fancyhead[RO]{\thepage \\ Name: \hspace{4cm} \,\\}
\fancyhead[LO]{BECA / IB Math 01-Linear functions\\* 22 September 2021}

\begin{document}

\subsubsection*{1.7 Do Now Quiz: Functions}
\begin{enumerate}
  \item More on the pyramid workout routine: Let $x$ be the set number with the number of repetitions (``reps") a function of $x$.
  \begin{center}
      Sample Bench Press Pyramid 
      (\href{https://www.bodybuilding.com/content/build-muscle-and-strength-with-pyramid-training.html}{Bill Geiger})\\
        Set 1: 135 lbs, 14 reps\\
        Set 2: 185 lbs, 12 reps\\
        Set 3: 205 lbs, 10 reps\\
        Set 4: 225 lbs, 8 reps\\
        Set 5: 245 lbs, 6 reps\\
        Set 6: 265 lbs, 4 reps
  \end{center}
\begin{enumerate}[itemsep=0.5cm]
  \item How many reps are planned for the second set, when $x=2$?
  \item Which set has the fewest reps? \\(express your answer in the form $x=$ a number)
  \item Explain what the ordered pair $(4,8)$ would refer to in this context.\vspace{1.5cm}
  \item Do the reps increase by a constant amount with each set? Explain. \\(If so, what is the slope, or rate of change?) \vspace{1.5cm}
\end{enumerate}

\item Consider the function $f(x)=50 - 10x$.
\begin{enumerate}
  \item Write down the independent variable.
  \item Calculate $f(1)$ \vspace{1cm}
  \item Show that $f(3.5)= 15$ \vspace{1.5cm}
  \item There is an $x$ for which $f(x)= -80$. \\ Find this value of $x$.
\end{enumerate} \vspace{2cm}

\newpage
\subsubsection*{Early finishers}
\item In the following two problems, solve for the value of $x$.
  \begin{multicols}{2}
    \begin{enumerate}
      \item   $\displaystyle \frac{1}{2}x-5=3 \frac{1}{2}$ \vspace{6cm}
      \item   $\displaystyle 4x-\frac{3}{4}=3+\frac{1}{4}x$  \vspace{6cm}
    \end{enumerate}
  \end{multicols}
    \vspace{3cm}

\item Given the linear function $f(x)=-\frac{2}{3}x+4$.
\begin{enumerate}
  \begin{multicols}{2}
    \item Find $f(0)$ \vspace{6cm}
    \item   $f(x)=0$. Find $x$. \vspace{6cm}
  \end{multicols}\vspace{4cm}
  \begin{multicols}{2}
      \item Plot the answers to the first two parts, (a) and (b), as points on the grid and label them as ordered pairs. 
      \item Draw a straight line through the points to represent the function.
      \item Which answer, (a) or (b), is the $x$-intercept. Which is the $y$-intercept?
      \begin{center}
      \begin{tikzpicture}[scale=0.8]
        %\draw [help lines] (-3,-2) grid (4,6);
        \draw [thick, ->] (-1.2,0) -- (7.4,0) node [below right] {$x$};
        \draw [thick, ->] (0,-1.2)--(0,6.4) node [left] {$y$};
        \foreach \x in {-1, 1,2, ..., 7} \draw (\x cm,1pt) -- (\x cm,-1pt) node[anchor=north] {$\x$};
        \foreach \y in {1, 2, 3, 4, 5} \draw (1pt,\y cm) -- (-1pt,\y cm) node[anchor=east] {$\y$};
      \end{tikzpicture}
      \end{center}
    \end{multicols}
\end{enumerate}

\item Simplify each expression. (Leave it in radical form if necessary, not a decimal.)
\begin{enumerate}
  \begin{multicols}{2}
  \item   $\sqrt{81}$ \vspace{1.5cm}
  \item   $\sqrt{27}$ \vspace{1.5cm}
  \end{multicols}
\end{enumerate}
\vspace{0.5cm}

\end{enumerate}
\end{document}