\documentclass{beamer}
\usepackage{geometry}
\usepackage[english]{babel}
\usepackage[utf8]{inputenc}
\usepackage{amsmath}
\usepackage{amsfonts}
\usepackage{amssymb}
\usepackage{tikz}
\usetikzlibrary{quotes, angles}
\usepackage{graphicx}
\usepackage{multicol}
%\usepackage{pgfplots}
%\pgfplotsset{width=10cm,compat=1.9}
%\usepackage{pgfplotstable}

\setlength{\headheight}{26pt}%doesn't seem to fix warning

\usepackage{fancyhdr}
\pagestyle{fancy}
\fancyhf{}

%\rhead{\small{24 March 2019}}
\lhead{\small{BECA / Dr. Huson / Geometry - Unit 13: Regents Review}}

\renewcommand{\headrulewidth}{0pt}

\title{10th Grade Geometry - Unit 13: Regents Review}
\subtitle{Bronx Early College Academy}
\author{Christopher J. Huson PhD}
\date{27 May 2019}

\begin{document}
\frame{\titlepage}
\section[Outline]{}
\frame{\tableofcontents}


\section{13.1 Scale \& applications of dilation Tuesday 28 May}
  \frame
  {
    \frametitle{GQ: How do we use scale factors?}
    \framesubtitle{CCSS: HSG.CO.D.12 Congruence, geometric constructions \hfill \alert{13.1 Tuesday 28 May}}

    \begin{block}{Do Now: Handout}
      \begin{enumerate}
        \item Using scale factors
        \item Real world situations
      \end{enumerate}
    \end{block}
    Guest teacher, Mr. Segal. Applications of scale factors in finance.\\[0.25cm]
    Homework: Problem set, test corrections due Thursday
  }

  \frame
  {
    \frametitle{GQ: How do we use scale factors?}
    \framesubtitle{Triangle similarity: for your notebook}
      Given  \[\triangle ABC \sim \triangle DEF \]
      Equivalently \[ \triangle ABC \rightarrow \triangle DEF \]
      Complete the three line segment correspondences, three scale factor ratios, \& three dilations.\\[0.5cm]
        \begin{multicols}{3}
          \renewcommand{\baselinestretch}{1.5}
          \begin{enumerate}
            \item $\overline{AB} \rightarrow \overline{DE} $
            \item $\overline{BC} \rightarrow$
            \item $\overline{AC} \rightarrow$
          \end{enumerate}
          \begin{enumerate}
            \item $k= \frac{DE}{AB}$
            \item $k=$
            \item $k=$
          \end{enumerate}
          \begin{enumerate}
            \item $DE= k \times AB$
            \item $EF= k \times $
            \item $DF= k \times $
          \end{enumerate}
        \end{multicols}

  What happens if $k=1$?
  }

\section{13.2 Similarity review Wednesday 29 May}
  \frame
  {
    \frametitle{GQ: How do we use scale factors?}
    \framesubtitle{CCSS: HSG.CO.D.12 Congruence, geometric constructions \hfill \alert{13.2 Wednesday 29 May}}

    \begin{block}{Do Now: Quadrilateral properties}
      \begin{enumerate}
        \item Given a list of features, identify the applicable quadrilateral
        \item Early finishers: Triangle congruency proofs
      \end{enumerate}
    \end{block}
    ASA proof of a parallelogram's congruent triangles, implications\\
    Pretest packet: volume, trig, analytic geometry
  }

\section{13.3 Similarity review Thursday 30 May}
  \frame
  {
    \frametitle{GQ: How do we use scale factors?}
    \framesubtitle{CCSS: HSG.CO.D.12 Congruence, geometric constructions \hfill \alert{13.3 Thursday 30 May}}

    \begin{block}{Do Now: Quadrilateral properties}
      \begin{enumerate}
        \item Given a list of features, identify the applicable quadrilateral
        \item Early finishers: Triangle congruency proofs
      \end{enumerate}
    \end{block}
    Notebook check: trapezoid area (10.1)
    Review for test\\[0.5cm]
    Homework: Study for \alert{exam tomorrow}
  }

\section{13.4 Similarity review Friday 31 May}
  \frame
  {
    \frametitle{GQ: How do we use scale factors?}
    \framesubtitle{CCSS: HSG.CO.D.12 Congruence, geometric constructions \hfill \alert{13.4 Friday 31 May}}

    Do Now handout: Volume, density, \& trig problems\\
    Classwork review
    \begin{block}{Assessment: Exam}
      \begin{enumerate}
        \item Sector areas and arc lengths; compound areas
        \item Volume formulas, compound shapes, density problems
        \item Unit conversions, rounding
        \item Trigonometric situations
        \item Solving for a missing input given a formula result
    \end{enumerate}
    \end{block}
    Quadrilateral packet project grade\\[0.5cm]
    Homework: packet
  }

\end{document}
