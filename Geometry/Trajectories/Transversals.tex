\documentclass[12pt, oneside]{article}
\usepackage[letterpaper, margin=1in, headsep=0.5in]{geometry}
\usepackage[english]{babel}
\usepackage[utf8]{inputenc}
\usepackage{amsmath}
\usepackage{amsfonts}
\usepackage{amssymb}
\usepackage{tikz}
\usetikzlibrary{quotes, angles}
\usepackage{graphicx}
%\usepackage{pgfplots}
%\pgfplotsset{width=10cm,compat=1.9}
%\usepgfplotslibrary{statistics}
%\usepackage{pgfplotstable}
%\usepackage{tkz-fct}
%\usepackage{venndiagram}

\usepackage{fancyhdr}
\pagestyle{fancy}
\fancyhf{}
\rhead{\thepage \\Name: \hspace{1.5in}.\\}
\lhead{BECA / Dr. Huson / Geometry 10th Grade\\* Learning trajectory: Transversals and parallel lines}

\renewcommand{\headrulewidth}{0pt}

\begin{document}
\subsubsection*{Transversals and parallel lines}
  \begin{enumerate}
  \item Corresponding angles
  \item Alternate interior
  \item Situations and configurations
  \begin{enumerate}
    \item Triangle $180^\circ$ sum of internal angles
    \item Parallelograms in both directions
    \item Triangle midlines
    \end{enumerate}
  \end{enumerate}

  \begin{enumerate}
    \subsubsection*{Corresponding angles}
    \item Given two parallel lines and a transversal, as shown. Apply the theorem ``If a transversal intersects two parallel lines, then corresponding angles are congruent."
    \begin{center}
    \begin{tikzpicture}
      \draw [<->, thick] (1,2)--(9,2);
      \draw [<->, thick] (0,0)--(8,0);
      \draw [<->, thick] (4,-1)--(5.5,3);
      \node at (4.5,0.3) [left]{$5$};
      \node at (4.5,0.3) [right]{$6$};
      \node at (4.3,-0.3) [left]{$7$};
      \node at (4.3,-0.3) [right]{$8$};
      \node at (5.2,2) [above left]{$1$};
      \node at (5.2,2) [above right]{$2$};
      \node at (5,2) [below left]{$3$};
      \node at (5,2) [below right]{$4$};
    \end{tikzpicture}
    \end{center}
      \begin{enumerate}
        \item State the angle corresponding with $\angle 7$. \bigskip
        \item Given $m\angle 6 = 80^\circ$ and $m\angle 2 = 2x^\circ$. Find $x$. \bigskip
        \item Given $m\angle 5 = 100^\circ$. Find $m\angle 3$.
      \end{enumerate}


  \end{enumerate}

\end{document}
