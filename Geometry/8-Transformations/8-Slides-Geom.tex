\documentclass{beamer}
\usepackage{geometry}
\usepackage[english]{babel}
\usepackage[utf8]{inputenc}
\usepackage{amsmath}
\usepackage{amsfonts}
\usepackage{amssymb}
\usepackage{tikz}
\usetikzlibrary{quotes, angles}
\usepackage{graphicx}

%\usepackage{pgfplots}
%\pgfplotsset{width=10cm,compat=1.9}
%\usepackage{pgfplotstable}

\setlength{\headheight}{26pt}%doesn't seem to fix warning

\usepackage{fancyhdr}
\pagestyle{fancy}
\fancyhf{}

%\rhead{\small{4 March 2019}}
\lhead{\small{BECA / Dr. Huson / Geometry - Unit 8 Transformational Geometry}}

\renewcommand{\headrulewidth}{0pt}

\title{10th Grade Geometry - Unit 8: Transformational Geometry}
\subtitle{Bronx Early College Academy}
\author{Christopher J. Huson PhD}
\date{4 March 2019}

\begin{document}
\frame{\titlepage}
\section[Outline]{}
\frame{\tableofcontents}


\section{Laptops - Geogebra class codes}
  \frame
  {
    \frametitle{GQ: How do we model with digital tools?}
    \framesubtitle{CCSS: HSG.CO.D.12 Congruence, geometric constructions \hspace{\stretch{1}} \alert{7.1 Tuesday 18 January}}

    GeoGebra Geometry App\\
    Enter \alert{N7BHK} for 10.1 or \alert{P9PNZ} for 10.2\\
    Set up account using your real name.\\
    Beginner Tutorials with Lesson Ideas\\
    Author: Tim Brzezinski\\[0.5cm]
    Homework: Complete Geogebra
  }

\section{7.10 Geogegra transformations intro. Tuesday 12 February}
  \frame
  {
    \frametitle{GQ: How do we apply translations to functions?}
    \framesubtitle{CCSS: HSG.CO.D.12 Congruence, geometric constructions \hfill \alert{7.10 Tuesday 12 February}}

    \begin{block}{Geogebra project: Create a transformations puzzle problem}
      \begin{enumerate}
        \item Start with a polygon
        \item Use Geogebra's tranformations tools
        \item List the transformation steps you used
        \item Rubric: correct, aesthetics, MLA
        \item Print out a color pdf to email me. (husonbeca@gmail.com)
      \end{enumerate}
    \end{block}
    Lesson: Geogebra tool pallette\\[0.5cm]
    Homework: Practice problems
  }

\section{7.15 Geogegra median partition 2:1 ratio. Tuesday 26 February}
  \frame
  {
    \frametitle{GQ: How do we use technology to explore geometric relationships?}
    \framesubtitle{CCSS: MP5 Use appropriate tools strategically: dynamic geometry software \hfill \alert{7.15 Tuesday 26 February}}

    Do Now: Practice analytic geometry skills on handout
    \begin{block}{Lesson: Geogebra project to measure the division of a median of a triangle by the centroid}
      \begin{enumerate}
        \item Start with a triangle, connect two midpoints and medians, intersecting at the centroid
        \item Use Geogebra's measurement tools
        \item Explain the resulting 2:1 ratio using text and symbols
        \item Assessment rubric: correct, aesthetics, MLA
        \item Print out a color pdf to email me. (husonbeca@gmail.com)
      \end{enumerate}
    \end{block}
    Homework: Pretest packet due Thursday \alert{(test Friday)}
  }

\section{8.1 Geogegra - Transformations project Tuesday 6 March}
  \frame
  {
    \frametitle{GQ: How do we use technology to explore geometric relationships?}
    \framesubtitle{CCSS: MP5 Use appropriate tools strategically: dynamic geometry software \hfill \alert{8.1 Tuesday 6 March}}

    \begin{block}{Lesson: Geogebra project showing various transformations}
      \begin{enumerate}
        \item Apply transformations to polygons (show at least two)
        \item Use Geogebra's formating tools
        \item Label with the transformation's specifics (e.g. center, factor)
        \item Rubric: correct, aesthetics, \alert{MLA \& email standards}
        \item Export a .png to email me. (husonbeca@gmail.com)
        \item Filename: Last-Title.png, email subject line message
      \end{enumerate}
    \end{block}
    \alert{Parent conferences this Thursday evening, Friday afternoon}\\
    Homework: Test corrections  (due tomorrow)
  }

\section{8.2 Dilation and similar triangles. Wednesday 7 March}
  \frame
  {
    \frametitle{GQ: How do we transform objects on the coordinate plane?}
    \framesubtitle{CCSS: HSG.CO.D.12 Congruence, geometric constructions \hfill \alert{8.2 Wednesday 7 March}}

    \begin{block}{Do Now Plotting transformations review review}
      \begin{enumerate}
        \item Handout
      \end{enumerate}
    \end{block}
    Lesson: Translation, reflection, rotation, dilation, composition, properties\\[0.5cm]
    Homework: Practice problems handout
  }

\section{8.3 Dilation and similar triangles. Thursday 8 March}
  \frame
  {
    \frametitle{GQ: How do we transform objects on the coordinate plane?}
    \framesubtitle{CCSS: HSG.CO.D.12 Congruence, geometric constructions \hfill \alert{8.3 Thursday 8 March}}

    \begin{block}{Do Now Analytic geometry review}
      \begin{enumerate}
        \item Point-slope form of linear equations
        \item Applications of slope, graphing linear equations
        \item The equation of a circle, deriving center and radius
      \end{enumerate}
    \end{block}
    Lesson: Midlines, medians, the centroid. Measuring with Geogebra, submissions standards\\[0.5cm]
    Homework: Practice problems handout
  }

  Pi Day


\section{8.4 Symmetry, "onto" transformations. Monday 11 March}
  \frame
  {
    \frametitle{GQ: How do we say that objects are mapped "onto" themselves?}
    \framesubtitle{CCSS: HSG.CO.D.12 Congruence, geometric constructions \hfill \alert{8.4 Monday 11 March}}

    \begin{block}{Do Now Analytic geometry practice}
      \begin{enumerate}
        \item Point-slope form of linear equations
        \item Applications of slope, graphing linear equations
        \item The equation of a circle, deriving center and radius
      \end{enumerate}
    \end{block}
    Lesson: SSS Similarity;
    \\Symmetry in terms of tranformations \emph{onto} oneself\\[0.5cm]
    Homework: Practice problems handout
  }

\section{8.5 Geogegra - Transformations project Tuesday 12 March}
  \frame
  {
    \frametitle{GQ: How do we use technology to explore geometric relationships?}
    \framesubtitle{CCSS: MP5 Use appropriate tools strategically: dynamic geometry software \hfill \alert{8.5 Tuesday 12 March}}

    \begin{block}{Lesson: Geogebra project showing various transformations}
      \begin{enumerate}
        \item Apply transformations to polygons (show at least two)
        \item Use Geogebra's formating tools
        \item Label with the transformation's specifics (e.g. center, factor)
        \item Rubric: correct, aesthetics, \alert{MLA \& email standards}
        \item Export a .png to email me. (husonbeca@gmail.com)
        \item Filename: Last-Title.png, email subject line message
      \end{enumerate}
    \end{block}
    \alert{Parent conferences this Thursday evening, Friday afternoon}\\
    Homework: Test corrections  (due tomorrow)
  }

\end{document}
