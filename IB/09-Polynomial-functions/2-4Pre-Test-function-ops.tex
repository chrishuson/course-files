\documentclass{article}
\usepackage[english]{babel}
\usepackage[utf8x]{inputenc}
\usepackage{amsmath}
\usepackage{amsfonts}
\usepackage{amssymb}
\usepackage{tikz}
\usepackage{graphicx}


\title{11.1 Tests}
\author{chris }
\date{September 2018}

\begin{document}


\noindent BECA / Huson / 11.1 IB Math SL \qquad \qquad Name:\\
26 September 2018
\subsection*{Chapter 1 Pre-Test: Function Operations}

\begin{enumerate}

\item For the function $f(x) = 2x-7$
    \begin{itemize}
        \item[(a)] What is the value of $f(3)$?
        \item[(b)] Solve for $x$ if $f(x)=0$.
        \item[(c)] Find  $f(1-x)$.
        \item[(d)] Find the inverse of $f(x)$,  $f^{-1}(x)$.
    \end{itemize}

\item For the function $g(x) = x^2-4$ with $x>0$
\begin{itemize}
    \item[(a)] Simplify the expression $g(x-3)$
	\item[(b)] Find  $g^{-1}(x)$.
\end{itemize}

\item For the functions $f(x) = 2-x^2$ and $g(x) = 2x-5$
\begin{itemize}
    \item[(a)] What is the value of $g(3)$?
	\item[(b)] Find $(f\circ g)(3)$.
	\item[(c)] Find $(f\circ g)(x)$.
\end{itemize}

\item Find the inverse of $\displaystyle f(x)= \frac {4x-2}{5}$

\item Given that $g(x) = \frac {1}{3} x+2$
\begin{itemize}
    \item[(a)] Find the inverse of $g(x)$.
	\item[(b)] Graph the function $g(x)$ and its inverse on the same axes, using the scale 1 unit equals 1 cm and labeling following IB conventions.
\end{itemize}

\item For the functions defined by $f(x) = 2x$ and $g(x) = x+4$
\begin{itemize}
    \item[(a)] Find an expression for $(f\circ g)(x)$.
	\item[(b)] Find an expression for $(g\circ f)(x)$.
	\item[(c)] Solve $(f\circ g)(x)=(g\circ f)(x)$.
\end{itemize}

\item Write down the domain and range of $f(x)= x^2-6$

\item Using a GDC to analyze the function $\displaystyle f(x)= \frac {3x+2}{x+1}$
\begin{itemize}
    \item[(a)] Write down the equations for the asymptotes.
	\item[(b)] Write down the domain and range of $f(x)$.
\end{itemize}

\item Consider the function $f(x) = x^3 - 4x^2 - 3x + 18$.
\begin{itemize}
    \item[(a)] Find the values of $f(x)$ for $a$ and $b$ in the table below:\\
	\begin{tabular}{|l|c|c|c|c|c|c|c|c|c|}
	\hline
	$x$ & -3 & -2 & -1 & 0 & 1 & 2 & 3 & 4 & 5\\
	\hline
    $f(x)$ & -36 & $a$ & 16 & $b$ & 12 & 4 & 0 & 6 & 28\\
	\hline
	\end{tabular}
	\item[(b)] Using a scale of 1 cm for each unit on the $x$-axis and 1 cm for each 5 units on the $y$-axis, draw the graph of $f(x)$ for $-3 \leq x \leq 5$. Label it clearly using IB conventions on the graph paper provided (other side).
\end{itemize}
\end{enumerate}

\end{document}
